\documentclass[a4paper, 11pt, oneside, german]{article}
\usepackage{gfsbaskerville}
% Load encoding definitions (after font package)
\usepackage[LGR,T1]{fontenc}
\usepackage{textalpha}
\usepackage{graphicx}
\graphicspath{ {./} }
\usepackage[figurename=]{caption}
\usepackage{listings}
\usepackage{subcaption}
\lstset{basicstyle=\ttfamily}
% Babel package:
\usepackage{babel}
\usepackage{cjhebrew}
\usepackage{yfonts}
% With XeTeX/LuaTeX, load fontspec after babel to use Unicode
% fonts for Latin script and LGR for Greek:
\ifdefined\luatexversion \usepackage{fontspec}\fi
\ifdefined\XeTeXrevision \usepackage{fontspec}\fi

% "`Lipsiakos" italic font `cbleipzig`:
\newcommand*{\lishape}{\fontencoding{LGR}\fontfamily{cmr}%
		       \fontshape{li}\selectfont}
\DeclareTextFontCommand{\textli}{\lishape}

\usepackage{amssymb}
\usepackage{booktabs}
\setlength{\emergencystretch}{15pt}
\usepackage{fancyhdr}
\usepackage{microtype}
\usepackage[titles]{tocloft}
\usepackage{sectsty}

\allsectionsfont{\frakfamily}
\sectionfont{\frakfamily\Huge}
\subsectionfont{\frakfamily\LARGE}
\subsubsectionfont{\frakfamily\LARGE}
\paragraphfont{\frakfamily\LARGE}

\usepackage{graphicx}
\graphicspath{ {./} }
\begin{document}
\frakfamily
\renewcommand{\contentsname}{
\frakfamily{Inhaltsverzeichnis}
}

\renewcommand{\cftsecfont}{\frakfamily}
\renewcommand{\cftsubsecfont}{\frakfamily}
\renewcommand{\cftsubsubsecfont}{\frakfamily}

% fix toc page numbers
\let\origcftsecfont\cft
\let\origcftsecpagefont\cftsecpagefont
\let\origcftsecafterpnum\cftsecafterpnum
\renewcommand{\cftsecpagefont}{\frakfamily{\origcftsecpagefont}}
\renewcommand{\cftsecafterpnum}{\frakfamily{\origcftsecafterpnum}}
\let\origcftsubsecpagefont\cftsubsecpagefont
\let\origcftsubsecafterpnum\cftsubsecafterpnum
\renewcommand{\cftsubsecpagefont}{\frakfamily{\origcftsubsecpagefont}}
\renewcommand{\cftsubsecafterpnum}{\frakfamily{\origcftsubsecafterpnum}}
\let\origcftsubsubsecpagefont\cftsubsubsecpagefont
\let\origcftsubsubsecafterpnum\cftsubsubsecafterpnum
\renewcommand{\cftsubsubsecpagefont}{\frakfamily{\origcftsubsubsecpagefont}}
\renewcommand{\cftsubsubsecafterpnum}{\frakfamily{\origcftsubsubsecafterpnum}}

\renewcommand\thefootnote{\frakfamily{\arabic{footnote}}}

\begin{titlepage} % Suppresses headers and footers on the title page
	\centering % Centre everything on the title page
	%\scshape % Use small caps for all text on the title page

	%------------------------------------------------
	%	Title
	%------------------------------------------------
	
	\rule{\textwidth}{1.6pt}\vspace*{-\baselineskip}\vspace*{2pt} % Thick horizontal rule
	\rule{\textwidth}{0.4pt} % Thin horizontal rule
	
	\vspace{1.5\baselineskip} % Whitespace above the title
	
	{\scshape\huge Beitr"age zur Geschichte und Kenntnis }
	
	\vspace{1\baselineskip} % Whitespace after the title block

	{\scshape\huge meteorischer Stein- und Metall-Massen, }

	\vspace{1\baselineskip} % Whitespace after the title block

 	{\scshape\huge und der Erscheinungen, welche deren }

	\vspace{1\baselineskip} % Whitespace after the title block

	{\scshape\huge Niederfallen zu begleiten pflegen.}

	\vspace{1.5\baselineskip} % Whitespace above the title

	\rule{\textwidth}{0.4pt}\vspace*{-\baselineskip}\vspace{3.2pt} % Thin horizontal rule
	\rule{\textwidth}{1.6pt} % Thick horizontal rule
	
	\vspace{1\baselineskip} % Whitespace after the title block
	
	%------------------------------------------------
	%	Subtitle
	%------------------------------------------------
	
	{\scshape Von D. Karl von Schreibers,} % Subtitle or further description
	
	\vspace*{1\baselineskip} % Whitespace under the subtitle
	
        {\scshape\footnotesize der "osterreichischen Erblande Ritter und Landstande in Nieder-"Osterreich, k. k. Rate und Direktor der Hof-Naturalien-Kabinette, Mitgliede der medizinischen Fakult"at und der k. k. Landwirtschafts-Gesellschaft in Wien; der k"onigl. Akademie der Wissenschaften zu M"unchen; der k"onigl. Gesellschaft der Wissenschaften zu G"ottingen; der ehemals kaiserl. Leopoldinisch-Karolinischen Akademie der Naturforscher zu Bonn; der k"onigl. Akademie n"utzlicher Wissenschaften zu Erfurt; der Soziet"at f"ur National-Industrie und der philomatsichen Gesellschaft zu Paris; der Gesellschaft f"ur K"unste und Wissenschaften zu Lille; der kaiserl. Gesellschaft der Naturforscher zu Moscow; der Gesellschaft naturforschender Freunde zu Berlin, und der naturforschenden Gesellschaften zu Jena, Leipzig, Hanau, Marburg; der mineralogischen Gesellschaften zu Jena, Petersburg, Dresden; der Werner'schen Soziet"at f"ur Naturkunde zu Edinburgh; der physisch-medizinischen zu Erlangen und der pharmazeutischen zu St. Petersburg; die niederrheinischen Gesellschaft f"ur Natur- und Heilkunde zu Bonn; der Soziet"at f"ur Forst- und Jagdkunde zu Dreissigacker, u. s. w. Mitgliede, und der mineralogischen Soziet"at zu Jena ordentlichem Assessor.} % Subtitle or further description

        \begin{figure}[h!]
            \centering
            \includegraphics[scale=0.5,keepaspectratio]{Figures/cover.png}
        \end{figure}
        
	%------------------------------------------------
	%	Editor(s)
	%------------------------------------------------
        \vspace*{\fill}

	{\scshape Wien. 1820.}
	
	{\scshape{Im Verlage von J. G. Heubner.}}
	
	\vspace{0.5\baselineskip} % Whitespace after the title block

    \scshape Internet Archive Online Edition  % Publication year
	
	{\scshape Namensnennung Nicht-kommerziell Weitergabe unter gleichen Bedingungen 4.0 International} % Publisher
\end{titlepage}
\setlength{\parskip}{1mm plus1mm minus1mm}
\clearpage
\tableofcontents
\clearpage
\vspace*{\fill}
{\begin{center}
Segnius irritant animos demissa per aures,\\
Quam quae sunt oculis subjecta fidelibus.\\
Horat.
\end{center}}
\vspace*{\fill}
\clearpage
\LARGE
\pagestyle{fancy}
\fancyhf{}
\cfoot{\frakfamily{\frakfamily{\thepage}}}
\section*{\frakfamily{Vorrede.}}
\paragraph{}
Die seltene Gelegenheit, sich von der Realit"at und den n"ahern Umst"anden eines ebenso wunderbaren als lange und vielfach bestrittenen Naturereignisses --- eines so genannten Steinregens pers"onlich "uberzeugen zu k"onnen, bot sich mir, ebenso unerwartet als h"ochst erw"unscht, im Jahre 1808 bei dem Steinfalle um Stannern in M"ahren dar. Wenn gleich nicht als Augenzeuge bei dem Vorfalle selbst zugegen, machten es mir doch die g"unstigen Umst"ande einer geringen Entfernung des Schauplatzes von Wien, und einer sehr fr"uhzeitigen und verl"asslichen Kunde davon, besonders aber die h"oheren Ortes erhaltene, in solchen F"allen h"ochst n"otige Vollmacht und die kr"aftige Unterst"utzung von Seite der Beh"orden, m"oglich, eine allen W"unschen und Forderungen entsprechende Untersuchung an Ort und Stelle, und wenige Tage unmittelbar nach dem Vorfalle selbst, vorzunehmen. Es konnte demnach umso weniger fehlen, dass ein in so vielfacher Beziehung h"ochst anziehender Gegenstand der Physik meine ganze Aufmerksamkeit, die bisher immer nur schwach und blo"s durch von Zeit zu Zeit bekannt gewordene, mehr oder weniger befriedigende Nachrichten von, in der Ferne vorgefallenen, "ahnlichen Begebenheiten angeregt wurde, auf sich zog, als derselbe vor Kurzem eben lebhaft und nachdr"ucklich wieder zur Sprache gebracht und das Interesse daf"ur durch die mannigfaltigen, von vielen angesehenen Physikern dar"uber vorgebrachten, ebenso seltsamen als widersprechenden Meinungen und Hypothesen, so allgemein und m"achtig in Anspruch genommen worden war.

Diese kr"aftige Anregung und vollends die erfolgreiche Benutzung jener Gelegenheit, welche so mannigfaltigen Stoff und so zahlreiche Materialien zu eigenen Beobachtungen, Erfahrungen und Reflexionen darbot, hatten nicht nur eine fortgesetzte, ernstliche Besch"aftigung mit diesem Gegenstande und eine Reihe von Untersuchungen, Arbeiten und Versuchen zur Folge; sondern veranlassten auch den Entschluss, Alles bis auf diese Zeit an Beobachtungen und Erfahrungen, an Erkl"arungen und Meinungen hier"uber bekannt gewordene, zu sammeln, zusammen zu stellen und einer Vergleichung und kritischen Beurteilung zu unterziehen, und alles aufzubieten, von den etwa noch vorhandenen Produkten fr"uherer, und den k"unftig vorkommenden, zeitweiliger Ereignisse der Art, so viele als m"oglich aufzubringen und die kaiserl. Sammlung hieran so vollst"andig als m"oglich zu machen.

Diese weit aussehenden Pl"ane und Vors"atze und jene, Zeit und Ruhe heischenden, mannigfaltigen Unternehmungen, wurden leider nur zu fr"uh und gewaltsam, durch die bald nach dieser Periode eingetretenen ung"unstigen Zeitverh"altnisse, die den literarischen Verkehr erschwerten und mir ganz andere Besch"aftigungen aufdrangen, unterbrochen, und zuletzt, durch die lange Fortdauer und Folgen derselben, zum Teil ganz in Vergessenheit gebracht. Inzwischen war doch bereits nicht nur eine ersch"opfende Benutzung jener Gelegenheit erzielt, die umst"andlichste und befriedigendste Untersuchung jenes Ereignisses zu Stande gebracht, und selbst das Wesentlichste der hierbei erhaltenen Resultate bekannt gemacht, sondern auch eine F"ulle neuer Ansichten und Aufkl"arungen gewonnen und eine Menge belehrender Versuche und erfolgreicher Untersuchungen als Vorarbeiten angestellt, welche zu interessanten Beobachtungen und Erfahrungen f"uhrten, die aber, nach einmal so gewaltsam abgerissenem Faden, zu dessen Wiederauffassung sich in langer Zwischenzeit weder Mu"se noch Veranlassung finden wollte, nur zum Teil und au"ser Zusammenhang, Bruchst"uckweise und auf indirekten Wegen, zur "offentlichen Kenntnis gebracht werden konnten.

Gl"ucklicheren Erfolg, als jene ung"unstigen Zeitumst"ande erwarten lie"sen, hatte mein Bestreben in Auftreibung und Erhaltung der materiellen Belege fr"uherer und in der Zwischenzeit vorgefallener Ereignisse; denn im Laufe von 10 Jahren war es mir doch gelungen, von 29 derselben, die noch vorhanden und irgendwo aufbewahrt oder eben zur Kenntnis gekommen waren,\footnote{\frakfamily{Es mochten deren damals, und ungef"ahr von der Mitte des 15ten Jahrhunderts her, nebst sechs Eisenmassen, bei vierzig derlei Steinmassen gewesen sein, von welchen sich, notorisch, ein oder das andere Bruchst"uck als sprechender Beleg, nach gen und Ort verschiedener, solcher Ereignisse, urspr"unglich im Besitze irgend eines bekannten Privat-Liebhabers befand, der es, seinen individuellen Ansichten gem"a"s, der Merkw"urdigkeit des mehr oder weniger beglaubigten und f"ur ein Wunder angesehenen Factums wegen, oder als ein Dokument der Leichtgl"aubigkeit der Menschen, als ein Kuriosit"ats-St"uck aufzubewahren f"ur gut fand; von welchen aber in der Zwischenzeit leider viele, ja die Mehrzahl, wie es mit Privat-Besitzungen, zumal solcher Art, zu gehen pflegt, vollends in Verlust geraten sind. und so kam es denn auch, das von beinahe hundert und zwanzig bedeutenden, zu ihrer Zeit ziemliches Aufsehen erregenden und hinl"anglich beurkundeten Steinf"allen, die demnach von gleichzeitigen Schriftstellern, Historikern und Chronikschreibern der Mit- und Nachwelt bekannt gemacht wurden, und die seit dem Anfange unserer Zeitrechnung bis zum Jahre 1806 sich ereignet hatten, nun kaum mehr neunzehn durch derlei authentische Belege sich bekr"aftigen lassen, und zwar au"ser jenem von Ensisheim, von 1492, wovon wir die lange Erhaltung des Beleges der kr"aftigen F"ursorge Maximilians, der ihn zu einem Kirchenschatz machte, zu verdanken haben, und dem h"ochst zuf"allig (in Laugiers H"anden in Paris) in einem kleinen Fragmente noch erhaltenen, von dem 1668 bei Verona Statt gehabten Steinfalle --- keiner von einem fr"uheren Datum als aus der zweiten H"alfte des 18ten Jahrhunderts; und die in der kaiserlichen Sammlung seit ihrem Niederfalle aufbewahrte Eisenmasse von Agram, 1751, und der Stein von Tabor, 1753 (au"ser welchem nur noch wenige Fragmente von diesem, doch sehr bedeutend gewesenen Steinfalle in anderweitigen Besitz zu sein scheinen), sind nebst jenen beiden, so viel bekannt, bereits die "altesten noch vorhandenen Belege der Art.}} charakteristische und zur Aufstellung geeignete St"ucke zu erhalten. Die kaiserl. Sammlung erwuchs somit zur ansehnlichsten und vollst"andigsten von der Art kostbarer und merkw"urdiger Natur-Produkte, indem dieselbe nun --- mit den bereits fr"uher schon vorhanden gewesenen\footnote{\frakfamily{Schon vor f"unfzehn Jahren, als ich die Direktion der k. k. Hof-Naturalien-Kabinette antrat, fanden sich deren bereits sieben --- und manche davon schon seit lange --- zwar gerade nicht als Belege der immer noch bezweifelten Ereignisse, die sich, mehr oder weniger befriedigenden, damals hier, so wie "uberhaupt noch ziemlich allgemein, wenig beglaubigten Nachrichten zu Folge, zu fr"uheren Perioden in verschiedenen L"andern zugetragen hatten, und f"ur deren Produkte sie ausgegeben waren, sondern vielmehr nur als seltsame Fossilien eines r"atselhaften Ursprunges und Herkommens, unter den Sch"atzen des Mineralreiches daselbst aufbewahrt. Namentlich waren es Musterst"ucke von jenen 1753 bei Tabor in B"ohmen, 1768 bei Mauerkirchen in Bayern, 1785 bei Eichst"adt in Franken und 1803 um L'Aigle in Frankreich gefallenen Steinen, und nebst der 1751 bei Agram in Kroatien niedergefallenen Metall-Masse, ein Bruchst"uck von der durch Pallas aus Sibirien bekannt gewordenen und von einer dieser sehr "ahnlichen, angeblich aus Norwegen herstammenden Eisenmasse. Und unstreitig war dieser Vorrat damals schon, als wohl kaum jemand an das Zusammensammeln dieser r"atselhaften Natur-Produkte noch dachte, der reichhaltigste und in Hinsicht der Gr"o"se und Vollkommenheit der St"ucke bereits der kostbarste in seiner Art, wie er denn auch, und zwar schon viel fr"uher --- 1798 -- Hrn. D. Chladni, der damals nur die sibirische Masse und den bei Mauerkirchen gefallenen Stein kannte, Gelegenheit verschaffte, sich in seinen bereits bekannt gemachten Mutma"sungen "uber die Natur und den Ursprung dieser Massen, durch die Wahrnehmung ihrer "ubereinstimmenden Abweichung von allen terrestrischen Fossilien und der auffallenden "Ahnlichkeit derselben unter sich, zu best"arken, und einige Jahre sp"ater --- 1801 --- des Hrn. v. Buchs Aufmerksamkeit erregte, und, auf dessen Mitteilung des Gesehenen, einer "ahnlichen, entscheidenden und zu jener Zeit noch sehr gewagten, auch lange nach der Hand noch lebhaft bestrittenen "Au"serung des Hrn. Pictet, in einer Versammlung des National Institutes zu Paris, zur Bekr"aftigung diente; so wie auch ich demselben die Kenntnis zu danken hatte, die mich ein ganz unerwartet vorgelegtes Bruchst"uck von jenen um Stannern gefallenen Steinen, auf der Stelle als identisch und folglich gleichen Ursprunges mit jenen Massen erkennen machte, und die mir Muth und Zuversicht gab, diese vorteilhafte Gelegenheit zur vollsten Selbst"uberzeugung und zur m"oglichsten "Uberzeugung Anderer zu benutzen und ohne Furcht mich zu kompromittieren, die Schritte zu machen, welche n"otig waren, um eine amtliche und f"ormliche Untersuchung des Factums, so schnell wie m"oglich, einzuleiten. Die sorgf"altige Aufbewahrung und Aufstellung dieser, teils zuf"allig (der Eisenmassen aus Sibirien und Norwegen und des Stein-Fragments von Eichst"adt) oder bei irgend einer Gelegenheit (der Metall-Masse von Agram und des Steines von Tabor) erhaltenen, teils selbst absichtlich und relativ um sehr hohe Preise beigeschafften (des Stein-Fragments von Mauerkirchen und des Steines von L'Aigle) zweideutigen Fossilien, zeugen "ubrigens von der Aufmerksamkeit und Werthschatzung, welche die Wiener Naturforscher diesen Natur-Produkten zu jener Zeit schon zollten, indes so manche von jenen oben erw"ahnten vierzig "ahnlichen, materiellen Belegen solcher Ereignisse, von welchen, notorisch, teils ein Fragment, meistens aber ein ganzer Stein, teils selbst die ganze niedergefallene Masse und zwar gew"ohnlich mit authentischen Nachrichten von glaubw"urdigen M"annern, oft selbst mit formlich abgefassten Urkunden, einem wissenschaftlichen Vereine zur Beurteilung, oder Kabinetten und "offentlichen Anstalten zur Aufbewahrung eingesendet worden waren, in Verlust gerieten; so dass nicht nur an diesen vermeintlich sichern Bestimmungspl"atzen sich gegenw"artig keine Spur mehr von denselben findet, sondern selbst nur von drei derselben kleine Fragmente in Privat-Besitz nachweisbar noch vorhanden sind. So kam einer von den bei Roa in Spanien 1438 gefallenen Steinen, in das k"onigl. Museum zu Madrid; einer von jenen aus der Gegend von Schleusingen 1552, in das herzogl. Museum zu Rudolstadt; der 39 Pfund schwere, 1581 in Th"uringen gefallene Stein (nebst der im Archive zu Dresden noch aufbewahrten Urkunde) und der $\mathfrak{\frac{1}{2}}$ Zent. schwere, 1647 bei Zwickau gefallene Stein, in die Kunstkammer nach Dresden; einer von jenen 1654 auf der Insel F"unen gefallenen, in das k"onigl. Naturalien-Kabinett zu Kopenhagen; der in demselben Jahrhunderte in Mailand gefallene Stein mit dem Settalianischen Kabinette, in welchem derselbe urspr"unglich aufbewahrt gewesen, in die Ambrosianische Bibliothek daselbst; der im Kanton Bern 1698 gefallene, in die dortige Stadt-Bibliothek; der Stein von Terranova in Kalabrien 1755, in die k"onigliche Bibliothek zu Neapel; jener von Sigena in Aragonien 1773, in das k"onigl. Museum zu Madrid (ein kleines Fragment davon befindet sich im k"onigl. Museum zu Paris; und der $\mathfrak{6\frac{1}{2}}$ Pfund schwere, 1775 bei Rodach gefallene Stein, in das herzogl Naturalien-Kabinett zu Coburg. So wurden mehrere von den vielen und gro"sen, 1668 im Veronesischen gefallenen Steinen, der damaligen Akademie zu Verona vorgelegt, und gegenw"artig scheint, wie bereits erw"ahnt, nur ein kleines Fragment mehr davon vorhanden zu sein, und so wurden Bruchst"ucke von den bei Nicorps in der Normandie 1750 und von jenen bei Lucé 1768 in Frankreich gefallenen Steinen, der Pariser Akademie eingeschickt, und nur von letzteren finden sich derzeit noch einige kleine Fragmente im Besitze von Privaten.}} --- 36\footnote{\frakfamily{N"amlich 27 Stein- und 9 Metall-Massen. Von ersteren m"ochten derzeit, 1820, im Ganzen etwa 40 --- notorisch und nachweisbar in H"anden bekannter Besitzer indes, wohl kaum mehr als 34 --- als materielle Belege von Ereignissen der Art --- deren doch dermal beinahe 150 seit unserer Zeitrechnung zur n"ahern Kenntnis kamen, und hinl"anglich beurkundet sind --- von letzteren etwa 12, gleichen, obgleich nicht faktisch erwiesenen Ursprunges, als St"ucke oder in Fragmenten, vorhanden und noch irgendwo aufbewahrt sein. Von ersteren besa"s das Pariser Museum 1815, nur 13; das britische Museum in London 1818, 11; und von den vorz"uglichsten Privat-Sammlern (zu welchen insbesondere auch Heuland und Sowerby in London geh"oren, deren Sammlungstand mir inzwischen zur Zeit nicht speziell genug bekannt ist) Klaproth 1810, 10; Lavater in Z"urich 1811, 10; Blumenbach 1812, 11; De Drée in Paris 1818, 26; Chladni 1819, 27.}} aufzuweisen hat, die in verschiedenen L"andern und zu verschiedenen Perioden, nach ganz verl"asslichen Nachrichten entweder sichtbar niedergefallen oder zwar blo"s zuf"allig aufgefunden, aber, nach aller Wahrscheinlichkeit und Analogie, allgemein auch als solche anerkannt sind; und zwar in so bedeutenden Massen, dass deren Gesamtgewicht beinahe drei Zentner erreicht.

Viele von diesen neu akquirierten, fr"uhzeitig erhaltenen und jene, fr"uher schon im kaiserl. Kabinette vorhanden gewesenen, so wie manche einzelne in hiesigen Privat-Sammlungen befindliche und mehrere von entfernten Besitzern gef"alligst mir zur Ansicht mitgeteilte, ausgezeichnete St"ucke, insbesondere aber die reiche Ausbeute von dem Steinfalle um Stannern und die vielen, besonders ausgezeichneten, frischen und vollkommenen Exemplare von daher, gaben gleich Anfangs zur Anfertigung von Abbildungen Veranlassung. Eine genaue und vorz"uglich in oryktognostischer Beziehung vorgenommene Untersuchung und Vergleichung dieser r"atselhaften Fossilien, wie sie bei diesem Vorrate m"oglich war, machte n"amlich auf so Manches aufmerksam, was ebenso wesentlich zu deren Erkenntnis als merkw"urdig an sich und dabei einer Versinnlichung bed"urftig und einer solchen auch f"ahig schien, dass naturgetreue Darstellungen umso zweckm"a"siger und erw"unschter erachtet wurden, als die Objekte selbst, ihrer Seltenheit und Kostbarkeit wegen, und gewisser Ma"sen blo"s als Einzelheiten existierend, nur von Wenigen besessen, von Vielen nicht einmal je gesehen werden k"onnen.

Mehr als siebzig derlei Original-Abbildungen waren bereits schon zu Anfang des Jahres 1809 von der Hand eines geschickten K"unstlers zu Stande gebracht, die, trotz der oft erprobten Schwierigkeit bei Darstellung anorganischer Natur-Produkte, allgemeinen Beifall fanden und den Wunsch erregten, dass eine preisw"urdige Vervielf"altigung derselben m"oglich sein m"ochte; allein die im gew"ohnlichen Wege auf Kupfer veranstalteten Proben zeigten nur zu bald die Schwierigkeiten der Ausf"uhrung und die Kostspieligkeit einer solchen Unternehmung; so dass der Zweck nur unvollkommen und einseitig zu erreichen gewesen w"are. Die Fortschritte, welche in dieser Zwischenzeit im Steindrucke gemacht wurden, die Vorteile, welche dieser gew"ahrt und der gute Erfolg, mit welchem man denselben bereits verschiedentlich zur Darstellung naturhistorischer Gegenst"ande anwendete, bestimmten mich, auch dieses Mittel zur Vervielf"altigung versuchen zu machen, und da der Versuch, wo nicht meinen W"unschen, doch den Erwartungen entsprach, viele Sachverst"andige befriedigte und das Wesentlichste erzielen zu lassen verhie"s; so fand ich mich umso bereitwilliger, der erneuerten Aufforderung mehrerer Wissenschaftsfreunde, und namentlich des Herrn D. Chladni, bei Gelegenheit der eben hier veranstalteten Herausgabe seines neuesten Werkes "uber diesen Gegenstand, zu entsprechen, und wenigstens eine Auswahl aus jener Sammlung von Abbildungen auf diesem Wege vervielf"altigen zu lassen und bekannt zu machen, als meine Verh"altnisse und Berufsgesch"afte bereits lange schon alle Hoffnung mir benommen hatten, den fr"uheren Plan zu einer umfassenderen Bearbeitung des Gegenstandes, je realisieren und selbe demnach ihrer urspr"unglichen Bestimmung gem"a"s benutzen zu k"onnen, dagegen eine so g"unstige Gelegenheit, wie die Erscheinung jenes Werkes war --- die eben sowohl zu meiner Beruhigung, als zum unbezweifelbaren Gewinn der guten Sache, jener Realisierung zuvor kam und sie nun vollends ganz entbehrlich machte - mir die Versicherung gab, sie einer vorteilhafteren Bestimmung widmen und, in solch empfehlender Begleitung, f"ur selbe eine willkommenere Aufnahme gew"artigen zu k"onnen.

W"ahrend einer Reihe von tumultuarischen und gesch"aftsvollen Jahren durch mannigfaltige, zum Teil sehr heterogene Berufs- und Wissenschafts-Anforderungen, ganz von diesem Gegenstande abgelenkt, mehrerer schriftlicher Aufs"atze verlustiget, des chaotischen Vorrates zahlloser Notaten kaum Meister, und all des Vergangenen im Einzelnen nur schwach mich besinnend war es anf"anglich meine Absicht nur, diese Abbildungen durch kurze Beschreibungen zu erl"autern, und dies umso mehr, als einerseits die gr"undliche und so vielseitig vollst"andige Bearbeitung des Gegenstandes in jenem Werke jeden weitern Kommentar entbehrlich, andererseits der Drang der Zeit, um der nun einmal gemachten Verhei"sung zu entsprechen, so wie der Mangel an erforderlicher Mu"se, Gesch"aftsfreiheit und Geistesruhe, um jene vorhandenen Ged"achtnisbehelfe benutzen und die volle Erinnerung wieder gewinnen zu k"onnen, der Zustandebringung eines solchen sehr entgegen waren.

Da inzwischen selbst diese beschr"ankte Behandlung des Gegenstandes nicht nur ein aufmerksames Studium jenes Werkes und eine Zurateziehung mehrerer anderer, sondern insbesondere auch, der h"aufigen in dieser Zwischenzeit neu erhaltenen, erst noch zu bearbeitenden Materialien wegen, eine erneuerte Durchsicht und Pr"ufung eigener fr"uherer Ausarbeitungen, eine weitere Verfolgung derselben und selbst eine Fortsetzung und Wiederholung von abgebrochenen und unbefriedigend gebliebenen einstmaligen Versuchen und Untersuchungen notwendig machte; so wurden bald wieder alle Ber"ucksichtigungspunkte, welche die Vielseitigkeit des Gegenstandes in physischer und philosophischer Hinsicht darbietet und die, jetzt noch wie vor, den Physikern so reichhaltigen Stoff zu eigenen Mutma"sungen und Ansichten, und so vielfachen Anlass zu Debatten und Kontroversen geben --- und wohl noch lange geben m"ochten --- mittel- oder unmittelbar angeregt, und, samt den einst im Verfolge jener umfassenderen, fr"uheren Bearbeitung des Gegenstandes erhaltenen Resultaten, Bruchst"uckweise wenigstens, ziemlich lebhaft wieder ins Ged"achtnis zur"uck gerufen.

Und die mit Erweckung des Erinnerungsverm"ogens wieder erwachte alte Vorliebe f"ur den Gegenstand und ein bei jener vergleichenden Rekapitulation und Nachholung des im Laufe eines vollen Dezenniums, zumal auf den soliden Wegen der Erfahrung, Beobachtung und Untersuchung, Geschehenen, in etwas geschmeicheltes Selbstgef"uhl, reitzten mich umso mehr, manche Resultate fr"uherer Forschungen und Versuche, und einige dadurch motivierte Reflexionen und Folgerungen bei dieser Gelegenheit unter einem bekannt zu machen, als ich, nach eigenem Gef"uhle im Verfolg der Ausarbeitung, besorgen zu m"ussen glaubte, dass einerseits die Trockenheit einer so einseitigen Behandlung des, gerade von der spekulativen und vern"unftelnden Seite am meisten anziehenden, Gegenstandes, rein deskriptiv, wie sie anf"anglich beabsichtigt war, zumal durch die, bei solchen Objekten doch unerl"assliche physiographische Kleinigkeitskr"amerei, den Leser anekeln, andererseits Manches, hie und da mit Nachdruck Angedeutete, unverst"andlich oder unerheblich, wenigstens deutungs- und beziehungslos erscheinen m"ochte.

Die Vielseitigkeit des Gegenstandes und die h"aufigen Ber"uhrungs- und Beziehungspunkte, welche das Materielle der Objekte in obigen R"ucksichten darbot, motivierten nun eine bedeutende Menge solcher Einstreuungs-Artikel, die, dem einmal angenommenen Plane der Bearbeitung und ihrer individuellen Bestimmung gem"a"s --- als Erl"auterung oder Deutung irgend eines in der Physiographie ber"uhrten Punktes zu dienen, oder um, auf Veranlassung eines solchen, irgend eine, das Objekt oder den Gegenstand im Allgemeinen betreffende, physische, chemische, philosophische oder historische Tat- und Erfahrungssache zur Kenntnis, oder endlich irgend eine vorgefasste Meinung oder gangbare Hypothese zur Berichtigung, oder eine neue Mutma"sung und Ansicht in Anregung zu bringen --- als Noten zum Text angebracht wurden.

Obgleich diese Zugabe solcher Gestalt weder den Plan noch den eigentlichen Zweck der Behandlung des Gegenstandes ab"anderte, sondern nur eine Ver"anderung des Titels und in der Art der Ank"undigung veranlasste: so ist damit doch, da dieselbe den Hauptgehalt bedeutend "uberwiegt, das Volumen des Werkes betr"achtlich "uber meine anf"angliche Absicht, und weit "uber die urspr"ungliche Berechnung des Verlegers herangewachsen, und ich f"ande mich "uber die Folgen davon --- die Verz"ogerung des Erscheinungs-Termines und die Erh"ohung des Preises --- verantwortlich, wenn ich mich nicht f"ur erstere, durch meine Verh"altnisse und die Anforderungen der Aufgabe unter oben geschilderten Umst"anden, entschuldiget, und gegen letztere "uberhaupt, durch jede Aufopferung von meiner Seite, vorhinein schon verwahrt zu haben, glauben k"onnte. Dagegen muss ich "uber den Wert des Gehaltes, der hierzu Veranlassung gab, sowie "uber jenen des Ganzen, das Urteil kompetenter Richter gew"artigen, hoffe aber hierbei auf jene Nachsicht rechnen zu d"urfen, auf welche die Natur des Gegenstandes und die vielseitigen und schwierigen Anforderungen desselben, dem regen Eifer seiner k"uhnen Verfechter bei so sehr beschr"ankten Kr"aften, den vollsten Anspruch geben:

\begin{quote}
Quod si deficiant vires, audacia certe\\
Laus erit; in magnis et voluisse sat est.\\
--- Propert.
\end{quote}

Wien, im Julius 1820.
\clearpage
\section{\frakfamily{Erste Tafel.}}
\begin{center}
Die Gediegeneisen-Masse

von 71 Pfund Wiener Kommerziell-Gewicht,\footnote{\frakfamily{Bekanntlich ist nebst dieser nur noch eine zweite, kleinere Masse von 16 Pfund als Produkt des vorausgegangenen Feuer-Meteors, der beobachteten Feuerkugel, niedergefallen, welche nicht nur im Niederfallen, und selbst bei der Lostrennung von jener gesehen, sondern auch gleichzeitig mit jener, und auf 2000 Schritt Entfernung von derselben, aufgefunden und aus der Versenkung gehoben wurde; "uber deren Aufbewahrung oder Verwendung aber urspr"unglich keine Nachricht gegeben ward, und von deren Nochvorhandensein auch bis jetzt keine weitere Kenntnis erlangt werden konnte.}}
\end{center}
\paragraph{}
welche am 26. Mai 1751 gegen 6 Uhr Abends bei dem Dorfe Hraschina in der Agramer Gespanschaft (etwa drei Meilen N. O. von Agram) in Kroatien, unter den gew"ohnlichen meteorischen Erscheinungen und im Angesichte mehrerer Augenzeugen aus der Luft gefallen, und drei Klafter tief in einen kurz zuvor gepfl"ugten Feldgrund eingedrungen war.

Es wurde diese Masse\footnote{\frakfamily{Es ist dieselbe umso interessanter und sch"atzbarer, als sie von den ohne dies sehr wenigen "ahnlichen Eisen-Massen, deren Niederfallen historisch und faktisch erwiesen ist (wie die, ihrer Beschaffenheit nach, zwar zweifelhaften, und wie es scheint, ganz in Verlust geratenen Miscolz in Ungarn 1559, und von Torgau 1561; die zwar noch --- in Gotha --- vorhandene, aber dem Fundorte nach zweifelhafte --- aus Sachsen --- von 1540 oder 1550 ? und nebst einigen, die seit unserer Zeitrechnung im Orient --- China, Japan, Persien --- gefallen sein m"ogen; jene, am zuverl"assigsten bekannte, 1621 zu Lahore in Indien gefallene, welche aber der mogolische Kaiser Dschehan-gir ganz verschmieden lie"s), die einzige noch vorhandene zu sein scheint; so wie sie die einzige von dieser Art ist, welche physisch und chemisch untersucht wurde, und durch den Befund ihres Gehaltes und ihrer physischen Eigenschaften, als Prototyp auf einen gleichen meteorischen Ursprung jener "ahnlichen Eisen-Massen, nach Analogie zu schlie"sen berechtigte, welche zuf"allig zu verschiedenen Zeiten und an verschiedenen, sehr entfernten Orten aufgefunden worden, bekannt und noch vorhanden sind, aber bei welchen es, ihre Herkunft zu erweisen, an historischen und faktischen Belegen fehlte (wie dies bei den, in dieser Beziehung problematischen Eisen-Massen aus S"ud- und Nord-Amerika, Brasilien, Afrika, Sibirien, B"ohmen, Ungarn u. s. w. der Fall ist). Auch war sie von den derben Gediegeneisen-Massen die erste, und "uberhaupt mit von den ersten Meteorolithen (mit dem Eisen aus Sibirien, dem Eichst"adter und Sieneser Meteor-Steine), welche auf Veranlassung der kaum bekannt geworden Untersuchungen Howards (1802) in Deutschland analytisch untersucht wurden, und zwar von Klaproth (der die Resultate seiner Untersuchungen zuerst in einer Vorlesung in der k"onigl. Akademie der Wissenschaften zu Berlin, und dann im neuen allgemeinen Journal der Chemie, B. 1, zu Anfang des Jahres 1803 bekannt machte), welchem zu diesem Ende ein kleines St"uck von dieser Masse (gleichzeitig mit einem St"ucke vom Eichst"adter Meteor-Steine) schon im Jahre 1802 von hier aus mitgeteilt worden war. Im Jahre 1808 wurde, soweit es ohne Beeintr"achtigung der Form und Ansicht der Masse geschehen konnte, ein gr"o"seres St"uck von etwa 20 Loth abges"agt, um zu technischen Versuchen zu dienen, die Hr. Direktor von Widmanst"atten auf meine Veranlassung vornehmen wollte, und welche zu merkw"urdigen Resultaten, und insbesondere zur h"ochst interessanten Entdeckung des kristallinischen Gef"uges, welches diesen Massen, wo nicht ausschlie"slich, doch vorzugsweise eigent"umlich und f"ur dieselben charakteristisch zu sein scheint, f"uhrten. Die durch Abs"agung jenes St"uckes an der Masse erhaltene Fl"ache wurde mit Salpeters"aure ge"atzt, um jenes Gef"uge oberfl"achlich darzustellen und die Entdeckung zu bew"ahren; von dem "Uberreste des abges"agten Stuckes wurden kleine Abschnitte nach London, Paris und Harlem mitgeteilt.}} ihrer Merkw"urdigkeit wegen, und als Beleg des wunderbaren Naturereignisses, von dem bisch"oflichen Konsistorium zu Agram, welches, aus eigenem Antriebe, durch Abgeordnete das Factum sogleich (am 2. Julius desselben Jahres) an Ort und Stelle amtlich und f"ormlich untersuchen lie"s,\footnote{\frakfamily{Es war diese eine der fr"uhesten Begebenheiten der Art (die erste, mit Ausnahme jener von Th"uringen 1581 und von Bern 1698, welche ebenfalls von den Lokal-Beh"orden legal untersucht, und durch eine ausgefertigte Urkunde dokumentiert wurden, wovon sich jene von der ersteren Begebenheit, nach Chladnis Versicherung, noch zur Zeit im Archive zu Dresden aufbewahrt befindet), welche einer amtlichen Untersuchung von einer Beh"orde wert geachtet und durch eine ausgefertigte formliche Urkunde der Nachwelt aufbewahrt, und die erste, von welcher diese selbst, wenn gleich gerade nicht mit der Absicht, das Factum beglaubigen zu machen, zur Publizit"at gebracht wurde (Stutz, Bergbaukunde B. 2, 1790); und es w"are in der Tat unbegreiflich, wie eine so unbefangene und reine, deutungs- und beziehungslose Darstellung von einer so achtbaren Beh"orde so wenig Aufmerksamkeit erregen, so wenig auf die "Uberzeugung wirken konnte, wenn nicht zu vermuten st"ande, dass sie durch jene Publizierung nur wenigen eigentlichen Physikern zur Kenntnis kam. Sie verdient umso mehr an einem schicklicheren Orte, wie bei einer andern Veranlassung geschehen soll, und im Original bekannt gemacht zu werden, als es die ausdr"uckliche Absicht der Aussteller und Einsender dieser, mit allen F"ormlichkeiten ausgestatteten, Urkunde war, nicht nur die Mitwelt von der Realit"at des Factums zu "uberzeugen, sondern auch diese "Uberzeugung durch ein authentisches Dokument auf die Nachwelt zu bringen.}} samt einer schriftlichen Urkunde, welche das Untersuchungs-Protokoll enthielt, noch in demselben Jahre an den kaiserl. Hof eingesendet, wo sie in der k. k. Schatzkammer zu Wien aufbewahrt, und in der Folge, bei "Ubertragung der naturwissenschaftlichen Gegenst"ande aus derselben, an das k. k. Hof-Naturalienkabinett abgegeben wurde.

Es hat dieselbe eine platt gedr"uckte, etwas verschobene, dreiseitige Gestalt, und zeigt demnach zwei Fl"achen und drei R"ander. Die eine dieser Fl"achen ist, schief von den R"andern aufsteigend, m"a"sig gew"olbt, nach oben sich verebnend, und durch mehr oder weniger unterbrochene, gebogene und wellenf"ormige, rippenartige, abgerundete Erhabenheiten, und durch gr"o"sere und kleinere, seichtere und tiefere, meistens rundliche oder ovale Vertiefungen und Eindr"ucke, welche von jenen begrenzt werden, sehr uneben; die andere entgegen gesetzte Fl"ache ist dagegen beinahe flach und eben, und zeigt nebst einigen kleineren und tieferen Eindr"ucken gegen die R"ander hin, nur drei gro"se, sehr seichte und breit verlaufende Vertiefungen, welche, idem sie durch flache Zwischenr"aume in einander "ubergehen, und gewisser Ma"sen zusammen h"angen, diese Fl"ache im Ganzen etwas ausgeh"ohlt erscheinen machen.

Die R"ander, unter welchen diese beiden Fl"achen zusammensto"sen, sind von der konvexen Fl"ache her schief nach Au"sen abgerundet, und nicht nur durch die rippenartigen Erhabenheiten, welche sich von daher "uber dieselben bis an die entgegen gesetzte Fl"ache fortsetzen, und durch "ahnliche Eindr"ucke, sehr uneben, sondern auch, zumal gegen die Mitte, sehr stark ausgeschweift und gewisser Ma"sen unterbrochen, so dass man ihre Richtung nur schwer bestimmen kann. Zieht man inzwischen nach den hervorragendsten Punkten eines jeden Randes eine, demselben parallel laufende, gerade Linie, und schlie"st man das solcher Gestalt erhaltene Dreieck durch Verl"angerung dieser Linien "uber die abgerundeten Ecken hinaus, bis sie sich ber"uhren; so fallen die Linien, welche den beiden Seitenr"andern oder den beiden l"angeren Schenkeln der dreieckigen Form der Masse entsprechen, auf die Grundlinie, welche --- die Masse in dieser Richtung betrachtet -- dem untern Rande entspricht, unter einem Winkel von beil"aufig 80$^{\circ}$ auf. Die dritte oder obere, dem untern Rande gegen"uberstehende Ecke der dreiseitigen Masse, f"allt au"ser das Mittel derselben, und --- die Masse von der konvexen Fl"ache betrachtet -- stark gegen den rechten Seitenrand hin, indem der linke Seitenrand bogenf"ormig sich gegen jenen hin"uberzieht, und sich mit demselben in eine gegen ihn gerichtete, etwas stumpfe Spitze vereinigt. Die ganze Masse verflacht sich mehr gegen die linke Seite hin, zumal nach oben an der Kr"ummung des Seitenrandes, der hier am d"unnsten, an einer Stelle beinahe schneidend, und da von der entgegen gesetzten Fl"ache etwas "ubergebogen ist; dagegen erhebt sich die rechte Seite hier mit dem Au"senrande und der Spitze, indem sie von der entgegen gesetzten Fl"ache gleichsam her"uber gedr"uckt erscheint, so dass dort, abgesehen von den an dieser Stelle befindlichen ziemlich gro"sen und tiefen Eindr"ucken, welche den "au"sersten Rand auch ziemlich d"unn machen, eine starke Abweichung von der horizontalen Ebene dieser Fl"ache bewirkt wird, und die Spitze des Dreiecks, oder vielmehr beinahe die ganze obere H"alfte der Masse, solcher Gestalt etwas verdreht erscheint. An dieser Fl"ache dagegen laufen die R"ander, abgesehen von den genannten Abweichungen und von den zuf"alligen Eindr"ucken, gr"o"sten Teils horizontal mit der Ebene derselben; nur gegen die eine untere Ecke, welche der Richtung der verdrehten Spitze entspricht, ist der Seitenrand schief abgerundet, und ebenfalls gegen die konvexe Fl"ache gedr"uckt, so dass es scheinen m"ochte, als wenn die Masse, in noch weichem Zustande! auf dieser ganzen Seite, im Auffallen einen gr"o"seren Widerstand gefunden h"atte.\footnote{\frakfamily{Es findet sich leider in der Urkunde nicht bemerkt, in welcher Lage diese Masse in ihrer Versenkung gefunden wurde, sondern es wird nur erw"ahnt, dass die Spalte (nicht Grube) in der Erde drei Klafter tief und eine Elle weit gewesen sei, nach welchen Ausdr"ucken zu mutma"sen k"ame, als w"are sie mit einen der R"ander eingedrungen und auf keine der Fl"achen aufgefallen, wie dies auch nach dem Schwerpunkte der Masse, der auf deren unteren Rand f"allt, der Fall gewesen sein musste, da eine rotierende Bewegung, zumal fl"achenw"arts, nach Form und Beschaffenheit derselben nicht wohl angenommen werden kann. Umso merkw"urdiger ist die auffallende Verschiedenheit der Oberflache der beiden Fl"achen. Es wird zwar in der Urkunde bemerkt, dass in den Vertiefungen der konvexen Fl"ache (also gerade der entgegen gesetzten) etwas Erde eingedr"uckt war; daraus kann aber noch nicht gefolgert werden, dass gerade die Masse auf diese Fl"ache auffiel, indem beide Fl"achen wohl in ziemlich gleich stark dr"uckenden Contact mit der Erde kamen, wenn die Masse mit einem Rande vorw"arts in dieselbe eindrang; dass sich aber nur an der einen Fl"ache Erde eingedr"uckt fand, mag von der starken Unebenheit ihrer Oberfl"ache herger"uhrt haben. Dass sich "ubrigens gegenw"artig keine Spur von Erde an der ganzen Masse mehr findet, mag wohl mit als Beweis dienen k"onnen, dass die Masse nicht im geschmolzenen oder gar fl"ussigen Zustande zur Erde gekommen sei, in welchem Falle die Erde wohl etwas mehr fixiert worden w"are.}}

Die gr"o"ste L"ange der Masse, von den hervorragendsten Punkten des rechten Seitenrandes, von der oberen Ecke oder Spitze bis zur hervorragendsten Erhabenheit am untern Rande dieser Seite gemessen, betr"agt $\mathfrak{15\frac{1}{2}}$ Zoll; am linken Rande nur 13 Zoll.

Die gr"o"ste Breite, von den hervorragendsten Erhabenheiten an beiden Seitenr"andern, etwa 3 Zoll ober dem untern Rande, betr"agt 12; im Mittel der Masse ist sie 8; am oberen Ende, etwa 3 Zoll unter der Spitze, von "ahnlichen Punkten gemessen, $\mathfrak{6\frac{1}{2}}$ Zoll.

Die gr"o"ste Dicke, von den erhabensten Stellen an beiden Fl"achen zusammen gemessen, betr"agt $\mathfrak{3\frac{3}{4}}$ Zoll; an Stellen, wo zuf"allig von beiden Fl"achen Vertiefungen zusammenfallen, "ubereinander zu liegen kommen, betr"agt sie kaum 2, hie und da selbst kaum 1 Zoll; wo dies nicht der Fall ist, kann man sie im Durchschnitt auf 3 Zoll annehmen. An den "au"sersten R"andern ist die Masse hie und da sehr d"unn, kaum $\mathfrak{\frac{1}{2}}$, selbst nur $\mathfrak{\frac{1}{4}}$ Zoll dick; an einer Stelle beinahe sogar schneidend scharf.

Die Vertiefungen und Eindr"ucke, welche sich auf der konvexen Fl"ache zeigen, haben zwar viele "Ahnlichkeit mit jenen, welche sich auf der Oberfl"ache der meisten Meteor-Steine finden, sind aber hier ungleich gr"o"ser, tiefer, h"aufiger und zusammen hangender, so dass die rippenartigen Erhabenheiten, welche sie begrenzen, gewisser Ma"sen ein unregelm"a"siges und verworrenes Netz bilden, und der Oberfl"ache ein zellenf"ormiges Ansehen geben. Manche dieser Vertiefungen haben im Mittelpunkte 5 bis 7, und wenn man das Niveau von den zun"achst liegenden h"ochsten Erhabenheiten nehmen will, 9 bis 15 Linien Tiefe bei einer Ausdehnung von $\mathfrak{1\frac{1}{2}}$ bis $\mathfrak{2\frac{1}{2}}$ Zoll im Durchmesser. In diesen gr"o"seren Vertiefungen, welche meistens einen mehr oder weniger rundlichen, aber mehrfach ausgeschweiften Umriss, und bald eine Grube, bald eine Erhabenheit zum unregelm"a"sigen Mittelpunkte haben, liegen die seichteren, $\mathfrak{\frac{1}{2}}$ bis 2, 3 Linien tiefen, daum- oder fingerartigen Eindr"ucke von verschiedener Gr"o"se, zu 3, 4 bis 5 unregelm"a"sig, bisweilen aber auch kreisf"ormig beisammen; inzwischen kommen solche Eindr"ucke auch einzeln oder isoliert au"ser den Vertiefungen vor. Die rippenartigen Erhabenheiten, welche durch diese Vertiefungen und Eindr"ucke gebildet werden, entsprechen der St"arke, H"ohe und Dicke nach, der Tiefe derselben und ihrer wechselseitigen Entfernung voneinander; und ihrer Ausdehnung und Richtung nach, nach welchen sie bald l"anger, bald k"urzer, bald wellenf"ormig, bald unter verschiedenen Winkeln gebogen erscheinen, der Lage und Form derselben, und ihrer wechselseitigen Verbindung unter sich. Demnach haben die Erhabenheiten zwischen aneinander grenzenden Hauptvertiefungen oft mehrere Linien H"ohe, und eine nicht minder betr"achtliche Dicke, zumal an ihrer Basis, und nicht selten ein paar Zoll L"ange, insofern ihr zusammen gedr"uckter, abgerundeter R"ucken nicht durch isolierte Eindr"ucke unterbrochen, breit gedr"uckt und gewisser Ma"sen gedoppelt wird; die Erhabenheiten dagegen, welche die seichteren, in den gr"o"seren Vertiefungen liegenden, Eindr"ucke begrenzen, sind nur sehr schwach, oft kaum merklich, und verflachen sich mit ihrer Basis nicht selten, ohne einen bedeutenden R"ucken oder eine Kante gebildet zu haben. Es finden sich jedoch einige Erhabenheiten auf dieser Fl"ache, welche nicht, wenigstens nicht unmittelbar, durch Eindr"ucke entstanden zu sein scheinen, da sie solche nicht geradezu begrenzen, und zapfen- oder zitzenf"ormig vorragen; und andere, welche zum Teil zwar durch Vertiefungen veranlasst worden zu sein scheinen, indem sie zwischen solchen liegen, auch rippenartig, wie die meisten, gestaltet, aber h"oher und st"arker sind, als sie, verm"oge der Ausdehnung und Tiefe jener, gerade zu sein h"atten.\footnote{\frakfamily{Wenn gleich im strengen Sinne der Kunstsprache diese Beschaffenheit der Oberfl"ache keineswegs zellig, "astig und zackig genannt werden kann, so ist sie doch, wenigstens dem Ansehen nach, im Ganzen einer solchen sich sehr ann"ahernd, und obgleich sie auch als solche hier nur auf die Oberfl"ache beschr"ankt ist, und ihr noch ein wesentlicher Umstand, n"amlich die Ausf"ullung der Zellen durch eine anscheinend fremdartige Substanz, ermangelt; so ist doch gerade durch sie eine "Ahnlichkeit dieser mit der sibirischen Eisen-Masse und eine Ann"aherung an dieselbe unverkennbar. Und so wie auf der andern Seite eine "ahnliche, und, wie mir deucht, ganz unbestreitbare Ann"aherung der eigentlichen Meteor-Steine an dieselbe, ja, wie ich zu behaupten wage, durch die stark eisenhaltigen (wie jene von Eichst"adt, Timochin, Tabor, bei welchen das Gediegeneisen nicht blo"s in zerstreuten K"ornern eingesprengt, sondern schon in mehr oder weniger zusammen h"angenden Zacken, und nur von noch vorwaltender erdiger Masse eingeh"ullt erscheint) ein wahrer "Ubergang in dieselbe (zumal, wenn man die dichteren, mehr erdigen Partien, die sich an manchem gr"o"seren St"ucke von der sibirischen Masse finden, oder die ungleich weniger "astigen und zelligen, vorgeblich aus Sachsen und Norwegen herstammenden, der sibirischen "ubrigens h"ochst "ahnlichen Eisen-Massen als Zwischenglieder betrachten will) Statt findet; so fehlt es vielleicht nur noch an ein paar Zwischengliedern (welche sich wohl noch finden m"ochten, und wozu sich z. B. gleich die Brasilianer Eisen-Masse eignen d"urfte, welche, obgleich im Ganzen dicht und derb, nach den neuesten Reiseberichten der Bayer'schen Naturforscher, die selbe an Ort und Stelle sahen, voll Gruben, L"ocher und oberfl"achlicher Eindrucke ist, die zum Teil mit eingekeilten Quarz?-Stucken erf"ullt sein sollen), um diesen auch hier sinnlich nachweisen zu k"onnen. Es findet eine ungleich gr"o"sere Verschiedenheit im "au"sern Ansehen sowohl, als im Aggregats- und Koh"asions-Zustande ja selbst im qualitativen und quantitativen Verh"altnisse der Gemeng- und Bestandteile zwischen manchen Meteor-Steinen Statt, als zwischen jenen Massen. Ein in der mineralogischen Diagnostik ge"ubtes Auge d"urfte zwischen einem etwas grobk"ornigen, eisensch"ussigen Sandsteine, und einem etwas dichten, porphyrartigen Bimssteine wohl kaum mehr Verschiedenheit auffinden k"onnen, als z. B. zwischen den Meteor-Steinen von Eichst"adt und von Stannern. Und doch l"asst sich zwischen diesen letzteren durch eine Reihe von Zwischengliedern, welche die allm"ahliche Ab"anderung des Aggregats- und Koh"asions-Zustandes, und die graduelle Zustandsver"anderung mancher einzelnen Gemengteile und deren allm"ahliches Hervortreten versinnlichen, ein augenscheinlicher "Ubergang nachweisen, welches zum Teil bei Erkl"arung der siebenten Tafel geschehen wird, und bei einer k"unftigen Veranlassung umst"andlicher geschehen soll.\\
\hspace*{6mm}Keine Verwandtschaft von Gattungen terrestrischer Fossilien versinnlicht wohl den Begriff einer Sippschaft (wie ich mich sehr bald "uberzeugte, und daher dieses Ausdruckes schon bei Gelegenheit meiner Beschreibung der m"ahrischen Aerolithen in Gilberts Annalen 1808 bediente, als ich zuerst auf die viel zu wenig beachtete Verschiedenheit der Meteor-Stein "uberhaupt, und auf die doch zwischen ihnen bestehende Verwandtschaft vorl"aufig aufmerksam machte), selbst ganz rein oryktognostisch genommen, deutlicher, und bei weitem keine zeigt so ausgedehnte Grenzen und so heterogen scheinende Extreme bei so allm"ahlichen "Ubergangen, als die Meteor-Massen, und bei keiner Verwandtschaftsstufe terrestrischer Fossilien ist die Konstruierung einer so genannten Suite, in Werners Sinne, zu ihrer vollst"andigen Erkenntnis notwendiger und an sich interessanter und lehrreicher.\\
\hspace*{6mm}Die Betrachtung der Meteor-Massen von dieser Seite, n"amlich von Seite ihrer so wesentlichen Verschiedenheit voneinander, welche bisher, wie nun auch Chladni bemerkt, so wenig ber"ucksichtigt wurde, obgleich noch weit auffallendere Beispiele, als das oben angef"uhrte (z. B. die unter sich sowohl als von allen "ubrigen noch weit mehr als jene, und in vielfachen Beziehungen abweichenden Meteor-Steine von Alais, Chantonnay, Erxleben, Langres), Aufmerksamkeit h"atten erregen sollen, --- und nach dieser ihrer Versippung unter einander: m"ochte wohl, wo nicht "uber den Ort, doch "uber die Art ihrer urspr"unglichen Entstehung und Bildung, und "uber manche, noch lange nicht befriedigend erkl"arte Erscheinungen bei ihrem Niederfalle, einiges Licht geben, und vielleicht selbst manche unserer geognostischen und oryktognostischen Ansichten berichtigen.}}

Die Vertiefungen und Erhabenheiten, welche an der entgegen gesetzten ebenen Fl"ache gegen den Rand zu liegen, zumal an der linken Seite (die Masse von dieser Fl"ache betrachtet) der oberen H"alfte, gleichen ziemlich jenen der vorigen Fl"ache, nur sind erstere seichter, minder ausgeschweift in ihrem Umrisse, und haben wenigere und breitere Eindr"ucke, oder gleichen vielmehr selbst blo"s aneinander sto"senden gr"o"seren Eindr"ucken, und die zwischen ihnen liegenden Erhabenheiten sind auch nur wenig erhaben und rippenartig, und verflachen sich mehr nach Art jener, welche einzelne seichte Eindr"ucke zu begrenzen pflegen. Die drei gro"sen ausgezeichneten Vertiefungen aber, welche in und gegen die Mitte, zumal der untern H"alfte, dieser Fl"ache liegen, unterscheiden sich sehr von allen "ubrigen, und zwar nicht nur durch ihre Gr"o"se, indem die gr"o"ste "uber 4 Zoll im Durchmesser misst, und durch ihre geringe Tiefe, indem eben diese Vertiefung an der tiefsten Stelle kaum 6 Linien unter die horizontale Ebene der Fl"ache reicht, sondern vorz"uglich dadurch, dass sie keinen runden, sondern einen unregelm"a"sigen, obgleich wenig ausgeschweiften Umriss, und sehr seichte, kaum merkliche, aber gro"se und breit verlaufende, gleichsam in einander flie"sende Eindr"ucke haben, und dass sie, einzelne Stellen ausgenommen, wo sie an tiefere Randeindr"ucke grenzen, von keinen rippenartigen Erhabenheiten begrenzt sind, sondern schief aufsteigend, allm"ahlich in die ziemlich horizontalen Ebenen, die zwischen und an ihnen liegen, und die an den meisten Stellen selbst etwas weniges ausgeh"ohlt sind, "ubergehen.\footnote{\frakfamily{Die auffallende Verschiedenheit dieser Fl"ache von der entgegen gesetzten, welche offenbar zeigt, dass auch solche Massen w"ahrend ihres Niederfallens noch eine wesentliche --- sei es auch nur eine oberfl"achliche --- Ver"anderung erleiden, wovon bei den Meteor-Steinen, wenn sie auch noch so kleine Bruchst"ucke der zerplatzten Feuerkugel sind, die um und um sie umgebende Rinde den Beweis liefert, w"are hier schlechterdings nicht zu erkl"aren, zumal sie nur einen Teil, wenn gleich den gr"o"seren, derselben betrifft, wenn man nicht ann"ahme, was auch h"ochst wahrscheinlich ist, dass diese Fl"ache, oder vielmehr blo"s jener Teil derselben, erst sp"ater gebildet worden, und zwar durch Lostrennung jenes zweiten zugleich herabgefallenen kleineren St"uckes, w"ahrend dem Niederfallen, entstanden sei. Da jedoch dieses St"uck nur 16 Pfund, demnach kaum den vierten Teil dieser vorhandenen Masse, gewogen haben soll, jener Teil dieser Fl"ache derselben aber, welchen sie nach obigem bedeckt haben m"usste, eine Ausdehnung von 10 bis 12 Zoll an L"ange und 4 bis 7 Zoll an Breite hat; so m"usste jenes St"uck sehr flach, und kaum einen Zou dick gewesen sein. Es hei"st nun zwar in der Urkunde, dasselbe sei viel kleiner als die eingesendete Hauptmasse gewesen, doch wird auch darin erw"ahnt, dass dasselbe eine bei 2 Ellen weite, also eine selbst noch gr"o"sere Spalte als jene, in die Erde gemacht habe, folglich wenigstens nach einer Richtung eine betr"achtliche Ausdehnung gehabt haben m"usse; auch erhellet aus der Urkunde, dass dasselbe zerst"uckelt worden sei, indem die Untersuchungs-Kommission nur einen Teil davon erhielt: das St"uck muss demnach wirklich sehr d"unn gewesen sein, sonst w"are eine Zerst"uckelung oder auch nur Teilung desselben, bei der bekannten au"serordentlichen Z"ahigkeit solcher Massen, nicht leicht m"oglich gewesen. Dass aber au"ser diesem St"ucke noch mehrere sich losgetrennt haben und unbeobachtet niedergefallen sein sollten, ist wohl nicht wahrscheinlich, da so viele Augenzeugen auf dem Platze waren, die Feuerkugel im Zerplatzen, und die beiden St"ucke, in welche sie sich teilte, im Niederfallen gesehen, und das eine St"uck selbst auf 2000 Schritt Entfernung (und eine noch gr"o"sere Entfernung vom Punkte des Niederfalls der Hauptmasse, folglich eine noch mehr parabolische Richtung in Falle eines St"uckes ist, bei der geringen H"ohe, auf der die Feuerkugel, wenigstens im Momente des Zerplatzens, gestanden zu haben scheint, und sei der allenfalls voraussetzbaren Projektions-Kraft gegen die Zentripetal-Kraft eines K"orpers von solchem spezifischen Gewichte, nicht wohl denkbar) sogleich aufgefunden wurde. Dass aber vollends die Hauptmasse im Momente des Auffallens auf festen Grund ihren Umfang ver"andert, sich abgeplattet, und demnach in die Breite und Lange ausgedehnt haben sollte, so dass jenes Stuck vor seiner urspr"unglichen Lostrennung einen weit kleineren Fleck zu bedecken gehabt h"atte, und folglich betr"achtlich kleiner gewesen sein d"urfte; diesem, an und f"ur sich, widerspricht nicht nur bei gegenw"artigem individuellen Falle die ganze Beschaffenheit der Masse in allen Beziehungen, sondern es streitet "uberhaupt eine Menge von Gr"unden gegen die, einer solchen Annahme zur Basis dienende, Voraussetzung, und, wie es scheint, ziemlich allgemein angenommene Meinung, als k"amen die Meteor-Massen jeder Art, Metalle wie Steine, in einem solchen Grade von Weichheit, ja selbst von Fl"ussigkeit zur Erde, wenn gleich andererseits nicht in Abrede gestellt werden kann, dass, wenigstens bei letzteren, die Gemengeteile, und vielleicht selbst die entfernteren Bestandteile, sich vor und in den Momente des Niederfallens in einem ganz andern Koh"asions-Zustande befinden m"ussen, als die Steine im Ganzen kurze Zeit nachher erkennen lassen.\\
\hspace*{6mm}Um obige Mutma"sung vollkommen zu bew"ahren, m"usste das in Frage stehende St"uck ebenso kontrastierende, und den Fl"achen unserer Hauptmasse respektiv entsprechende Oberfl"achen zeigen; und es w"are demnach sehr zu w"unschen, dass wenigstens ein Teil davon noch aufgefunden werden m"ochte, zu welchen Ende neuerdings Nachforschungen eingeleitet worden sind.}}

Auf der konvexen Fl"ache zeigen sich nebst ein paar zarten, engen, wahrhaften Nissen oder Spr"ungen, welche sich "uber Erhabenheiten und Vertiefungen eine bedeutende Strecke von mehreren Zollen fortsetzen, teils gerade, teils gebogen und gezackt verlaufen, und, soweit es sich durch ein hinein gestecktes Blatt Papier messen l"asst, wenigstens einige Linien tief sind, --- merkw"urdige Einschnitte, die das Ansehen haben, als w"aren sie absichtlich durch ein mei"sel- oder hakenartiges Werkzeug hervor gebracht worden, aber keineswegs daher r"uhren k"onnen, da sie viel zu h"aufig und regelm"a"sig vorkommen, keinen der Sch"arfe eines schneidenden Instrumentes, sondern vielmehr einen der Beschaffenheit der Oberfl"ache der Eindr"ucke entsprechenden Grund, und abgerundete, der Beschaffenheit der Oberfl"ache der Erhabenheiten entsprechende, klaffende R"ander erkennen lassen. Es zeigen sich diese Einschnitte vorz"uglich queer "uber dem R"ucken, seltener nach der L"ange der rippenartigen Erhabenheiten, nur wenige finden sich in der Tiefe der seichteren Eindr"ucke, und setzen sich "uber diese, wenn sie von Erhabenheiten kommen, auch nicht weit und seichter fort. Nur wenige haben die L"ange von 1 bis 2 Zollen, die meisten nur von 3 bis 6 Linien, bei einer ziemlich gleichf"ormigen Tiefe von etwa $\mathfrak{\frac{2}{12}}$ bis $\mathfrak{\frac{3}{12}}$ Linie, und einer "ahnlichen Weite, die nach der Tiefe, und meistens nach den beiden Enden hin, etwas abnimmt. Wenn man mit einem feinen spitzigen Instrumente die Tiefe verfolgt, so gelangt man nach Hinwegschaffung des den Naum ausf"ullenden, gelben ockrigen Pulvers, oft erst in einer Tiefe von einer halben, ja beinahe ganzen Linie, auf den Grund, der entweder von dem Rindenh"autchen der Oberfl"ache bedeckt ist, das sich in jedem Falle an den W"anden bis hinab zieht, oder bisweilen Farbe und Glanz des Metalle zeigt. Sie laufen meistens schnurgerade, nur "au"serst wenige sind nach den Kr"ummungen der Erhabenheiten etwas gebogen, und stehen meist einzeln, weit voneinander entfernt, inzwischen doch auch bisweilen paarweise gen"ahert, sich parallel oder in einen Winkel zusammensto"send. H"ochst merkw"urdig ist, dass diese Einschnitte, so abgebrochen in ihrem Verlaufe und so zerstreut auf der Oberfl"ache der Masse sie auch erscheine, doch beinahe ohne Ausnahme in drei bestimmten, leicht erkennbaren Richtungen streichen, und einen Parallelismus und eine Winkeldurchkreuzung zeigen, welche, insoweit sie bestimmt und verglichen werden k"onnen, dem kristallinischen Gef"uge der Masse, von dem in der Folge die Rede sein wird, zu entsprechen, oder wenigstens mit demselben in einigem Zusammenhange zu stehen scheinen.

Die Masse hat im Ganzen eine schw"arzlich-braune Farbe, welche kaum im geringsten die metallische Natur derselben verr"at. Alle Vertiefungen und Eindr"ucke, sowohl die auf der konvexen als auch die gegen den Rand der ebenen Fl"ache und auf allen R"andern liegenden, so wie auch die Einschnitte auf der konvexen Fl"ache, sind matt, und ihre Farbe zieht sich aus dem schw"arzlich-braunen ins graue, erd- und rostbr"aunlich-gelbe, hie und da ins rostr"otlich-gelbe; die gro"sen Vertiefungen aber mit ihren Eindr"ucken, und die zwischen und an ihnen liegenden flachen, nur etwas ausgeh"ohlten Stellen auf der ebenen Fl"ache, haben eine sehr matte, rostbraune Farbe. Alle Erhabenheiten dagegen, welche diese Vertiefungen und Eindr"ucke an beiden Fl"achen und an allen R"andern begrenzen, die klaffenden R"ander der Einschnitte an der konvexen Fl"ache, dann der "au"serste Rand, mit welchem die gro"sen Vertiefungen auf der ebenen Fl"ache in die angrenzenden flacheren Stellen "ubergehen, endlich der ganze linke Seitenrand dieser Fl"ache mit der oberen Spitze, welcher gleichsam gegen die konvexe Fl"ache hin"uber gedr"uckt erscheint, haben eine fettige, etwas wachs"ahnliche, gl"anzende, br"aunlich-schwarze Farbe, welche an den "au"sersten Kanten, zumal auf dem R"ucken der rippen- und zapfenartigen Erhabenheiten, hie und da in eine rein metallisch eisengraue gleichsam "ubergeht. Der Art Stellen von rein metallischem Ansehen und Glanze, deren Farbe aus dem eisengrauen bald mehr ins Zink-, bald mehr ins Silberwei"se sich zieht, finden sich von verschiedener Gr"o"se und Ausdehnung, doch meistens nur sehr klein, hin und wieder auch selbst in den Eindr"ucken der konvexen Fl"ache, am meisten aber und von ausgezeichnet silberwei"ser Farbe am "ubergebogenen untern Seitenrande der ebenen Fl"ache, der "ubrigens mit "au"serst d"unner, glatter, schwarzbrauner Rinde bedeckt ist. Die ganze Oberfl"ache der Masse, jene der gro"sen Vertiefungen und der angrenzenden flachen Stellen der ebenen Fl"ache ausgenommen, erscheint dem blo"sen Auge beinahe glatt, bei n"aherer Betrachtung mit einer Lupe aber erscheint sie, und zwar in allen Vertiefungen und Eindr"ucken, "au"serst fein gek"ornt, chagrinartig rau; an allen, dunkleren und gl"anzenderen, Erhabenheiten und Stellen dagegen mehr glatt und nur zart aderig, metallische Ramifikationen bildend, die sich ziemlich weitschichtig, und meistens von dem R"ucken der Erhabenheiten "uber die Verflachung zu beiden Seiten gegen die Eindr"ucke, welche sie begrenzen, hin verlaufen; an den rein metallischen, gl"anzenden Stellen erscheint sie aber vollkommen glatt und spiegelich. Betrachtet man diese letzteren Stellen genauer, so ersieht man bald, dass sie von einer "au"serst zarten Decke oder Rinde entbl"o"st sind, welche wie ein d"unnes Oberh"autchen die ganze Masse umkleidet, sich "uber alle Vertiefungen und Erhabenheiten ziemlich gleichf"ormig ausdehnt, und jenes geaderte oder chagrinartig raue Ansehen der "ubrigen Oberfl"ache hervor bringt, und die hier an diesen Stellen, wie ihre Ausrandung zeigt, welche einen offenbaren gewaltsamen Bruch, bisweilen aber auch eine nat"urliche Begrenzung erkennen l"asst, entweder zuf"allig oder absichtlich abgerieben oder abgeschlagen, bisweilen aber auch in ihrer urspr"unglichen Bildung unterbrochen worden ist.\footnote{\frakfamily{Unverkennbar ist die "Ahnlichkeit dieser Rinde, und "uberhaupt der Beschaffenheit der Oberfl"ache in dieser Beziehung, mit jener Meteor-Steine, zumal aus der Suite der stark eisenhaltigen, wie z. B. der Steine von Eichst"adt, Timochin, Tabor, Barbotan, L'Aigle \emph{zc.}, und in gewisser Beziehung der von Chantonnay, Erxleben und Ensisheim.}}

So wenig auffallend jene verschiedenartige Beschaffenheit der Oberfl"ache, und insbesondere ihre Rauigkeit, und die Existenz dieser zarten Rinde, an der konvexen Fl"ache sowohl, als auch an den, in den "ubrigen Beziehungen derselben entsprechenden und gleichartigen Stellen der entgegen gesetzten ebenen Fl"ache erscheinen, um so auffallender und in die Augen springender zeigen sie sich hier auf jenem Teile dieser Fl"ache, der auch in den "ubrigen R"ucksichten so wesentlich von der Beschaffenheit der ganzen "ubrigen Oberfl"ache abweicht, und hier erscheint alles gleichsam nach einem vergr"o"serten Ma"sstabe.

Die k"ornig-raue Oberfl"ache der drei gro"sen, und selbst einiger an dieselben grenzender kleinerer Vertiefungen zum Teil, so wie auch der zwischen jenen und an und um dieselben liegenden ebeneren Stellen, spricht sich hier dem unbewaffneten Auge, so wie dem Gef"uhle, sehr deutlich aus, und ebenso auffallend erscheinen die glatten, rein metallischen, eisengrauen, nur durch neu entstandenen Eisenrost hie und da matt angeflogenen Flecken von betr"achtlichem Umfange, die sich vorz"uglich auf den ebeneren Stellen finden, und welche die urspr"ungliche Bedeckung durch eine "ahnliche k"ornig-raue (hier ganz unverkennbare, meist zuf"allig, und wohl noch mehr absichtlich abgeschlagene, oder vielmehr abgesch"alte) Rinde umso deutlicher erkennen machen, da sie an allen diesen Stellen durchgehend von ansehnlicher Dicke ist, die selten weniger als eine halbe Linie, gew"ohnlich $\mathfrak{\frac{3}{4}}$ Linien betr"agt.

Schon mit freiem Auge kann man hier erkennen, dass die Rauigkeit dieser Rinde durch kleine und "au"serst kleine rundliche Erhabenheiten oder W"arzchen hervorgebracht wird, welche unordentlich dicht aneinander geh"auft, bisweilen in kurze Schn"ure einzeln aneinander gereihet, oder hie und da zu feinen Adern, und, wie wohl selten, zu gr"o"seren Tropfen oder Flecken von verschiedener Form zusammengeflossen sind. Mit der Lupe betrachtet, erscheinen diese Erhabenheiten als einzelne, gleichsam aus der Masse ausgeschwitzte, aufsitzende Tr"opfchen mit konvexer, etwas rauer, gewisser Ma"sen getr"aufter Oberfl"ache, von schwarzer Farbe und pechartigem Glanze, die an einander gereiheten oder mehr oder weniger zu Adern zusammen geflossenen aber etwas abgeplattet, und die Zwischenr"aume sind mit einem erd- oder ockerbr"aunlichen Zement ausgef"ullt, welches, da diese sowohl an sich als zusammen genommen mehr Raum ausf"ullen als jene Tr"opfchen, eine solche raue Oberfl"ache im Ganzen rostbraun erscheinen machen.

Ein mittelst eines Mei"sels losgetrenntes Bl"attchen solcher Rinde, das sich, wenn die Kontinuit"at einmal unterbrochen ist, an solchen Stellen sehr leicht von der glatten, selbst spiegelichen Oberfl"ache der Masse absch"alen l"asst, zeigt an den Bruchstellen gar keine schlackige oder por"ose Beschaffenheit, sondern vielmehr, und zwar an den Bruchr"andern, eine zart- und ziemlich geradfaserige Textur nach der Dicke des Blattes. Die Fasern scheinen durch ein "ahnliches ockerartiges, br"aunlich und r"otlich-gelbes Zement verbunden, oder vielmehr selbst (durch Einwirkung der Luft, welche zwischen die Rauigkeiten der Oberfl"ache eingedrungen sein konnte) in eine solche ockerartige Substanz verwandelt worden zu sein, und gehen, in zarte B"undel zusammen geh"auft, kolbenf"ormig und meistens etwas nach einer Richtung gebogen, in jene schwarze Tr"opfchen an der Oberfl"ache "uber. Ein von den Randerhebungen jener rauen Vertiefungen und Stellen, und von dem R"ucken der sie begrenzenden Erhabenheiten (wo, wie bereits oben erw"ahnt worden, die Rinde immer d"unner, obgleich hier nie so papier- oder vielmehr schneidig d"unn, wie an jenen der konvexen Fl"ache, schw"arzer, etwas gl"anzend und beinahe glatt, auch dichter und fester erscheint) auf gleiche Art abgenommenes Rindenbl"attchen (was jedoch wegen der geringen Dicke und des st"arkeren Zusammenhanges mit der Oberfl"ache hier schwerer und nur in kleinen Fragmenten bewirkt werden kann), zeigt an den Bruchr"andern eine "ahnliche, aber etwas zartere und mehr eine k"ornig-faserige Textur, und eine beinahe Zinkwei"se Farbe mit starkem, rein metallischem Glanze (wahrscheinlich weil hier wegen Dichtheit der Rinde an der Oberfl"ache die Luft nicht einwirken konnte), und die Fasern gehen unmittelbar in die eigentliche schwarze Rinde, die hier keine W"arzchen mehr erkennen l"asst, indem diese bereits in ein H"autchen zusammengeflossen zu sein scheinen, "uber. An einigen Stellen, zumal an solchen, wo die raue Rinde der Vertiefungen in die glatte der Erhebungen "ubergeht, und hier noch eine betr"achtliche Dicke hat, erscheint sie an ihren Bruchr"andern gewisser Ma"sen stratifiziert, und zwar in drei, obgleich nur sehr schwach angedeuteten, ziemlich gleich dicken, horizontalen Schichten, wovon die oberste die eigentliche Rinde, und die unterste das Blatt, welches unmittelbar auf der Masse auflag, bildet, und welche beide in die mittlere, etwas dickere, ohne merkliche Unterbrechung der Richtung der Fasern, "ubergehen, nur durch eine "au"serst zarte horizontale Linie von derselben getrennt scheinen, und sich blo"s durch etwas schw"acheren metallischen Glanz und etwas ver"anderte Farbe von ihr unterscheiden, indem erstere mit schwarzen Rindeteilchen, die andere mehr oder weniger mit gelben oder braunen Ocherteilchen gemengt ist. Hie und da, sowohl an dem einen als dem andern Rindenbl"attchen, erscheinen die Fasern an den Bruchr"andern bisweilen Speis- oder Messinggelb angelaufen, und diese Stellen zeigen gleichsam den "Ubergang vom metallischen Zustande derselben in den ockrigen, nach dem verschiedenen Grade der Einwirkung der Luft; wie sie sich denn auch meistens dort finden, wo die raue Rinde in die glatte "ubergeht, folglich der Luft ein geringerer Zutritt gestattet wurde. Die Fl"ache, mit welcher diese Rindenbl"attchen "uberhaupt auf der Oberfl"ache der Masse aufsitzen, ist ganz dicht und ziemlich glatt, nur etwas unregelm"a"sig streifig, von mattem, metallischem Ansehen, und eisengrau, mit einem schwachen oberfl"achlichen Farbenanfluge von Blau, Roth und Messinggelb, wie Eisen, das l"angere Zeit an der Luft gelegen hat. Nur da, wo ein solches Bl"attchen sehr d"unn eine erhabene Stelle bedeckte, und sehr dicht und fest aufsa"s, erschien jene Fl"ache mehr oder weniger Zinkwei"s und metallisch gl"anzend.\footnote{\frakfamily{So sehr das ganze "au"sere Ansehen dieser, so wie aller "ahnlichen Massen meteorischen Ursprunges (selbst der Meteor-Steine), und insbesondere das kristallinische Gef"uge des Eisens, aus dem sie bestehen, oder das sie enthalten, unwiderlegbar einen urspr"unglich fl"ussigen Zustand derselben voraus setzen; so widersprechen doch eben dieselben, insbesondere aber die dem Meteor-Eisen ganz eigent"umlichen, und von jenen der, durch die bekannten Schmelz-Prozesse erhaltenen Produkte der Kunst, so sehr abweichenden physischen Eigenschaften (der hohe Grad von Dehnbarkeit und Z"ahigkeit, der sich bei gro"ser Hitze verliert, indem bekanntlich alles Meteor-Eisen gerade dann erst br"uchig wird), und vor Allem der Umstand, dass diese Massen von mechanisch eingemengtem, ganz unver"anderten Schwefeleisen so ganz durchdrungen sind (wie dies von den Meteor-Steinen hinl"anglich bekannt ist, von den Eisen-Massen aber bei Gelegenheit der Erkl"arung der achten und neunten Tafel bemerkt werden wird), der, wie es scheint ziemlich allgemein angenommenen, Meinung, als w"are dieser fl"ussige Zustand auf dem so genannten trockenen Wege, durch Hitze, hervor gebracht, und das Produkt eines gew"ohnlichen Schmelz-Prozesses. Noch mehr aber streiten diese Grunde und manche Erscheinungen beim Niederfalle dieser Massen und Steine, und vorzugsweise bei dieser Eisen-Masse, gegen die beinahe ebenso allgemein gefasste, und selbst von unserm Chladni unterst"utzte Meinung, als ginge dieser Schmelz-Prozess w"ahrend des Niederfalls in unsrer Atmosph"are noch fort (oder beg"onne vielmehr wohl gar in selber), und als k"amen sie, und als w"are namentlich diese Masse, in wahrhaft durch Hitze geschmolzenem Zustande, selbst tropfbar fl"ussig (wie wenigstens unser G"u"smann behaupten wollte) bis zur Erde gekommen. Der Umstand, dass hier, laut Urkunde, die Augenzeugen beide Massen, jede in Gestalt einer feurigen, verwickelten Kette (aus welchen G"u"smann gegliederte Z"uge von einer Klafter L"ange machte), wollen --- aber doch --- herab fallen gesehen haben (die G"u"smann aus der Hohe sich ergie"sen lasst), mochte wohl einer optischen T"auschung, einem Licht-Ph"anomene zugeschrieben werden d"urfen; und jener, das die Erde, worein sie fielen, rauchte und wie ausgebrannt und gr"unlich aussah, konnte, wenn ja alles w"ortlich und als wahr und richtig bezeichnet angenommen werden soll, wohl mit mehr Grund einer zur Zeit unbekannten Einwirkung der Massen auf dieselbe zugeschrieben werden, als gerade ihrer Hitze, die doch nur die der Erde beigemengten animalischen und vegetabilischen Teile verbrennen und rauchen machen konnte, indes nicht einmal eines bemerkten Geruches erw"ahnt wird. Die flache, wie hingegossene Gestalt, und die wellenf"ormigen Unebenheiten aber, welche, nebst obigen Punkten der Urkunde, unser Chladni als ganz deutliche Beweise, dass die Materie in geschmolzenem Zustande zur Erde kam, geltend machen zu m"ussen meint, scheinen mir gerade dagegen zu sprechen. Die Masse m"usste, meines Bed"unkens, ungleich flacher und platter, wenn sie hingeflossen, oder mehr oder weniger konisch sein, wenn sie (wie G"u"smann will) in die Erde eingegossen worden, und beides ohne alle Verspritzung, was kaum denkbar ist, und doch der Fall war, vorgegangen w"are.\\
\hspace*{6mm}Geradezu aber, und besonders in diesem individuellen Falle, spricht gegen eine solche Annahme: dass die Massen so tief in die Erde gedrungen waren, da sie doch von den Seiten her keinen Widerstand fanden sich auszubreiten, und dass dieses Auffallen von solcher H"ohe und gewaltsame Eindringen hei"ser und fl"ussiger Metall-Massen ohne alle Verspritzung erfolgte; dass ferner an der ganzen einen gro"sen Masse keine Spur sich findet von fest anklebender Erde, oder, was bei einer solchen Voraussetzung wohl der Fall sein m"usste, von eingekneteten und eingeschmolzenen Sandteilchen und Steinchen, mit welchen sie doch in Contact gekommen sein muss; dass endlich keine der Massen beim Ausgraben warm befunden wurde, ein Umstand, den man anzumerken gewiss nicht unterlassen h"atte, wenn er vorhanden gewesen w"are. Leider wird in der Urkunde nicht bemerkt, wann die Massen eigentlich ausgegraben wurden; aber eben daraus und aus der ganzen Erz"ahlung l"asst sich abnehmen, dass es auf der Stelle (bei der kleineren Masse hei"st es auch wirklich: sogleich) oder doch in jedem Falle noch an demselben Tage geschah. Nun aber ereignete sich das Factum Abends um 6 Uhr, und da man wohl schwerlich das Eintreten der Nacht wird abgewartet haben; so geschah die Ausgrabung wohl h"ochst wahrscheinlich innerhalb den ersten zwei Stunden. Eine durch Hitz geschmolzene und im Fl"usse sich befindende Eisen-Masse von solchem Volumen w"urde aber wohl kaum in 24 Stunden soweit ausgek"uhlt gewesen sein, dass man sie h"atte ber"uhren k"onnen.}}

Noch sind an dieser ebeneren Fl"ache der Masse zwei Stellen bemerkenswert: die eine befindet sich am oberen Teile an der rechten Seite derselben am aufsteigenden Rande der gr"o"sten Vertiefung, in Gestalt einer Aush"ohlung oder Grube von rundlichtem Umrisse, und 6 bis 7 Linien Weite nach Au"sen, welche gleichsam durch einen starken, von oben und von der Seite her nach Innen und gegen die Vertiefung wirkenden Eindruck hervor gebracht worden zu sein scheint, indem die eine Wand sehr schief und mit sanft verlaufendem Rande einw"arts l"auft, die entgegen gesetzte aber etwas schief aufw"arts steigt, und gegen die Vertiefung hin einen aufgeworfenen abgerundeten Rand bildet. Diese Grube verenget sich etwas gegen ihren Grund, welcher einen ovalen Umriss von $\mathfrak{4\frac{1}{2}}$ Linie L"ange zu $\mathfrak{2\frac{3}{4}}$ Linie Breite hat, und geht in eine Tiefe von 2 bis 3 Linien, die aber kaum unter das Niveau der tiefsten Stelle jener gro"sen Vertiefung reicht. Die Seitenw"ande dieser Grube haben ein raues, ockriges Ansehen, den Grund aber schlie"st eine glatte, nur etwas por"os scheinende, matt metallisch gl"anzende, eisengraue Ebene von graphit"ahnlichem Ansehen, welche in der Mitte etwas verbrochen ist, und hier wieder eine ockrige Beschaffenheit zeigt.

Die zweite Stelle befindet sich ganz am untern Rande der Masse, wo ein St"uck einst gewaltsam und absichtlich abgebrochen worden zu sein scheint. Es zeigt sich hier ein rauer, etwas hakiger, zerkl"ufteter, und durch Rost verunstalteter Bruch; an einer kleinen Stelle daselbst aber ein deutlicher, wenigstens zweifacher Durchgang von Bl"attern von betr"achtlicher Dicke, metallischem Ansehen und Glanze, und lichtstahlgrauer, ins silberwei"se fallender Farbe.

Auf der konvexen Fl"ache sind zwei, gegen die obere Ecke der Masse isoliert stehende Erhabenheiten, auf $\mathfrak{\frac{1}{4}}$ bis $\mathfrak{\frac{1}{2}}$ Zoll Tiefe, und auf der ebenen Fl"ache ist ein St"uck von betr"achtlicher Ausdehnung (bei 5 Zoll lang, 1 bis $\mathfrak{2\frac{1}{4}}$ Zoll breit und bei $\mathfrak{\frac{3}{4}}$ Zoll dick) von der Oberfl"ache der Masse am Rande der abgerundeten Ecke der rechten Seite, zum Behufe technischer und analytischer Versuche, abges"agt worden, wo nun das Innere der Masse zu Tage liegt. Die solcher Gestalt erhaltenen Abschnittsfl"achen zeigten roh eine dichte, derbe Masse von metallischem Glanze, und lichtstahlgrauer, ins silberwei"se fallender Farbe, deren Dichtheit und Gleichf"ormigkeit im Gef"uge nur hie und da durch zarte gezackte Risse und kleine Kl"ufte, und noch mehr durch h"aufig und zerstreut eingemengte, meist mikroskopisch kleine k"ornige Partikelchen von metallischem Ansehen, st"arkerem Glanze und wei"serer Farbe, --- welche, mit mehr und weniger ockriger Substanz verbunden, zum Teil auch jene Risse und Kl"ufte erf"ullen, --- unterbrochen erschien. Eine der kleinen Abschnittsstellen auf der konvexen Fl"ache der Masse, welche mit dem Gerbstahl poliert wurde, zeigt eine spiegelnde Oberfl"ache von beinahe silberwei"ser, ins stahlgraue fallender Farbe, oder vielmehr einer Farbe, welche jener des polierten Platins sehr "ahnelt. Die beiden andern durch jene Abschnitte erhaltenen Fl"achen wurden mit Salpeters"aure ge"atzt, um das merkw"urdige kristallinische Gef"uge darzustellen, das sich bei dieser Behandlung am deutlichsten ausspricht, und wovon bei der Erkl"arung der darauf Bezug habenden Tafeln insbesondere die Rede sein wird.

Obgleich von diesen, im Vergleich zur Dicke der Masse, nur oberfl"achlichen Stellen nicht geradezu auf eine durchaus gleiche Beschaffenheit im Innern geschlossen werden kann, welches "uberzeugend zu machen ohne wesentliche Beeintr"achtigung der, gerade im ganzen Zusammenhange, so merkw"urdigen Form und Beschaffenheit dieser Masse nicht geschehen konnte; so berechtiget doch zu dieser Annahme einerseits die "Ubereinstimmung des absoluten Gewichtes mit dem Volumen derselben, nach dem bekannten spezifischen Gewichte, andererseits die bereits gemachte Erfahrung bei "ahnlichen Massen, wenn gleich nicht faktisch erwiesenen, doch unbezweifelbar gleichen meteorischen Ursprunges (den Elbogner und Lénartoer Gediegeneisen-Massen) welche teils, beinahe durch ihre Mitte, teils selbst nach mehrfachen Richtungen durchschnitten wurden, und durchaus eine, im Wesentlichen, gleichf"ormige Beschaffenheit zeigten.\footnote{\frakfamily{Man wird die Umst"andlichkeit in der Beschreibung dieser, an sich sowohl als ihrer vielseitigen Beziehungen wegen, h"ochst merkw"urdigen Masse, dem Bestreben zu Gute halten, jedem entfernten Forscher, der sie nie, vielleicht keine "ahnliche je zu Gesicht bekommen d"urfte, die m"oglichst vollkommenste anschauliche Kenntnis (wozu die bildliche Darstellung, der Unvollkommenheit der Kunst und der Beschaffenheit des Gegenstandes wegen, leider nur wenig beitragen konnte) von derselben zu verschaffen, und ihn in den Stand zu setzen, "uber so manch R"atselhaftes und Paradoxes, das uns die Erkl"arung des Ursprungs und der Bildung meteorischer Massen, und der meisten ihr Erscheinen und Niederfallen begleitenden Umst"ande, so schwer, ja unm"oglich zu machen scheint, und wor"uber, vorz"uglich was die Eisen-Massen betrifft, diese als Prototyp und als zur Zeit einzige, erwiesener Massen, meteorischen Ursprungs, einiges Licht geben kann, eine Mutma"sung fassen, oder wenigstens die zum Teil ziemlich widersprechenden Folgerungen und Behauptungen, zu welchen die mehr oder weniger genaue, richtige und unbefangene authoptische Betrachtung und Beurteilung derselben bei Andern bereits Veranlassung gegeben hat, und ohne Zweifel in der Folge noch geben wird, pr"ufen und w"urdigen zu k"onnen. Eine ersch"opfende Genauigkeit bei Beschreibung dieser Masse schien mir umso notwendiger, als eine vor der Hand sehr unbedeutend und ganz unwesentlich scheinende Kleinigkeit in der Folge bei Auffassung oder Beurteilung, Verteidigung oder Widerlegung einer Ansicht oder Erkl"arung, oft wichtig und entscheidend sein kann, die schwierige Behandlung eines so massiven Klotzes aber eine oftmalige Wiederholung "ahnlicher Betrachtungen, vernachl"assigter Nachforschungen wegen, nicht wohl gestattet. Dagegen glaubte ich die Bekanntmachung der Resultate der analytisch-chemischen und physisch-technischen Untersuchungen f"ur eine k"unftige Veranlassung versparen zu sollen, da dieser Gegenstand eigentlich nicht zum Zweck der gegenw"artigen geh"ort, und eine Ausarbeitung voraussetzt, die nur mangelhaft und unvollkommen h"atte zu Stande gebracht werden k"onnen, da es an der ben"otigten Mu"se gebrach, indem sie nicht nur eine Wiederholung und Erneuerung aller fr"uher (1808) gemachten, peremtorisch abgebrochenen, sondern eine Menge ganz neu anzustellender Versuche, wozu die in dieser Zwischenzeit erhaltenen Materialien Stoff genug lieferten, notwendig gemacht h"atte.}}

Die bildliche Darstellung zeigt diese merkw"urdige Masse, von der konvexen Fl"ache betrachtet, in nat"urlicher Gr"o"se.
\clearpage
\section{\frakfamily{Zweite Tafel.}}
\subsection{\frakfamily{Tabor.}}
\paragraph{}
Einer der gr"o"sten Steine von dem sehr bekannten und ziemlich ergiebig gewesenen Steinregen,\footnote{\frakfamily{Ungeachtet der Ergiebigkeit dieses Steinregens, indem sich derselbe doch "uber einen Fl"achenraum von einer halben Stunde in der L"ange, und einer Viertelstunde in der Breite erstreckte, und derselbe Beobachter von seinem Standplatze aus, von wo er den einen Stein fallen sah, noch deren vier in das Getreide niederfallen h"orte (die folglich in seiner N"ahe, und die Steine daher im Durchschnitt "uberhaupt ziemlich dicht gefallen sein m"ussen), und viele der Steine gro"s und von bedeutendem Gewichte waren (von 5 bis 13 Pfund), und obgleich die Begebenheit zu jener Zeit viel Aufsehen erregte, und durch Zeitungs- und wissenschaftliche Nachrichten bekannt gemacht wurde; so scheinen doch gegenw"artig nur wenige Belege mehr davon, und meistens nur in Bruchst"ucken, nachweisbar vorhanden zu sein. Au"ser einigen Privaten in Prag, und vielleicht noch an einigen Orten in B"ohmen, und Hrn. Chladni, sind meines Wissens nur das Universit"ats-Museum in Pesth, die De Drée'sche Sammlung in Paris, und das Mus. britan. in London (welches das von Born beschriebene St"uck mit dessen Sammlung durch Grevilles Verm"achtnis erhielt), im Besitze von solchen.}} der sich am 3. Julius 1753 um 8 Uhr Abends bei Tabor (eigentlich um Strkow, n"achst Plan, einem zur Herrschaft Seltsch geh"origen, eine Stunde von Tabor entfernten Dorfe) in B"ohmen ereignete, von beinahe 5 Pfund am Gewichte, und welcher im Momente der Begebenheit, vor einem, nach der Hand als Augenzeuge amtlich vernommenen Knechte (Math. Wondruschka) auf 30 Schritte Entfernung niederfiel, und ohne sich merklich zu versenken, blo"s die Erde aufwarf, auch sogleich von dem Beobachter aufgehoben und der Ortsobrigkeit "ubergeben wurde.

Es wurde dieser Stein von dem damaligen, zu jener Zeit in Tabor, der Kreisstadt des Bechiner Kreises, residierenden k"onigl. B"ohmischen Kreishauptmanne, Grafen Vinc. v. Wratislaw, gleich nach der Begebenheit, die derselbe aus eigenem Antriebe amtlich und f"ormlich an Ort und Stelle untersuchte, mit einem umst"andlichen Berichte an das k"onigl. B"ohmische Kammer-Pr"asidium zu Prag, und von diesem an die k. k. allgemeine Hofkammer nach Wien eingesendet.

Der Stein ist vollkommen ganz, und um und um mit Rinde bedeckt, die nur an einigen kleinen Stellen etwas abgesto"sen, und hie und da abgebrochen worden ist.

Es zeichnet sich derselbe besonders durch eine anscheinende Regelm"a"sigkeit\footnote{\frakfamily{Diese Regelm"a"sigkeit, auf die ich bereits in meinen Aufs"atzen in Gilberts Annalen, 1808, aufmerksam gemacht habe, und die nun auch Hr. D. Chladni bew"ahrt und einer Beachtung wert gefunden hat, ist umso merkw"urdiger, da hierin eine "Ubereinstimmung oder doch eine auffallende Ann"aherung zwischen vielen Steinen, nicht nur von einer und derselben Begebenheit (demnach zwischen Bruchst"ucken ein und desselben Meteors), sondern auch von, nach Zeit und Ort, sehr verschiedenen Ereignissen, und selbst zwischen solchen Statt findet, die sowohl in ihrem Aggregats- als Koh"asions-Zustande, als sogar im qualitativen und quantitativen Verh"altnisse der n"achsten und wesentlichsten Bestand- und Gemengteile bedeutend voneinander abweichen (und kaum k"onnen dies in diesen Beziehungen irgendwelche mehr als z. B. die Steine von Tabor und von Stannern), und da dieselbe auf einen Grund-Typus hinzudeuten scheint, der jenem sehr nahe kommt, welcher der "ahnlichen Bildung (Struktur, Absonderungs-Zerspaltungsform --- Figurierung ---) einiger terrestrischer, der Trapp-Formation angeh"origen Fossilien, welchen die Meteor-Steine in mehrfachen Beziehungen "uberhaupt sehr verwandt sind, zum Grunde liegt.}} in seiner Form aus. Er bildet n"amlich eine deutliche, nur etwas verschoben und ungleichseitig vierseitige, abgestumpfte niedere Pyramide,\footnote{\frakfamily{Da jener Regelm"a"sigkeit kein Kristallisation-Gesetz zum Grunde liegen kann, und demnach die vorkommenden Fl"achen und Kanten keineswegs mit wahren Kristallisation-Fl"achen und Kanten verglichen werden d"urfen, wie sie denn auch ihrer zuf"alligen Beschaffenheit, der Eindr"ucke und Verdr"uckungen wegen, wenigstens nicht mit der geh"origen Genauigkeit, weder geometrisch gedeutet, noch goniometrisch bestimmt werden k"onnen; so durfte die Darstellung und Beschreibung der Formen auch nur deskriptiv, nach der auffallendsten und am leichtesten zu versinnlichenden "Ahnlichkeit mit einer bekannten geometrischen Figur, keineswegs aber kristallologisch genommen werden. Wollte man letzteres, so m"usste man die Form dieses Steines als eine verschobene und ungleich vierseitige S"aule mit schief aufgesetzter Endfl"ache betrachten. Bemerkenswert scheint "ubrigens doch zu sein, dass zwei Seitenkanten an diesem Steine, mit m"oglichster Genauigkeit an gleichen Punkten gemessen, einen gleichen Winkel von beil"aufig 98$^{\circ}$, und darin eine "Ubereinstimmung mit "ahnlichen Kanten von drei verschiedenen s"aulenf"ormigen Basalten des Kabinettes zeigten, die damit verglichen wurden; so wie sich auch ein ganz "ahnlicher Winkel von einer Seitenkante am n"achst zu beschreibenden Steine von L'Aigle, und ein "ahnlicher am Steine von Lissa fand. "Uberhaupt messen die Winkel der sch"arfern Kanten dieses Steines zwischen 75 und 95$^{\circ}$, und die der stumpferen zwischen 105 und 125$^{\circ}$. Zwei Steine von diesem Ereignisse, welche der um die Geschichte desselben so verdiente, in der Zwischenzeit verstorbene D. Mayer, und einer, welchen Graf Thun in Prag besa"s, und welche mir die gef"alligen Besitzer einst zur Ansicht einschickten, hatten ebenfalls eine ziemlich regelm"a"sige Gestalt. Der eine, 10 Loth schwer, war rhomboidal; der andere, der nur 3 Qu"antchen wog, bildete eine vollkommene, scharfkantige, nur etwas schiefe, sonst fast gleichseitig dreiseitige Pyramide; und der dritte, von 1 Pfund 10 Loth, einen sehr verschobenen Rhombus, dem in der Folge zu beschreibenden Steine von Lisa sehr "ahnlich.}} deren Grundfl"ache $\mathfrak{4\frac{1}{2}}$ Zoll in L"ange und Breite, die obere Endfl"ache 3 Zoll in beiden Durchmessern, und deren H"ohe bei 3 Zoll misst.

Die Grundfl"ache ist fast ganz eben, und nur an einem Rande, wo die Kante schief und etwas ungleich abgestumpft ist, von der horizontalen Ebene abweichend. Sie zeigt mehrere gro"se l"anglichte, aber sehr seichte Eindr"ucke.

Zwei Seitenfl"achen, welche beinahe senkrecht auf die Grundfl"ache aufgesetzt sind, und mit derselben stumpfe und etwas abgerundete und geschweifte Grundkanten bilden, sind kleiner als die beiden andern, etwas konvex, haben wenige, kleine, ziemlich seichte Eindr"ucke, und sto"sen in eine sehr abgerundete gemeinschaftliche Seitenkante zusammen.

Die beiden andern gr"o"seren Seitenfl"achen erheben sich unter einem ziemlich spitzen Winkel schief von der Grundfl"ache, und sto"sen in eine ziemlich scharfe gemeinschaftliche Kante zusammen, welche mit den Kanten der Grundfl"ache eine starke hervorspringende Ecke bildet. Die eine dieser Fl"achen, die gr"o"ste von allen, ist sehr gew"olbt, und hat nur sehr wenige rundliche seichte Eindr"ucke; die durch sie mit der Grundfl"ache gebildete Grundkante ist schief und ungleich abgestumpft, die mit der ansto"senden Seitenfl"ache gebildete Seitenkante stumpf zugerundet. Die andere oder vierte Seitenfl"ache ist etwas konkav, sonst flach und eben, und zeigt nur einen gro"sen, aber sehr seichten, sanft verlaufenden, und einen ovalen, starken, tiefen Eindruck, in dessen Grunde ein gro"ses Korn Metall steckt. Die von dieser Fl"ache mit der Grundfl"ache gebildete Grundkante ist abgerundet, gegen die eine Ecke hin aber ziemlich scharf, "ubrigens ungleich, etwas geschweift und eingedr"uckt im Verlaufe; die mit der ansto"senden Seitenfl"ache gebildete Seitenkante ist aber, im ganzen etwas gebogenen Verlaufe, ziemlich scharf.

Die obere Endfl"ache entspricht der Form nach der Grundfl"ache, nur ist sie kleiner, und der Richtung der Seitenfl"achen nach, wovon zwei fast senkrecht unter einem Winkel von beinahe 90, zwei aber schief unter etwa 75$^{\circ}$ von der Grundfl"ache aufsteigen, aus dem Mittel geschoben. Sie ist "ubrigens ziemlich stark vertieft, und hat viele, zum Teil gro"se und ziemlich tiefe Eindr"ucke. Die von den Seitenfl"achen her mit derselben gebildeten Endkanten sind alle etwas geschweift, verdr"uckt, gebogen, und unregelm"a"sig im Verlaufe, aber doch ziemlich scharf, nur die von der konvexen gro"sen Seitenfl"ache her gebildete, ist stark verdr"uckt und etwas breit abgerundet.

Die Rinde ist durchaus gleichf"ormig dieselbe, und so wie sie bei Meteor-Steinen von "ahnlicher Beschaffenheit der Masse, bei einem solchen Aggregats-Zustande und einem gleichen qualitativen und quantitativen Verh"altnisse der Bestand- und Gemengteile, zumal bei einem "ahnlichen bedeutenden Gehalte an Gediegeneisen, durchgehends gefunden wird; n"amlich: von schw"arzlich-brauner, hie und da, mehr oder weniger, mit eisengrau und ockergelb und br"aunlich gemischter Farbe, sehr schwachen, matten, hie und da schimmernden, stellenweise matt metallischem Glanze, und ziemlich glatter, nur hie und da fein und verworren, kurz und runzlicht-aderiger, gr"o"sten Teils aber klein und platt k"orniger, narbiger oder warziger Oberfl"ache, mit ziemlich h"aufig eingestreuten eisengrauen, metallisch gl"anzenden Punkten, und gr"o"seren oder kleineren Flecken, als den vorragenden und abgeplatteten Spitzen und Zacken des eingemengten Gediegeneisens. Ihre Dicke betr"agt $\mathfrak{\frac{1}{12}}$ bis $\mathfrak{\frac{2}{12}}$, selten $\mathfrak{\frac{3}{12}}$ einer Linie. Ihre H"arte ist bedeutend, indem sie mit dem Stahle leicht und ziemlich wacker Funken gibt. Sie wirkt an allen Stellen sehr kr"aftig auf die Magnetnadel, und setzt eine ziemlich empfindliche auf einen halben Zoll Entfernung lebhaft in Bewegung.

Sie gleicht in allen diesen Eigenschaften am meisten jener der Meteor-Steine von Eichst"adt, Timochin, Barbotan, L'Aigle, Apt, Charsonville, Berlanguillas, Toulouse \emph{zc.}

An einigen Stellen, namentlich an drei Ecken der Grundfl"ache, und an einer der oberen Endfl"achen, und auf zwei Pl"atzen an den Grundkanten dieses Steines, zeigt sich etwas unvollkommene Rinde, das ist, Rinde, die sich nicht vollkommen ausgebildet hat, keine vollkommen zusammenhangende Kruste bildet, und die Steinmasse nicht ganz bedeckt, sondern nur in Tropfen, oder in, aus solchen zusammen geflossenen Adern oder Flecken dieselbe teilweise deckt.\footnote{\frakfamily{Es scheint nicht, dass die unvollkommene Rinde an diesen Stellen der sp"ateren Entstehung derselben, durch Lostrennung oder Absprengung eines St"uckes, und folglich dem Mangel des ben"otigten Zeit-Moments zu ihrer Bildung, welches am gew"ohnlichsten wohl der Fall sein d"urfte, sondern vielmehr der individuellen Beschaffenheit und dem besonderen Mengungsverh"altnisse der Grundmasse an diesen Stellen, welche der Rindenbildung mehr Widerstand leisteten, zuzuschreiben sei, wie denn auch diese Stellen nur sehr klein sind, und keinen Verlust der Masse erkennen lassen. Ich verweise "ubrigens hinsichtlich dieser Beschaffenheit der Rinde, welche sich mehr oder weniger beinahe auf jedem einzelnen Meteor-Steine findet, wie ich zuerst bemerkt habe, und welche umso merkw"urdiger ist, da sie uns am ersten "uber die h"ochst r"atselhafte, und zur Zeit noch gar nicht befriedigend erkl"arte Entstehung und Bildung der Rinde an den Meteor-Massen "uberhaupt Aufschluss geben k"onnte, auf die Erkl"arung von Fig. 3 und 4 der sechsten Tafel, und hinsichtlich der mannigfaltigen Beschaffenheit derselben "uberhaupt auf jene s"amtlicher Darstellungen auf der vierten, f"unften und sechsten Tafel, und im Allgemeinen auf meinen Aufsatz in Gilberts Annalen B. 31, und bitte damit zu vergleichen, was, hinsichtlich ihrer Entstehung und Bildung, Hr. Professor v. Scherer an denselben Orte, und Hr. D. Chladni in seinem neuesten Werke vorgebracht haben.}}

Die Abbildung des Steines, welche die Versinnlichung der auffallend regelm"a"sigen Form und der Beschaffenheit seiner Oberfl"ache zum Zwecke hat, ist von einer Ansicht desselben genommen, in welcher sich erstere und ihre "Ahnlichkeit mit einer bekannten Figur, insbesondere aber ihre "Ubereinstimmung mit andern "ahnlich gestalteten Meteor-Steinen am deutlichsten ausspricht. Der Stein ist diesem zu Folge auf seiner Grundfl"ache (ihn als Pyramide betrachtend) liegend, von der einen breiten, konvexen Seitenfl"ache etwas gewendet vorgestellt, um den ganzen Umriss, eine zweite Seitenfl"ache mit der verl"angerten Kante und der vorspringenden Ecke, und die obere Endfl"ache ersichtlich zu machen.

\subsection{\frakfamily{L'Aigle.}}
\paragraph{}
Einer von den gr"o"seren Steinen von dem besonders ergiebigen Steinregen,\footnote{\frakfamily{Im strengeren Sinne; denn es fielen doch zwischen zwei und drei Tausend Steine auf einen Fl"achenraum von h"ochstens 2 franz"os. Quadrat-Meilen, und zwar auf drei Explosions-Punkte beschr"ankt, die zusammen wohl kaum den f"unften Teil dieses Fl"achenraums betroffen haben m"ochten. Das Gesamtgewicht, nach einem "ahnlichen Ma"sstabe, wie bei dem Ereignisse von Stannern, gesch"atzt, d"urfte wohl 30 bis 40, vielleicht 50 Zentner betragen haben, da viele der Steine 3 bis 5, mehrere selbst zwischen 10 und 17 Pfund wogen. Au"serdem, dass dieses Ereignis, eben dieser Ergiebigkeit und der g"unstigen Umst"ande wegen, --- dass sich dasselbe n"amlich in einer so bewohnten und kultivierten Gegend, und bei hellem Tage zutrug, --- nicht nur das meiste Aufsehen in neuester Zeit erregte, und die schlummernde, bisher nur von Zeit zu Zeit durch minder bedeutende Vorf"alle "ahnlicher Art, und oft aus weiter Ferne her, schwach angeregte Aufmerksamkeit auf diese wunderbaren, und wie sich‘s bei Erwachung dieser bald zeigte (denn noch in demselben Jahre wurden drei "ahnliche beobachtet, und eine davon selbst noch innerhalb den Grenzen des alten Frankreichs, --- bei Apt, Departement Vaucluse, Oktober 1803 ---), keineswegs so seltenen Naturerscheinungen, erweckte, sondern auch nicht wenig beitrug, durch eine, auf Veranlassung des National-Instituts in Paris, von einem ber"uhmten Physiker (Biot) an Ort und Stelle vorgenommene legale und wissenschaftliche Untersuchung und Bew"ahrung des Factums, den noch ziemlich allgemein vorherrschenden Unglauben an die Realit"at solcher Begebenheiten zu verscheuchen; so ist es auch, aus eben diesen Gr"unden und durch den Spekulations-Geist eines Pariser Mineralien-H"andlers (Lambotin), dasjenige, wovon die meisten Belege erhalten wurden und in die Welt kamen.}} der sich am 26. April 1803, Nachmittags gegen 1 Uhr, zu L'Aigle (Departement de l'Orne, der ehemaligen Normandie) in Frankreich (etwa 25 franz. Meilen westlich von Paris) ereignete, von beinahe 2 Pfund am Gewicht.

Es ward derselbe noch im Laufe desselben Jahres, in welchem sich die Begebenheit zutrug, in Wien zu Kaufe geboten, und von dem damaligen Direktor, Abbé St"utz, f"ur das k. k. Mineralien-Kabinett angekauft.

Er ist vollkommen ganz und um und um "uberrindet, nur ist er hie und da an den Kanten etwas abgesto"sen, und eine Ecke ist abgebrochen, die sich aber dabei befindet.

Obgleich dieser Stein auf den ersten Anblick sehr unregelm"a"sig geformt zu sein scheint, die Fl"achen sehr uneben und ungleich, und die Kanten sehr verdr"uckt sind; so ist doch bei n"aherer Betrachtung desselben eine bestimmte, und, wie es scheint, nur zuf"allig verunstaltete Grundform unverkennbar, und auffallend die "Ubereinstimmung mit dem vorher beschriebenen Steine von Tabor.

Er bildet n"amlich ebenfalls eine verschoben und ungleichseitig vierseitige, abgestumpfte, niedere Pyramide, deren Grundfl"ache etwas "uber 3 Zoll, die obere Endfl"ache $\mathfrak{2\frac{1}{2}}$ Zoll, in beiden Durchmessern, und deren H"ohe beinahe $\mathfrak{2\frac{1}{2}}$ Zoll misst.

Die Grundfl"ache ist sehr gew"olbt, und ebenfalls durch Abstumpfung einer Kante, die aber hier besonders stark ist, so dass gleichsam eine neue Fl"ache durch dieselbe gebildet wird, sehr, und umso mehr verunstaltet, als auch die gegen "uberstehende Kante einiger Ma"sen abgestumpft und stark verdr"uckt ist. "Ubrigens hat diese Fl"ache nur wenige seichte Eindr"ucke.

Von den Seitenfl"achen sind ebenfalls zwei aneinandersto"sende klein, fast senkrecht, etwas konvex, und haben nur wenige breite, seichte Eindr"ucke. Die beiden andern gr"o"seren erheben sich unter einem etwas spitzigen Winkel schiefer, und sto"sen in eine ziemlich scharfe gemeinschaftliche Kante zusammen, welche mit den Kanten der Grundfl"ache ebenfalls eine hervorspringende Ecke bildet. Die eine dieser Fl"achen ist ebenfalls konvex, und ihr entspricht die abgestumpfte Kante der Grundfl"ache; die andere ist konkav: gerade wie beides am vorhin beschriebenen Steine von Tabor der Fall ist. Auch diese beiden Fl"achen haben nur sehr wenige kleine und seichte Eindr"ucke.

Die obere Endfl"ache entspricht zwar der Form nach, obgleich sie ziemlich scharf begrenzt ist, nicht der Grundfl"ache, da diese durch Abstumpfung und Verdr"uckung der Kanten sehr verunstaltet ist; dagegen vollkommen der gleichnamigen am Steine von Tabor: drei Schenkel des auf "ahnliche Art verschobenen ungleichseitigen Vierecks, welches dieselbe bildet, sind n"amlich ziemlich gleich, der vierte aber ist viel k"urzer; "ubrigens ist sie kleiner als die Grundfl"ache, und ebenfalls, durch ungleiche Erhebung der Seitenfl"achen von der Grundfl"ache, aus dem Mittel geschoben. Sie ist stark vertieft, und hat viele, meistens ziemlich tiefe, zum Teil zusammen gedr"angte, aber kleine Eindr"ucke.

Auch die oberen Endkanten stimmen an beiden Steinen darin "uberein, dass die von der konvexen Seitenfl"ache mit der oberen Endfl"ache gebildete, die stumpfeste, die von der konkaven die sch"arfste, die beiden andern etwas abgerundet sind.

Das Winkelma"s der meisten Kanten, insoweit dasselbe einiger Ma"sen bestimmbar ist, f"allt zwischen 80 u. 115$^{\circ}$.\footnote{\frakfamily{Herr Graf v. Fries allhier besitzt zwei Steine von diesem Ereignisse, wovon der eine, beinahe vollkommen ganze und "uber 3 Pfund schwere, in seiner Form auffallend mit dem hier beschriebenen "ubereinstimmt, selbst in dem Umstande, dass zwei Seitenfl"achen mit einer Ecke verl"angert sind; der andere aber von 24 Loth am Gewichte, obgleich unvollkommen, sich doch auch jener Form sehr n"ahert.}}

Die Rinde ist genau und in jeder Beziehung dieselbe, wie bei dem Steine von Tabor, nur im Ganzen etwas glatter, mehr klein und platt narbig als aderig, und etwas lichter braun, mit mehr br"aunlichen und gelblichen Ockerflecken, aber fast ohne Spur von Gediegeneisen. Ihre Dicke ist im Ganzen fast noch etwas geringer; an H"arte und Wirkung auf den Magnet kommt sie aber genau mit jener am Taborer-Steine "uberein.

An mehreren kleinen Stellen der Grundkanten, an den Kanten und an einer Ecke der oberen Endfl"ache, und an der gr"o"seren Ecke der Grundfl"ache, zeigt sich unvollkommene Rinde; aber nur an der letzteren Stelle scheint sie die Folge eines Verlustes an Masse, durch sp"atere Lostrennung eines St"uckes, zu sein.

Die Darstellung dieses Steines hat gleiche Zwecke, wie jene des vorhin beschriebenen Steines von Tabor, demnach sind dabei auch gleiche R"ucksichten genommen, und derselbe auf seiner --- angenommenen --- Grundfl"ache liegend, von der einen breiteren, gew"olbten Seitenfl"ache, etwas gewendet, vorgestellt worden, um den ganzen Umriss, die andere breite konkave Seitenfl"ache mit der verl"angerten Kante und der vorspringenden Ecke, und die obere Endfl"ache zur Ansicht zu bringen.

\subsection{\frakfamily{Eichst"adt.}}
\paragraph{}
Ein verschoben vierseitig pyramidales Bruchst"uck, 7 Loth schwer, von dem am 19. Februar 1785, nach 12 Uhr Mittags, bei Eichst"adt in Franken, so viel bekannt, einzeln gefallenen Steine von 5 Pfund 22 Loth am Gewicht, welches um das Jahr 1789 von dem Domherrn v. Hompesch zu Eichst"adt, dem damaligen Direktors-Adjunkten des k. k. Mineralien-Kabinettes, Abbé St"utz, mitgeteilt wurde, der es daselbst niederlegte.\footnote{\frakfamily{Es wurde dieses Stuck, wegen des offenbaren Gehaltes an Gediegeneisen, als des merkw"urdigsten Gemengteiles desselben, und mit ihm, aus gleichem Bestimmungsgrunde, der Stein von Tabor (so wie in der Folge der Stein von L'Aigle, und das Bruchstuck vom Mauerkirchner Meteor-Steine), der Agramer Eisen-Masse, und den vorhandenen St"ucken vom sibirischen Eisen, beigesellt, und die ganze Suite, bei der eben um jene Zeit vorgenommenen neuen systematischen Einrichtung des Kabinettes, mit der Suite der Magneteisen-Steine vereinigt, in einen Schrank eingereihet.\\
\hspace*{6mm}Die Erhaltung dieses St"uckes gab zu einem Aufs"atze Veranlassung, welchen Abbé St"utz noch in demselben Jahre, 1789, in Form eines Briefes, in das eben angefangene periodische Werk eines von Born und Trebra gestifteten montanistischen Vereines (Bergbaukunde 2. Band, Leipzig 1790) einr"ucken lie"s, und welcher nicht nur die fr"uheste umst"andlichere Nachricht von diesem Ereignisse, sondern auch die durch dasselbe angeregte und motivierte Bekanntmachung der h"ochst merkw"urdigen Urkunde "uber die Agramer Eisen-Masse, und zugleich auch eine Mutma"sung "uber den wahrscheinlichen Ursprung solcher angeblich aus der Luft gefallenen Massen enth"alt, die den damaligen Ansichten und dem allgemein herrschenden Unglauben --- wenigstens an eine urspr"unglich "uberirdische Entstehung derselben --- entsprechend, und in dieser Voraussetzung gerade bei diesen zwei dem Verfasser n"aher bekannt gewordenen Vorfallen (Agram n"amlich und Eichst"adt, als wo nur einzelne Massen fielen) wirklich am annehmbarsten war. Eine Mutma"sung, die "ubrigens schon 20 Jahre fr"uher von den Pariser Akademikern, mit Lavoisier an ihrer Spitze ausging, und 12 Jahre sp"ater noch (1802) von einem bekannten franz"osischen Physiker (Patrin) bei Gelegenheit der Howard'schen Resultate und Folgerungen, und gegen dieselben, verteidiget wurde.\\
\hspace*{6mm}Bruchstucke von diesem Eichst"adter Steine geh"oren "ubrigens zu den seltensten und am wenigsten bekannten von allen Meteorolithen neuerer Zeit, indem die Total-Masse so unbedeutend war, und die Begebenheit selbst erst sp"at allgemeiner bekannt wurde. (N"amlich lange nach St"utz, 1805 erst, gab Prof. Pickel zu Eichst"adt Nachricht davon in v. Molls Annalen.) Ein gro"ses Stuck davon befindet sich am Berg-Collegium in M"unchen, ein kleines besitzt Herr v. Moll daselbst, und kleine Fragmente finden sich meines Wissens in den durch Vollst"andigkeit in dieser Partie ausgezeichneten Sammlungen des Marquis De Drée in Paris, und des j"ungst verstorbenen L. R. Lavaters in Z"urich. Klaproth opferte ein erhaltenes Bruchst"uck der Analyse, und Chladni suchte vergebens ein Fragment f"ur seine Sammlung aufzutreiben.}}

Obgleich dieses Bruchst"uck, dem Gewichte nach, nur den 26sten Teil des ganzen Steines betr"agt, so l"asst sich doch aus den noch daran vorhandenen nat"urlichen, mit Rinde bedeckten Fl"achen welche ohne Zweifel Seitenfl"achen waren --- und aus deren Richtung, so wie aus der gemeinschaftlichen Kante, in welche dieselben zusammen sto"sen, nicht nur auf eine regelm"a"sige, sondern selbst auf eine vierseitig pyramidale, und somit den vorhin beschriebenen Steinen von Tabor und L'Aigle sehr "ahnliche Form, welche dieser Stein, als ganz, gehabt haben d"urfte, mit aller Wahrscheinlichkeit schlie"sen.

Die beiden "uberrindeten Fl"achen erheben sich n"amlich schief unter einem Winkel von 72$^{\circ}$ von der angenommenen breiteren, freilich hier gebrochenen, Grundfl"ache (wie dies bei einer der gr"o"seren schiefern Seitenfl"achen des Taborer Steines wirklich beil"aufig auch der Fall ist), und verschm"alern sich offenbar nach oben, lassen also keinen Zweifel "uber die urspr"unglich pyramidale Form des Steines.

Sie sto"sen ferner unter einem Winkel von 116$^{\circ}$ beil"aufig, in eine gemeinschaftliche Kante zusammen (bemerkenswert, dass am Taborer Steine bei einer stumpf abgerundeten gemeinschaftlichen Kante zweier Seitenfl"achen ein "ahnlicher Winkel von 115$^{\circ}$ vorkommt); verl"angert man sich nun diese beiden Seitenfl"achen, wovon hier nur ein Teil, und zwar im Mittel, von 12 und 15 Linien vorhanden, nach ihrer offenbaren Richtung bis an ihre h"ochst wahrscheinliche urspr"ungliche Grenze von Ausdehnung in die Breite, d. i. auf etwa 4 Zoll (welche Gr"o"se\footnote{\frakfamily{Nach St"utz Nachricht, die sich auf eine schriftliche Mitteilung des B. Hompesch gr"undet, hatte der Stein ungef"ahr einen halben Schuh im Durchmesser. (Chladni gibt, wahrscheinlich aus einem kleinen Versehen im Niederschreiben, einen Schuh an.) Dieses kann, nach den Gewichtsverh"altnissen, nur insofern gegr"undet sein, als man damit den l"angsten meinte, etwa von einer Ecke queer zur entgegen gesetzten gemessen, und dann m"usste selbst noch, wie oben erw"ahnt, eine Ecke etwas verl"angert gewesen sein, und wenn der Stein wirklich pyramidal war, dessen H"ohe kaum mehr als 3 Zolle betragen haben.}} der Stein, als Cubus genommen, nach seinem absoluten Gewichte und dem spezifischen = 3,7 beil"aufig gehabt haben m"ochte); so kommt, wenn man kein sehr ungleichseitiges Prisma, oder ganz willk"urlich, eine polyedrische Gestalt sich denken will --- wogegen dieser so sehr regelm"a"sige Teil des Ganzen, und in gewisser Beziehung das angegebene Ma"s des Steines selbst, streitet -- ein verschobenes Viereck heraus, das h"ochst wahrscheinlich ungleichseitig war, und eine vorspringende Ecke hatte, weil sonst- nach obigen Gewichtsverh"altnissen -- beinahe bei keiner andern denkbaren Form des Steines, mit welcher sich die Gestalt dieses Bruchst"ucks vereinigen lie"se, ein Durchmesser von 6 Zoll (wie doch ausdr"ucklich angegeben wird) sich ergeben k"onnte.

Die beiden "uberrindeten Fl"achenreste sind "ubrigens fast ganz flach und eben, besonders die eine; die andere hat nur ein paar etwas seichte Stellen, die man kaum Eindr"ucke nennen kann.

Die Rinde ist im Ganzen wie an den Steinen von Tabor und L'Aigle, nur etwas dunkler schwarzbraun, und mehr kurzaderig-runzlich als narbig, und am "ahnlichsten jener an den Steinen von Timochin und Tipperary. Sie ist merklich dicker als an irgendeinem mir bekannten Meteor-Steine (auch hierin kommt, wenigstens stellenweise, die an den Steinen von Timochin und Tipperary ihr am n"achsten), zumal an einer dieser Fl"achen, wo sie beinahe eine halbe Linie erreicht.\footnote{\frakfamily{St"utz gibt aus Versehen, weil er wahrscheinlich verga"s die Betrachtung mit einer Handlupe, die wohl drei bis vier Mahl vergr"o"sert haben mag, angestellt zu haben, die Dicke auf 2 Linien an.}}

Ihre H"arte ist etwas geringer als die der Rinde der Steine von Tabor und L'Aigle, doch gibt sie ziemlich leicht am Stahle Funken; dagegen wirkt sie merklich st"arker auf die Magnetnadel, und setzt dieselbe fan auf $\mathfrak{\frac{3}{4}}$ Zoll Entfernung in Bewegung. (Auch in diesen beiden Eigenschaften steht ihr die Rinde an den Steinen von Timochin und Tipperary am n"achsten.) Und sie gibt dadurch nicht allein, sondern auch durch h"aufige, etwas erhabene eisengraue metallische Punkte und kleine Flecke von abgeplatteten Spitzen und Zacken, den starken Gehalt dieses Steines an Gediegeneisen zu erkennen.\footnote{\frakfamily{Es ist dieser Meteor-Stein nicht nur der Gehaltreichste an Gediegeneisen, wie dies auch das spezifische Gewicht bew"ahrt (das nach meiner Wiegung zwischen 3,680 und 3,730 schwankt, und worin ihm nur die Steine von Tipperary nach Higgins, und von Timochin nach Klaproth gleich zu kommen scheinen, und die Steine von Charsonville nach Hauy, und von Tabor nach eigener Wiegung --- denn der Bournon'schen Gewichtsangabe zu 4,28 liegt offenbar ein Versehen oder der Umstand zum Grunde, dass das gewogene St"uck zuf"allig ein gro"ses Eisenkorn einschloss --- nahe kommen), sondern er enthalt dasselbe auch in den gr"o"sten, massivsten (obgleich immer noch sehr zarten), und hie und da wirklich "astig verbundenen und zusammen h"angenden Zacken, wie sich am deutlichsten an einer abgeschliffenen Fl"ache erkennen l"asst. Es bedarf in der Tat wohl kaum mehr eines Zwischengliedes, um den "Ubergang der Masse dieses Steines in jene des sibirischen Eisens (zumal in die dichteren, weniger zelligen, und mehr erdig-ockrigen Partien desselben, und der angeblich norwegischen und s"achsischen Massen im Ganzen) sinnlich nachzuweisen, umso weniger, als in derselben bereits auch der olivinartige Gemengteil (wof"ur man, nach "au"serem Ansehen, Art der Einmengung, nach den physischen Eigenschaften und chemischen Bestandteilen, das mandelsteinartig eingemengte, gleichsam in rundlichte Zellen eingeschlossene und meist von Gediegeneisen umgebene Fossil --- das sich mehr oder weniger und in verschiedenen Graden von Ausbildung, wie bei Erkl"arung der siebenten Tafel gezeigt werden wird, in allen Meteor-Steinen findet --- zu erkennen nicht anstehen kann) so sehr pr"adominiert, dass derselbe mit den Metallteilchen gut $\mathfrak{\frac{2}{3}}$ der Gesamtmasse betr"agt.}} 

Die Darstellung dieses Bruchst"uckes ist, der Absicht gem"a"s, und nach den bereits erw"ahnten R"ucksichten, von den beiden mit Rinde bedeckten nat"urlichen Fl"achenresten, und von der gemeinschaftlichen Kante, in welche sie zusammensto"sen, genommen.

\subsection{\frakfamily{Siena.}}
\paragraph{}
Ein Bruchst"uck, oder vielmehr h"ochst wahrscheinlich (nach Gr"o"se, Form, Richtung und Ausdehnung der vorhandenen, nat"urlichen, mit Rinde bedeckten Fl"achen) wenigstens die H"alfte eines (urspr"unglich etwa 3 bis 4 Loth schwer gewesenen) mittelgro"sen Steines, von 7 Qu"antchen am Gewichte, von dem am 16. Junius 1794, Abends nach 7 Uhr, bei Siena im Toskanischen Statt gehabten betr"achtlichen Steinniederfalle.\footnote{\frakfamily{Es ist dieses einer der Steinniederfalle neuerer Zeit, von welchem die Produkte ziemlich bekannt und verbreitet wurden, obgleich man die Realit"at der Begebenheit, trotz einer gepflogenen legalen Untersuchung, und die Herkunft und den "uberirdischen Ursprung der Steine zur Zeit des Ereignisses selbst, sehr bezweifelte. Allein die Begebenheit machte gro"ses Aufsehen, da sie bedeutend war (es fielen einige hundert, aber meist nur kleine, oder doch nur mittelgro"se, einige Lothe, auch nur wenige Qu"antchen schwere Steine --- nur einzelne wenige wogen 3 bis 7 Pfund --- auf einen Fl"achenraum von 2 bis 3 italienischen Meilen), und sich bei Tage und vor vielen Augenzeugen ereignete; von angesehenen Gelehrten, Tata, Soldani, Spallanzani, viel dar"uber geschrieben wurde, und mehrere angesehene und gelehrte Engl"ander, Thomson, Hamilton, Lord Bristol, sich eben damals in Italien befanden, welche dem Gegenstande, der zu gro"sen Debatten Veranlassung gab, noch mehr Zelebrit"at im Auslande verschafften. Es finden sich demnach Belege von diesem Ereignisse in vielen Sammlungen, namentlich im Mus. brit. zu London, der De Drée'schen Sammlung zu Paris, und in jenen Chladnis, Lavaters, Blumenbachs, Klaproths \emph{zc.}}}

General Tihavsky, der sich eben damals zur Zeit des Ereignisses in Neapel befand, erhielt dieses St"uck von dem ebenfalls da anwesenden gelehrten Engl"ander Thomson, welchem es von Soldani aus Siena zugeschickt wurde, und brachte es bei seiner R"uckkehr mit nach Wien; aber erst als der Steinfall bei Stannern die Aufmerksamkeit der Physiker, zumal in Wien, neuerdings und so m"achtig in Anspruch nahm, ward es zur Sprache gebracht, und von dem gef"alligen Besitzer auf mein Ansuchen dem kaiserlichen Kabinette zum Geschenke gemacht.

Es ist zwar an diesem Steine an zwei Stellen, und zwar, wie es scheint, mit bestimmter Vorsicht, Masse abgeschlagen worden, und die beiden solcher Gestalt entstandenen, ziemlich gro"sen, und unter einem Winkel von 85$^{\circ}$ zusammen sto"senden frischen Bruchfl"achen lassen zwar an und f"ur sich ihre urspr"ungliche Gestalt, Beschaffenheit, Richtung und Ausdehnung nicht wohl erraten; doch l"asst sich aus der Form des vorhandenen St"uckes, und den drei mit Rinde bedeckten Seitenfl"achen, und der noch ganz vollkommenen Endspitze, mit aller Wahrscheinlichkeit darauf schlie"sen, und es scheint nach dieser Ansicht die eine dieser Bruchfl"achen die vierte gr"o"sere gew"olbte Seitenfl"ache, die andere die untere End- oder Grundfl"ache des Steines gewesen zu sein. Und bei dieser Annahme erscheint die urspr"ungliche Form dieses Steines nicht nur sehr regelm"a"sig als verschobene und ungleichseitig vierseitige Pyramide mit durch drei Fl"achen zugespitzter Endspitze, sondern auffallend "ubereinstimmend mit jener des auf der vierten Tafel vorgestellten gro"sen Steines von Stannern, umso mehr, als die Grundfl"ache ebenfalls ein "ahnlich verschobenes Viereck mit einer stark vorspringenden Ecke gebildet zu haben scheint, und die Endspitze durch eine "ahnliche Richtung und Ausdehnung der Zuspitzungsfl"achen ebenfalls aus dem Mittel ger"uckt ist, und durch die zwei breiteren gegen "uber stehenden Zuspitzungsfl"achen zu einer Kante gebildet wird.

Die vorhandenen, mit Rinde bedeckten Seitenfl"achen, stehen ziemlich senkrecht auf der als Grundfl"ache betrachteten Bruchfl"ache: die eine, breiteste, ist fast eben; die n"achste, kleinste von allen, welche mit voriger unter einem Winkel von etwa 80$^{\circ}$ jene gemeinschaftliche Kante bildet, auf welche die Zuspitzungsfl"ache aufgesetzt ist, ist etwas konkav; die dritte, welche unter einem sehr stumpfen Winkel von beinahe 135$^{\circ}$ mit letzterer zusammenst"o"st, ist etwas gew"olbt. Die eine auf die Kante aufgesetzte Zuspitzungsfl"ache bildet ein auf eine Ecke gestelltes Rhomboid, ist die kleinste und etwas vertieft; die beiden andern sind breiter und gr"o"ser, sehr unregelm"a"sig gestaltet, und, zumal die eine, fast ganz eben. Sie sto"sen unter einem Winkel von 90$^{\circ}$ in die gemeinschaftliche Endkante zusammen. Alle Fl"achen haben nur wenige, kaum bemerkbare, seichte, kleine Eindr"ucke.\footnote{\frakfamily{Die kaiserl. Sammlung besitzt au"ser diesem noch zwei vollkommen ganze, obgleich nur sehr kleine Steine von dieser Begebenheit. Der eine, um und um mit vollkommener, und nur an einer Ecke mit unausgebildeter Rinde bedeckte, der nur ein Qu"antchen wiegt, zeigt der Form nach, trotz seiner Kleinheit, eine auffallende "Ahnlichkeit mit den beschriebenen Steinen von Tabor und von L'Aigle, indem er, selbst hinsichtlich der gew"olbten Grundfl"ache, und der einen stark vorspringenden Ecke, eine "ahnliche, verschoben und ungleichseitig vierseitige, abgestumpfte, niedere Pyramide bildet. Der andere, etwas gr"o"sere, von $\mathfrak{2\frac{1}{4}}$ Qu"antchen am Gewichte, der nur an einem Ende etwas verbrochen ist, und an einer Fl"ache und an zwei andern kleinen Stellen unvollkommene Rinde zeigt, hat eine Form, die sich jener des n"achst zu beschreibenden Steines von Lissa sehr n"ahert. Die Rinde an diesen beiden Steinen, die vielleicht lange dem Einfl"usse der Witterung ausgesetzt waren, zeigt, obgleich sie eben so d"unne, zart und rissig ist wie an dem oben beschriebenen, durch das ganz matte Ansehen und eine mehr braune, mit Rostflecken gemengte Farbe, einige "Ahnlichkeit mit jener der Steine von L'Aigle.}}

Die Rinde ist besonders zart und d"unne, beinahe kohlschwarz, etwas ins Graue ziehend, von wenigem und mattem, aber etwas seidenartigen, stellenweise schimmernden Glanze, und von gar keinem Ansehen, das einen Metallgehalt verriete. Sie ist "ubrigens sehr zart rau, fein und eng, kurz und verworren, runzlicht-aderig, und voll zarter Risse, welche unregelm"a"sige Felder bilden. Sie hat die meiste "Ahnlichkeit einerseits mit der Rinde an den Steinen von Lissa, Agen, York, andererseits mit jener an den Steinen von Parma und Benares, und zeigt "uberhaupt von dem geringen Metallgehalt der Masse, welchen auch das spezifische Gewicht vermuten l"asst (3,3 bis 3,4). Sie gibt am Stahle nur schwer und schwache Funken, und wirkt auch nur schwach auf die Magnetnadel, kaum auf $\mathfrak{\frac{2}{12}}$ Linie Entfernung.

Die Abbildung zeigt diesen Stein auf die eine, als untere End- oder Grundfl"ache betrachtete Bruchfl"ache aufgestellt, von der gemeinschaftlichen Kante, in welche die einen zwei mit Rinde bedeckten Fl"achen zusammensto"sen, und auf welche die eine Zuspitzungsfl"ache aufgesetzt ist, die mit den beiden andern breiteren, welche schief auf den Seitenfl"achen aufsitzen, die kantige Endspitze bildet.
\clearpage
\section{\frakfamily{Dritte Tafel.}}
\subsection{\frakfamily{Lissa.}}
\paragraph{}
Der gr"o"ste und einzig ganz und vollkommen erhaltene von den vier bei Lissa (zwischen den D"orfern Strattow und Wustra, 4 bis 5 Meilen O. N. O. von Prag) im Bunzlauer Kreise in B"ohmen am 3. September 1808, Nachmittags um halb 4 Uhr, gefallenen und im Falle beobachteten und aufgefundenen Steinen.

Er wiegt 5 Pfund 19 Loth.

Es wurde derselbe von vier Augenzeugen, in deren N"ahe er niederfiel, im Auffallen beobachtet, gleich aufgehoben und an das Oberamt zu Lissa abgeliefert, welches, nachdem es am 8. September eine f"ormliche Untersuchung des Factums vorgenommen, und eine offizielle Anzeige davon an das k"onigl. Kreisamt zu Bunzlau erstattet hatte, denselben bis zu der in Folge des kreisamtlichen Berichtes, von Seite des k"onigl. B"ohmischen Landes-Guberniums veranlassten wissenschaftlichen Untersuchung, welche am 17. November Statt fand, aufbewahrte, und dann an die Untersuchungs-Kommission abgab, von welcher derselbe mit den diessf"alligen Berichten nach Wien eingesendet wurde.\footnote{\frakfamily{Bruchst"ucke von Steinen, als Belege dieser Begebenheit neuester Zeit, m"ochten wohl zu den seltensten und am schwersten zu erhaltenden geh"oren. Denn f"urs erste war der Steinfall von sehr geringer Bedeutung, es fielen n"amlich nur vier Steine, die zusammen kaum 18 Pfund wogen, und wenn gleich unter den gew"ohnlichen tumultuarischen Erscheinungen, doch ohne gro"ses Aufsehen zu erregen, und nur vor wenigen Augenzeugen; so wie denn auch die ganze Begebenheit schwerlich beachtet worden, noch weniger zur "offentlichen Notiz gekommen sein w"urde, wenn nicht, erst drei Monate fr"uher, und zwar kaum auf 20 Meilen Entfernung, eine "ahnliche, der Steinfall bei Stannern, Statt gehabt, oder vielmehr, wenn nicht diese vorausgegangene Begebenheit durch die veranlassten amtlichen Untersuchungen, die selbst zu jener Zeit noch im Gange waren, und sich sogar, einiger Nebenerscheinungen wegen, "uber die Grenzen B"ohmens erstreckten, die Aufmerksamkeit der Lokal-Beh"orden, und selbst des Landvolks in jener Gegend aufgeregt gehabt h"atte. Andererseits wurden die gefallenen Steine nur wenig zerst"uckelt, und erhielten bald eine fixe Bestimmung. II. k. k. HH. die Erzherzoge Rainer und Johann erhielten gro"se Bruchst"ucke f"ur H"ochstderen Sammlungen, ebenso Se. Excellenz Herr Graf v. Wrbna; kleine St"ucke blieben zum Angedenken in Kloster zu Lissa, in den H"anden einiger Beamten, und im Besitze des Hrn. D. Reu"s von Bilin. Diese m"ochten, mit den beiden St"ucken der kaiserl. Sammlung, allein schon "uber 10 Pfund am Gewichte betragen. Von dem Reste befinden sich, meines Wissens, kleine Fragmente in den Sammlungen Chladnis, Klaproths und De Drées, und ein Bruchst"uck von etwa 7 Loth in der Sammlung der mineralogischen Gesellschaft zu Jena.}} Dieser Stein ist, bis auf einige kleine Stellen an den sch"arfern Kanten, wo die Rinde etwas abgesto"sen ist, und zwei Ecken, wo urspr"unglich ein St"uck abgeschlagen worden war, doch so, dass die Form des Steines keineswegs gelitten, und der Verlust der Masse kaum 5 bis 6 Loth betragen haben mag, vollkommen ganz und durchaus mit der gew"ohnlichen Rinde bedeckt.

Seine Gestalt ist nicht minder auffallend regelm"a"sig als jene der beschriebenen Steine von Tabor und von L'Aigle, und noch mehr die "Ahnlichkeit, die hierin zwischen allen dreien Statt findet.

Er bildet n"amlich ebenfalls eine deutliche, verschoben und ungleichseitig vierseitige, stark abgestumpfte, niedere Pyramide, die nur etwas mehr als an den beiden vorigen in die Breite gezogen ist, so dass die beiden Endfl"achen ein mehr l"anglichtes Viereck bilden.

Die gr"o"sere End- oder Grundfl"ache\footnote{\frakfamily{Den Stein von dieser Ansicht und bei dieser Haltung betrachtet, in welchen sich n"amlich dessen Regelm"a"sigkeit und die "Ahnlichkeit mit einer geometrischen Figur am auffallendsten ausspricht und am deutlichsten beschreiben und darstellen l"asst.\\
\hspace*{6mm}Herr Bergrath Reu"s, welchem bei Gelegenheit der wissenschaftlichen Untersuchung des Factums, zu welcher derselbe beauftragt wurde, und bei Ansicht dieses Steines die Regelm"a"sigkeit der Form desselben nicht entgangen war, ob er gleich durch keine "ahnliche Beobachtung aufmerksam gemacht worden zu sein scheint, betrachtete den Stein kristallologisch, folglich in einer andern Haltung, n"amlich der L"ange nach, die beiden Endfl"achen als Seitenfl"achen nehmend, und beschreibt ihn demnach --- kristallographisch (in Gehlens Journal f"ur Chemie, Physik und Mineralogie, B. 8. S. 447, 1809) als eine unregelm"a"sige f"unfseitige S"aule (die beiden Abstumpfungfl"achen der Grundkanten als einzelne Seitenfl"achen betrachtend), mit sehr ungleichen Seitenfl"achen, und an welcher eine Endfl"ache schief angesetzt (eine der schm"alern gewolbtern Seitenfl"achen), die andere mit zwei sehr ungleichen Fl"achen zugesch"arft ist (die, jener gegen "uber stehende, keineswegs gedoppelte, sondern blo"s durch gro"se und tiefe Eindr"ucke verdr"uckte und verunstaltete Seitenfl"ache).}} hat "uber 6 Zoll im l"angeren, und $\mathfrak{4\frac{1}{2}}$ Zoll im schm"alern Durchmesser, die kleinere oder obere Endfl"ache $\mathfrak{4\frac{1}{2}}$ zu 3 Zoll, und die Seitenfl"achen haben $\mathfrak{3\frac{1}{2}}$ Zoll H"ohe.

Die Grundfl"ache ist sehr unregelm"a"sig, und durch viele, zum Teil ziemlich gro"se und tiefe Eindr"ucke, vorz"uglich aber durch starke Abstumpfung der beiden Grundkanten der gegen"uberstehenden breiteren Seitenfl"achen sehr verunstaltet, indem durch diese gewisser Ma"sen zwei schiefe Fl"achen gebildet werden, die fast in der Mitte der Grundfl"ache zusammensto"sen. (Es ist bemerkenswert, dass die st"arkere Abstumpfung, gerade wie beim Taborer und L'Aigler Steine, dieselbe breite und gew"olbte Seitenfl"ache trifft; besonders auffallend aber ist "ubrigens die "Ahnlichkeit hinsichtlich der doppelten Abstumpfung und der Gew"olbtheit der Grundfl"ache mit dem letzteren.)

Von den Seitenfl"achen sind ebenfalls zwei gr"o"ser und breiter; auch ist die eine davon konvex, und durch viele ziemlich gro"se und tiefe Eindr"ucke sehr verunstaltet; die andere konkav, mit sehr wenigen kleinen seichten Eindr"ucken. Diese beiden Fl"achen, welche in Hinsicht der Beschaffenheit ihrer Oberfl"ache zweien aneinandersto"senden am Taborer Steine so "ahnlich sind, grenzen hier nicht aneinander, sondern stehen sich gegen "uber, und sind mehr senkrecht als schief aufgestellt. Die von beiden mit der Grundfl"ache gebildeten Kanten sind, wie bereits bemerkt, stark schief abgestumpft; die mit der oberen Endfl"ache gebildeten aber ziemlich scharf. Von den beiden andern Seitenfl"achen, die etwas schiefer aufsteigen, ist die eine ziemlich gew"olbt, hat viele kleine, nicht sehr tiefe Eindr"ucke, aber eine gro"se und ein paar kleine Vertiefungen, die von einem bruchst"uckweisen Verluste der Masse (durch sp"atere Lostrennung oder Absprengung) vor der Rindenbildung herzur"uhren scheinen, und welche diese Fl"ache sehr verunstalten; die andere ist m"a"sig gew"olbt, sonst eben, und wenige seichte Eindr"ucke abgerechnet, besonders glatt. Beide bilden mit der Grundfl"ache sehr zugerundete, mit der oberen Endfl"ache dagegen besonders scharfe Kanten. Die gemeinschaftliche Seitenkante, in welche jene letztere ebenere Seitenfl"ache mit der angrenzenden, konkaven, breiteren Seitenfl"ache zusammensto"st, und welche besonders scharf ist (der Winkel = 80-85$^{\circ}$), bildet mit den Grundkanten dieser Fl"achen ebenfalls eine stark hervorspringende Ecke, wie dies bei den Steinen von Tabor und von L'Aigle der Fall ist.

Die obere Endfl"ache bildet ein ziemlich regelm"a"siges, l"anglichtes, verschobenes Viereck, entspricht ziemlich dem Mittel der Grundfl"ache, ist aber wegen schiefer Richtung der Seitenfl"achen betr"achtlich kleiner, fast flach, nur etwas konkav, und durch viele aber kleine und sehr seichte Eindr"ucke uneben gemacht. Sie gleicht jener am Taborer und L'Aigler Steine auch darin, dass drei Schenkel des Vierecks bedeutend gr"o"ser sind als der vierte; "ubrigens ist sie l"anglichter.

Das Winkelma"s schwankt, obgleich es sich wegen starker Ungleichheit, Eindr"uckung und Verdr"uckung der Kanten nur an wenigen Stellen approximativ bestimmen l"asst, nur zwischen 80 und 110$^{\circ}$.\footnote{\frakfamily{Ein kleines, 3 Loth schweres Bruchst"uck eines urspr"unglich ebenfalls bei 5 Pfund schwer gewesenen, aber in mehrere St"ucke zerschlagenen Steines von diesem Ereignisse, zeigt die Reste von zwei "uberrindeten Fl"achen, wovon die eine besonders flach, eben und glatt ist, und, von einer als Basis angenommenen Bruchfl"ache, unter einem Winkel von etwa 84$^{\circ}$, die andere, etwas unebenere, vertieftere, eingedr"ucktere, und, der Rinde nach, rauere, unter 60$^{\circ}$ aufsteigt, und welche, unter einem Winkel von beil"aufig 65 bis 70$^{\circ}$, in eine besonders scharfe gemeinschaftliche Kante zusammen sto"sen, die wieder von derselben Basis unter einem Winkel von 50 bis 55$^{\circ}$ aufsteigt, daher wohl die hervor springende Ecke jenes Steines gebildet hat, der nach diesen Indizien h"ochst wahrscheinlich eine "ahnliche verschoben vierseitige Pyramidal-Form, wie der beschriebene, gehabt haben d"urfte.\\
\hspace*{6mm}Der Stein im Besitze Sr. k. H. des Erzherzogs Johann, im Johanneo zu Gr"atz, --- welcher 1 Pfund 7 Loth wiegt, und beinahe vollkommen ganz ist, obgleich er dem ersten Anblicke nach nur ein gro"ses Bruchst"uck zu sein scheint, indem eine gro"se Fl"ache nur mit sehr unvollkommener Rinde bedeckt, oder vielmehr gleichsam nur angeflogen ist, --- stellt ein etwas verschobenes vierseitiges Prisma vor; und das Bruchst"uck in der Sammlung Sr. Excellenz des Hrn. Grafen v. Wrbna, von 22 Loth am Gewicht, l"asst wenigstens auf eine "ahnliche rhomboidale Form des Steines, von dem es abgeschlagen wurde, schlie"sen.}}

Die Rinde h"alt, dem Aggregats-Zustande und dem quantitativen Verh"altnisse der Gemengteile gem"a"s, nach welchen diese Steine gleichsam ein Verbindungsglied zwischen zwei darin, und folglich dem "au"sern Ansehen nach ziemlich stark abweichenden Reihen von Meteor-Steinen bilden, das Mittel zwischen jener an den Steinen von Tabor, L'Aigle, Eichst"adt \emph{zc.}, und jener der Steine von Siena, Parma, Benares \emph{zc.}, am "ahnlichsten ist sie aber der Rinde an den Steinen von York und Agen, mit welchen diese Steine auch in obigen Beziehungen die meiste "Ahnlichkeit haben.\footnote{\frakfamily{Ich behalte mir vor, bei einer andern Veranlassung "uber diese Reihenbildung, "Ahnlichkeit und "Uberg"ange der verschiedenen Meteor-Steine umst"andlicher zu sprechen, und verweise inzwischen auf die Erkl"arung der siebenten Tafel.}}

Sie ist n"amlich hier, und namentlich an diesem Steine, schwarz, beinahe kohlschwarz, ohne allem metallischockerbraunen Ansehen, im Ganzen zwar mehr matt als gl"anzend, aber doch stellenweise von einem seidenartigen Schimmer, und, obgleich sehr zart, doch mehr runzlicht als narbigt, oder warzig rau. Obgleich sie beim ersten Anblick in diesen Beziehungen gleichf"ormig "uber den ganzen Stein ausgedehnt zu sein scheint; so zeigt doch eine genauere Betrachtung und Vergleichung einige Verschiedenheit. An einer H"alfte dieses Steines, und zwar an der oberen Endfl"ache, an der breiten konkaven, und der kleineren verunstalteten Seitenfl"ache (welche Fl"achen, nach obiger Beschreibung, auch in Betreff der "ubrigen Beschaffenheit ihrer Oberfl"ache mit einander "ubereinstimmen), zeigt sie sich ganz auf die beschriebene Weise; an der Grundfl"ache dagegen, der breiten konvexen und der andern kleineren, ebenfalls gew"olbten Seitenfl"ache (die ihrer "ubrigen Beschaffenheit nach wieder mit einander "ubereinstimmen), erscheint sie mehr braun, mit einem schwachen, etwas ins Kupferrote ziehenden Schimmer, zumal in den Eindr"ucken, im Ganzen aber matter und glatter, wenigstens weniger aderig; auch scheint sie hier etwas d"unner zu sein. Eine kleine, aber kaum beschreibbare Abweichung, zeigt in allen Beziehungen die eine kleinere, am meisten gew"olbte und ebenste Seitenfl"ache, so dass demnach dieser Stein, hinsichtlich seiner Oberfl"ache, eine dreifache Verschiedenheit, gewisser Ma"sen drei Seiten, zeigt.\footnote{\frakfamily{Von dieser, wie mir deucht, h"ochst merkw"urdigen, und von mir zuerst an den Steinen von Stannern beobachteten Verschiedenheit der Oberflache sowohl, als insbesondere der Rinde an ein und demselben Steine, wird bei Beschreibung der in dieser Beziehung besonders ausgezeichneten ganzen Steine von Stannern, und bei Erkl"arung der Figuren auf der f"unften und sechsten Tafel, die Rede sein. Zeigt sich gleich an diesem Steine von Lissa diese Verschiedenheit, zumal der Rinde, nicht so auffallend (wie es auch bei ihrer Beschaffenheit in Allgemeinen als Folge des Aggregats-Zustandes und des qualitativen und quantitativen Verh"altnisses der Gemeng- und Bestandteile, und insbesondere des Eisengehaltes wegen nicht anders sein kann, und noch weniger bei jenen Meteor-Steinen der Fall ist, deren Gehalt an --- Gediegen --- Eisen noch weit betr"achtlicher befunden wird); so zeigt sie sich doch, was in anderer Hinsicht nicht minder merkw"urdig ist, wie es auch von ganz anderen Ursachen herr"uhrt, um so auffallender zwischen den einzelnen Steinen von dieser Begebenheit. An dem einen kleinen Bruchstucke der Sammlung n"amlich ist die Rinde noch weit schwarzer, noch mehr seidenartig schimmernd, zumal an der einen Flache, und, "au"serst zart zwar, aber sehr ausgezeichnet, runzlicht-aderig, und "uberhaupt der Rinde der Steine von Parma und Benares gar sehr "ahnlich; dagegen die Steine in Besitze Sr. k. H. des Erzherzogs Johann, und Sr. Excellenz des Hrn. Grafen v. Wrbna, eine Rinde zeigen, die beinahe ganz jener an den Steinen von Tabor, L'Aigle u. s. w. "ahnlich, matt, braun und weit glatter ist. Und ebenso der Rinde entsprechend und mit gleicher Ann"aherung, ist auch die innere Beschaffenheit und das Ansehen der Masse im Bruche an diesen Steinen verschieden. Diese Verschiedenheit, sowohl in Hinsicht der Beschaffenheit der Oberfl"ache und Rinde, als auch der Masse im Innern, die offenbar von einer Verschiedenheit im Aggregats- und Koh"asions-Zustande, und wenigstens des quantitativen Verh"altnisses der Gemengteile abh"angt, findet sich "ubrigens nicht blo"s bei den Steinen von dieser Begebenheit, sondern auch bei mehreren andern, namentlich bei jenen von Stannern und Siena, insbesondere auch bei jenen von L'Aigle (wie auch Hr. Chladni bemerkte), und mochte vielleicht bei den meisten gefunden werden, wenn man Gelegenheit hatte, so viele Stein und Bruchst"ucke von ein und demselben Ereignisse vergleichen zu k"onnen, wie es bei diesen der Fall war; und sie findet sich nicht blo"s bei verschiedenen einzelnen Steinen desselben Niederfalles, ob sie gleich auch als Bruchstucke einer Hauptmasse, der Feuerkugel, betrachtet werden, sondern bisweilen selbst bei Bruchstucken eines und desselben Steines, so dass sich solche oft un"ahnlicher sind, wie dies vorz"uglich bei obigen Steinen von Lissa und bei manchen von L'Aigle der Fall ist, als Bruchst"ucke von Steinen von, nach Zeit und Ort, sehr entfernten Begebenheiten.}}

Die Dicke der Rinde ist "ubrigens im Ganzen, wie an den meisten Meteor-Steinen, etwa zwischen $\mathfrak{\frac{1}{12}}$ bis $\mathfrak{\frac{3}{12}}$ Linien. Nur an einzelnen kleinen Stellen, hie und da an den Kanten, zeigt sich eine Spur von unvollkommener, unausgebildeter Rinde, wo die Masse des Steines mehr oder weniger ver"andert (etwas gebr"aunt) zu Tage liegt, und es das Ansehen hat, als wenn die fl"ussige Rindenmasse "uber diese Stellen sich nicht h"atte ausbreiten, nicht zusammenflie"sen k"onnen. In einem kleinen, aber tiefen Eindrucke an einer der Fl"achen, findet sich eine solche Stelle, wo die Masse des Steines ganz und gar unver"andert ist, und den frischesten Bruch zeigt, indes doch der sie begrenzende Rindenrand deutlich erkennen l"asst, dass es kein k"unstlicher Bruch ist.

Ihre H"arte ist kaum geringer als die der Rinde an den Steinen von Tabor und L'Aigle; aber auf die Magnetnadel wirkt sie bedeutend schw"acher.

Die Abbildung stellt den Stein nach der Ansicht und Haltung, nach welchen die Beschreibung genommen, auf der gr"o"seren End- oder Grundfl"ache liegend vor, so dass, mit dem ganzen Umrisse, die eine ausgezeichnetere, breitere, konkave Seitenfl"ache, die obere Endfl"ache, und zum Teil noch die zwei kleinen Seitenfl"achen, wovon die eine mit der vorderen die verl"angerte Kante und vorspringende Ecke bildet, zu ersehen sind.
\clearpage
\section{\frakfamily{Vierte Tafel.}}
\subsection{\frakfamily{Stannern.}}
\paragraph{}
Der gr"o"ste\footnote{\frakfamily{Au"ser einem, von Joseph Wurschy von Neustift, in derselben Gegend, in einem W"aldchen, etwa 2500 Klafter n"ordlich von der Kirche von Stannern, gefundenen Steine (Nr. 61 des Planes), welcher 13 Pfund gewogen haben soll, aber in kleine St"ucke zerschlagen wurde, lie"s sich, trotz allen mittel- und unmittelbaren lang fortgesetzten Nachforschungen, kein "ahnlicher an Gr"o"se weiter nachweisen. Die n"achsten an Gewicht waren schon Steine zwischen 3 und 5 Pfund, und deren m"ochten wohl kaum mehr als jene 6 bis 7 gefallen und aufgefunden worden sein, welche der Plan nachweiset.}} von den bei Stannern in M"ahren, am 22. Mai 1808, Morgens gegen 6 Uhr, gefallenen Steinen,\footnote{\frakfamily{Obgleich dieser Steinfall gerade keiner von den bedeutendsten war, indem nach den genauesten Nachforschungen, die wohl bei keiner Begebenheit der Art so umst"andlich und fortgesetzt angestellt wurden, kaum mehr als 100 Steine zu einem Gesamtgewicht von h"ochstens 150 Pfund gefallen sein d"urften; so sind doch die Belege davon ebenso, und beinahe allgemeiner noch, wenigstens zweckm"a"siger, verbreitet, als jene vom Steinregen zu L'Aigle, der doch in jeder Beziehung zwanzig bis drei"sig Mahl ergiebiger war. Man hat dies den Einleitungen zu verdanken, welche bei diesem Ereignisse zur geh"origen Untersuchung des Factums, zum Einsammeln, und dann zu einer zweckm"a"sigen (unentgeldlichen) Verteilung der entbehrlichen Steine und Bruchst"ucke an die bekanntesten "offentlichen Sammlungen, und an die vorz"uglichsten Privat-Sammler und Schriftsteller aus dieser Partie in ganz Europa, getroffen worden sind, und es w"are wohl sehr zu w"unschen, dass von den Regierungen aller Staaten bei "ahnlichen Ereignissen auf gleiche Art verfahren werden m"ochte. Auf diese Weise k"onnte sehr leicht eine "ahnliche (gewiss sehr wichtige, und wie wir "uberzeugt zu sein glauben, in der Folge sicher noch zu sehr bedeutenden Aufschl"ussen f"uhrende) Zusammenstellung der Produkte (der ausgezeichnetsten, und in irgend einer Beziehung merkw"urdigen Steine und Bruchst"ucke) eines jeden vorfallenden Ereignisses der Art, an einen bestimmten, zweckm"a"sigen Platz (an irgend einer "offentlichen wissenschaftlichen Anstalt im Staate), und eine "ahnliche Verbreitung und Verteilung der entbehrlichen St"ucke an andere "ahnliche Pl"atze ("offentliche Museen und Privat-Sammlungen) --- womit einerseits die nicht minder wichtige und notwendige, gr"o"stm"oglichste und vollst"andigste Zusammenstellung solcher Produkte von verschiedenen Ereignissen, an verschiedenen Orten, und zur ausgebreitetsten Benutzung, andererseits eine sichere und dauernde Aufbewahrung derselben f"ur Mit- und Nachwelt erzielt w"urde --- bewirkt, und damit am meisten zur seinerzeitigen Aufkl"arung dieser, in so vielfachen Beziehungen r"atselhaften, Naturerscheinung beigetragen werden. Der bisherigen Vernachl"assigung solcher Ma"sregeln ist es zuzuschreiben, dass wir von achtzig bis hundert Tausend "ahnlichen Ereignissen, die sich, nach einem h"ochst wahrscheinlichen Kalk"ul, seit unserer Zeitrechnung blo"s in Europa zugetragen haben m"ochten, kaum von einem Hundert derselben hinl"anglich beglaubigte Nachrichten, und von diesem kaum von drei und drei"sig (und diese beinahe ausschlie"slich von Ereignissen aus der neuesten Zeit, von den letzten 70 Jahren) nachweisbare, materielle Belege besitzen, und dass wir, nach Jahrtausenden, jetzt in diesem Jahrhunderte erst, nicht nur die ersten Schritte zur Aufkl"arung zu machen, sondern selbst noch den Unglauben an die Realit"at dieser ebenso auffallenden als wunderbaren Ph"anomene, die sich seit Menschengedenken, und keineswegs so selten, auf unserem Planeten ereigneten und immerfort ereignen, zu bek"ampfen haben.}} welcher ganz erhalten wurde.

Es ward derselbe erst gegen Ende des Monats Julius jenes Jahres, also zwei Monate nach dem Ereignisse, indem er in ein Kornfeld gefallen war und da verborgen blieb, von Katharina Pauser und ihrem Manne, einem Tagl"ohner von Neustift, im Beisein noch einiger Arbeitsleute, auf dem Felde des Neustifter Bauers, Jacob Achatzi, N. N. O. vom Markte Stannern, und zwar bei 3000 Klafter von der Kirche, fast am "au"sersten Ende (kaum 250 Klafter vom "au"sersten Punkte, wo noch ein Stein gefallen war) des befallenen Fl"achenraums gegen N. (Situations-Plan Nr. 59), zuf"allig w"ahrend des Kornschneidens aufgefunden.

Er steckte fest in der Erde, und nur eine Ecke desselben ragte hervor, welche die Aufmerksamkeit des Tagl"ohnerweibes auf sich zog, indem es das geschnittene Korn zusammenraffte und in Garben band. Die Erde war sehr trocken und fest, und der Mann hatte M"uhe, den Stein herauszubringen. Im Herausheben brach die in der Erde stecken gebliebene Spitze, oder vielmehr die eine obere Ecke ab. Das Gewicht desselben ward beil"aufig auf $\mathfrak{9\frac{3}{4}}$ Pfund gesch"atzt, wie es sich auch im Plane angegeben findet; der Stein wiegt aber wirklich 11 Pfund und 10 Loth Wiener Kommerzial-Gewicht.

Au"ser einigen feinen und seichten Rissen, und hie und da etwas abgeschlagenen Kanten und Ecken, ist derselbe vollkommen ganz und durchaus mit Rinde bedeckt.

Es stellt derselbe eine wenig verschobene, und beinahe gleichseitig vierseitige Pyramide vor, deren etwas aus der Mitte ger"uckte Endspitze durch drei neue, auf den Seitenfl"achen aufsitzende, unvollkommene Fl"achen schief zugespitzt ist. Die Grundfl"ache ist fast eben, und hat wenige gro"se, seichte, breit verlaufende Eindr"ucke. Die von ihr mit den fast senkrecht aufsteigenden Seitenfl"achen gebildeten Kanten sind meistens etwas verdr"uckt und abgerundet, eine jedoch ist sehr scharf, und bildet einen Winkel von 90$^{\circ}$. Eine Seitenecke ist besonders hervorspringend, und nur wenig abgerundet und auffallend ist die "Ahnlichkeit der Grundfl"ache dieses Steines, zumal in Hinsicht dieses Umstandes, mit jener der zuvor beschriebenen Steine von Tabor, L'Aigle, und selbst von Lissa, so wie die der Form des Steines im Ganzen, mit jener des Steines von Siena.

Zwei Seitenfl"achen, welche unter einem Winkel von beil"aufig 100$^{\circ}$ in eine ziemlich scharfe Kante zusammensto"sen, die mit den Kanten der Grundfl"ache jene hervorspringende Ecke bildet, sind fast ganz flach und eben, nur etwas vertieft, und haben sehr wenige seichte, sanft verlaufende Eindr"ucke. Die zwei entgegen gesetzten Seitenfl"achen sto"sen in eine stumpfere und verdr"uckte gemeinschaftliche Kante zusammen, und bilden "ahnliche Kanten mit den vorigen Seitenfl"achen und mit der Grundfl"ache. Sie sind konvex, zumal die eine derselben, und durch h"aufigere, zum Teil tiefe Eindr"ucke, sehr uneben.

Die drei unvollkommenen Zuspitzungsfl"achen, wovon die eine, gr"o"sere, fast gerade auf der einen gew"olbteren Seitenfl"ache aufsitzt, und mit derselben eine sehr verdr"uckte, undeutliche Kante unter einem sehr stumpfen Winkel bildet, die beiden andern, kleineren, aber auf den etwas vertieften Seitenfl"achen schief, und so aufgesetzt sind, dass sie mit jener eine au"ser die Mitte fallende Zuspitzungs-Endkante bilden, --- wovon die abgebrochene Spitze die eine Ecke ausmachte, --- haben die Beschaffenheit der Oberfl"ache mit den korrespondierenden Seitenfl"achen gemein.

Die Rinde\footnote{\frakfamily{Was die merkw"urdige Beschaffenheit der Rinde an den Meteor-Steinen von Stannern im Allgemeinen, die auffallende Verschiedenheit derselben, nicht nur an verschiedenen einzelnen Steinen, sondern selbst oft, und zwar sogar gew"ohnlich an einem und demselben Steine, und die gro"se Mannigfaltigkeit hinsichtlich der besonderen Beschaffenheit ihrer Oberflache, und was endlich die Folgerungen betrifft, die sich aus der genauen vergleichenden Betrachtung derselben ziehen lassen; so verweise ich auf das, was Herr Professor von Scherer und ich im 31. Bande von Gilberts Annalen dar"uber umst"andlich vorgebracht haben, und wozu die gegenw"artigen Darstellungen (zumal die Figuren der f"unften und sechsten Tafel) gewisser Ma"sen als versinnlichende Belege dienen sollen.}} ist fast durchaus "uber den ganzen Stein von gleicher, und zwar von der gew"ohnlichsten Beschaffenheit, wie sie an diesen Steinen "uberhaupt zu sein pflegt, ziemlich gleich dick, dicht und fest, etwas fettig, und ziemlich stark gl"anzend, rein dunkelschwarz, und von der rauen, einfach und verworrenen, runzlicht-aderigen Art (A. a. 2. Gilberts Annalen Bd. 31, S. 56); nur an den gew"olbteren, unebeneren Fl"achen n"ahert sie sich der blattf"ormig gezeichneten (ebendas. A. a. 3), und ist hier matter, etwas weniger schwarz, und, wie es scheint, etwas d"unner.

Sie ist nirgends abgesprungen, aber auch an keiner Stelle zeigt sich, trotz der bedeutenden Oberfl"ache dieser gro"sen Masse, eine Spur von der unvollkommenen Art (ebendas. S. 58. D.).

Viele Runzeln und Adern, zumal an den Kanten, sind stark erhaben, scharf und falten"ahnlich. S"aume der Rinde finden sich an diesem Steine nirgendwo, wohl aber an den sch"arfern Kanten, wo die stark aderige Rinde von zwei Fl"achen zusammensto"st, deutliche N"ahte.

Die Dicke derselben weicht, so wie "uberhaupt bei diesen Steinen im Allgemeinen, nicht von der gew"ohnlichen Dicke der Rinde an andern Meteor-Steinen ab, und betr"agt im Ganzen $\mathfrak{\frac{1}{12}}$ bis $\mathfrak{\frac{2}{12}}$ Linie.

Ihre H"arte ist nur sehr gering, und nur schwer, und blo"s an einzelnen Stellen (an diesem Steine wohl an gar keiner) lassen sich der Rinde dieser Steine "uberhaupt mit dem Stahle einzelne schwache Funken entlocken; eben so wenig zeigt sie eine merkliche Wirkung auf die Magnetnadel; nur gepulvert bleiben einzelne Atome an der Spitze h"angen.

Es zeigt sich zwar allenthalben an diesem Steine, in den Furchen und Vertiefungen des Adergeflechtes der Rinde, etwas Erde\footnote{\frakfamily{Diese Erde l"asst sich inzwischen selbst da, wo sie am festesten an- und eingedr"uckt zu sein scheint, doch ziemlich leicht und ohne Verletzung der zartesten Adern und Runzeln, mit einer scharfen B"urste wegb"ursten, und mit einem nassen Schwamme vollends rein wegwaschen, so dass keine Spur in irgendeiner Beziehung von ihrem fr"uheren Dasein zur"uckbleibt. Ein Umstand, der wohl, mit manchen andern Beobachtungen, sehr gegen die Annahme des fl"ussigen oder doch weichen Zustandes der Rinde, selbst noch im Momente des Auffallens der Steine, streitet.}} eingedr"uckt, --- was bei dem tiefen und gewaltsamen Eindringen des Steines in das Erdreich, und bei den wiederholten Regeng"ussen, welche in der ziemlich langen Zwischenzeit bis zu dessen Auffinden Statt hatten, wohl nicht anders sein konnte, --- am meisten jedoch an den konvexen Fl"achen, auf welche der Stein auch, verm"oge seines Schwerpunktes, aufgefallen sein m"usste, falls dieser nicht etwa durch eine rotierende Bewegung des Steins im Falle, --- welcher inzwischen einerseits die Beschaffenheit der Rinde, wenn diese als fl"ussig, andererseits die Form des Steines, wenn die Masse weich gedacht werden soll, --- widerspr"ache, turbiert worden w"are.\footnote{\frakfamily{Ich bemerke das noch sichtliche Ankleben von Erde an diesem, wie insbesondere an allen folgenden ganzen Steinen von Stannern, absichtlich mit Genauigkeit, weil dasselbe hier --- wo es sich "ubrigens, der eigent"umlichen Rauigkeit der Oberfl"ache wegen, auch deutlicher zeigen und l"anger erhalten konnte als an irgend einem andern Meteor-Steine --- mit vollkommenster Verl"asslichkeit, die wahren Auffallspunkte der einzelnen Steine --- inzwischen aber auch jene Stellen, welche bei tieferem Eindringen derselben in den Grund nebenher noch mit Erde in Ber"uhrung kamen, --- bezeichnet, indem die meisten dieser Steine (nur den eben beschriebenen und die beiden folgenden kleinsten ausgenommen) unmittelbar w"ahrend der Begebenheit, oder doch nur wenige Tage nach dem Ereignisse, in welcher Zwischenzeit noch keine Ab"anderung in der urspr"unglichen Lagerung derselben, noch eine zuf"allige Ver"anderung mit der umgebenden Erde Statt gefunden haben konnte, aufgehoben und unmittelbar aus der ersten Hand, von dem Auffinder selbst, erhalten worden waren.}}

Der Stein ist, auf der Grundfl"ache liegend, so dargestellt, dass sich die eine Seitenfl"ache mit der aufsitzenden Zuspitzungsfl"ache in gerader, die zwei ansto"senden Seitenfl"achen aber, wovon die eine mit ersterer die etwas verl"angerte Seitenkante und die vorspringende Ecke bildet, in schiefer Ansicht zeigen.
\clearpage
\section{\frakfamily{F"unfte Tafel.}}
\subsection{\frakfamily{Erste Figur.}}
\paragraph{}
Einer der kleineren, aber vollkommen ganz erhaltenen Steine von dem Ereignisse bei Stannern, der 5 Loth 1 Qu"antchen wiegt, und sich durch eine besonders regelm"a"sige Form auszeichnet.

Er ward durch das von der Untersuchungs-Kommission veranlasste absichtliche Aufsuchen der gefallenen Steine, am 28. Mai von einem Landmanne zwischen dem Markte Stannern und dem Dorfe Lang-Pirnitz, oder vielmehr ganz nahe an diesem letzteren Orte, im s"udlichen Teile des befallenen Fl"achenraums (und zwar etwa 2600$^{\circ}$ s"udlich von der Kirche von Stannern, und kaum 1500$^{\circ}$ vom "au"sersten Punkte, wo noch ein Stein in diesem Teile gefallen war, dagegen "uber 5000$^{\circ}$ von der Fallstelle des vorhin beschriebenen Steines entfernt) aufgefunden. (Situations-Plan Nr. 19.)

Es ist derselbe vollkommen ganz, um und um "uberrindet, und bildet eine unvollkommene, dreiseitige Pyramide, deren "Ahnlichkeit, obgleich sie sich, durch Abrundung und Abstumpfung der Ecken und Kanten zum Teil beinahe einer Kugelform n"ahert, mit der Form des gro"sen, zuvor beschriebenen Steines unverkennbar, und umso auffallender ist, als sich an der Grundfl"ache dieses Steines, durch Abplattung und Breitdr"uckung einer Ecke, die Tendenz zu einer "ahnlichen (vielleicht urspr"unglich gewesenen und nur abge"anderten) verschoben und ungleichseitig vierseitigen Pyramidal-Form, die an jenem ausgesprochen ist (an den Figur 2 und 5 dieser Tafel vorgestellten Steinen aber auch nur in einem "ahnlichen Grade angedeutet erscheint), nicht verkennen l"asst.

Die stark konvexe und unebene Grundfl"ache des Steines stellt n"amlich ein ungleichschenkliches Dreieck vor, dessen R"ander mit den drei ziemlich senkrecht aufsteigenden, fast ganz ebenen, nur etwas vertieften Seitenfl"achen stumpfe Kanten bilden, und dessen stumpfe Ecken den abgerundeten Seitenkanten entsprechen. Die eine dieser Ecken ist aber gleichsam platt und breit gedr"uckt, und geht, zugerundet, unmittelbar in eine ebenfalls breit gedr"uckte und abgerundete, beinahe zu einer vierten Seitenfl"ache gestaltete Seitenkante "uber, die bogenf"ormig, allm"ahlich sich verd"unnend, gegen die Endspitze verl"auft.

Nach dem andern Ende des Steines verschm"alern sich die Seitenfl"achen, und endigen sich in eine etwas nach einer Fl"ache hin- und selbst etwas "ubergebogene (folglich ebenfalls, und zwar sehr stark, au"ser das Mittel der Grundfl"ache fallende) dreiseitige, ziemlich scharfe Spitze, die durch zwei sehr unvollkommene und ungleiche, schief auf die Seitenfl"achen aufgesetzte Fl"achen zugespitzt, und gewisser Ma"sen kantig zugesch"arft wird.\footnote{\frakfamily{Die von mir in Gilberts Annalen von diesem Steine schon fr"uher gegebene Beschreibung, B. 31, S. 36, D., spricht die Form desselben nicht deutlich genug aus.}}

Nur auf der Grundfl"ache finden sich einige einzelne, ziemlich seichte und kleine Eindr"ucke.

Die Rinde ist "uber den gr"o"sten Teil des Steines, und eigentlich durchaus eine und dieselbe, und zwar von gleicher zart strahlig-aderige- Beschaffenheit (A. b. 1. Gilberts Annalen, Bd. 31, S. 57), und einem, mit dieser Beschaffenheit stets verbundenen, stellenweise (wo n"amlich die oberste Schichte abgesprungen oder abgesto"sen ist, was bei dieser Art Rinde gew"ohnlich Statt findet) matten, im Ganzen aber starken, seidenartigen, schimmernden Glanze, und beinahe kohlschwarzer Farbe. Die feinen erhabenen Strahlen sind zwar kurz und oft unterbrochen, und verwirren sich oft hin und wieder, zumal bei ihrem Ursprunge, wo sie ein Geflecht bilden; doch scheinen sie von der Spitze aus "uber die Seitenfl"achen gegen die Grundfl"ache hin ihre Hauptrichtung zu nehmen, an deren Kanten, zumal von zwei Fl"achen her, sie sich verdicken, anh"aufen und als ein gezackter, ziemlich scharf abgeschnittener Rand enden, ohne einen Saum oder eine Naht zu bilden.

Auf der Grundfl"ache zeigt sich zwar dieselbe Rinde, ihrer Hauptbeschaffenheit nach, allein nur in Spuren, denn die oberste Schichte, die auf den Seitenfl"achen nur hie und da an kleinen Stellen abgesto"sen ist, scheint hier ganz zu fehlen, und ihre Oberfl"ache erscheint beinahe matt, nur wenig schimmernd, und mehr braun als schwarz. Allein bei Betrachtung unter der Lupe zeigt sich, dass die obere Schichte doch nicht abgerieben oder abgesto"sen ist, --- in welchem Falle solche Stellen ganz matt, por"os und gleichsam schwammig erscheinen, --- sondern dass sie nur in einer andern Modifikation vorhanden ist, n"amlich Statt Runzeln und Adern, gro"sen Teils blo"s erhabene Punkte und K"orner bildend.

Von eingedr"uckter Erde zeigt sich an der schm"ahlern, der gebogenen, breit gedr"uckten Seitenkante entgegen gesetzten Seitenfl"ache die meiste Spur, aber auch hier nur in den zarten Zwischenr"aumen der erhabenen, scharfen Adern, und, wie am gew"ohnlichsten, in den vertieften mikroskopischen Punkten und Poren der Oberfl"ache.

Die Abbildung, welche diesen Stein als Musterst"uck solcher von geringerer Gr"o"se bei vollkommener Integrit"at und von ausgezeichneter Form darstellen soll, zeigt denselben, auf einer Seitenfl"ache liegend und mit der Endspitze nach unten gekehrt, um mit dem so viel als m"oglich ganzen Umrisse die gew"olbte Grundfl"ache, und die eine, breiteste, Seitenfl"ache --- gegen welche die Spitze gebogen ist --- mit ihren Seitenr"andern --- wovon der eine die gebogene, breit gedr"uckte Kante bildet --- ersichtlich zu machen.

\subsection{\frakfamily{Zweite Figur. a. b.}}
\paragraph{}
Ebenfalls einer von den kleineren, bei Stannern gefallenen Steinen, 4 Loth 1 Qu"antchen wiegend, welcher ganz erhalten worden ist, und eine auffallend regelm"a"sige Form zeigt.

Es wurde derselbe, am andern Tage nach dem Ereignisse, von einem Landmanne auf einem Haberfelde zwischen Lang- und Klein-Pirnitz, ebenfalls im s"udlichen Teile des befallenen Fl"achenraums (und zwar etwa 2400$^{\circ}$ s"udlich von der Kirche von Stannern, beil"aufig 700$^{\circ}$ "ostlich von der Fallstelle des vorhin beschriebenen "ahnlichen Steines, und ziemlich in gleicher Entfernung vom "au"sersten Fallpunkte in S.), flach aufliegend und einen starken Zoll tief in das Erdreich eingedrungen gefunden, und am 28. Mai mir selbst zu Lang-Pirnitz, wo ich auf der Fahrt nach Stannern angehalten hatte, um vorl"aufige Erkundigungen einzuziehen, auf mein Verlangen "uberlassen. (Situations-Plan Nr. 16.)

Er ist vollkommen ganz und durchaus "uberrindet, nur eine Ecke ist etwas abgesto"sen, und ein kleines St"uck der oberen Endspitze abgeschlagen; der Verlust an Masse kann indes kaum 2 Qu"antchen betragen.

Es stellt derselbe eine etwas verschobene, aber ziemlich gleichseitig dreiseitige, oder vielmehr eine ungleichseitig vierseitige, etwas verl"angerte Pyramide vor. Er zeigt n"amlich eigentlich zwar nur drei ziemlich gleich breite Seitenfl"achen; allein eine derselben ist, durch eine, obgleich nur unvollkommene Kante, die sich aber an der Grundfl"ache doch durch eine deutliche Ecke ausspricht, der L"ange nach in zwei sehr ungleiche H"alften geteilt.

Diese solcher Gestalt geteilte Seitenfl"ache ist im Ganzen etwas konvex, und durch verh"altnism"a"sig sehr gro"se Eindr"ucke sehr uneben, ja durch einen besonders gro"sen und tiefen gegen die Basis hin, welcher beinahe einem Verluste an Masse, durch sp"atere Lostrennung oder Absprengung eines St"uckes (wenn diesem nicht zum Teil die Gleichf"ormigkeit der Rinde widerspr"ache) zugeschrieben werden k"onnte, gewisser Ma"sen verunstaltet. Die beiden andern Seitenfl"achen, welche mit dieser beiderseits unter einem ziemlich stumpfen Winkel, in eine sehr stumpfe, verdr"uckte und ausgeschweifte, unter sich aber in eine beinahe schneidend scharfe, aber im Verlaufe, durch Eindr"ucke von den Fl"achen her, mehrere Male gebogene gemeinschaftliche Kante, unter einem ziemlich spitzen Winkel zusammen sto"sen, sind ziemlich flach, eher etwas vertieft, und haben zwar ziemlich viele, aber nur seichte und breit verlaufende Eindr"ucke, die mehr den Unebenheiten einer nat"urlichen Bruchfl"ache des Steines, als den gew"ohnlichen Eindr"ucken gleichen. Nach dem einen Ende zu verschm"alern sich die Seitenfl"achen allm"ahlich, und gehen, nachdem sich die eine unvollkommenere Kante, welche die konvexe Seitenfl"ache teilte, mit der n"achsten zu vereinigen scheint, in die Spitze "uber, welche, obgleich sie abgebrochen ist und urspr"unglich fehlt, nach der Richtung der Fl"achen stumpf und dreiseitig, und etwas gegen die konvexe Fl"ache gebogen gewesen sein d"urfte.

Die Grundfl"ache ist fast flach, nur etwas vertieft, sonst vollkommen eben, und bildet ein sehr ungleichseitiges, verschobenes Viereck, indem jeder Seitenfl"ache --- selbst den beiden sehr ungleich geteilten H"alften der einen konvexen --- eine Kante, und jeder Seitenkante --- selbst der unvollkommenen, jene Fl"ache teilenden --- eine, wenn gleich stumpfe, Ecke entspricht. Die mit den beiden H"alften der konvexen Seitenfl"achen gebildeten Kanten sind sehr stumpf, jene mit den zwei andern Seitenfl"achen aber ziemlich scharf, und da diese Seitenfl"achen mit ihrer gemeinschaftlichen Kante sich nach diesem Ende des Steines hin betr"achtlich verl"angern; so erh"alt die Grundfl"ache dadurch eine ganz schiefe Richtung gegen die viel k"urzere konvexe Seitenfl"ache, und die durch jene verl"angerte Seitenkante mit den beiden Grundkanten der Seitenfl"achen gebildete Ecke springt bedeutend vor, und scheint (da sie verbrochen ist) ziemlich scharf gewesen zu sein.

Die Rinde ist an diesem Steine besonders merkw"urdig, und zeigt eine wesentliche und auffallende Verschiedenheit nach den verschiedenen Fl"achen desselben.

Auf der konvexen Seiten- und der mit dieser auch im "Ubrigen "ubereinstimmenden Grundfl"ache ist sie von der sehr rauen, runzlicht-aderigen Art (A. a. 1. Gilberts Annalen Bd. 31, S. 56), mit dem gew"ohnlichen Glanze, der durch matte Stellen --- wo n"amlich die oberste Schicht abgesprungen ist --- unterbrochen wird, und von mehr brauner als schwarzer Farbe. Auf den beiden andern Fl"achen dagegen ist sie ganz glatt, sehr dicht, fest und gleichf"ormig, sehr schwach aderig, und nur sehr undeutlich und unvollkommen blattf"ormig gezeichnet, pechschwarz und sehr fettig gl"anzend. (B. 2. ebendas. S. 57.) Von der konvexen Seiten- und der Grundfl"ache "uber die Kanten her, bildet die dortige rauere Rinde auf die Rinde dieser Fl"achen her"uber undeutliche und nicht scharf begrenzte S"aume.\footnote{\frakfamily{Diese h"ochst merkw"urdige, und, wie mir d"aucht, f"ur die Erkl"arung der Bildung der Rinde sowohl, als der Formierung (Vereinzelung) der Steine sehr wichtige Eigenheit derselben, S"aume zu bilden, spricht sich am deutlichsten an dem gleich zu beschreibenden, und vorzugsweise deshalb ("ubrigens auch der Gro"se, Vollkommenheit und Form wegen) auf derselben Tafel Figur 5 abgebildeten Steine aus, mit welchem dieser, und zwar nicht nur in der Form, --- sogar in den einzelnen Unregelm"a"sigkeiten derselben, --- sondern auch in der ganzen Beschaffenheit und Art der "Uberrindung, die auffallendste "Ahnlichkeit und "Ubereinstimmung zeigt.}}

Von unvollkommener Rinde zeigt sich keine Spur an diesem Steine, und von eingedr"uckter Erde nur etwas an der Grundfl"ache, auf welche der Stein, verm"oge seines nat"urlichen Schwerpunktes auch aufgefallen sein m"usste.

Figur 2. a. stellt diesen merkw"urdigen Stein, auf den beiden glatten Seitenfl"achen und ihrer gemeinschaftlichen Kante liegend, von der konvexen, unebenen und unvollkommen geteilten Seitenfl"ache, und der mit derselben in schiefer Richtung verbundenen Grundfl"ache vor;

Figur 2. b. zeigt denselben aber, auf jener Seitenfl"ache ruhend, von der gemeinschaftlichen, schneidend scharfen Kante, in welche die beiden andern Seitenfl"achen zusammensto"sen.

\subsection{\frakfamily{Dritte Figur.}}
\paragraph{}
Einer der kleinsten, und doch vollkommen "uberrindeten Steine von dem Ereignisse bei Stannern, von kaum $\mathfrak{2\frac{1}{2}}$ Qu"antchen am Gewichte.

Es ward derselbe einige Zeit nach der Begebenheit, in Folge nachtr"aglicher amtlicher Aufforderung an das Landvolk jener Gegend, die etwa noch verborgen liegenden Steine aufzusuchen und abzuliefern, an das k. k. Kreisamt zu Iglau eingebracht, und von diesem mit mehreren andern eingesendet.

Da dieser Stein zu klein und unbedeutend schien, so ward weder der Finder namentlich angezeigt, noch in dem sp"aterhin aufgenommenen Situations-Plane die Stelle angedeutet, wo derselbe aufgefunden wurde; indessen doch in dem Einbegleitungsschreiben bemerkt: dass derselbe aus der Gegend von Lang-Pirnitz, demnach aus dem s"udlichen Teile des befallenen Fl"achenraums, eingebracht worden sei.

Er ist vollkommen ganz, und nur an einer Seite etwas abgeschlagen, so dass der Verlust an Masse etwa ein halbes Qu"antchen betragen haben m"ochte; au"ser dieser Stelle ist er um und um mit Rinde bedeckt.

Er bildet eine etwas verdr"uckte, verschoben aber ziemlich gleichseitig vierseitige, sehr abgestumpfte und niedere Pyramide, und gleicht mit dieser Form, die sich ziemlich deutlich auf den ersten Blick ausspricht, und im Kleinen, sehr dem Steine von Tabor; nur ist die Grundfl"ache (so wie bei den Steinen von Lissa und L'Aigle) durch eine sehr starke schiefe Abstumpfung einer Kante, wodurch die Fl"ache gleichsam in zwei H"alften geteilt wird, verunstaltet, wodurch sich die Form von dieser Seite mehr jener des Steines von L'Aigle n"ahert. Die beiden Endfl"achen, den Stein in dieser Haltung betrachtet, sind sehr uneben, sonst ziemlich flach; die obere kleinere und etwas aus dem Mittel ger"uckte zeigt einige kleine, aber ziemlich tiefe Eindr"ucke; die untere gr"o"sere, mehr l"anglicht viereckigte, wird durch die neue, durch die Abstumpfung gebildete, sehr stumpfe Kante, welche die Fl"ache der Quere nach in zwei ziemlich gleiche H"alften teilt, gew"olbt gemacht. Eine der Seitenfl"achen ist beinahe senkrecht aufgesetzt, ganz flach und eben; die gegen"uberstehende etwas schief, konvex und uneben; die dritte sehr schief aufsteigende etwas konkav, und dies eigentlich durch ein paar verh"altnism"a"sig sehr gro"se, aber seichte und sehr breit verlaufende Eindr"ucke; und die vierte, dieser gegen"uberstehende, ist die verbrochene. Alle diese Fl"achen bilden sowohl unter sich als vorz"uglich mit der Grundfl"ache, am wenigsten mit der oberen Endfl"ache, ziemlich scharfe Kanten. Die Rinde scheint auf den ersten Anblick "uber den ganzen Stein durchaus von ganz gleicher Beschaffenheit zu sein, ist es wohl auch im Wesentlichen, zeigt aber doch bei n"aherer Betrachtung einige untergeordnete Modificationen.\footnote{\frakfamily{Wenn man sich die Rinde bei ihrer Entstehung, w"ahrend ihrer Bildung, und wenigstens einige Zeit w"ahrend des Falles des Steines, in einem mehr oder weniger fl"ussigen Zustande denken will (und das muss man wohl, wenigstens bei den Meteor-Steinen von Stannern), und zumal, wenn man (was man, wie mir d"aucht, weniger muss noch soll) die Rinde-bildende Potenz aus der Luft selbst (den durch Kondensation ausgepressten und durch Reibung erzeugten W"armestoff) nehmen will; so muss jeder Stein an seinen verschiedenen Fl"achen oder Seiten, je nachdem sie, nach dessen Richtung im Falle (wenn auch eine Achsenbewegung dabei Statt f"ande, welcher jedoch, ohne der Form zu erw"ahnen, die Beschaffenheit der Rinde, diese als fl"ussig angenommen, an den meisten Steinen offenbar widerspricht), mehr oder weniger dem Luftstrome entgegen gestellt waren, wenigstens eine zwei-, ja wohl dreifache kleine, untergeordnete Modifikation der Rinde, wenn diese auch "uber den ganzen Stein von einer und derselben Hauptbeschaffenheit sein sollte --- was sie in einzelnen F"allen auch wohl sein kann --- erkennen lassen. Und dies scheinen wirklich die Meteor-Steine von Stannern, deren Rinde, verm"oge ihrer ganz eigent"umlichen Natur und Beschaffenheit, vorzugsweise, ja bis jetzt beinahe ausschlie"slich geeignet ist diese Modifikationen auszusprechen, zu best"atigen. Ein anderes ist es um jene Hauptverschiedenheiten der Rinde, deren ich in meinem Aufs"atze, in Gilberts Annalen Bd. 31, erw"ahnt und vier aufgestellt habe; diese r"uhren von ganz anderen Ursachen her (von der urspr"unglichen Form und der individuellen Beschaffenheit der Oberflache der Steine; von der Kraft und der Dauer des Rinde-bildenden Prozesses, die durch Hohe, Richtung und Schnelligkeit des Falles bei den verschiedenen einzelnen Steinen mannigfaltig verschieden sein, ja selbst bei ein und demselben Steine durch wiederholte Zerplatzungen oder Lostrennung einzelner Stucke im Falle wieder abge"andert werden k"onnen u. s. w.); und es k"onnen deren an einem und demselben Steine (wie dies der vorhin beschriebene bew"ahrt, und die Figur 5 und Tafel 6 Figur 3 und 4 abgebildeten noch deutlicher zeigen) ebenfalls zwei auch drei vorkommen. Dass jede derselben nach obigem ihre eigenen Modifikationen haben m"usse, ergibt sich von selbst, und welche Komplikationen aus dem zuf"alligen Zusammentreffen mehrerer von diesen und jenen notwendig entstehen m"ussen, dies l"asst sich denken.}}

An der oberen End- und der einen schiefen Seitenfl"ache ist sie n"amlich von einer ganz eigenen Art, die gleichsam das Mittel h"alt zwischen der strahlig und runzlich-aderigen. Der Grund ist matt und etwas graulich-schwarz, und die Adern, welche mehr vereinzelt stehen, verl"angert und nur selten etwas ramifiziert sind, wenig zusammenh"angen, und daher kein eigentliches Netz oder Geflecht bilden, sind pechschwarz oder pechbraun, mit einem "ahnlichen fettigen Glanze. Sie sind ziemlich stark und grob, so dass die Oberfl"ache ziemlich rau erscheint; unter der Lupe erscheinen sie aber wie gek"ornt, und aus einzelnen mehr oder weniger dicht aneinander gereiheten und zusammenflie"senden K"ugelchen oder Tr"opfchen gebildet, wie kleine Perlenschn"ure (sehr "ahnlich der unvollkommenen Rinde D. 2; aber nicht auf frischer Bruchfl"ache, sondern auf schon "uberrindetem Grunde; eine Anomalie, die ich bei keinem Steine von Stannern wieder finde). An der untern End- oder Grundfl"ache und an der konvexen Seitenfl"ache zeigt sich dagegen die Rinde zwar von einer "ahnlichen, aber schon mehr ausgesprochenen, dichter strahlig-aderigen Beschaffenheit, von dunkelschwarzer Farbe, und starkem, etwas seidenartigem Glanze (fast genauso, wie die Rinde an den Seitenfl"achen des zuvor beschriebenen, und Figur 1 abgebildeten Steines), und zeigt offenbar einen "Ubergang in oder vielmehr aus jener zuvor beschriebenen. An der ebenen Seitenfl"ache endlich erscheint sie beinahe kohlschwarz, von fettigem, etwas schillerndem Glanze, und zart runzlicht und verworren, klein und sehr dicht-aderig, unverkennbar als Modifikation oder h"oherer Grad der letzteren.

Sie bildet "ubrigens nirgendwo S"aume oder N"ahte, aber unvollkommen, und zwar im h"ochsten Grade (D. 3), findet sie sich an ein paar "au"serst kleinen Stellen, und auf einem, verh"altnism"a"sig, bedeutend gro"sen Flecke an der oberen Endfl"ache. An allen diesen Stellen scheint aber blo"s die bereits gebildet gewesene Rinde, nicht aber ein St"uck der Masse des Steines, abgesprungen zu sein. Nur an der Grundfl"ache zeigt sich etwas Spur von Erde.

Dieser h"ochst merkw"urdige Stein ist auf seiner Grundfl"ache liegend vorgestellt, um dessen obere Endfl"ache --- welche am regelm"a"sigsten ist, und seine Form am besten ausspricht --- die eine gew"olbte Seitenfl"ache von vorne, und die schiefe von der einen Seite, zur Ansicht zu bringen.

\subsection{\frakfamily{Vierte Figur.}}
\paragraph{}
Der kleinste, und doch vollkommen ganze und durchaus "uberrindete, bei Stannern gefallene Stein, der kaum 56 Gran wiegt.

Auch dieser ward erst einige Zeit nach dem Ereignisse eingebracht, und vom k. k. Kreisamte zu Iglau mit der Anzeige, dass er ebenfalls in der N"ahe des Dorfes Lang-Pirnitz, also im s"udlichen Teile des befallenen Fl"achenraums, aufgefunden worden sei, hierher eingesendet; im Situations-Plane aber eben so wenig, wie vom vorigen Steine, der scheinbaren Unbedeutendheit wegen, der Finder genannt, oder die Fallstelle angegeben.

Es zeigt derselbe einen eif"ormigen Umriss, da er aber sehr plattgedr"uckt, und bei einer L"ange von 11 und einer Breite von 8 Linien, im gr"o"sten Durchmesser, an der dickesten Stelle kaum $\mathfrak{4\frac{1}{2}}$ Linie misst, eine mandelf"ormige Gestalt. Diese Gestalt n"ahert sich jedoch --- indem die beiden Fl"achen auf der einen Seite in eine scharfe Kante zusammen sto"sen, an der entgegen gesetzten aber durch einen ziemlich breiten Rand verbunden sind, der eine dritte, obgleich weit schm"alere, Fl"ache bildet --- einem ungleichseitigen Prisma, und damit auffallend, obgleich im winzig Kleinen, der Form des n"achst zu beschreibenden, Figur 5 abgebildeten, Steines; nur dass an diesem Steine das Prisma von zwei Fl"achen her stark zusammen gedr"uckt, und die dritte Fl"ache die schm"alste ist, und dass diese sich allm"ahlich in den Rand der andern Seite, der gemeinschaftlichen Kante der beiden andern Fl"achen, verliert, ohne mit denselben Endfl"achen zu bilden.

Die beiden gr"o"seren Fl"achen sind etwas konvex, allm"ahlich gegen ihre gemeinschaftliche, fast schneidend scharfe, Kante schief abnehmend, und, zumal die eine, durch ziemlich tiefe ungleichf"ormige Eindr"ucke, die ebenso, wie an den gleichartigen Fl"achen jenes Steines, mehr den Unebenheiten einer nat"urlichen Bruchfl"ache als gew"ohnlichen Eindr"ucken gleichen, sehr uneben; die schmale Fl"ache ist noch weit gew"olbter und unebener, zumal nach einem Ende hin, wo ein verh"altnism"a"sig betr"achtliches St"uck der Steinmasse sich gleichzeitig losgetrennt zu haben scheint, und eine bedeutende Vertiefung zur"uck lie"s.

Die Rinde scheint auch an diesem Steine durchaus von einerlei Beschaffenheit zu sein, und ist auch wirklich von einerlei, und zwar von der glatten Art, von dunkelschwarzer Farbe und starkem fettigem Glanze, ganz "ahnlich jener an den konkaven Fl"achen des zuvor beschriebenen und Figur 2 b, und des n"achst zu beschreibenden, Figur 5 abgebildeten, Steines; nur scheint sie fast durchaus d"unner zu sein; denn sie zeigt einen Grad von Durchscheinenheit, der selten vorkommt, so dass der, wie es scheint, schwerer in Rinde umwandelbare wei"se Gemengteil der Steinmasse in Gestalt einzelner gelblicher und br"aunlicher K"orner durchscheint; und die Adern sind etwas st"arker und faltenartiger, doch ohne die Oberfl"ache rau zu machen oder ein Netz zu bilden. Offenbar zeigt sich aber auch an diesem, doch so kleinen Steine eine Modifikation oder Abstufung der Hauptbeschaffenheit der Rinde; denn unverkennbar ist sie an der einen breiten Fl"ache, dichter, dunkler und gl"anzender, und von hier ist sie auch in Gestalt eines unvollkommenen Saumes "uber den scharfen Rand auf die entgegen gesetzte Fl"ache, und zum Teil auch "uber den stumpferen Rand der auf einer Seite an dieselbe grenzenden schmalen Fl"ache, welche in allen Beziehungen mehr mit jener "ubereinstimmt, "ubergeflossen. Von unvollkommener Rinde findet sich keine Spur, und nur gegen das eine etwas dickere und breitere Ende des Steines zeigt sich etwas Erde an der schmalen Fl"ache.

Die Abbildung zeigt den Stein im ganzen Umrisse auf der einen breiten, st"arker "uberrindeten Fl"ache liegend, mit dem scharfen --- der schmalen Fl"ache entgegen gestellten --- Rande nach vorne gekehrt, um den Rindensaum auf der einen einiger Ma"sen ersichtlich zu machen.

\subsection{\frakfamily{F"unfte Figur.}}
\paragraph{}
Einer der gr"o"sten Steine von dem Ereignisse bei Stannern, 3 Pfund 18 Loth wiegend.

Er ward am Tage (29. Mai) der an Ort und Stelle abgehaltenen Untersuchungs-Kommission, bei angeordneter Aufsuchung der gefallenen und ganz au"ser Acht gelassenen Steine, von einem Bauersweibe auf einem Felde zwischen Stannern und dem Dorfe Falkenau, beinahe im Mittelpunkte des befallenen Fl"achenraums (und zwar etwa 600$^{\circ}$ "ostlich von der Kirche von Stannern, und etwa 3000$^{\circ}$ vom "au"sersten Punkte in N., und etwa 4000$^{\circ}$ vom "au"sersten Punkte in S., wo die entferntesten Steine gefallen waren), auf ziemlich festem Boden, flach aufliegend und nur sehr wenig in die Erde eingedrungen, gefunden. (Situations-Plan Nr. 45.)

Es ist derselbe vollkommen ganz, und durchaus mit Rinde bedeckt; nur an ein paar kleinen Stellen ist diese etwas abgeschlagen, und an dem einen Ende ist ein kleines St"uck ausgebrochen, doch so, dass der Verlust an Masse kaum auf 1 Loth angeschlagen werden kann.

Der Umriss des Steines ist eif"ormig, mit stark abgestumpften Enden; er bildet aber eigentlich ein vollkommenes, nur etwas ungleichseitig dreiseitiges, gegen die beiden Enden verschm"alertes Prisma, und stellt solcher Gestalt ein Segment eines Eies vor.

Er zeigt n"amlich drei Hauptfl"achen, die unter ziemlich spitzen Winkeln zusammensto"sen, und ziemlich scharfe Kanten bilden, und von welchen die etwas breitere konvex, und die beiden andern ein wenig vertieft sind. Nach den beiden Enden hin verschm"alern sich diese Fl"achen, aber ungleich, so dass die konvexe mit einer der konkaven mehr nach dem einen, die andere konkave mehr nach dem andern Ende zu abnimmt. Die beiden Enden sind stark abgestumpft, und durch eine Fl"ache geschlossen, so dass man diese als End-, jene als Seitenfl"achen betrachten kann.

Die eine dieser Endfl"achen, die man als die gr"o"sere und regelm"a"sigere, als die Grundfl"ache dieses Steines ansehen mag, ist flach, nur etwas vertieft, und bildet ein vollkommenes, aber stark verschobenes, und sehr ungleichseitiges Viereck; drei Ecken desselben entsprechenden Seitenkanten, und folglich die R"ander den Seitenfl"achen, die vierte Ecke aber, welche etwas stumpfer ist, f"allt gegen die Mitte der breiteren Seitenfl"ache, von deren Teilung durch eine Kante sich der Anfang zeigt. (Die Tendenz zur vierseitigen S"aule, oder, da das andere Ende schm"aler zul"auft, und dort die Endfl"ache kleiner und ganz unregelm"a"sig ist, zur ungleichseitig vierseitigen Pyramide, ist unverkennbar, und besonders auffallend die "Ahnlichkeit und "Ubereinstimmung dieses Steines mit dem zuvor beschriebenen, und Figur 2. a. b. abgebildeten, ungleich kleineren, nur mit dem Unterschiede, dass dieser gegen das eine Ende ungleich mehr verschm"alert ist, und daher eine vollkommen konische Gestalt hat.)

Die beiden etwas vertieften Seitenfl"achen, die in eine gemeinschaftliche, scharfe, etwas verdr"uckte und wellenf"ormig ausgeschweifte Kante (genauso wie an jenem Steine; auch ist sie l"anger als wenigstens eine der beiden andern Seitenkanten, und bildet an der Grundfl"ache die vorspringendste Ecke) zusammen sto"sen, haben nur wenige, und sehr seichte, aber ziemlich gro"se und breit verlaufende, unf"ormliche Eindr"ucke, die (ebenso) mehr den Unebenheiten einer nat"urlichen Bruchfl"ache, als den gew"ohnlichen Eindr"ucken gleichen, und bilden zum Teil, oder liegen in gr"o"seren, st"arkeren Vertiefungen, welche von ungleichf"ormiger, aber gleichzeitiger Lostrennung einzelner St"ucke der Masse herzur"uhren scheinen.

Die breitere, konvexe Seitenfl"ache, welche mit jenen Fl"achen ziemlich scharfe, und hie und da besonders d"unne, "ubrigens sehr verdr"uckte und ausgeschweifte Kanten bildet, zeigt weit wenigere und noch seichtere Eindr"ucke von gew"ohnlicher Art, so dass sie fast eben erscheint; nur gegen das obere Ende hin ist sie durch betr"achtliche Vertiefungen (die wahrscheinlich ebenfalls durch eine ungleichf"ormige, aber auch mit der Entstehung der ganzen Fl"ache gleichzeitige Lostrennung einzelner St"ucke entstanden sein m"ogen) gewisser Ma"sen verunstaltet.

Die obere Endfl"ache, welche mit den Seitenfl"achen sehr undeutliche und unvollkommene Kanten bildet, entspricht der Beschaffenheit der Oberfl"ache nach, vollkommen dem oberen, verunstalteten Teile der konvexen Seitenfl"ache; die untere Endfl"ache aber (die mit den Seitenfl"achen ziemlich scharfe, und nur mit der einen sehr schmalen konkaven eine platt und sehr breit gedr"uckte Kante bildet), hat nur einige sehr seichte Eindr"ucke, und zeigt in jeder Beziehung eine, obgleich nur wenig bedeutende Abweichung von allen "ubrigen Fl"achen.

Die Rinde ist an diesem Steine (sowie an jenem Fig. 2. a. b.) ganz besonders merkw"urdig, und zeigt (genauso wie an diesem in jeder Hinsicht) eine sehr wesentliche und auffallende Verschiedenheit nach den verschiedenen Fl"achen, oder vielmehr nach den Seiten des Steines.

An den beiden konkaven Seitenfl"achen ist sie n"amlich von gleicher, und zwar von der (in Gilberts Annalen Bd. 31, S. 57, sub B. 2. beschriebenen) glatten, nur sehr schwach aderigen Art; sehr dicht, fest und gleichf"ormig, pechschwarz und sehr fettig gl"anzend. Nur hie und da zeigen sich wenig erhabene kleine Adern und Ramifikationen, die nur selten zusammen hangen, und nur eine Anlage zu blattf"ormigen Zeichnungen, ohne bestimmte Richtung, bemerken lassen. Sie bildet weder Nahte noch Saume.

An der konvexen Seitenfl"ache dagegen, auch an dem oberen, verunstalteten Teil derselben, ist die Rinde besonders ausgezeichnet, von der sehr rauen, runzlicht und faltig-aderigen Art (ebendas. S. 56, A. a. 1.), zwar dicht und fest, aber sehr ungleichf"ormig, da fleck- und stellenweise die oberste, raue Schichte derselben fehlt, wo sie lockerer, por"os, matt, und mehr braun als schwarz erscheint; sonst beinahe kohlschwarz, und von ziemlich starkem, nur durch jene Stellen unterbrochenen, aber mehr seidenartigen, schillernden Glanze. Abgesehen von der Erhabenheit einzelner starker Runzeln und Falten, hat sie im Ganzen keine betr"achtlichere Dicke als jene an den entgegen gesetzten Fl"achen. Die erhabenen, ziemlich scharfen Adern, Runzeln und Falten, bilden ein ziemlich enges, unregelm"a"siges Netz, oder ein verworrenes Adergeflecht; aber, obgleich einige mehr verl"angerte Adern, zumal hie und da auf den R"ucken der Erhabenheiten, welche einige Vertiefungen begrenzen, ausgezeichnet und besonders scharf sind; so sind dieses doch keine N"ahte, da sie immer nur von einer Seite her gebildet werden, und keine bestimmte Richtung haben. Dagegen ist die Rinde an allen Kanten dieser Fl"ache, sowohl gegen die beiden andern Seitenfl"achen, als auch gegen die untere End- oder Grundfl"ache hin, obgleich hier schw"acher und undeutlicher, angeh"auft, verdickt, und "uber die Kanten selbst geflossen, so dass sie an jenen Fl"achen eine Art von Saum bildet, der wie eine doppelte Lage von Rindenmasse, "uber eine Linie breit, sich auf dieselben und "uber die denselben eigent"umliche Rinde hinein zieht, genau dem Laufe der Kanten folgt, und ziemlich scharf abgeschnitten endigt. An einer der Seitenkanten ist diese Saumrinde, und zwar gerade an den zwei hervorragendsten Stellen, auf einen halben Zoll L"ange, wieder gegen die Fl"ache zur"uckgedr"uckt, gerade als wenn der Stein mit diesen Punkten gegen einen harten K"orper gesto"sen w"are, der die (noch nicht ganz erstarrte ?) Rinde zur"uckgebogen h"atte. Die Beschaffenheit der Rinde an der oberen Endfl"ache stimmt ganz, so wie die "ubrige Beschaffenheit der Oberfl"ache des Steines hier, mit der Beschaffenheit beider an dieser konvexen Fl"ache "uberein; jene dagegen an der untern End- oder Grundfl"ache weicht hierin, obgleich nicht sehr auffallend, von jener an beiden Seiten des Steines ab. Sie ist n"amlich bei weitem nicht so glatt und fettig gl"anzend, wie die an den konkaven Seitenfl"achen; aber auch nicht so rau und runzlicht-aderig und schimmernd, wie die an der konvexen, sondern "uberhaupt mehr von der gemein-aderigen, an den Steinen von Stannern am gew"ohnlichsten vorkommenden Art (A. a. 2). Von der konvexen Fl"ache her bildet die dortige, "uber die Kante auf diese Fl"ache nur etwas "ubergeflossene Rinde, auch nur einen undeutlichen, unvollkommenen Saum; von den beiden konkaven Fl"achen her aber steht die Rinde gleichsam an den Rand der gemeinschaftlichen Kante frei an, und von jener dieser Fl"ache geschieden.

Von unvollkommener Rinde findet sich an diesem ganzen gro"sen Steine nur eine und selbst etwas zweideutige Spur, an einer kleinen Stelle auf der scharfen gemeinschaftlichen Kante der konkaven Seitenfl"achen.

Nur an diesen letzteren Fl"achen, auf welche der Stein auch wirklich aufgefallen zu sein scheint, da er namentlich auf denselben liegend gefunden wurde, und zum Teil an den beiden Endfl"achen, findet sich Erde in die kleinen, seichten, ohne dieselbe kaum sichtbaren, Zwischenr"aume des schwachen Adergeflechtes, und in die vertieften Punkte und Poren der Rinde an- und eingedr"uckt; auf der konvexen Fl"ache dagegen, die bei dem geringen Eindringen des, doch "uber 3 Zoll dicken, Steines in das Erdreich hoch genug "uber dasselbe hinausragte, so dass nicht leicht ein Regenguss Erde dar"uber schlemmen konnte, zeigt sich trotz der Rauigkeit der Oberfl"ache keine Spur davon. (An dem Steine Fig. 2 finden sich erstere, sonst ganz gleich beschaffene Fl"achen, ganz rein von Erde, wovon sich hier "uberhaupt nur etwas an der Grundfl"ache zeigt.)

Die Abbildung zeigt diesen besonders ausgezeichneten Stein, auf der konvexen Seitenfl"ache liegend (wie Figur 2. b. den "ahnlichen), von den beiden glatten, konkaven Seitenfl"achen und ihrer gemeinschaftlichen Kante (nur gest"urzt, die Grundfl"ache nach oben, und etwas gewendet), um den merkw"urdigen Rindensaum, von der konvexen Fl"ache her, an einem Seitenrande (in das n"otige Licht gebracht) ersichtlich zu machen.
\clearpage
\section{\frakfamily{Sechste Tafel.}}
\subsection{\frakfamily{Erste Figur.}}
\paragraph{}
Einer von den gro"sen Steinen von dem Steinfalle bei Stannern, 2 Pfund 12 Loth schwer.\footnote{\frakfamily{In dem, dem Situations-Plane angeschlossenen Verzeichnisse der aufgefundenen Steine, wird das Gewicht, wahrscheinlich weil nur nach Erinnerung gesch"atzt, da das Verzeichnis mehrere Monate sp"ater aufgenommen wurde, nur auf 2 Pfund angegeben.}} Vollkommen ganz, und durchaus "uberrindet.

Es ward derselbe am 29. Mai im Verfolg des angeordneten Aufsuchens der gefallenen Steine, ganz nahe an dem Marktflecken Stannern selbst, ziemlich im Mittelpunkte des befallenen Fl"achenraumes (und zwar kaum 500$^{\circ}$ westlich von der Kirche von Stannern, und bei 3000$^{\circ}$ vom "au"sersten Punkte in N., und bei 4000$^{\circ}$ vom "au"sersten Punkte in S., wo die entferntesten Steine gefallen waren) aufgefunden, und an die Untersuchungs-Kommission abgegeben. (Situations-Plan Nr. 26.)

Die Gestalt dieses Steines scheint beim ersten Anblicke h"ochst unregelm"a"sig zu sein, denn viele gro"se, unf"ormliche und zum Teil ziemlich tiefe Eindr"ucke, die offenbar vom Verlust an Masse durch sp"atere Zersprengung und gleichzeitige Lostrennung mehrerer St"ucke, vor der Rindebildung im Ganzen, herr"uhren, verunstalten die Fl"achen, verdr"ucken die Kanten und unterbrechen deren Richtung, so dass der Stein, zumal derselbe gegen das eine Ende hin etwas verschm"alert, und hier von zwei Seiten her stark zusammen gedr"uckt ist, keilf"ormig erscheint. Inzwischen ist doch die verschoben dreiseitige oder die unvollkommen und sehr ungleichseitig vierseitige prismatische Grundgestalt unverkennbar, und an der einen Endfl"ache deutlich ausgesprochen, und die "Ahnlichkeit mit den oben beschriebenen und Figur 2 und 5 auf der vorigen Tafel abgebildeten Steinen nachweisbar. Man kann n"amlich vier Seiten- und zwei Endfl"achen, die zum Teil von ziemlich scharfen, wenn gleich sehr ausgeschweiften, und hie und da unterbrochenen Kanten begrenzt werden, deutlich unterscheiden.

Zwei der Seitenfl"achen, die sich gegen "Uberstehen, sind sich fast ganz gleich; sie sind breiter als die "ubrigen, und l"anglich-viereckig. Gegen das eine, obere Ende sind sie nur wenig verschm"alert, aber vertieft und abgeplattet, weil der Stein hier so zusammengedr"uckt ist, dass er kaum einen Zoll dick erscheint; gegen das andere, untere Ende sind sie etwas mehr verschm"alert, aber konvexer, wie denn der Stein hier 3 Zoll dick ist. Sie sind sehr uneben, voll gro"ser, zum Teil ziemlich tiefer, aber sanft sich verlaufender Eindr"ucke.

Die dritte, zwischen jenen liegende Seitenfl"ache ist gegen das untere Ende fast so breit, wie die beiden vorhergehenden, nach oben aber sehr verschm"alert, weil der Stein von den andern Seiten her so stark zusammengedr"uckt ist. In der Mitte ist sie etwas vertieft, sonst flach und ebener als jene, da sie nur wenige, zwar gro"se, aber sehr seichte Eindr"ucke von gew"ohnlicher Art hat. Die vierte, der letzteren gegen "uber liegende Seitenfl"ache endlich ist unvollkommen, oder gleicht vielmehr einem breit gedr"uckten Rande. Sie ist sehr schmal, sehr uneben und konvex, und bildet mit den beiden breiten Seitenfl"achen, zwischen welchen sie liegt, undeutliche, sehr verdr"uckte, unterbrochene und abgerundete Kanten. Sie gleicht sehr der unvollkommenen vierten Fl"ache des Figur 2 a auf der vorigen Tafel dargestellten Steines, und noch mehr, ihrer ganzen Beschaffenheit nach, der schmalen Fl"ache an dem daselbst Figur 4 abgebildeten Steine.

Die obere Endfl"ache ist undeutlich und unbestimmbar, oder vielmehr sie erscheint, weil der Stein von zwei Seiten her so sehr zusammen gedr"uckt ist, blo"s als ein breiter, abgerundeter Rand. Die untere Endfl"ache dagegen bildet eine vollkommene, etwas verschobene und ungleichseitig vierseitige Fl"ache, deren ziemlich spitze Ecken den Seitenkanten, und die ziemlich scharfen Kanten den Seitenfl"achen entsprechen, und die mit der gleichnamigen Fl"ache der oben beschriebenen Steine, Figur 2 und 5, gro"se "Ahnlichkeit zeigt. Sie ist ziemlich stark ausgeh"ohlt, und durch etwas kleinere, aber st"arkere Vertiefungen als die "ubrigen Fl"achen, uneben gemacht.

Nach dieser Beschaffenheit der Oberfl"ache l"asst dieser Stein drei Verschiedenheiten nach seinen verschiedenen Fl"achen erkennen, wovon die eine Seitenfl"ache f"ur sich die eine, die derselben entgegen gesetzte, schmale, in Verbindung mit der untern Endfl"ache, die andere, und die beiden breiten, sich gegen"uberstehenden Seitenfl"achen zusammen, die dritte zeigen.

Die Rinde ist an diesem Steine ganz besonders ausgezeichnet und merkw"urdig. Sie ist durchaus von derselben Hauptbeschaffenheit und von gleicher, und zwar von der rauen, ganz vollkommen blattf"ormig gezeichneten Art (A. a. 3. Gilberts Annalen Bd. 31, S. 56), zeigt aber doch, nach den verschiedenen Fl"achen, die sie bedeckt, unverkennbar eine zwei-, zum Teil dreifache Modifikation.

An der untern Endfl"ache, von wo aus die blattf"ormigen Zeichnungen ihre Richtung nach aufw"arts "uber die Seitenfl"achen nehmen, erscheint sie noch sehr unvollkommen und undeutlich blattf"ormig, mehr verworren, runzlicht-aderig, von etwas graulich-schwarzer Farbe, und etwas mattem, fettigem Glanze; an der schmalen, konvexen Seitenfl"ache ist sie von gleicher Farbe und "ahnlichem Glanze, aber schon deutlich blattf"ormig gezeichnet, und die Bl"atter streichen gerade nach aufw"arts gegen das obere Ende, und bald schlagen sie, bald die Bl"atter der angrenzenden breiten Seitenfl"achen "uber die abgerundeten Kanten. An den beiden breiten Seitenfl"achen ist sie schon ausgezeichnet blattf"ormig, und die Richtung der Bl"atter geht von der Grundfl"ache nach aufw"arts und etwas schief, gr"o"sten Teils gegen die schmale Seitenfl"ache hin. Ihre Farbe zieht sich mehr ins Pechschwarze, und ihr Glanz ist etwas st"arker und mehr fettig, und beides umso mehr, je mehr sie sich dem oberen Ende und der vierten Seitenfl"ache des Steines n"ahert. Auf dieser letzteren endlich ist sie besonders ausgezeichnet und gro"sbl"atterig, und die Bl"atter streichen, wie vom Mittel der untern Endfl"ache aus, schief aufw"arts in entgegen gesetzter Richtung nach den angrenzenden breiteren Seitenfl"achen, und mehr oder weniger selbst "uber die gemeinschaftlichen scharfen Kanten, so dass sie hier teils am Rande frei anstehen, teils, auf jene Fl"achen ganz "uberschlagend, einen mehr oder weniger deutlichen und unterbrochenen Saum auf denselben bilden. Die Rinde ist "ubrigens auf dieser Fl"ache von einer sehr dunkel-, fast kohlschwarzen Farbe, mehr seidenartigem, schimmerndem Glanze, und von zarterer Beschaffenheit (die Adern sind n"amlich viel feiner und sch"arfer), und n"ahert sich "uberhaupt sehr der strahlig-aderigen Art; auch scheint sie, wo nicht im Ganzen, d"unner, doch gleichf"ormiger zu sein, wenigstens ist sie nicht so, wie an den "ubrigen Fl"achen, stellenweise, am Rande der Bl"atter --- zumal wo sich dieser "uber den R"ucken von Erhabenheiten zieht --- verdickt, angeh"auft, und gleichsam wie "Olfarbe mit einem groben Pinsel hingeschmiert. An der oberen Endfl"ache, die eigentlich, wie bereits erw"ahnt, blo"s einen stumpfen, abgerundeten Rand bildet, ist die Rinde bis gegen das Mittel derselben hin dick, fast wulstig (hie und da wohl auf eine halbe Linie) angeh"auft, besonders glatt, und von den breiten Fl"achen her, wie erstarrendes Pech, gleichsam angeflossen, und zwar scheint es, den abgesto"senen Stellen nach, die obere glatte Schichte zu sein, welche sich hier verdickt hat.

Von unvollkommener Rinde findet sich nur an einer "au"serst kleinen, kaum bemerkbaren Stelle, auf jeder der breiten Seitenfl"achen, eine Spur, wo offenbar die Rinde nicht zusammengeflossen war.

Angedr"uckte Erde zeigt sich nur an der schmalen konvexen Fl"ache, und gegen die untere H"alfte der an sie grenzenden breiten Seitenfl"achen, welches auch die Stellen sind, auf welche der Stein, kraft seines nat"urlichen Schwerpunktes, gefallen sein sollte.\footnote{\frakfamily{Und dieser Richtung im Falle scheint die Modifikation der Rinde auf den verschiedenen Fl"achen sehr zu entsprechen.}}

Die Abbildung stellt diesen ausgezeichneten Stein, auf der schmalen Seitenfl"ache liegend, etwas schief gewendet vor, und zeigt die eine ebenere Seitenfl"ache fast in gerader, eine der breiteren in schiefer, und die untere Endfl"ache in saucierter Richtung.\footnote{\frakfamily{In Gilberts Annalen Bd. 31 ist von diesem Steine Tafel 1 Fig. 2 bereits eine verkleinerte und skizzierte Abbildung, aber von einer der breiten Seitenfl"achen, gegeben worden; ich habe daher absichtlich hier eine andere Fl"ache zur Darstellung gew"ahlt.}}

\subsection{\frakfamily{Zweite Figur.}}
\paragraph{}
Einer der gr"o"seren von den bei Stannern gefallenen Steinen, 1 Pfund 12 Loth wiegend. Vollkommen ganz und um und um mit Rinde bekleidet.

Dieser wurde am 28. Mai in Folge der gemachten Aufforderung, die gefallenen Steine aufzusuchen, von einem Landmanne ebenfalls in der N"ahe des Marktes Stannern, eigentlich bei dem Dorfe Sorez, noch mehr im Mittelpunkte des befallenen Fl"achenraumes, als der vorhin beschriebene (und zwar kaum 1000$^{\circ}$ O. S. O. von diesem, und etwa 600$^{\circ}$ in gleicher Richtung von der Kirche von Stannern, und beinahe in ganz gleichem Abstande von den beiden "au"sersten Fallstellen in N. und S.), aufgefunden. (Situations-Plan Nr. 35.)

Er stellt eine zwar etwas unvollkommene, aber nur wenig verschobene, ungleichseitig vierseitige Pyramide vor, deren abgeflachte Spitze stark aus dem Mittel gedr"uckt und auf eine Seite "ubergebogen ist.

Die Grundfl"ache, welche ganz flach und beinahe vollkommen eben und platt, ohne alle Eindr"ucke und Vertiefungen ist (ein Fall, der bei einer Fl"ache von solcher Ausdehnung h"ochst selten an einem Steine vorkommt), bildet ein etwas verschobenes und ungleichseitiges Viereck, dessen ziemlich, und gewisser Ma"sen ausgezeichnet gerade laufende scharfe, fast schneidende Kanten, ebenso vielen, ziemlich senkrecht aufgesetzten, nach oben verschm"alerten und nach einer Seite hingebogenen Seitenfl"achen, und dessen Ecken ebenso vielen, ziemlich scharfen, aber sehr verdr"uckten und ausgeschweiften Seitenkanten entsprechen. Eine Ecke der Grundfl"ache ist ziemlich spitzig, und die ihr diagonal gegen"uberstehende etwas abgest"utzt; eine dritte Ecke ist st"arker, und die ihr entgegen gesetzte vierte noch mehr abgest"utzt, so dass durch letztere die Grundfl"ache beinahe f"unfseitig gemacht wird. Diese Abstumpfungen gehen etwas schief von unten nach aufw"arts und au"sen, und bilden Dreiecke, deren Basis auf der Grundfl"ache ruht, und deren spitzer oberer Winkel sich allm"ahlich in die Seitenkante verliert. Solcher Gestalt wird die vierseitige Form der Pyramide durch sie nicht ver"andert, und die Grundfl"ache zeigt immer noch eine gro"se "Ahnlichkeit mit jener der meisten bereits beschriebenen Steine, so wie die Form im Ganzen, welche den Grund-Typus deutlich genug ausspricht, mit jener mehrerer derselben.

Die obere Endfl"ache ist nur unvollkommen, und eigentlich die horizontale Fortsetzung einer schief aufsteigenden Seitenfl"ache.

Zwei aneinandergrenzende Seitenfl"achen sind, zumal die eine, breiter als die andern, und ziemlich stark gew"olbt; die beiden andern, gegen deren gemeinschaftliche, sehr verdr"uckte und beinahe ganz verschwundene Kante (welcher auch die am st"arksten abgest"utzte Ecke der Grundfl"ache entspricht) die abgeflachte Endspitze hingedr"uckt und "ubergebogen ist, sind bedeutend schm"aler und etwas vertieft.

Die an diesem Steine auf allen Fl"achen, au"ser der ganz ebenen Grundfl"ache, vorkommenden Eindr"ucke, sind von ganz eigener Art, wie ich sie an keinem Steine von Stannern (deren ich doch, mit Inbegriff der gr"o"seren Bruchst"ucke, bei 100 zu Gesicht bekam), noch an irgendeinem Meteor-Steine, wieder fand, ausgenommen --- obgleich nicht ganz so deutlich ausgesprochen --- an der Grundfl"ache des n"achst zu beschreibenden. Sie sind n"amlich verh"altnism"a"sig sehr klein, aber tief und grubenartig, nicht so breit wie gew"ohnliche Eindr"ucke und sanft verlaufend, sondern ziemlich scharf gerandet, gleichsam kantig, wie von grobk"ornigen oder br"ockligen Absonderungen entstanden, und geben der Oberfl"ache, da sie ziemlich h"aufig sind, ein klein-wellenf"ormiges Ansehen. Auf den beiden schm"alern, konkaven Seitenfl"achen zeigen sie schon eine Modifikation; sie sind n"amlich hier gr"o"ser, aber seichter und mehr breit verlaufend, auch minder zahlreich. Die obere Endfl"ache stimmt hierin mit den andern Seitenfl"achen "uberein.

Auch die Rinde ist an diesem Steine von eigent"umlicher, und der seltenen, strahlig- und netzartig-aderigen Art, aber durchaus, "uber den ganzen Stein, von einerlei Hauptbeschaffenheit, die nur eine Haupt- und eine dieser letzteren untergeordnete Modifikation erkennen l"a"st.\footnote{\frakfamily{Und diese Modifikationen zeigen eine "Ubereinstimmung mit der Beschaffenheit der Oberfl"ache und mit der Richtung, welche die Fl"achen im Niederfallen des Steines, kraft dessen individuellen Schwerpunktes, h"ochst wahrscheinlich gehabt haben m"ochten.}}

Auf der ebenen Grundfl"ache ist sie n"amlich ausgezeichnet auseinanderlaufend strahlig; die ziemlich erhabenen, zarten und scharfen runzelartigen Adern laufen, wenig geschl"angelt und fast gar nicht ramifiziert, von einem k"ornig-rauen Mittelpunkte --- der aber nicht ganz im Mittel der Fl"ache liegt --- strahlenf"ormig auseinander und gegen die Kanten hin. Die Zwischenr"aume zwischen diesen, eben nicht sehr gedr"angten Strahlen, sind durch zartere Runzeln und Adern, die zum Teil "Aste derselben sind, und durch erhabene Punkte und Tr"opfchen rau. "Ubrigens ist die Rinde hier beinahe kohlschwarz, und von einem ziemlich starken, schimmernden, seidenartigen Glanze. An allen "ubrigen Fl"achen dagegen erscheint sie netzartig-aderig, das ist, die sehr erhabenen und scharfen, zwar strahlenf"ormig verl"angerten, aber als Folge der Unebenheiten verschiedentlich und stark gebogenen und gekr"ummten Adern bilden durch ihre Verbindung unter sich ein unregelm"a"siges, weitschichtiges Netz, dessen Maschen oder Zwischenr"aume ebenfalls durch zartere, k"urzere Adern und Runzeln rau sind. An den Erhabenheiten, welche die Vertiefungen begrenzen, sowie an den meisten Kanten, bildet die Rinde ziemlich hohe und scharfe, zart gefaltete N"ahte, welche der Oberfl"ache ein ganz eigent"umliches und besonders raues Ansehen geben.\footnote{\frakfamily{Die Erhabenheit und Sch"arfe der Adern und N"ahte der Rinde, insbesondere an diesem Steine, sprechen wohl sehr gegen die vermeintliche Fl"ussigkeit derselben, die selbst noch im Momente des Auffallens der Steine Statt haben soll; so wie andererseits die Form und die Sch"arfe der Kanten, nicht nur an diesem, sondern an den meisten Steinen, gegen die pr"asumierte Weichheit, Plastizit"at, teigige Schmelzung (\emph{fusion pateuse}) der Steinmasse in demselben Momente zu streiten scheinen; obgleich nicht in Abrede zu stellen ist, das sie sich eben so wenig mit dem h"ochst spr"oden, leicht br"uchigen und fast zerreiblichen Zustande, in welchem, wenigstens die Steine von Stannern, selbst sehr kurze Zeit nach ihrem Falle befunden worden sind, und sich noch befinden, vereinbaren lassen, und mit welchem letzteren "uberhaupt die vollkommene Integrit"at so vieler, mitunter ansehnlicher und ziemlich gewichtiger Steine im offenbarsten Widerspruche steht.}} "Ubrigens hat die Rinde hier eine mehr ins Graue ziehende schwarze Farbe, und einen etwas schw"acheren, aber noch mehr schimmernden, seidenartigen Glanz.

An den beiden konkaven Fl"achen zeigt sich insofern eine kleine Modifikation von dieser letzteren Beschaffenheit der Rinde, dass sie hier etwas dunkler schwarz ist (gleichsam im "Ubergange von jener der Grundfl"ache in jene der andern Seitenfl"achen), schw"achere N"ahte, minder raue Zwischenr"aume, und, wenigstens gegen die Endspitze hin, eine schwache Anlage zu blattf"ormigen Zeichnungen zeigt.

"Ubergeflossen oder S"aume bildend findet sich die Rinde an diesem Steine nirgendwo, und unvollkommen (und zwar im h"ochsten Grade, aber nur als Folge einer oberfl"achlichen Absprengung eines "au"serst kleinen St"uckes derselben) zeigt sie sich nur auf einer sehr kleinen Stelle auf einer der konkaven Fl"achen.

An der Grundfl"ache, an einer der konvexen und an einer konkaven Seitenfl"ache, gegen welche letztere die Endspitze gebogen ist, zeigt sich stellenweise etwas eingedr"uckte Erde.

Dieser, durch die seltene Art von "Uberrindung besonders ausgezeichnete Stein, ist von seiner --- in dieser Beziehung merkw"urdigsten --- Grundflache, die zugleich dessen Form am besten erkennen macht, dargestellt.\footnote{\frakfamily{In Gilberts Annalen Bd. 31, Tafel 2 Figur 1. 2., ist bereits von diesem Steine eine skizzierte Darstellung von zwei Ansichten gegeben worden, und zwar die eine von den beiden gew"olbten Seitenfl"achen mit ihrer gemeinschaftlichen Kante, die andere von der Grundflache genommen.}}

\subsection{\frakfamily{Dritte Figur.}}
\paragraph{}
Ebenfalls einer der gr"o"seren von den bei Stannern gefallenen Steinen, von 1 Pfund 7 Loth am Gewichte, welcher am 29. Mai, auch nahe bei Stannern selbst, zwischen den D"orfern Sorez und Falkenau, demnach ebenfalls im Mittelpunkte des befallenen Fl"achenraumes (und zwar nur etwa 500$^{\circ}$ mehr n"ordlich als der letztbeschriebene, und etwa 300 "ostlich von der Kirche von Stannern) aufgefunden und der Kommission "ubergeben wurde. (Situations-Plan Nr. 43.)

Es ist derselbe vollkommen ganz, so, wie er wirklich zur Erde gefallen, obgleich er, bei oberfl"achlicher Betrachtung, das Ansehen hat, als w"are ein betr"achtliches St"uck davon nach der Hand gewaltsam abgeschlagen, und die k"unstlich erzeugte Bruchfl"ache durch absichtliche oder zuf"allige Beschmutzung so ver"andert worden, dass sie nicht mehr vollkommen einer ganz frischen der Masse gleichet. Diese Vermutung findet noch "uberdies in der offenbaren Verunstaltung der Form, deren urspr"ungliche gr"o"sere Regelm"a"sigkeit noch unverkennbar ist, durch Verlust an Masse, eine auffallende Bekr"aftigung. Es hat mit dieser Vermutung insoweit auch die vollste Richtigkeit, dass jenes Bruchansehen und diese Formverunstaltung wirklich von einem sp"ateren, nach der urspr"unglichen Bildung (Individualisierung) dieses Steines und nach dessen totaler Inkrustierung Statt gehabten Verluste an Masse herr"uhre; allem es zeigt sich bei n"aherer Betrachtung unwiderleglich, dass dieser Verlust noch vor dem wirklichen Niederfallen oder Auffallen des Steines, und w"ahrend seines Zuges durch die Luft, durch nat"urliche Absprengung und Lostrennung eines St"uckes entstanden sein m"usse, indem die vermeintlich k"unstliche Bruchfl"ache wirklich mit wahrer, obgleich nicht vollkommen ausgebildeter Rinde bedeckt erscheint.\footnote{\frakfamily{Dieser Stein war es auch, an dem ich jene, f"ur die in jeder Beziehung so schwierige Erkl"arung der Bildung der Rinde an den Meteor-Steinen, gewiss sehr wichtige Beobachtung, n"amlich "uber das Vorkommen derselben in verschiedenen Graden von Unvollkommenheit, oft selbst an ein und demselben Steine, zuerst machte, und zu machen nicht wohl verfehlen konnte, da sie an diesem Steine so ausgesprochen und in die Augen springend ist, und welche so wie die ebenso vorkommenden Hauptverschiedenheiten und Modifikationen derselben, wie mir deucht wohl unbestreitbar, eine stufenweise und allm"ahliche --- ich will gerade nicht behaupten, langsame, aber doch wiederholte, fortgesetzte, und w"ahrend der ganzen Periode des Falles der einzelnen Steine fortdauernde --- Bildung der Rinde voraussetzen. Es war mir dann ein Leichtes, dieses, gar nicht ungew"ohnliche Vorkommen der Rinde, in an sich schwerer erkennbaren Graden, nicht nur an den meisten Meteor-Steinen von Stannern, sondern auch an jenen von andern Ereignissen, deren Rinde, ihrer Natur nach, weit weniger geeignet ist, diesen Zustand erkennen zu lassen --- daher er auch bis dahin (1808), und wie es scheint, noch bis jetzt von niemand beobachtet wurde --- aufzufinden und nachzuweisen.}}

So unregelm"a"sig die Form dieses Steines nun auch ist, so ist doch in seiner Begrenzung durch wahre Fl"achen, und in deren Verbindung, Ausdehnung und Richtung, der Grund-Typus zur verschoben vierseitigen Pyramide, und damit die "Ahnlichkeit mit den meisten der beschriebenen Steine deutlich genug noch ausgesprochen, und man m"usste diesen Stein, trotz dessen starker Abplattung und anscheinender Zurundung, nach zwei End- und vier Seitenfl"achen beschreiben, z"oge man auch nur die verschiedene Beschaffenheit seiner Oberfl"ache und die Modifikationen der Rinde in Betrachtung.

Die eine, bedeutend gr"o"sere Endfl"ache, stellt ein verschobenes, aber ziemlich gleichseitiges Viereck vor, dessen Ecken abgestumpft, und mehr oder weniger zugerundet, und dessen ziemlich gerade laufende R"ander, die mit den mehr oder weniger schief aufsteigenden Seitenfl"achen ziemlich scharfe Kanten bilden, ausgeschweift sind. Sie ist in der Mitte etwas gew"olbt, sonst ziemlich flach, und durch sehr viele kleine, aber ziemlich tiefe, grubenartige Eindr"ucke auf eben die Art und ebenso sehr uneben, wie die Seitenfl"achen des zuvor beschriebenen Steines.

Drey aneinandergrenzende Seitenfl"achen sind sehr niedrig. Die eine steigt beinahe senkrecht; die andere, unter einem ziemlich spitzen Winkel in eine deutliche, ziemlich scharfe Kante mit ihr zusammensto"sende, etwas schief; die dritte, unter einem sehr stumpfen Winkel, mit ersterer eine sehr undeutliche, ganz abgerundete Kante bildende, noch mehr schief von der Grundfl"ache in die H"ohe. Alle haben nur wenige, seichte, aber gro"se und breit verlaufende Eindr"ucke von gew"ohnlicher Art.

Die vierte Seitenfl"ache ist, zumal in ihrem Mittel, wo sich der obere Rand in eine stumpfe Spitze verliert --- von der eine ziemlich erhabene scharfe Kante bis zum Rande der Basis l"auft, und diese Fl"ache der L"ange nach in zwei H"alften teilt, auch gewisser Ma"sen eine f"unfte unvollkommene Ecke an der Grundfl"ache bildet --- betr"achtlich h"oher als jene, und erhebt sich zwischen den beiden schiefern Seitenfl"achen, mit welchen sie in etwas undeutliche Kanten zusammen st"o"st, beinahe senkrecht von der Grundfl"ache Sie ist sehr uneben, ihre Unebenheiten r"uhren aber nicht von gew"ohnlichen Eindr"ucken her, sondern stellen nat"urliche Unebenheiten einer Bruchfl"ache der Steinmasse selbst vor.

Die obere Endfl"ache endlich steigt von zwei Seitenfl"achen --- der einen etwas schiefen und der senkrechten, niederen --- mit welchen sie unter einem sehr stumpfen Winkel in etwas undeutliche Kanten zusammen sto"st, eine Strecke lang schief aufw"arts, als wenn sie eine gew"olbte Fl"ache bilden wollte, wird aber bald durch eine neue Fl"ache unterbrochen, die wie von einer zuf"alligen, sp"ateren und gewaltsamen Abschlagung der Endspitze entstanden zu sein scheint. Diese Fl"ache hat einen rundlichen Umriss, der aber doch einiger Ma"sen den Seitenfl"achen und Kanten entspricht, erhebt sich schief gegen den Rand und die Spitze der einen senkrechten h"oheren, und st"o"st mit der vierten schiefen Seitenfl"ache mit einem ziemlich scharfen kantenartigen Rand zusammen. Sie sieht ebenso rau und uneben aus, wie die eine hohe Seitenfl"ache, und folglich wie eine gew"ohnliche Bruchfl"ache der Steinmasse, indes ihre Basis gegen die zwei ersteren Seitenfl"achen hin, hinsichtlich ihrer Beschaffenheit und Eindr"ucke, ganz diesen gleicht. So verschieden solcher Gestalt die Oberfl"ache dieses Steines nach den verschiedenen Fl"achen desselben erscheint; so verschieden und offenbar in "Ubereinstimmung mit jenen Verschiedenheiten zeigt sich auf eine h"ochst merkw"urdige Weise die Beschaffenheit der Rinde an demselben.

Auf der gr"o"seren End- oder Grundfl"ache desselben ist sie n"amlich genau und in allen Beziehungen, so wie an den Seitenfl"achen des vorhin beschriebenen Steines, von der dichten, festen, rauen, netzartig-aderigen Art (A. b. 2), mit sehr erhabenen Adern, h"aufigen, scharfen N"ahten und sehr rauen Zwischenr"aumen; nur zieht sich hier die Farbe mehr ins Pechschwarze, und der seidenartige Glanz n"ahert sich mehr dem fettigen; auch scheint die Rinde hier d"unner zu sein, indem an einigen Stellen, zumal gegen die eine raue Seitenfl"ache hin, die untere braune Schichte, und auf der ganzen Oberfl"ache der, wie es scheint, schwerer in Rinde umwandelbare, wei"se Gemengteil der Steinmasse (wie an dem Tab. 5 Fig. 4 vorgestellten Steine) in Gestalt einzelner und zusammen geh"aufter, wei"ser, gelblicher und br"aunlicher K"orner, die kaum die Gr"o"se der Hanf- oder Hirsek"orner haben, durchscheint.

An den drei, aneinandergrenzenden, auch sonst gleichartigen Seitenfl"achen dagegen ist sie von der gew"ohnlichsten einfach-aderigen Art (A. a. 2), von dunkelschwarzer Farbe und von dem gew"ohnlichen fettigen Glanze. Doch zeigt sich auch hier eine kleine Modifikation, indem an einer derselben, und zwar an der am schiefsten aufsteigenden (auch unebeneren) die Rinde glatter, gl"anzender, anscheinend d"unner, und mit einer Anlage zur bl"atterigen Zeichnung sich zeigt; und was besonders merkw"urdig ist, auf ihr, vom Rande der Grundfl"ache her, die Rinde "ubergeflossen erscheint und einen Saum bildet, indes sie an den beiden andern Fl"achen von jener Fl"ache her gleichf"ormig "uber die R"ander oder Kanten fortl"auft. An der oberen, mit der neuen Bruchfl"ache gebildeten Endkante steht die Rinde dieser Fl"ache angeh"auft, gleichsam als ein aufrechtstehender, ziemlich scharfer Rand an.

An der vierten h"oheren Seitenfl"ache erscheint die Rinde sehr ungleichf"ormig, da sie sehr oft in der Bildung unterbrochen worden zu sein scheint; hin und wieder ist sie deutlich aderig und rau; hie und da aber, zumal an der einen H"alfte, wo auch an der Endkante von der Grundfl"ache her ein Saum gebildet wird, zeigt sich eine Anlage zur blattf"ormig gezeichneten. Sie ist "ubrigens sehr dicht, schwarz und fettig-gl"anzend, und an den erhabensten Stellen und Punkten, so auch an der Teilungskante, dick und kompakt. An den tiefen Stellen ist sie d"unner, und fehlt an manchen Pl"atzen sogar ganz, wo die Grundmasse mit br"aunlicher Farbe zum Vorschein kommt. In dieser zeigt sich der wei"se Gemengteil der Steinmasse in Gestalt von wei"sen K"ornern, und es werden auf ihr nur einzelne oder zusammen gruppierte, und mehr oder weniger ineinander geflossene schwarze Tr"opfchen Rinden-Substanz dem freien Auge sichtbar. (Niedrigster Grad der unvollkommenen Rinde. D. 1.)

An der oberen Endfl"ache endlich, das ist, insoweit eine solche, au"ser der neuen Bruchfl"ache, vorhanden ist, und von den beiden Seitenfl"achen gebildet wird, ist die Rinde ganz genau von derselben Beschaffenheit in jeder Beziehung wie an diesen letzteren, und zieht sich auch von denselben geradezu, ohne alle Unterbrechung der Adern, auf diese Fl"ache her"uber; nur dass sie hier hin und wieder etwas abgerieben ist.

Ganz anders zeigt sich nun die Rinde an jener sp"ater entstandenen Bruchfl"ache, die im Ganzen ein raues, mattes, erdgrau-br"aunliches Ansehen hat. Hier ist in dem br"aunlichen Grunde der wei"se Gemengteil nicht nur noch der Farbe nach erkennbar, und nur selten gelblich oder br"aunlich, sondern selbst hie und da noch ganz erdig und fast kreidewei"s, und die Rinden-Substanz zeigt sich nur, vorz"uglich auf dem R"ucken der scharfen, gleichsam kantigen Erhabenheiten, wie ausgeschwitzte Tropfen, die entweder einzeln dastehen, oder zu Perlenschn"uren, Adern oder kleinen Flecken und Streifen zusammengeflossen sind. Gegen die R"ander hin ist die Rinden-Substanz h"aufiger, an den R"andern selbst aber ist sie von den angrenzenden Fl"achen her angeh"auft, und bildet einen deutlichen Abschnitt, so dass gegen die beiden aderigen Seitenfl"achen hin, wo die konvex sich erhebende Endfl"ache in diese Bruchfl"ache sich allm"ahlich verliert, durch die Rinde selbst erst ein scheinbarer Rand gebildet wird. (Mittlerer Grad der unvollkommenen Rinde. D. 2.).\footnote{\frakfamily{Es zeigt dieser Stein demnach eine f"unffache Verschiedenheit der Rinde an seinen verschiedenen Fl"achen, wovon drei, n"amlich die an den drei niederen Seitenfl"achen und der Basis der oberen Endflache; dann die der vierten hohen Seitenflache und der neuen Bruchfl"ache; endlich die der Grundfl"ache --- wenn sie nicht etwa Modifikation dieser letzteren ist --- Hauptverschiedenheiten zu betrachten kommen, von welchen der Grund haupts"achlich in der ungleichzeitigen Entstehung der Fl"achen, und folglich der ungleichen Dauer des Rindebildungs-Prozesses zu suchen sein d"urfte: --- zwei aber, n"amlich die an der einen schiefern Seitenflache von jener der beiden andern, und die an der vierten hohen Seitenflache von jener der neuen Bruchfl"ache, wohl nur Modifikationen vorstellen, die von der Richtung des Steines im Falle, und von der dadurch abge"anderten Einwirkung des Luftstromes, herr"uhren m"ochten.}}

An der Grundfl"ache sowohl als an allen Seitenfl"achen, ist hie und da etwas, obgleich nur "au"serst wenig, Erde noch anklebend.

Die Abbildung stellt diesen lehrreichen Stein auf seiner Grundfl"ache liegend und so vor, dass nebst den drei niederen Seitenfl"achen die obere Endfl"ache mit der unvollkommen "uberrindeten Bruchfl"ache ganz zur Ansicht kommt.\footnote{\frakfamily{In Gilberts Annalen Bd. 31, Taf. 3, Fig. 2, ist bereits auch von diesem Steine eine Darstellung versucht worden, die aber durch die Kolorierung sehr verunstaltet worden ist.}}

\subsection{\frakfamily{Vierte Figur.}}
\paragraph{}
Ein mittelgro"ser Stein von der Begebenheit bei Stannern, 1 Pfund 1 Loth wiegend, welcher am Tage des Ereignisses selbst, und zwar ebenfalls ganz nahe bei Stannern, auch zwischen den D"orfern Sorez und Falkenau, demnach ebenfalls im Mittelpunkte des befallenen Fl"achenraumes (und zwar kaum mehr als 100$^{\circ}$ s"udlich vom vorhin beschriebenen entfernt) aufgefunden, und dem Pater Caplan in Stannern "uberbracht wurde, der ihn am 29. Mai der Kommission "uberreichte. (Situations-Plan Nr. 40.)

Auch dieser Stein ist vollkommen ganz, und so wie er zur Erde gekommen, erhalten worden, obgleich derselbe noch ungleich mehr als der vorhin beschriebene, auch selbst bei n"aherer, ja wohl ganz naher Betrachtung, das Ansehen eines gro"sen Bruchst"uckes, oder der H"alfte eines entzwei geschlagenen Steines hat, wof"ur er auch lange Zeit von mir und jedermann gehalten wurde, indem eine ganze Seite desselben eine beinahe ganz frische, nur etwas dunkler gef"arbte, gleichsam beschmutzte, Bruchfl"ache zeigt.\footnote{\frakfamily{Schwerlich w"urde ich selbst diese Fl"ache f"ur das, was sie wirklich ist, so bald erkannt haben, wenn nicht der zuvor beschriebene Stein, und "ahnliche, mancherlei Abstufungen der unvollkommenen Rinde aufs klarste aussprechende Stellen an vielen andern, mich aufmerksam gemacht h"atten.}}

Seine Gestalt ist unregelm"a"sig und schwer zu beschreiben; doch bilden alle bestimmbaren Fl"achen, und selbst die scheinbar frische Bruchfl"ache, ein verschobenes Viereck, und am ganzen Steine lassen sich noch acht Ecken, acht End- und vier Seitenkanten am vollkommensten nachweisen, so dass sich die Grund- oder urspr"ungliche Absonderungsgestaltung leicht denken, und die "Ahnlichkeit in der Total-Form mit den meisten der zuvor beschriebenen Steine wieder nicht verkennen l"asst.

Die Oberfl"ache aller vollkommen "uberrindeten Fl"achen --- wovon wieder zwei der an einander grenzenden Seitenfl"achen etwas gew"olbt, die zwei andern etwas vertieft sind, die als Grundfl"ache zu betrachtende aber, welche der neueren Bruchfl"ache gegen "uber gestellt ist, flach und ziemlich eben erscheint --- hat wenige, aber gro"se und breit verlaufende Eindr"ucke gew"ohnlicher Art; ein paar tiefere, sch"arfer begrenzte, sind nicht sowohl blo"sen Eindr"ucken, als vielmehr einem Verluste der Masse durch --- mit der Individualisierung des Steines und der Bildung der Rinde im Ganzen --- gleichzeitige Lostrennung einzelner kleiner St"ucke zuzuschreiben.

Die Rinde ist fast durchaus dieselbe, wenigstens von einer und derselben Hauptbeschaffenheit an allen diesen Fl"achen, und ganz und in jeder Beziehung von der gew"ohnlichsten, einfach und verworren-aderigen Art, wie z. B. an den Seitenfl"achen des vorhin beschriebenen Steines. Sie zeigt weder S"aume noch N"ahte, bildet aber hie und da ziemlich lange, scharfe und erhabene Adern, die eine ziemliche Strecke "uber eine Kante oder den R"ucken von Erhabenheiten laufen, doch keine bestimmte Richtung haben.

An einer ziemlich gro"sen, stark hervorragenden, sehr unebenen Stelle, eigentlich an der ganzen einen gew"olbten Seitenfl"ache, zeigt sich --- als Modifikation --- eine Anlage zur blattf"ormig gezeichneten Rinde; auch scheint da die matte untere Schichte br"aunlich durch, und in ihrer N"ahe zeigen sich an den Kanten der angrenzenden Fl"achen Anh"aufungen von Rinde, von diesen letzteren her, die sich S"aumen n"ahern. "Ubrigens ist die Rinde von der gew"ohnlichen dunkelschwarzen Farbe, und dem gemeinen, ziemlich starken, etwas fettigen Glanze.

Das Merkw"urdigste an diesem Steine ist nun jene dem unbewaffneten Auge ganz rindenlos erscheinende neuere Bruchfl"ache, welche die gr"o"ste und gewisser Ma"sen regelm"a"sigste am Steine ist.

Es bildet dieselbe, obgleich sie sich auch "uber einen Teil einer angrenzenden Fl"ache ausdehnt, ein ziemlich gleichseitiges, nur etwas verschobenes Viereck, welches von drei Seiten her durch die anstehende Rinde der angrenzenden Fl"achen, auf der vierten aber durch die scharfe Bruchkante der Steinmasse, ausgeschweift zwar nach den vorkommenden Unebenheiten der Fl"achen, aber scharf begrenzt wird. Ihre ziemlich spitzen Ecken entsprechenden Seitenkanten, und die scharfen R"ander den Seitenfl"achen des Steines, und sie hat ganz das Ansehen, als w"are ein noch Mahl so gro"ser Stein zerspalten worden, und habe durch einen besonders gl"ucklichen, ziemlich ebenen und geraden Bruch diese Bruchfl"ache gegeben. Sie ist sehr uneben, aber nicht von der Art, wie die "uberrindeten Fl"achen zu sein pflegen (durch meist rundlichte, allm"ahlich sich erhebende, und sanft in die Erhabenheiten breit verlaufende, sondern durch sehr ungleichf"ormige und winklige, von senkrechten, oder nur wenig schiefen und ziemlich scharfkantigen Erhabenheiten begrenzte Vertiefungen), vielmehr sieht sie gerade so aus wie eine frische k"unstliche Bruchfl"ache der Steinmasse, hat aber weder das frische Ansehen, noch ganz die Farbe einer solchen, sondern ist schmutzig oder br"aunlich-grau, hie und da mit bl"aulichwei"s und aschgrau gemischt. Die Masse scheint dichter, fester und weniger rau zu sein, und wenn man sie mit der einfachen Lupe betrachtet, so sieht man hier und da, zumal an den erhabenen Stellen, an den Kanten der scharfen Erhabenheiten, und der durch Risse getrennten Partien, die angefangene Erzeugung der schwarzen Rinden-Substanz in Gestalt kleiner Tropfen, Perlenschn"ure oder Einfassungen. An den R"andern st"o"st die Rinde der vollkommen inkrustierten Seitenfl"achen dicht an, so dass, wie gesagt, durch dieselbe eigentlich der wahre Rand dieser Fl"ache selbst erst gebildet wird; und obgleich diese Rinde hier scharf abgeschnitten und nicht viel dicker ist, als an einer k"unstlichen Bruchfl"ache, so zeigt sie doch keine Spuren eines Bruches; denn sie ist da eben so dicht und gl"anzend, wie an der Oberfl"ache, und l"asst die zweite untere, por"ose, matte Schichte nicht erkennen. (Haupt-Kriterium eines solchen, vor dem wirklichen Niederfalle und noch in der Luft entstandenen, nat"urlichen Bruches von einem k"unstlichen.) Offenbar ist sie an einigen Stellen, zumal gegen jene Seitenfl"ache hin, wo die Rinde sehr kompakt, schwarz und aderig ist, von daher wie "ubergeflossen oder "ubergedr"uckt, wenigstens weiter fortschreitend, so dass sie einen betr"achtlichen Saum oder eine Einfassung auf dieser Fl"ache, "uber die Kante her, bildet. An einer scharfen Ecke erstreckt sich diese Einfassung bis auf $\mathfrak{1\frac{1}{2}}$ Linie weit auf diese Fl"ache hinein; die Steinmasse ist in der angrenzenden Gegend auch dunkler, und zeigt h"aufigere Tropfen.

Eine, dieser ganz "ahnliche, aber ungleich kleinere Fl"ache, findet sich an demselben Steine gegen den unteren Rand der einen Seitenfl"ache (die von jener Fl"ache unter einem Winkel von beil"aufig 100$^{\circ}$ abweicht), mitten in der Rinde, gerade als wenn hier ein Zoll gro"ses (aber allem Ansehen nach nur sehr d"unnes) St"uck der Steinmasse, das etwa urspr"unglich eine hervor stehende Ecke oder eine Erhabenheit gebildet haben mochte --- nachdem die Hauptfl"ache und "uberhaupt der ganze Stein bereits "uberrindet war --- und zwar ganz gleichzeitig mit jenem St"ucke, das obige neuere Bruchfl"ache bildete, mit Gewalt abgesprengt worden w"are, und als wenn, hier wie dort, das Rinden bildende Agens (der Rindenbildungs-Prozess) nicht mehr Intensit"at oder Zeit genug gehabt h"atte, die erzeugte Bruchfl"ache vollkommen zu inkrustieren (was wohl unwiderleglich, wirklich und w"ortlich der Fall gewesen sein muss).

Diese beiden Fl"achen zeigen die unvollkommene Rinde in ihrem h"ochsten Grade (D. 3), und zwar von bedeutender Ausdehnung, wie ich sie, aber meistens nur auf sehr kleinen Stellen vorkommend, auf den meisten der beschriebenen Steine nachgewiesen habe.\footnote{\frakfamily{Dieser kostbare Stein zeigt demnach eine zweifache Hauptverschiedenheit der Rinde, und zwar gerade die extremsten Punkte von ihrer Ausbildung beisammen, die wohl die entferntesten Zeit-Momente der Rindebildungs-Periode, und die heterogensten Wirkungsgrade des Rindebildungs-Prozesses zu bezeichnen scheinen --- und eine, auch wohl zwei Modifikationen; erstere n"amlich an der einen gew"olbtern Seitenfl"ache, als Modifikation der dunkleren, raueren, an den "ubrigen vollkommen "uberrindeten Fl"achen vorkommenden Rinde; und letztere etwa an einer der, an jene gro"se Bruchfl"ache angrenzenden, obiger gerade entgegen gestellten Seitenfl"achen, wor"uber sich zum Teil jener Bruch fortsetzte, die Masse aber schon weit dunkler, und die Rinde bereits in Flecken und Streifen (D. 1) sich zeigt.}}

Von eingedr"uckter Erde zeigt sich etwas an der, der neuern Bruchfl"ache entgegen gesetzten, als Grundfl"ache betrachteten, und an der gr"o"seren, gew"olbten Seitenfl"ache.

Die Abbildung zeigt diesen belehrenden Stein, auf einer Seitenfl"ache aufgestellt, von jener merkw"urdigen, gro"sen, neueren Bruchfl"ache, und zwar so, dass das Licht von jener Seite einf"allt, wo sich die scharfe Kante und Ecke mit dem "ubergeschlagenen Rindensaume befindet.\footnote{\frakfamily{Auch von diesem Steine, und von derselben Ansicht genommen, findet sich in Gilberts Annalen Bd. 31, Taf. 3, Fig. 1, eine fr"uhere Abbildung, die aber durch die Kolorierung gar sehr an Deutlichkeit verloren hat.}}

\subsection{\frakfamily{F"unfte Figur.}}
\paragraph{}
Ein $\mathfrak{3\frac{1}{2}}$ Loth wiegendes Bruchst"uck eines gro"sen, urspr"unglich 4 Pfund schwer gewesenen Steines von Stannern, welcher am Tage der Begebenheit selbst, von dem Oberj"ager von Iglau, gegen den Ort Teschen zu, am westlichen Teile des befallenen Fl"achenraumes von dessen Mittelpunkte, und zwar am entferntesten Punkte daselbst (etwa 1300$^{\circ}$ westlich von der Kirche von Stannern, und bei 3400$^{\circ}$ s"ud-westlich vom "au"sersten Punkte in N., und bei 4500$^{\circ}$ nord-westlich vom "au"sersten Punkte in S., wo die entferntesten Steine gefallen waren) gefunden, aber zerschlagen, und wovon nur die gr"o"sere H"alfte, von 2 Pfund 12 Loth am Gewichte, am 29. Mai an die Untersuchungs-Kommission abgegeben wurde. (Situations-Plan Nr. 63.)

Es zeigte diese gr"o"sere H"alfte des Steines, au"ser den frischen Bruchfl"achen, gr"o"sten Teils eine sehr raue, grob-runzlicht-aderige Rinde von dunkelschwarzer Farbe, und dem gew"ohnlichen fettigen Glanze, die aber sehr h"aufig und bedeutend fleck- und stellenweise abgerieben oder abgesprungen, das ist, von der obersten schwarzen, gl"anzenden Schichte entbl"o"st, und hier braun, matt und zart por"os war (A. a. 1. Gilberts Annalen Bd. 31, S. 56 im ausgezeichnetsten Grade). Da dieses St"uck "ubrigens nichts Auszeichnendes hatte, so ward dasselbe zum Behufe der beabsichtigten Versuche, und um mehrere Mitteilungen machen zu k"onnen, in viele Bruchst"ucke zerschlagen, wovon nun dieses eines ist, welches f"ur die Sammlung zur"uckbehalten wurde.

Es zeigt dasselbe, von der einen konvexen Au"senseite, die oben beschriebene Rinde im vollkommensten Grade, von der andern aber eine frische Bruchfl"ache von der gew"ohnlichen Beschaffenheit der Masse dieser Steine; nur mit dem Besondern, dass auf derselben, zwar nur gegen den Rand des Bruches, und folglich gegen die "au"sere Rinde hin, aber doch hie und da beinahe einen halben Zoll tief von der Oberfl"ache einw"arts, und zwar an Stellen, wo an dieser vor dem Zerbrechen des Steines gar keine Risse oder Spr"unge der Masse zu beobachten waren, ziemlich gro"se Flecke von Rinden-Substanz mitten in oder dermal vielmehr auf der ganz unver"anderten Steinmasse zur Ansicht kommen.

Diese Flecke liegen zum Teil dicht an der Oberfl"ache, und h"angen mit der "au"sern Rinde wirklich zusammen, als wenn diese hineingeflossen w"are; einige liegen aber weiter ab, ganz isoliert, und sind von durchaus unver"anderter Steinmasse, selbst von eingestreuten, metallisch gl"anzenden Kies-Br"ockeln und Punkten umgeben. Einige derselben sind gl"anzend schwarz, wie die "au"sere Rinde, viele matt schwarz, wie die untere Schichte derselben zu sein pflegt, die meisten aber sind mehr oder weniger von der Steinmasse bedeckt, die beim Zerschlagen des Steines daran festblieb.

Die Gr"o"se und Gestalt dieser Flecke ist sehr verschieden, ihr Umriss ist aber nie rundlich, sondern vielmehr winkelig und vieleckig; ihr Rand scharf begrenzt und wie gebrochen, und ihre Dicke betr"agt nicht mehr als die der Au"senrinde. Eingeknetet in die Masse sind diese Flecke keineswegs, denn sie erscheinen nur als d"unne Lagen, und verursachen, dort wo sie sich finden, eine gleichsam schalige oder schiefrige Absonderung der Steinmasse.\footnote{\frakfamily{Obgleich ich mich zur Zeit au"ser Stande f"uhle, von der Bildung der Rinde an den Meteor-Steinen "uberhaupt, und insbesondere von der Entstehung derselben im Innern der Steinmasse, sowohl in Gestalt solcher Flecken (in welcher sie jedoch am seltensten, und wohl nie weit von der Oberfl"ache entfernt vorkommen, und f"uglich noch der Einwirkung des Rinde bildenden Agens von Au"sen her zuzuschreiben sein d"urfte), als in Gestalt eingestreuter Punkte (in welcher sie inzwischen nur bei sehr lockeren Meteor-Steinen, z. B. bei jenen von Chassigny (Langres) deutlich, weniger bei den Steinen von Stannern, und bei beiden selbst h"ochst problematisch (ob nicht Chrom-Eisen oder Eisen-Oxyd?), bei Meteor-Steinen von festem Koh"asions-Zustande und dichtem Gef"uge meinen Untersuchungen nach, selbst nicht als Spur erscheint), als vollends in Form von Adern, G"angen, Schichten und Lagen (deren Substanz man f"ur einerlei mit jener der Rinde zu halten geneigt scheint, und von welcher bei Erkl"arung der n"achsten Tafel die Rede sein wird), eine befriedigende Erkl"arung zu geben; so muss ich doch freim"utig gestehen, dass ich der Ansicht meines Freundes Chladni, von der Bildung der Rinde "uberhaupt, und dieser im Innern (insofern ihr Vorkommen darin wirklich Statt findet) insbesondere, durchaus nicht beistimmen kann. Die Gegenwart des Schwefels (dessen Anwesenheit in der Steinmasse, wenigstens in gebundenem Zustande, "ubrigens nicht in Abrede gestellt werden kann), den Hr. Chladni als das Haupt-Material betrachtet, aus welchem die Rinde gebildet wurde, gibt sich in derselben auf keine Weise zu erkennen; weder durch die chemische Analyse, noch durch eine leichte Schmelzbarkeit (die im Gegenteile sehr schwer ist, da sie wenigstens 6 bis 9$^{\circ}$ Wedgwd. Hitze fordert, und die wohl, wenn man den Rindebildungs-Prozess durch Hitze geschehen lassen wollte, sehr gegen die, obgleich nur durch ein paar F"alle, in Anregung gebrachte Abf"arbung der Steine, streiten m"ochte), weder durch den Geruch bei Erhitzung, noch durch den geringsten Grad von Wirkung auf das Elektrometer, wenn gerieben oder erw"armt; so wie andererseits die Mannigfaltigkeit der Rinde bei verschiedenen Meteor-Steinen, und die offenbare Abh"angigkeit derselben von den Gemeng- und Bestandteilen der Steinmasse, gegen ein solches allgemeines Haupt-Material streitet. Die Gleichf"ormigkeit der Rinde, zumal hinsichtlich der Dicke, auf sonst gleichartigen, wenn gleich sich noch so sehr entgegen gesetzten Fl"achen, an ein und demselben Steine, und die "Ubereinstimmung hierin bei allen Meteor-Steinen im Augemeinen; die unwiderleglich von der Beschaffenheit der Oberfl"ache abh"angigen Hauptverschiedenheiten derselben an ein und demselben Steine; die offenbare, allm"ahliche und stufenweise Ausbildung derselben; und der unverkennbare "Ubergang ihrer Massenteilchen in jene der Steinmasse, und umgekehrt, wo beide sich im Kontakte befinden (wie sich aus der mikroskopischen Betrachtung ergibt) u. s. w., lassen sich wohl schlechterdings nicht durch eine "Ubergie"sung oder Bespritzung von Au"sen her erkl"aren. Endlich lasst sich das, nach meinen Beobachtungen nur h"ochst selten (meiner "Uberzeugung nach bisher nur an diesem einzigen beschriebenen Bruchst"ucke) und nie tief im Innern eines Steines sich zeigende wirkliche Vorkommen von Rinde in Gestalt von Flecken, deren Form, Beschaffenheit und Zusammenhang mit der Steinmasse (nach obigem), so wie die Art des mehr als problematischen Vorkommens derselben in Adern, G"angen und Lagen (wovon seines Ortes) wohl nicht mit der Idee einer Einknetung und Zusammenklebung vereinigen, als welche einerseits einen ziemlich tumultuarischen (G"ahrungs-) Prozess bei jedem einzelnen Steine nach dessen Individualisierung, Bildung und bereits schon ein Mahl vollendeter Inkrustierung, andererseits ein h"aufiges Zusammentreffen, Zusammenpassen und Wiedervereinen der bereits mit Gewalt losgetrennten und weit weg und aus einander geschleuderten Steine und Bruchst"ucke voraussetzen, mit welchen die Regelm"a"sigkeit und "Ubereinstimmung so vieler Steine in der Form (der Grund-Typus), die Beschaffenheit der Fl"achen und Kanten (welche beide Umstande schlechterdings keinen solchen Grad von Weichheit nach einmal geschehener Inkrustierung denken lassen), der entfernte Niederfall der einzelnen Steine voneinander (der meistens einen Zwischenraum von 2 bis 300, oft 1000 und mehr Klafter betr"agt) u. s. w., im offenbarsten Widerspruch zu stehen scheinen.\\
\hspace*{6mm}Eher k"onne ich der Meinung meines Freundes, des Hrn. Prof. v. Scherer (welcher fr"uher gleichzeitig und zum Teil gemeinschaftlich mit mir "uber diesen Gegenstand arbeitete, und seine Bemerkungen "uber die Beschaffenheit und wahrscheinliche Entstehung der Rinde an den Steinen von Stannern, in einem gleichzeitigen Aufs"atze in Gilberts Annalen Bd. 31 bekannt machte), beipflichten, nach welcher die Rinde in einem Nu, und gleichsam mit Blitzesschnelle, und zwar im Momente der Vereinzelung, Individualisierung der Steine, "uber alle zugleich, und "uber deren ganzen Umfang auf ein Mahl, nur mit verschiedener Intensit"at der wirkenden Potenz, demnach mit einigen Modifikationen, gebildet wurde, und jene Potenz in der Elektrizit"at zu suchen sein mochte; wenn sich darnach einige Eigenheiten derselben, z. B. die vielen und auffallenden Hauptverschiedenheiten und h"aufigen stufenweisen Modifikationen und "Uberg"ange der Rinde (deren, wie gezeigt worden ist, immer an einem und demselben, oft sehr kleinen Steine, mehrere, 2 bis 5, deutlich unterschieden, aber nicht wohl begreiflich von einer so vielfachen Verschiedenheit der Intensit"at, der sie auf ein Mahl erzeugenden Potenz, abgeleitet werden k"onnen), befriedigend erkl"aren lie"sen; wenn ihr nicht ferner einige Erscheinungen bei dem Ereignisse selbst, z. B. das bei diesem, so wie "uberhaupt bei allen "ahnlichen Ereignissen, wo viele Steine fielen, ganz einstimmig gleichartig beobachtete, fortgesetzte, einem kleinen Gewehr- oder Pelotonfeuer "ahnliche Getose nach den Haupt-Detonationen (welches wohl nur von einem wiederholten, sukzessiven Zerplatzen und Zerspringen der einzelnen Steine w"ahrend ihres Falles hergeleitet werden kann); das so ausnehmend schiefe und sanfte Auffallen mancher einzelner, ziemlich gro"ser Steine, so dass sie kaum merklich die Erde streiften und eine Strecke fortrollten (welches eine horizontale Wurfbewegung voraussetzt, die sich mit der H"ohe, auf welcher die Hauptzerplatzung vorging, der gegenwirkenden Schwerkraft wegen, schlechterdings nicht vertr"agt, daher eine sp"atere Zerplatzung eines einzelnen Steines im Falle, auf minderer H"ohe, und die Lossprengung eines St"uckes davon in solcher Richtung vorausgesetzt werden muss) u. s. f. --- in Wege stunden; und wenn es endlich nicht ganz an allen Wahrnehmungen fehlte (worauf insbesondere und mit Vorbedacht bei der Untersuchung der Begebenheit zu Stannern alle Rucksicht genommen wurde), die das Spiel oder den Einfluss der Elektrizit"at bei diesen Ereignissen nur einiger Massen bewahren konnten. Dagegen bin ich mit diesen beiden scharfsinnigen Physikern vollkommen einverstanden, wenn sie behaupten, die Rinde der Meteor-Steine sei das Produkt eines Prozesses, das mit keinem Produkte der uns bekannten nat"urlichen und k"unstlichen Schmelz-Prozesse (wenn jener Rinde bildende ja in die Reihe solcher zu stellen sein sollte) einige "Ahnlichkeit habe, weshalb wir uns auch zur Zeit keinen richtigen Begriff von ihrer Bildung machen k"onnen.}}
\clearpage
\section{\frakfamily{Siebente Tafel.}}
\paragraph{}
Die Abbildungen auf dieser Tafel haben die Darstellung und Versinnlichung der inneren Beschaffenheit der Steinmasse einiger, der in dieser Beziehung ausgezeichnetsten Meteor-Steine, des Aggregats-Zustandes derselben und ihrer wesentlichsten Gemengteile zum Zwecke, und in Hinsicht dieser letzteren insbesondere, die Darstellung des allgemeinsten, auffallendsten und sehr wesentlichen, n"amlich des mehr oder weniger kugelichten, porphyrartig in der "ubrigen Steinmasse erscheinenden, erdigen Gemengteiles, und zwar in den verschiedenen Graden seiner Ausbildung, die von einer kaum erkennbaren Ausscheidung bis zu dessen ausgesprochenstem Zustande --- als olivinartige Substanz im sibirischen Eisen --- "Uberg"ange nachweisen lassen, und deren sich oft mehrere, nicht nur in verschiedenen Steinen eines und desselben Niederfalles, sondern selbst in einem und demselben Bruchst"ucke, beisammen finden.\footnote{\frakfamily{Um eine deutliche Ansicht von dem so sehr verschiedenen Aggregats-Zustande der Steinmasse sowohl, als insbesondere von dem so sehr abweichenden, wechselseitigen, quantitativen Verh"altnisse der Gemengteile, und von deren mannigfaltigen Beschaffenheit und Zustand zu gewinnen, ist es durchaus notwendig, an jedem Meteor-Steine oder an einem Bruchst"ucke von demselben, eine Bruchfl"ache schleifen und polieren zu lassen; doch muss dieses mit der Vorsicht geschehen, dass bei der Behandlung so wenig Feuchtigkeit und so wenig Schmirgel, oder sonstiges Schleif- oder Polier-Pulver, als nur immer m"oglich, angewendet, letzteres aufs vollkommenste sogleich weggewaschen, und das St"uck dann schnell und gut getrocknet werde, um das eigent"umliche Ansehen nicht durch eine fremdartige Substanz, oder durch bef"orderte Oxydation des enthaltenen Eisens, mehr oder minder verunstaltet zu erhalten.}}

\centerline{*\hspace{15mm}*\hspace{15mm}*\hspace{15mm}*\hspace{15mm}*}
\bigskip

Alle bisher bekannten, eigentlichen Meteor-Steine, sind gemengte Massen, und alle authoptisch mir davon bekannten 34,\footnote{\frakfamily{Namentlich Bruchst"ucke von Steinen von den Vorf"allen bei Ensisheim, Verona, Tabor, Laponas (Bresse), Lucé, Mauerkirchen, Sigena, Eichst"adt, Charkow, Barbotan, Siena, York, Salés, Benares, L'Aigle, Apt, Eggenfeld, Glasgow, Doroninsk, Alais, Timochin, Weston, Parma, Stannern, Lissa, Tipperary, Charsonville, Berlanguillas, Toulouse, Erxleben, Chantonnay, Limerick, Agen und Chassigny (Langres), als von welchen auch "ahnliche Belege sich notorisch im Besitze "offentlicher Sammlungen oder bekannter Privat-Eigent"umer befinden. Es sollten und werden wohl auch von noch mehreren Vorf"allen neuerer Zeit, vielleicht von 20 bis 30 au"ser obigen, derlei Belege vorhanden sein und sich in den H"anden von Privat-Besitzern befinden, die aber leider nicht verl"asslich bekannt sind.}} nach Zeit und Ort des Niederfallens verschiedene, nur mit Ausnahme jener von Alais, Erxleben, Chassigny (Langres), und zum Teil jener von Chantonnay, welche ein ganz eigent"umliches Ansesehen, selbst im Ganzen\footnote{\frakfamily{Ein Ansehen, wodurch sie sich nicht nur unter sich, sondern auch von allen "ubrigen bisher bekannten Meteor-Steinen so sehr unterscheiden, dass man sie wohl nicht leicht f"ur solche erkennen m"ochte, wenn nicht einerseits ihre faktisch erwiesene Herkunft und die Haupt-Resultate der chemischen Analyse, und andererseits selbst einige, wenn gleich nur einseitige, und oft nur in "Uberg"angen nachweisbare, oryktognostische Verwandtschaft hinsichtlich einzelner Gemengteile, oder irgend einer Zustandsver"anderung der Masse bei andern, unbezweifelbaren Meteor-Steinen, f"ur sie das Wort sprechen und gewisser Ma"sen B"urgschaft leisten m"ochten. (So z. B. das stellenweise Filzig-Faserige der Grundmasse der Steine von Eggenfeld, Mauerkirchen, Benares, Parma, Siena, und das zum Teil Unausgesprochene und Undeutliche des kugelichten Gemengteiles bei so vielen Meteor-Steinen, f"ur jene von Stannern; --- die individuelle Beschaffenheit dieses letzteren Gemengteiles bei vielen andern Meteor-Steinen, und die "Ahnlichkeit darin mit der Hauptmasse jener von Chassigny, f"ur diese; --- die "Ahnlichkeit der Substanz der in vielen Meteor-Steinen vorkommenden Adern und G"ange, f"ur die von Chantonnay, und zum Teil von Alais; --- endlich die bei manchen Meteor-Steinen hie und da erscheinenden spatartigen, schillernden Stellen, f"ur jenen von Erxleben.)}} haben, und beziehungsweise auch der von Stannern, lassen viererlei Gemengteile, selbst dem freien Auge, und gew"ohnlich sehr deutlich ausgesprochen, erkennen.

Zwei dieser Gemengteile sind erdiger, zwei davon metallischer Natur.

Der eine erdige Gemengteil hat ein mehr oder weniger mattes, mageres, und, nach der verschiedenen Feinheit und Gleichf"ormigkeit des Korns --- das vom groben bis zum "au"serst feinen, dem unbewaffneten Auge kaum unterscheidbaren, abweicht --- und nach dem mehr oder minder dichten und festen Koh"asions-Zustande --- der vom leicht zerreiblichen bis zum schwer zersprengbaren und Funkengeben geht --- und insofern derselbe nicht --- was jedoch selten und nur stellenweise der Fall ist --- eine besondere, faserige, sp"atige oder bl"atterige Textur zeigt, ein mehr oder minder raues, sandsteinartiges Ansehen, und eine lichter oder dunkler aschgraue, selten ins Wei"se oder Gelbliche, meistens ins Bl"auliche ziehende Farbe.

Es kann dieser Gemengteil, r"ucksichtlich der "ubrigen, seiner Gleichf"ormigkeit wegen, und da er meistens mehr oder weniger, und oft sehr bedeutend "uber alle "ubrigen zusammen, oder doch "uber jeden derselben einzeln genommen, an Menge vorwaltet, als Haupt- oder Grundmasse angesehen werden, und dies umso f"uglicher, als alle "ubrigen Gemengteile aus dieser Masse gebildet oder ausgeschieden worden, aus ihr entstanden oder hervor gegangen sein d"urften, als zu welchem Schlusse nicht nur die physiologisch-oryktognostischen, sondern insbesondere die physisch-chemischen Untersuchungen, auf deren Resultate geh"origen Ortes hingedeutet werden wird, zu berechtigen scheinen.

Die Abweichungen dieser Grundmasse in obigen Eigenschaften, obgleich sie in den extremsten Gliedern sehr auffallend sind, gehen durch Zwischenglieder so allm"ahlich in einander "uber, dass zuletzt aller Abstand verschwindet; besonders merkw"urdig aber ist, dass mehrere dieser Abweichungen, zumal in Dichtheit und Farbe, und zwar oft in einem sehr merklichen Grade, nicht selten bei Steinen von einem und demselben Ereignisse, ja selbst bei Bruchst"ucken eines und desselben Steines vorkommen, so dass sich solche, zumal wenn "ahnliche Abweichungen hinsichtlich der "ubrigen Gemengteile, wo sie noch weit gew"ohnlicher und ungleich mannigfaltiger sind, zugleich Statt finden, oft mehr voneinander unter sich, als von Bruchst"ucken ganz anderer, nach Zeit und Ort des Niederfallens sehr verschiedener, Meteor-Steine unterscheiden.\footnote{\frakfamily{Diess ist z. B. vorz"uglich bei den Steinen von Chantonnay, L'Aigle, Barbotan, Weston, Charsonville, Agen, Lissa, und zum Teil selbst bei denen von Stannern der Fall, und manche Bruchstucke eines einzelnen dieser Steine sind sich weit un"ahnlicher, als es oft Bruchst"ucke von Steinen von Eichst"adt und Timochin, von Apt und Berlanguillas, von York, Glasgow und Toulouse, von Tipperary und Limerick, von Siena und Parma gegen einander sind, ja oft sind jene manchen von diesen mehr "ahnlich, als sie es unter sich selbst sind.}}

Im Bruche gibt diese Masse nach dem verschiedenen Koh"asions-Zustande --- wenn dieser oder vielmehr der durch die "ubrigen Gemengteile vermittelte Aggregats-Zustand nicht so locker ist, dass sie br"ocklig oder sandsteinartig k"ornig zerf"allt, was jedoch h"ochst selten der Fall ist --- gr"o"sere oder kleinere, unbestimmt eckige und ziemlich scharfkantige, und an den "au"sersten Kanten bisweilen selbst etwas durchscheinende Bruchst"ucke, und geschliffen nimmt sie nicht selten einen bedeutenden und andauernden Grad von Politur an.

Nach obigem Ma"sstabe ist die Masse auch mehr oder weniger leicht, wenn ganz rein, meistens sehr leicht, zu Pulver zu sto"sen, und zuletzt zum feinsten Pulver zerreiblich.

Das gr"obere Pulver unter dem Mikroskope betrachtet, zeigt, auch bei vollkommen erdigem Ansehen der Masse im Ganzen (wie bei den Steinen von Siena, Benares, Stannern), ein Gemenge von mehr oder weniger kristallinischen, durchscheinenden, zum Teil durchsichtigen, unbestimmt eckigen, ziemlich scharfkantigen K"ornern, von kristallwei"ser, gelblicher, gelblichgr"uner und gr"unlicher, in einander "ubergehenden Farben, meistens in gr"o"serer, und von halb kristallinischen, teils halb durchscheinenden, teils ganz undurchsichtigen, grauen, blau- und rauchgrauen Partikelchen, gew"ohnlich in geringerer Menge. Erstere scheinen in diese, diese in andere, meistens doch nur in einem sehr geringen Verh"altnisse, oft nur einzeln vorhandene, schwarze, gl"anzende kleine Massen "uberzugehen, die ein etwas schlackiges und der Kohlenblende "ahnliches Ansehen haben. Gew"ohnlich zeigt sich noch eine vierte Art von Massenteilchen in jenem Gemenge, obgleich meistens nur in sehr geringer Menge, bisweilen jedoch vorwaltend, als wei"se oder grauliche, mehr erdige, undurchsichtige, oder doch nur schwach und teilweise durchscheinende, dem verwitterten Feldspate "ahnliche Teilchen, welche, oft innig mit den Partikelchen der zweiten Art verbunden, in andere "ubergehen, die eigentlich nicht mehr der Hauptmasse anzugeh"oren scheinen, und von welchen bei Gelegenheit des einen metallischen Gemengteiles der Steinmasse (des Gediegeneisens und der damit verbundenen Rostflecke) die Rede sein wird.

Die kleinen schwarzen Massen sind etwas schwerer zu Pulver zu sto"sen, und lassen beim Zerreiben gew"ohnlich ein kleines Metallteilchen zur"uck, das sich auf dem Ambosse, obgleich etwas schwer, fletschen l"asst, auch werden sie von der Magnetnadel angezogen; die grauen Partikelchen werden es nur in so ferne, als sie mit jenen oft innig verbunden sind; die kristallinischen durchsichtigen aber gar nicht.

Aus dieser Beschaffenheit\footnote{\frakfamily{Obige Beschreibung ist das Resultat einer m"uhsamen, schon 1808 vorgenommenen, vergleichenden, mikroskopischen Betrachtung von zehn verschiedenen Meteor-Steinen, die mir damals zu Gebote standen (namentlich des von Eichst"adt; der von Tabor, Barbotan und L'Aigle; von Ensisheim und Lissa; und der von Siena, Mauerkirchen, Benares und Stannern), welches wohl als allgemein geltend angesehen werden kann (da ich in dieser Zwischenzeit keine Mu"se fand, diese Untersuchungen weiter fortzusetzen), indem es aus der Vergleichung von so vielen, in den wesentlichen Beziehungen so sehr voneinander abweichenden Steinen, die nach meiner Ansicht vier "Ubergangesreihen in der Sippschaft bilden, abgezogen ist.}} der Massenteilchen dieses einen, die Grundmasse der Meteor-Steine konstituierenden Gemengteiles, so wie aus jener, gleich zu beschreibenden des zweiten erdigen Gemengteiles, die sich bei manchen Meteor-Steinen noch weit deutlicher, und selbst im Ganzen schon, ohne mikroskopische Untersuchung der integrierenden Massenteilchen ausspricht (wie bei den Steinen von Erxleben und Chassigny), und aus den offenbaren "Uberg"angen beider in einander, so wie aus den Resultaten der Analysen,\footnote{\frakfamily{Abgesehen von den metallischen Gemengteilen, stimmt bekanntlich nicht nur das qualitative, sondern selbst das quantitative Verh"altnis der chemischen Bestandteile der Steinmasse der meisten bisher bekannten Meteor-Steine ziemlich genau mit jenem des terrestrischen Olivins zusammen. Kieselerde ist ebenso wie bei diesem der vorwaltendste Bestandteil, der in der Regel wohl nur zwischen 30 und 50 Perzent abweicht, und Talkerde ist h"ochst wahrscheinlich ein ebenso best"andiger, nur im quantitativen Verh"altnisse etwas mehr, zwischen 2, im Allgemeinen doch wohl nur zwischen 10 und 30 Perzent variierender Bestandteil. Der sehr unbest"andig scheinende Gehalt an Alaun und Kalkerde (im Allgemeinen von 1 bis 3 Perzent --- mit Ausnahme der Steine von Stannern, wo er auf Rechnung jenes an Talkerde eingetreten zu sein scheint ---) ist doch viel zu gering, als dass er f"ur entscheidend und f"ur etwas mehr geltend gemacht werden k"onnte, als h"ochstens vielleicht f"ur eine Ann"aherung an ein anderes, mit dem Olivin geognostisch verwandtes Fossil, n"amlich den Augit.\\
\hspace*{6mm}Obgleich ferner der eine als Grundmasse angenommene Gemengteil nur h"ochst selten, selbst kaum \emph{en gros}, ganz rein und f"ur sich (nach oben beschriebener Beschaffenheit der Massenteilchen aber auf keine Weise vollkommen abgeschieden) chemisch untersucht werden kann; so fand sich doch, wo dieses einiger Ma"sen m"oglich war (wie bei den Steinen von Benares durch Howard und Bournon), ein h"ochst unbedeutender Unterschied selbst im quantitativen Verh"altnisse der Bestandteile zwischen diesem und dem andern, doch sehr ausgeschiedenen, und schon mehr als Olivin ausgesprochenen Gemengteil, n"amlich in diesem nur um 2 Perzent Kieselerde mehr, und 3 Perzent Talkerde weniger als in der Grundmasse.\\
\hspace*{6mm}Von dem Verh"altnisse dieses olivinartigen Gemengteiles in den Meteor-Steinen zur olivinartigen Substanz im sibirischen Eisen --- und von jenem dieser zum terrestrischen Fossil dieses Namens, wird gleich bei Beschreibung des ersteren die Rede sein.}} ergibt sich nicht nur die nahe Verwandtschaft, oder vielmehr die Identit"at beider, sondern auch die wahre Natur der Steinmasse im Ganzen, als Olivin in verschiedenen Graden von Ausbildung und Charakterisierung, wof"ur sie bereits auch Hausmann und Stromeyer erkannt und ausgesprochen haben.

Der zweite erdige Gemengteil der Steinmasse hat teils ein mattes, von der Grundmasse zum Teil oft nur wenig verschiedenes, mageres, meistens aber doch glatteres, dichteres Ansehen, und unterscheidet sich von derselben gew"ohnlich mehr oder weniger, obgleich oft nur allm"ahlich und "ubergehend, durch ein weit feineres gleichf"ormigeres Korn, gr"o"sere Festigkeit und H"arte, die vom Wacker-Feuerschlagen und Glasritzen nur bis zum Leichtzersprengbaren herabsinkt, und durch eine dichtere Textur, die bis ins Sp"atige und Kristallinische geht, und mit welcher der Glanz, ein Mittel zwischen Fett- und Glasglanz, zunimmt, und die Undurchsichtigkeit bis ins Durchscheinende, und selbst ins Durchsichtige "ubergeht.

Die Farbe geht aus dem verschiedenen Grau der Grundmasse, mit der sie inzwischen oft ganz gleich, nur meistens etwas lichter oder dunkler ist, ohne merklicher Abh"angigkeit von, und ohne regelm"a"sige "Ubereinstimmung mit obigen Eigenschaften, unter vielen und allm"ahlichen Abstufungen (Nuances) von H"ohe und Tiefe, licht und dunkel, und in sehr mannigfaltigen, ebenso allm"ahlich in einander "ubergehenden Modifikationen (Teintes) der Hauptfarben, aus dem Gelblichen oder Graulichen, einerseits, obgleich seltener, ins Wachs- und Honiggelbe, andererseits und am gew"ohnlichsten ins Lauch-, seltener ins Spargel- und Pistazien-, am h"aufigsten ins Oliven- und "Ol-, bis ins Schw"arzlich-Gr"une, und aus dem Bl"aulich-Grauen ins Perl- und Schiefer-Graue- und ins Lavendel- bis ins Schw"arzlich-Blaue.

Es zeigt sich dieser Gemengteil bald mehr, bald weniger ausgeschieden, sch"arfer oder schw"acher begrenzt, und nach Verh"altnis obiger Eigenschaften, zumal nach den verschiedenen Graden seiner Dichtheit und der Intensit"at und Beschaffenheit der Farbe, mehr oder weniger ausgesprochen und von der Grundmasse ausgezeichnet, bisweilen aber auch kaum erkennbar von derselben geschieden, aus ihr oder in sie gleichsam "ubergehend, mehr oder minder h"aufig, in Massen von sehr verschiedener Gr"o"se und Gestalt, und h"ochst ungleichf"ormig in der Grundmasse verteilt.

Bei weitem am gew"ohnlichsten ist das quantitative Verh"altnis dieses Gemengteils zur Grundmasse nur gering, nur h"ochst selten n"ahert sich dasselbe der H"alfte, gew"ohnlich betr"agt es zwischen $\mathfrak{\frac{1}{5}}$ bis $\mathfrak{\frac{1}{10}}$ von der Gesamtmasse, oft aber auch noch weit weniger, und nicht selten findet sich dieser Gemengteil nur in einzelnen, wenigen, sehr zerstreuten Massen, scheint aber, wenn gleich oft sehr undeutlich ausgesprochen, nie ganz zu fehlen\footnote{\frakfamily{So finden sich z. B. in der lockern, leicht zerreiblichen Meteor-Masse von Alais rundlichte K"orner von betr"achtlicher Dichtheit und H"arte eingemengt.}}; dagegen scheint er bisweilen, obgleich nur h"ochst selten, entweder ganz innig mit der Grundmasse gemengt zu sein, oder dieselbe beinahe ganz zu vertreten, und ausschlie"slich ganze einzelne Steine eines und desselben Meteors, und selbst ganze Meteor-Massen zu bilden.\footnote{\frakfamily{Wie dies bei den merkw"urdigen Steinen von Erxleben und Chassigny der Fall ist, die sich eben dadurch von allen bisher bekannten Meteor-Steinen so sehr unterscheiden, dass au"ser den zart eingesprengten Metallteilchen in dem einen, ersteren, auch gar keine "Ahnlichkeit mit irgend einem andern bekannten Meteor-Steine nachzuweisen w"are, wenn nicht doch hie und da in einem oder dem andern die ausgezeichnete Masse jener Steine, aus der ihr Ganzes besteht, wenigstens als einzelner Gemengteil erschiene. Und so auffallend demnach, sowohl nach den Resultaten der von mir neuerlichst vorgenommenen mikroskopischen Untersuchung der Massenteilchen, als noch mehr nach jenen der chemischen Analyse der Steinmasse beider (nach Klaproth und Stromeyer von dem einen, nach Vauquelin vom andern) einerseits die "Ahnlichkeit im Wesentlichen der Beschaffenheit und des Gehaltes mit allen "ubrigen Meteor-Steinen ist; noch umso mehr auffallend ist wohl andererseits nach denselben die ganz besondere "Ubereinstimmung hierin gerade zwischen diesen beiden Steinen, da sie doch unter sich, nach allen "au"sern und physischen Merkmahlen (das spezifische Gewicht allein ausgenommen, welches bei beiden ziemlich gleich ist, = 3,600 nach Klaproth bei jenem von Erxleben, und = 3,550 nach eigener Wiegung, bei jenem von Chassigny, obgleich dieser keine Spur weder von Gediegeneisen, noch von Kies oder Schwefeleisen zeigt, die beide in jenem h"aufig vorhanden sind), beinahe noch mehr als von allen andern Meteor-Steinen abweichen. (Inzwischen gerade nicht mehr als ihre beiderseitige Masse zu tun pflegt, wenn sie als isolierter Gemengteil, einzeln oder vereint, in einem andern Meteor-Steine vorkommt.)}}

Selten sind diese Massen bedeutend gro"s, und ebenso selten ganz unf"ormlich oder vieleckig gestaltet; gew"ohnlich, zumal bei h"oheren Graden von Dichtheit und bedeutender Intensit"at von Farbe, sind sie nur klein, h"ochstens von einigen Linien im gr"o"sten Durchmesser, und dann meistens ziemlich spitzeckig und scharfkantig, ungleichseitig dreieckig, rhomboidal und trapezoidal, oder scharf gerandet und oval, oder mehr oder weniger zugerundet; am h"aufigsten aber und zwar, obgleich gerade nicht immer im Verh"altnisse mit der Dichtheit und Farbe, doch stets bei den h"ochsten Graden derselben, und vorzugsweise bei den gr"unen Farben-Tinten, sehr und selbst "au"serst klein, und vollkommen zirkelrund.

Im letzteren Falle, zumal wenn der Koh"asions-Zustand der Grundmasse an und f"ur sich nicht sehr bedeutend ist, ist der Aggregats-Zustand zwischen diesem Gemengteil und jener so locker, dass diese Massen, umso mehr, wenn sie vollkommen kugelicht sind, beim Zerbrechen oder Zerschlagen des Steines (wo sie sonst, bei minder vollkommener Ausscheidung und festerem Zusammenhalte der Steinmasse, mitbrechen oder halbkugelicht "uber die Bruchfl"ache vorragend, sitzen bleiben) teils von selbst aus der Grundmasse herausfallen, teils mit leichter M"uhe aus derselben heraus gebrochen werden k"onnen, und dann, ihrem Volum und ihrer Form entsprechende Gruben (runde Zellen, wie der Olivin im sibirischen Eisen), deren Boden und W"ande verdichtet, und gleichsam abgegl"attet sind, und wahren Absonderungsstellen gleichen, zur"uck lassen, so dass es wirklich das Ansehen hat, als w"aren diese Kugeln in die "ubrige Masse eingeknetet worden.\footnote{\frakfamily{Ich kann nicht umhin, hier auf eine ganz "ahnliche Bildung und Absonderung, gleichzeitig entstandener und gleichartiger, oder doch nur wenig ver"anderter Massen terrestrischer Fossilien hinzuweisen, n"amlich auf jene, in dieser Beziehung h"ochst merkw"urdigen, kugelichten Basalte, Thon- und Klingstein-Porphyre, welche, zumal letztere, im Innern ihrer Grundmasse "ahnliche, oft vollkommen sph"arische Kugeln, von 4 bis 5 Zoll in Durchmesser, von vollkommen homogener Natur, nur etwas in der Farbe ver"andert, und von gr"o"serer Dichtheit und Feinheit im Korne als die Hauptmasse, eben so fest eingeschlossen, oder mehr oder weniger scharf abgesondert, und nicht selten eben es vollkommen ausgeschieden und lose, mit gegl"atteter Oberfl"ache und verdichteten W"anden der Gruben, eingeschlossen enthalten.}} Die Kugeln selbst sind in diesem Zustande meistens vollkommen sph"arisch, und haben eine mehr oder weniger dunkle, gr"unlich oder br"aunlich-graue Farbe, einen schwachen, etwas fettigen, meistens nur schimmernden Glanz, und eine sehr glatte Oberfl"ache, indes sie sonst, auf niederer Stufe von Ausbildung und Ausscheidung, wenn sie auch aus der Grundmasse hervorragen, mehr uneben und gleich dieser gef"arbt, ganz matt und rau sind, indem sie von Massenteilchen derselben, die innig mit ihrer Oberfl"ache zusammen hangen, bedeckt erscheinen. Nach den verschiedenen, sehr mannigfaltigen und sehr abweichenden Graden von Dichtheit und Festigkeit, sind die Massen dieses Gemengteiles, mehr oder weniger, leicht zersprengbar, aber nie zerreiblich, im Gegenteile nicht selten ziemlich schwer zersprengbar, und in dem Ma"se, als dieselben dadurch und durch die "ubrigen Eigenschaften von der Beschaffenheit der Grundmasse sich unterscheiden, und vollkommen ausgeschieden erscheinen, zeigt sich der Bruch, der im unvollkommensten Zustande noch rau und erdig, doch immer dichter ist als jener der Grundmasse, immer feiner, dichter, ebener, glatter, und geht endlich in einen vollkommen dichten, flachmuschlichen "uber. Sie zerspringen nach allen Richtungen (und erscheinen auch so von selbst, oft in viele kleine St"ucke, zersprungen auf geschliffenen Fl"achen) in unbestimmt eckige, ziemlich scharfkantige, meistens ganz undurchsichtige, nicht selten aber auch mehr oder weniger an den Kanten durchscheinende, bisweilen ganz durchscheinende, und, obgleich nur selten und einzeln, selbst ganz durchsichtige Bruchst"ucke, von einem schwachen, etwas fettigen Glanze, der sich mit zunehmender Durchscheinenheit, zumal bei lichtern, gr"unlichen und gelblichen Farben, immer mehr und mehr dem Glasglanze n"ahert; und in diesem Zustande geben dergleichen Bruchst"ucke nicht nur ziemlich leicht Funken am Stahle, sondern ritzen auch etwas das gemeine Glas.\footnote{\frakfamily{Es wollen Manche an Massen dieses Gemengteiles in Meteor-Steinen --- so wie an der olivinartigen Substanz im sibirischen Eisen --- (wovon seines Ortes die Rede sein wird) wo nicht eine vollkommene und ausgesprochene Kristall-Form, doch wenigstens einzelne, wahre Kristallisation-Fl"achen beobachtet haben. (So Calmelet und Gillet de Laumont, eine prismatische Form mit rhomboidaler Grundlage, die sogar ganz mit einer Ab"anderung aus der Kristall-Suite des Augits (\emph{Pyroxene H.}) "ubereinstimmen soll, in einem Steine von Chassigny; so Chladni etwas Kristall"ahnliches, als ein regelm"a"siges Parallelogramm, in einem Steine von Siena, und Kristallisation-Fl"achen an einer bedeutend gro"sen Masse dieses Gemengteiles in seinem Bruchst"uckchen vom Steine von Eggenfeld.) Ich habe mich von der Gr"undlichkeit dieser Angaben noch nicht vollkommen "uberzeugen k"onnen, und was ich zur Zeit von solchen angeblichen Kristall-Formen und angenommenen Kristallisation-Fl"achen (namentlich beim sibirischen Eisen) gesehen habe, kann ich vor der Hand blo"s als Absonderungsfl"achen erkennen.}}

Nach den verschiedenen Graden von Zersprengbarkeit lassen sich die Massen dieses Gemengteiles auch mehr oder weniger leicht, nie aber so leicht wie die Grundmasse, im Gegenteile meistens schwer, und gew"ohnlich sehr schwer, oft nur auf einem Ambosse, zu Pulver sto"sen, und nur selten, und dann erst, wenn schon sehr verkleinert, vollends zerreiben. Die Massenteilchen erscheinen unter dem Mikroskope, nach der verschiedenen Beschaffenheit, die sie urspr"unglich in ihrem Zusammenhange, in allen obigen vielseitigen Beziehungen, von Farbe, Durchscheinenheit u. s. w. zeigten, h"ochst mannigfaltig, doch zeigen sie, solcher Gestalt verkleinert und einzeln, immer lichtere und fast durchaus mehr ins Gr"unliche ziehende Farben, und mit diesen einen h"oheren Grad von Durchscheinenheit und scharfkantigere Bruchfl"achen, alles aber im Gro"sen in einem geringeren Grade als die oben beschriebenen Massenteilchen der Grundmasse, zumal als jene der mehr kristallinischen ersterer Art, von denen sie sich "ubrigens noch durch ein minder kristallinisches Ansehen und durch einen mehr fettigen Glanz unterscheiden, "ubrigens aber, und zwar durch die halbkristallinischen Massenteilchen zweiter Art der Grundmasse, in dieselben "uberzugehen, oder aus denselben hervor gegangen zu sein scheinen. Sie zeigen "ubrigens, sowohl in diesem als im konkreten Zustande, eben so wenig als jene, wenn nicht durch zuf"allig eingemengte Metallteilchen vermittelt, die geringste Wirkung auf die Magnetnadel.

Alle obigen, so mannigfaltigen Verschiedenheiten im Ansehen, Verhalten und Vorkommen, so wie das so sehr abweichende quantitative Verh"altnis dieses Gemengteiles, scheinen in keinem absoluten Wechselverh"altnisse mit oder in einer direkten Abh"angigkeit von der physischen Beschaffenheit der Grundmasse (von der Dichtheit, Farbe u. s. w. derselben) zu stehen; wohl aber scheint das quantitative Verh"altnis der entfernteren Bestandteile (zumal der Talk- und Kieselerde) der Steinmasse im Ganzen, darauf einigen Einfluss zu haben\footnote{\frakfamily{Bei allen Meteor-Steinen, bei welchen dieser Gemengteil h"aufiger, auch wohl nur deutlicher ausgesprochen, oder in einem vollkommeneren Zustande vorkommt (wie bei jenen von Eichst"adt, Tabor, Benares, Eggenfeld), scheint (insofern auf alle Analysen in dieser Beziehung anzugehen ist) die Talkerde in einem gr"o"seren Verh"altnisse = 17 bis 23 Perzent vorhanden zu sein. Am auffallendsten ist dies bei den Steinen von Erxleben und Chassigny, deren ganze Masse aus diesem Gemengteil, in einem ziemlich ausgesprochenen Zustande, besteht, und von welchen der Gehalt an Talkerde mit 23,58 bis 26,50 und 32 Perzent ausgewiesen wird. Es ist zwar von manchen noch der Gehalt als bedeutend (so von jenen von Apt mit 14, von Lissa mit 22, von Yorkshire mit 24?) angegeben, wo doch dieser Gemengteil \emph{en masse} nur selten und schwach ausgesprochen erscheint. Allein hier mag es an der Unvollkommenheit der Ausscheidung, und an der innigeren Verbindung der Massenteilchen liegen, welche letztere dieses auch (wenigstens bei den Steinen von Lissa) bew"ahren. Der sehr abweichende Gehalt dieses Gemengteiles sowohl als "uberhaupt der ganzen Steinmasse, an Eisen, und wohl auch der veschiedene Zustand, in welchem sich dasselbe in beiden befindet, durfte vielleicht den wesentlichsten Einfluss auf die meisten Zustandsverschiedenheiten haben.}}; das Meiste d"urfte jedoch wohl von besonderen Zustandsver"anderungen der Steinmasse im Ganzen abh"angen.\footnote{\frakfamily{Diess scheinen wohl jene in obiger Note ber"uhrten F"alle, wo die Ausscheidung und der Zustand dieses Gemengteiles dem quantitativen Verh"altnisse der Bestandteile der Steinmasse nicht entspricht, und "uberhaupt die so mannigfaltigen Zustandsverschiedenheiten desselben, die oft weder mit dem Gehalte, noch mit der Beschaffenheit der Steinmasse in irgendeinem Kausal-Verh"altnisse stehen, insbesondere aber die Steine von Erxleben und Chassigny, zu best"atigen.}} Sehr merkw"urdig aber ist, dass dieser Gemengteil, sollte er auch in einem noch so geringen Verh"altnisse vorhanden sein, in einem und demselben Steine sich h"ochst selten, wenn je, durchaus von ganz einerlei Beschaffenheit findet, abgesehen selbst von Form und Gr"o"se; dass er im Gegenteile gew"ohnlich, selbst in einem und demselben Bruchst"ucke eines Steines, sollte dieses auch nur ein paar Zoll Oberfl"ache bieten, wenigstens in zwei oder drei, oft aber in noch mehreren, und nicht gar selten in einer ganzen Suite von Zustandsver"anderungen in allen oben angef"uhrten Beziehungen erscheint: vom unvollkommensten, kaum von der Grundmasse unterscheidbaren Zustande, bis zum vollkommenst ausgebildeten, scharf geschiedenen, vollkommen glasartigen; und nicht minder merkw"urdig ist es, dass er sich ebenso und oft in einzelnen Zustandsverschiedenheiten, ganz ausnehmend "ahnlich, bei, nach Zeit und Ort des Niederfallens, sehr verschiedenen, "ubrigens im Ganzen oder in andern Beziehungen mehr oder minder sich "ahnlichen, Steinen zeigt, und solcher Gestalt einerseits die Unterscheidung solcher, sich oft ganz "ahnlicher Steine oder Bruchst"ucke verschiedener Abkunft --- die sonst durch ihn, gerade der vielen Modifikationen wegen, in welchen er vorkommen kann, am leichtesten w"are --- sehr schwer und unsicher macht; andererseits aber einen und oft ausschlie"slichen Anhaltspunkt zur Wiedererkennung und Nachweisung einer Analogie zwischen sonst gar sehr heterogen scheinenden Massen darbietet; so wie er denn auch die Homogenit"at der Materie, die Gleichf"ormigkeit des Bildungs-Prozesses und die Allgemeinheit der Herkunft aller dieser Massen bew"ahrt, und den vorz"uglichsten Charakter der nat"urlichen Versippung derselben begr"undet. Und so wie einerseits diese mannigfaltigen Modifikationen und die unverkennbaren "Uberg"ange derselben in einem und demselben Bruchst"ucke, so wie die "Ubereinstimmung darin in verschiedenen, der Grundmasse und allen Beziehungen nach oft sehr voneinander abweichenden Steinen, und das allm"ahliche, oft kaum erkennbare Hervortreten dieses Gemengteiles aus der Grundmasse --- die Homogenit"at desselben mit dieser bew"ahren, welche auch die Analyse best"atiget,\footnote{\frakfamily{Wie bereits in einer fr"uheren Note gezeigt worden ist.}} und auf eine blo"se Zustandsver"anderung der Masse, durch welche diese Umbildung oder Ausscheidung in verschiedenen Graden bewirkt wird, schlie"sen lassen; so scheint wohl andererseits auch aus denselben, so wie aus der Suite der oryktognostischen Merkmahle,\footnote{\frakfamily{Gef"uge, Festigkeit, Harte, Bruch, Bruchstucke, Durchscheinenheit, Glanz, und vollends die Farbenreihe, die, wie vorz"uglich die Massenteilchen zeigen, Gr"un immer zum Typus hat, welche den mannigfaltigen Zustandsverschiedenheiten und ihren allm"ahlichen "Ubergangen entsprechen.}} und den Resultaten der physischen\footnote{\frakfamily{Das spezifische Gewicht kann der Kleinheit der Massen wegen nicht wohl bestimmt werden, auch muss dasselbe nach den verschiedenen Zustandsver"anderungen notwendig abweichen, und nach dem sehr abweichenden Gehalte an verlarvtem sowohl, als selbst an mechanisch eingemengtem metallischen Eisen (der bei diesem Gemengteile in den Meteor-Steinen gew"ohnlich ungleich gro"ser ist, als bei der olivinartigen Substanz im sibirischen Eisen) sehr verschieden sein. Das spezifische Gewicht der olivinartigen Substanz im sibirischen Eisen (= 3,263 bis 3,3 nach Bournon) stimmt aber ganz genau mit jenem des terrestrischen Olivins "uberein (= 3,225 nach Werner; 3,265 nach Klaproth). Die Schmelzbarkeit, die Graf Bournon mit einem K"ugelchen aus einem Steine von Benares erprobte (wo dieser Gemengteil zwar besonders ausgeschieden, aber eben in keinem hohen Grade von Ausbildung vorkommt), ist ebenso schwer, wie die der olivinartigen Substanz im sibirischen Eisen und die des terrestrischen Olivins.}} und chemischen\footnote{\frakfamily{Insofern die Zustandsverschiedenheiten dieses Gemengteiles von dem Mischungsverh"altnisse abh"angen, insofern mag wohl auch dieses sehr mannigfaltig sein, inzwischen wich dasselbe nach Howards Analyse bei einer Masse der Art aus einem Steine von Benares nur h"ochst unbedeutend von jenem ab, welches er bei Zerlegung der olivinartigen Substanz aus dem sibirischen Eisen erhielt, und zwar --- wohl zu bemerken --- selbst weniger, trotz der Verschiedenheit beider Massen im "Au"sern, als das von Klaproth bei derselben Substanz gefundene von dem seinigen. (Howard erhielt n"amlich aus dem kugelichten Gemengteile des Steines von Benares 50 Perzent Kiesel- und 15 Perzent Talkerde, und aus der olivinartigen Substanz des sibirischen Eisens 54 Perzent Kiesel- und 26 Perzent Talkerde; Klaproth dagegen aus derselben Substanz von ersterer 41, von letzterer aber 38 Perzent. Den Hauptunterschied macht der Gehalt an Eisen, wovon Howard aus der kugelichten Masse 34 Perzent, aus letzterer nur 16, und Klaproth 18 Perzent erhielt.) Und noch unbedeutender ist die Abweichung im Mischungsverh"altnisse zwischen dieser und dem terrestrischen Olivin (in welchem die Kieselerde 50 bis 52, die Talkerde 37 bis 38, und das Eisen 10 bis 12 Perzent betragt); auffallend dagegen die nahe "Ubereinstimmung darin zwischen allen drei Substanzen und der Gesamtmasse der Steine von Erxleben und Chassigny. (Klaproth erhielt aus ersterem --- nebst etwas Kalk- und Alaunerde, Nickel, Mangan, Chrom und Schwefel --- $\mathfrak{35\frac{1}{2}}$ Perzent Kiesel- und $\mathfrak{26\frac{1}{2}}$ Perzent Talkerde und 31 Perzent regulinisches Eisen; Stromeyer aus demselben --- nebst den gleichen Nebenbestandteilen und $\mathfrak{\frac{3}{4}}$ Perzent Natrum --- $\mathfrak{36\frac{1}{3}}$ Perzent Kiesel- $\mathfrak{23\frac{1}{2}}$ Perzent Talkerde und $\mathfrak{24\frac{1}{2}}$ Perzent metallisches und $\mathfrak{5\frac{1}{2}}$ Perzent oxydulirtes Eisen. Vauquelin fand im letzteren --- ohne Nebenbestandteile, au"ser 2 Perzent Chrom --- 33 Perzent Kiesel- 32 Perzent Talkerde und 31 Perzent Eisenoxyd.)}} Untersuchungen, die vollkommenste Identit"at dieses Gemengteiles, trotz dessen anscheinender Verschiedenheit, nach den verschiedenen Graden seiner stufenweisen Ausbildung, nicht nur in allen eigentlichen Meteor-Steinen, sondern auch mit der olivinartigen Substanz\footnote{\frakfamily{Schon Graf Bournon hat auf diese Identit"at aufmerksam gemacht. Und so wie einzelne Massen dieses Gemengteiles in den Meteor-Steinen Zustandsverschiedenheiten zeigen, die ganz vollkommen und in allen Beziehungen jenen der olivinartigen Substanz im sibirischen Eisen entsprechen --- wobei bemerkenswert ist, dass solche oft in Steinen vorkommen, wo dieser Gemengteil im Allgemeinen gerade nicht am vollkommensten ausgesprochen ist (wie z. B. in jenen von Siena und Eggenfeld, in welchen Bournon und Chladni auch vollkommen durchsichtige, glasartige, gelblich-gr"une Massen desselben beobachteten) --- ebenso finden sich bei dieser (wie an seinem Orte erw"ahnt werden wird) Zustandsverschiedenheiten und "Uberg"ange, die sich in manche jenes Gemengteiles verlaufen. Nur ist das Verh"altnis gerade entgegengesetzt, und Zustandsverschiedenheiten, die hier am h"aufigsten vorkommen, sind dort am seltensten, und umgekehrt.}} im sibirischen Eisen, hervor zu gehen, und man kann demnach wohl ohne Anstand diesen Gemengteil, von welcher Beschaffenheit er auch immer in den Meteor-Steinen erscheinen mag, insofern er nur in einer der ihm zukommenden Eigenschaften von der Grundmasse sich unterscheidet und erkennbar ausgeschieden erscheint (\emph{a potiori}) mit gleichem Namen bezeichnen.\footnote{\frakfamily{Kugelicht kann man ihn im Allgemeinen nicht wohl nennen, da er bei weitem nicht immer, kaum vorherrschend, in dieser Form vorkommt.}}

Hinsichtlich der Steinmasse im Ganzen modifiziert dieser Gemengteil, nach seinem verschiedenen quantitativen Verh"altnisse, nach dem Grade seiner Ausbildung, der Art seiner Ausscheidung und seines Zusammenhanges mit der Grundmasse, und nach seinen so mannigfaltig abweichenden Eigenschaften, nicht nur oft den Koh"asions- und Aggregats-Zustand, sondern bestimmt auch damit und durch die Form und Begrenzung seiner einzelnen Massen, das Gef"uge und "au"sere Ansehen derselben, welches, wie sich am deutlichsten auf geschliffenen Fl"achen ausspricht, wo derselbe nach dem Grade seiner Dichtheit und Festigkeit eine bedeutende Politur annimmt, bald Granit- oder Porphyrartig, bald\footnote{\frakfamily{Aber nur beziehungsweise, der scheinbaren Einknetung wegen.}} Breccie- oder ganz vollkommen Mandelsteinartig, bald Marmorartig erscheint.

Von den beiden metallischen Gemengteilen erscheint der eine, und zwar auf frischen, rohen Bruchfl"achen der Steinmasse, mehr oder weniger h"aufig, und mehr oder minder deutlich ausgesprochen, dem Gesicht und Gef"uhl erkennbar, in Gestalt einzelner, hervorragender, gr"o"serer oder kleinerer, mehr oder weniger rundlichter und glatter, oder eckiger, rauer K"orner, oder ebenso beschaffener, gr"oberer oder feinerer Zacken, die zum Teil mit anklebenden erdigen Massenteilchen bedeckt, fest von der Masse eingeschlossen, innig mit ihr verbunden und gleichsam verwachsen sind, und von mehr oder weniger licht eisen- oder stahlgrauer Farbe und metallischem, obgleich meistens nur schwachem Glanze.

Geritzt geben diese K"orner oder Zacken die Geschmeidigkeit und Weichheit der Materie zu erkennen, und dabei einen stark gl"anzenden, lichtern, ins Silberwei"se ziehenden Strich.

Mit Gewalt aus der Masse gebrochen, worin sie bei weitem gr"o"sten Teils ohne Verbindung unter sich eingeschlossen, bisweilen aber doch durch feine "Aste einiger Ma"sen mit einander verbunden (wie z. B. in den Steinen von Eichst"adt, Timochin, Tabor) zu sein scheinen, lassen sie sich auf einem kleinen Amosse sehr leicht --- obgleich nicht immer gleich, oder wenigstens nicht gleichf"ormig leicht --- \footnote{\frakfamily{Gew"ohnlich blieb, zumal von jenen mikroskopischen, ganz eingeh"ullten Eisenk"ornern, von welchen bei den Massenteilchen der Grundmasse die Rede war, ein kleiner Eindruck auf meinem st"ahlernen Ambosse (ein Ingredienz von Dumotiezs \emph{Nécessaire minéralogique}) zur"uck. Wahrscheinlich r"uhrt diese partielle Spr"odigkeit und H"arte (die "ubrigens allem Meteor-Eisen, auch in den derbsten Massen, aus demselben Grunde --- wie seines Ortes gezeigt werden wird --- eigen ist) von mikroskopisch bei- oder eingemengtem Schwefeleisen her (welche Vermutung hier, so wie dort, wo sie noch durch "uberzeugendere Grund unterst"utzt werden kann, die etwas schwerere, wenigstens ungleichf"ormig leichte, Schmiedbarkeit, und die etwas schwierige Schwei"sbarkeit, so wie die Entwicklung von Schwefel-Wasserstoffgas bei der Aufl"osung dieses Eisens in S"auren, best"atigen). Vielleicht zum Teil auch von dessen Oxydation oder Verbindung mit Chrom; denn die Verbindung mit Nickel scheint demselben vielmehr den h"oheren Grad von Weichheit, Z"ahigkeit und Dehnbarkeit zu geben, worin derbe Massen im Ganzen jedes Schmiede-Eisen "ubertreffen. Auch Klaproth fand bei seinen Analysen dem, mittelst des Magnets ausgezogenen, Gediegeneisen, immer sehr viel Schwefelkies fest adh"arierend.}}, ohne zu rei"sen oder zu springen, zu den d"unnsten Bl"attchen strecken, fletschen, deren meistens sehr gezackter Rand die urspr"unglich uneben und zackig gewesene Oberfl"ache und Gestalt des Eisenteilchens bezeichnet. Bey diesem Fletschen springen nicht nur die fest angeklebt gewesenen erdigen Massenteilchen ab,\footnote{\frakfamily{Diese scheinen oft mehr blo"s oberfl"achlich anzukleben, und aus Vaquelins und Klaproths Beschreibung ihres Verfahrens bei den von ihnen vorgenommenen Analysen (indem sie gew"ohnlich bei der Aufl"osung des aus der gepulverten Meteor-Steinmasse mit dem Magnete ausgezogenen, und sorgf"altig von allen erdigen Teilchen gereinigten Eisens, noch 10 bis 20 Perzent erdige Bestandteile erhielten, und wie ersterer bei dem Steine von Charsonville ausdr"ucklich bemerkt, es sehr schwer h"alt, das Gediegeneisen ganz von der Talkerde zu reinigen), und vollends aus Laugiers neuester Zerlegung des sibirischen Eisens (nach welcher dieses, von allen erdigen Teilchen mechanisch vollkommen gereiniget, 16 Perzent Kiesel- und 15 Perzent Talkerde enthalten soll), scheint hervor zu gehen, dass die erdigen Bestandteile als Metalle oder Metalloide mit dem metallischen Eisen in irgend einem Verh"altnisse chemisch verbunden sind. Eine Mutma"sung, die durch das auffallend geringe spezifische Gewicht, durch die, wie scheint, schw"achere Wirkung auf den Magnet, und durch die Grunde, welche Klaproth bestimmt haben, alles in den Meteor-Steinen vorkommende Metall als regulinisch anzunehmen, wovon in der Folge die Rede sein wird, noch mehr Gewicht erhalt. Bekanntlich erhielt auch Daniell bei Untersuchung des Gusseisens, und Berzelius bei der Analyse eines gemeinen Schwefelkieses, Kieselerde (Silicium).}} sondern es zeigt sich gew"ohnlich auch ein schwarzes Pulver, das mehr oder weniger dem Magnete folgt. (H"ochst wahrscheinlich Eisen-Oxydul oder Schwefeleisen, welches letztere, wie seines Ortes gezeigt werden wird, nicht nur dem sibirischen, sondern selbst den dichtesten und derbsten Meteor-Eisenmassen h"aufig eingemengt ist.) Es zeigen sich "ubrigens jene K"orner, Zacken und gefletschten Bl"attchen sehr wirksam (doch wie mir d"aucht gefunden zu haben,\footnote{\frakfamily{Ich will dies vor der Hand noch nicht als ausgemacht behaupten, bis ich im Stande bin, durch genauere Versuche, die eine eigene Vorrichtung notwendig machen, die Beobachtung zu bewahren.}} etwas schw"acher als gew"ohnliches weiches Eisen und selbst als derbes Meteor-Eisen) auf die Magnetnadel, und bew"ahren sich durch alle diese Eigenschaften, so wie durch die Resultate der Analyse, als regulinisches Eisen.\footnote{\frakfamily{Und zwar stets mit Nickel legiert, so dass diese Verbindung als charakteristisch f"ur alles Meteor-Eisen im regulinischen Zustande angenommen wird, und es daher sehr befremdend w"are, wenn die Steine von Agen, nach Vauquelin, da sie doch sichtlich bedeutend viel Gediegeneisen f"uhren, keine Spur von jenem Metalle enthielten. Nach den neuesten, zum Teil absichtlich in dieser Beziehung vorgenommenen Analysen Stromeyers, scheint das Mischungsverh"altnis dieser Metall in den verschiedenen Meteor-Steinen und Eisenmassen ziemlich gleichf"ormig, n"amlich in ersteren von 7 bis 10, in letzteren zwischen 10 und 11 Perzent des Nickels zum Eisen, und im Allgemeinen jenes des Nickels bedeutend h"oher zu sein, als es bisher von den meisten Analytikern angegeben wurde. (So hatte Klaproth in der Total-Masse des Steines von Erxleben nur $\mathfrak{\frac{1}{4}}$ Perzent Nickel gefunden, indes Stromeyer $\mathfrak{1\frac{1}{2}}$ Perzent fand, und ersterer in der Masse eines Steines von Timochin, bei einem, selbst ausgewiesenen, Gehalte von $\mathfrak{17\frac{1}{2}}$ Gran regulinisch vorhanden gewesenen Eisens, kaum $\mathfrak{\frac{1}{2}}$ Perzent; so fand derselbe im sibirischen Eisen nur $\mathfrak{1\frac{1}{2}}$, im mexikanischen $\mathfrak{3\frac{1}{4}}$, im Elbogner $\mathfrak{2\frac{1}{2}}$, im Agramer $\mathfrak{3\frac{1}{2}}$ Perzent Nickel. Howard, Vauquelin und N. A. Scherer geben bei den von ihnen vorgenommenen Analysen h"ochst ungleichf"ormige, zum Teil viel zu gro"s, zum Teil viel zu gering scheinende Verh"altnisse von diesem Metalle an.) Dass, wie Stromeyer meint, Nickel mit Eisen, als Oxyd, auch in den erdigen Gemengteilen chemisch enthalten sei, ist deshalb, im Allgemeinen wenigstens unwahrscheinlich, weil Vauquelin in den Steinen von Chassigny, und Moser und Klaproth in jenen von Stannern durchaus keine Spur davon auffinden konnten, jene aber, welche Vauquelin in letzteren bemerkte, wohl in dem vorhandenen Schwefeleisen, und der Gehalt, den Howard davon in den abgesondert zerlegten erdigen Gemengteilen der Steine von Benares fand, ohne Zweifel in den mikroskopisch eingemengten Gediegeneisen- und Kiesteilchen enthalten gewesen sein d"urfte.}}

Das spezifische Gewicht dieses Eisens ist bedeutend geringer als jenes vom Roh- und Stabeisen sowohl, als insbesondere vom derben Meteor-Eisen.\footnote{\frakfamily{Von ersteren kann man im Durchschnitt wohl 7,2 bis 7,7 annehmen, von letzteren fand ich dasselbe, und namentlich vom mexikanischen, kroatischen, b"ohmischen, ungarischen und peruanischen Eisen zwischen 7,600 und 7,830. H"ochst merkw"urdig ist, dass jenes vom sibirischen Eisen gleichsam das Mittel zwischen letzteren und jenem des Eisens aus Meteor-Steinen halt; ich fand dasselbe = 7,540, nur etwas geringer als Karsten, der es mit 7,573, wie es auch Werner und Hausmann annahm, angibt. Graf Bournon gibt es mit 6,487 an, diesem mochte aber wohl eine Irrung zum Grunde liegen.}} Graf Bournon fand es bei jenem aus einem Steine von Tabor = 6,146, und ich bei einem gro"sen Korne und einem Bl"attchen aus einem Steine von L'Aigle zwischen 6,00 und 6,60.

Auf geschliffenen und polierten Fl"achen erscheint dieser Gemengteil noch ungleich deutlicher, da durch Schnitt und Politur die kleinsten Metallteilchen rein und spiegelicht gl"anzend zur Ansicht kommen. Er zeigt sich hier nun erst in seiner wahren Menge,\footnote{\frakfamily{Es ist sehr zu beklagen, dass die Analytiker bisher so wenig R"ucksicht auf den Gehalt der Meteor-Steine an mechanisch eingemengtem Gediegeneisen genommen, und denselben gew"ohnlich nur im Ganzen, bald, wie Vauquelin alles als Oxyd, wie es aus der ganzen Masse durch die Operation erhalten wurde, bald, wie Klaproth alles als regulinisch, nach Kalk"ul, angegeben haben; so dass man weder von dem Zustande, in welchem sich das Eisen in der Steinmasse befand, ob ganz rein und gediegen, oder mehr oder weniger mit Schwefel vererzt, als Kies- oder Schwefeleisen, oder mehr oder weniger mit Oxygen verbunden, als Oxyd oder Oxydul, noch weniger von den gegenseitigen quantitativen Verh"altnissen etwas erfahrt, wie dies bis jetzt beinahe nur aus Stromeyers musterhaften, leider nur wenigen Analysen, zu ersehen ist. Es w"are zu w"unschen, dass sie jedes Mahl das mechanisch eingemengte Gediegeneisen, so genau und rein wie m"oglich, aus der fein gepulverten Steinmasse von bestimmtem Gewichte mittelst einer Magnetnadel ausziehen, und dieses f"ur sich angeben und untersuchen m"ochten.}} die gew"ohnlich nicht gering ist, so dass er nach einer beil"aufigen, oberfl"achlichen Absch"atzung bisweilen $\mathfrak{\frac{1}{3}}$ oder $\mathfrak{\frac{1}{5}}$, d. i. 0,20 bis 0,30 (wie z. B. in den Steinen von Eichst"adt, Timochin, Tabor, Charsonville \emph{zc.}) der ganzen Masse betr"agt, meistens aber doch nur den zehnten oder zwanzigsten Teil des Ganzen ausmachen m"ochte, d. i. 0,10 bis 0,05 (wie in den Steinen von L'Aigle, Lissa \emph{zc.}), und oft auch in "au"serst geringer Menge, so dass er kaum $\mathfrak{\frac{1}{50}}$ der Masse betr"agt, = 0,02 (wie in den Steinen von Mauerkirchen, Siena, Benares \emph{zc.}), ja selbst noch weniger (wie in den Steinen von Parma, Eggenfeld), zuletzt ganz und gar fehlt (wie in den Steinen von Chassigny und Stannern).\footnote{\frakfamily{Hinsichtlich des merkw"urdigen Wechselverh"altnisses, welches zwischen dem Gehalte der Meteor-Steine an solcher Gestalt mechanisch eingemengtem --- ganz ausgeschiedenem --- regulinischen Eisen und jenem an Eisen in mehr oder weniger geschwefeltem und oxydierten Zustande zu bestehen, und der ganz besonderen Verbindung, in welcher ersteres (?) mit den erdigen Bestandteilen der Steinmasse verbunden, vollkommen verlarvt, vorzukommen scheint, verweise ich auf die Bemerkungen bei der Abhandlung der "ubrigen Gemengteile (des Schwefeleisens und des Eisenoxydes).}}

Es zeigt sich derselbe hier teils, zumal wenn h"aufig vorhanden --- und in diesem Falle meistens ziemlich gleichf"ormig verteilt --- in zarten, "au"serst feinen, zum Teil mikroskopischen Punkten (den Ausg"angen senkrecht gegen die Oberfl"ache stehender Zacken), teils in gr"o"seren oder kleineren, st"arkeren oder schw"acheren, mehr oder weniger zahnigen oder zackigen, und klein- und fein"astigen, gebogenen, winkeligen Adern, Linien und Flecken (den Durchschnitten mehr oder weniger horizontal gegen die Oberfl"ache liegender Zacken), die bisweilen durch zarte Zweige, mehr oder minder vollkommen, einzeln wenigstens, mit einander verbunden sind\footnote{\frakfamily{Ein, obgleich nur noch h"ochst unvollkommener und sehr unterbrochener Zusammenhang, der aber doch schon einige "Ahnlichkeit mit dem Eisengerippe der sibirischen, zumal der s"achsischen und jener angeblich aus Norwegen stammenden Eisenmasse zeigt.}}; teils --- obgleich seltener, und meistens nur, wo der Gehalt im Ganzen geringe --- in einzelnen betr"achtlich gro"sen, rundlichten, ovalen, keilf"ormigen, mehr oder weniger dreieckigen, gew"ohnlich scharf begrenzten, und gar nicht zackigen Flecken von einigen Linien im Durchmesser (den Durchschnitten von gr"o"seren, gew"ohnlich so gestalteten, und meistens platt gedr"uckten K"ornern oder Massen, in welchen sich dieser Gemengteil, bisweilen von Erbsen- bis Haselnuss-Gr"o"se, und von 20 bis 30 und mehr Gran am Gewichte --- wie in den Steinen von Ensisheim, L'Aigle, Barbotan, Salés \emph{zc.} eingemengt findet.\footnote{\frakfamily{Auf solche, oft ganz im Innern der Steinmasse verborgen liegende, gr"o"sere Eisenmassen, und "uberhaupt auf das mechanisch eingemengte Gediegeneisen, wenn es im Ganzen nicht sehr h"aufig vorhanden ist, indem dasselbe sonst sehr ungleichm"a"sig verteilt zu sein pflegt, muss bei Bestimmung des spezifischen Gewichtes, so wie bei der Analyse eines Bruchst"uckes, besondere R"ucksicht genommen werden.}}

Auf blo"s geschnittenen, rohen, noch unpolierten Fl"achen zeigt sich die Farbe dieser Eisenteilchen --- die hier ihre Weichheit und Geschmeidigkeit durch Erhabenheit und durch Streifung ihrer Oberfl"ache (welche das nicht ganz gleichf"ormig vorr"uckende, schneidende Instrument, Rad oder S"age, bewirkte) bew"ahren, und bisweilen gek"ornt, k"ornig angeh"auft, fast wie getr"auft erscheinen --- mehr oder weniger licht eisen- stahl- oder zinkgrau, und der Glanz, rein metallisch zwar, aber etwas schwach; auf polierten Fl"achen dagegen zieht sich erstere mehr oder weniger ins Silberwei"se, und letzterer wird sehr stark und spiegelnd.

Diese Eisenteilchen kommen "ubrigens in beiden erdigen Gemengteilen eingestreut vor, in dem olivinartigen doch offenbar ungleich weniger und zarter, und, wie es beinahe scheint (namentlich bei den Steinen von Charsonville, Apt, Toulouse), um so sparsamer, je unvollkommener derselbe ausgesprochen, und je mehr "ahnlich er noch der Grundmasse ist\footnote{\frakfamily{Howard gibt inzwischen, da er doch beide erdige Gemengteile des Steines von Benares m"oglichst getrennt und f"ur sich analysierte, in beiden ein ganz gleiches Verh"altnis von Eisenoxyd an, n"amlich 34 Perzent.}}; aber, wie es auf der andern Seite scheint, zumal im vollkommeneren Zustande desselben (wie bei den Steinen von Eichst"adt, Timochin, Benares), mehr um ihn herum angeh"auft, die Massen desselben gleichsam umgebend, einschlie"send.\footnote{\frakfamily{Wieder eine Ann"aherung des Entwickelungszustandes der Steinmasse der Meteor-Steine und ihrer Gemengteile an die sibirische Eisenmasse.}} Am h"aufigsten m"ochten sie wohl oft, in dem ganz besonderen Zustande, fest in die offenbar ver"anderten Massenteilchen der Grundmasse (wie oben bei dieser erw"ahnt worden ist) eingeh"ullt vorkommen.

Von diesem metallischen Gemengteile h"angt, obgleich nicht ganz ausschlie"slich (da das eingemengte, und selbst, wie scheint, das chemisch mit den erdigen Teilchen verbundene, mehr oder weniger oxydierte Eisen, zumal aber das eingemengte Schwefeleisen, bisweilen einige Wirkung zeigen), die Wirkung der Meteor-Steine im Ganzen auf die Magnetnadel ab, die demnach nach dem so sehr abweichenden quantitativen Verh"altnisse desselben von dem ganz Unmerklichen bis ins sehr Starke geht, und jener des massiven Gediegeneisens sich n"ahert. Es modifiziert derselbe ferner, verm"oge seines verschiedenen quantitativen Verh"altnisses (wobei jedoch der Gehalt der Masse an Eisen-Oxyd --- an verh"ulltem oder gar verlarvtem Eisen --- und an Schwefeleisen zu ber"ucksichtigen kommt), nicht nur das spezifische Gewicht der verschiedenen Meteor-Steine, und selbst --- der oft sehr ungleichen, daher wohl zu ber"ucksichtigenden Verteilung und Einmengung wegen --- der Bruchst"ucke eines und desselben Steines, sondern auch insbesondere, und nach Ma"sgabe der Beschaffenheit der Teilchen (ob gr"ober oder feiner, glatter oder zackiger), durch mechanische Zusammenhaltung und durch eine (vielleicht erst in der Folge durch Oxydation in der Atmosph"are) vermittelte innigere Verbindung aller Gemengteile unter sich, die Dichtigkeit und Festigkeit, den Koh"asions- und Aggregats-Zustand der ganzen Masse.

Es ist demnach dieser Gemengteil, zumal derselbe in so vielseitigen Beziehungen, insbesondere im quantitativen Verh"altnisse, und in Gr"o"se, Form und Verbindung seiner einzelnen Massen, so auffallende Verschiedenheiten zeigt, f"ur die verschiedenen Meteor-Steine sehr charakterisierend, wie er denn auch das Ansehen derselben, zumal auf polierten Fl"achen, sehr mannigfaltig modifiziert, und manche oft ausschlie"slich dadurch erkennbar und unterscheidbar macht.

In Begleitung dieses Gemengteiles, und zwar wo nicht ausschlie"slich, doch vorzugsweise nur desselben,\footnote{\frakfamily{Die F"alle, wo dasselbe auch bei deutlich ausgesprochenen Schwefeleisen-Massen Statt hat, scheinen mir gr"o"sten Teils zweifelhaft. So viel ist gewiss, dass die Erscheinung verh"altnism"a"sig h"ochst selten ist bei Steinen, die wenig Gediegeneisen, und doch viel, selbst sehr viel Schwefeleisen enthalten, wie die von Parma, Mauerkirchen, Siena, Benares, und gar nicht, wo das Gediegeneisen ganz fehlt, wie bei jenen von Chassigny und Stannern, obgleich bei letzteren der Gehalt an Schwefeleisen nicht unbedeutend ist. Inzwischen glaubte Klaproth doch diese Erscheinung der Verwitterung der Kiespunkte (des fein eingesprengten Schwefeleisens) zuschreiben zu sollen.}} und wo nicht urspr"unglich, doch stets in der Folge, wenn die Steine einige Zeit der atmosph"arischen Luft ausgesetzt waren,\footnote{\frakfamily{Klaproth, der die Meinung hegte, dass die Meteor-Massen und ihre Gemengteile durchaus keiner Einwirkung des Oxygens ausgesetzt waren, als etwa der momentanen w"ahrend des schnellen Durchzuges durch unsere Atmosph"are --- vor der sie "ubrigens auch durch die blitzschnell erzeugte oberfl"achliche Rinde sogleich gesch"utzt wurden --- und daher durchaus keinen Oxydations-Zustand irgend eines Gemengteiles annehmen zu d"urfen glaubte, schrieb diese Erscheinung ausschlie"slich der sp"ateren Einwirkung der atmosph"arischen Luft zu; inzwischen scheinen, abgesehen von den leicht zu machenden Einw"urfen gegen jene vorgefasste Meinung im Allgemeinen, mehrere Beobachtungen auch gegen diese daher r"uhrende Folgerung zu sprechen. Mehrere ganze und durch vollkommene "Uberrindung vor dem Eindringen der atmosph"arischen Luft gesch"utzte, und dabei ziemlich dichte und kompakte Steine, die ich selbst zu zerschlagen Gelegenheit hatte, und f"unf verschiedene, von welchen w"ahrend meiner Anwesenheit in Paris betr"achtlich Stucke --- freilich nach Steinschneiderart, aber schnell und mit m"oglichster Verwahrung gegen Durchn"assung --- abgeschnitten wurden, zeigten im Innern ihrer Masse dieselbe Erscheinung. Dasselbe beobachtete Bergrath Reu"s an einem von ihm zerschlagenen Steine von Lissa, kaum noch drei Monat nach dem Falle. Dagegen zeigen Bruchst"ucke von mehreren, an Gediegeneisen sowohl als Schwefeleisen ziemlich reichhaltigen Steinen, die seit vielen Jahren der atmosph"arischen Luft, und selbst h"aufiger Betastung ausgesetzt, auch an einer Fl"ache angeschliffen und poliert worden waren, noch bis zur Stunde keine Spur davon.}} erscheinen auf unvollkommen "uberrindeten, urspr"unglich oder sp"aterhin zuf"allig von Rinde entbl"o"sten Fl"achen, zumal auf frischen Bruchfl"achen, und zwar nach Ma"sgabe der Menge, Gr"o"se und Gestalt der Eisenteilchen (vielleicht auch nach der individuellen Beschaffenheit derselben),\footnote{\frakfamily{Wie einige an Gediegen- und Schwefeleisen ziemlich reichhaltige Steine, z. B. die von Erxleben, Tipperary, Limerick, und zum Teil selbst einige von Lissa, zu beweisen scheinen, die kaum eine Spur zeigen.}} mehr oder weniger h"aufige, gr"o"sere oder kleinere, verschieden gestaltete Flecke, oft nur zarte Punkte, von matter, licht ockergelber, durch eine Reihe von Abstufungen ins Gelblich- und R"otlichbraune, bis ins Dunkelbraune verlaufender Farbe, wahre Rostflecke, die sowohl durch die verh"altnism"a"sige Menge, als zum Teil auch durch die Gestaltung und Gr"o"se, umso mehr ein charakteristisches Merkmal f"ur viele Meteor-Steine abgeben, als sie, zumal auf rohen, unpolierten Fl"achen, weit mehr als die Metallteilchen selbst auffallen, auf polierten Fl"achen aber der ganzen Steinmasse ein ausgezeichnetes, marmoriertes Ansehen geben, so wie sie wohl auch den Zusammenhalt und Koh"asions-Zustand derselben als vermittelndes Bindungsmittel zu verst"arken scheinen.

Es ist bemerkenswert, dass diese Rostflecke, wie es scheint, nie auf den, zumal polierten, Fl"achen der Eisenteilchen, auch wenn sie Jahre lang der Luft ausgesetzt waren --- wobei sie kaum etwas von ihrer Politur einb"u"sten --- (wie denn auch das Meteor-Eisen "uberhaupt, vielleicht wegen der Verbindung mit Nickel, nicht so leicht rostet, und auch mehr der Einwirkung der S"auren widersteht, als gemeines Eisen), sondern immer nur auf ihrer rauen Oberfl"ache und am Rande derselben, insbesondere aber in den erdigen, von mehreren Eisenteilchen eingeschlossenen Zwischenr"aumen,\footnote{\frakfamily{Es scheint demnach, dass es nicht die Gediegeneisenteilchen selbst sind, welche diese Oxydation erlitten haben, sondern vielmehr die Atome von Eisen-Oxydul und vielleicht von Schwefeleisen, welche jene einh"ullen.}} die sie oft ganz durchdringen, erscheinen, und dass sie oft, wie mir d"aucht, in Folge der Zeit, einen etwas fettigen Glanz, unvollkommen bl"atterige oder schalige Absonderungen, und ein Pflinz- oder Eisenspatartiges, bisweilen fast Harz"ahnliches Ansehen gewinnen.

Die Massenteilchen dieser Flecke, die bei manchen an Gediegeneisen sehr reichhaltigen Meteor-Steinen (wie z. B. bei jenen von Eichst"adt und Timochin) beinahe die gr"o"sere H"alfte der Grundmasse betreffen, zeigen sich unter dem Mikroskope teils als erdige, ockrige, gelbe, pommeranzen- und r"otlich-gelbe, zum Teil aber als spatartige, und dann gl"anzende oder schillernde, ins Dunkelgelbe und Rotbraune ziehende bis ins Glasige, und dann ins Rothe verlaufende, kleine, gleichsam zusammen gekittete Massen, die mit mikroskopisch zarten Metallteilchen gemengt sind, und zum Teil in jene Feldspatartigen Massenteilchen, welche bei der Grundmasse erw"ahnt worden sind, "ubergehen. Sie werden von der Magnetnadel lebhaft angezogen, und lassen sich "au"serst schwer, zum Teil gar nicht zersto"sen, halten selbst am Ambosse mehrere starke Schl"age aus, und geben dann ein mehr oder weniger feines, lichter oder dunkler gelbes oder r"otlich gelbes, erdiges Pulver, das zum Teil noch retractorisch ist, und ein oder mehrere Bl"attchen hartes und sehr z"ahes, metallisch gl"anzendes, licht eisengraues Gediegeneisen.\footnote{\frakfamily{Aus dieser Beschaffenheit der Massenteilchen scheint wohl hervor zu gehen, dass diese Rostflecke kein Erzeugnisse einer schnellen und oberfl"achlichen, und blo"s durch die atmosph"arische Luft bewirkten Oxydation des Gediegeneisens, und noch weniger die Folge einer blo"sen Verwitterung des Schwefeleisens sein k"onnen; dagegen geben vielmehr Ansehen, Glanz, H"arte, Spr"odigkeit, und die Eigenschaft im Wasser nicht merklich, oder doch nur zum Teil, die Farbe zu "andern, Veranlassung, dieselben mit Eisenspat zu vergleichen.}}

Der andere metallische Gemengteil, der wohl nie ganz fehlen m"ochte,\footnote{\frakfamily{Au"ser etwa bei den Steinen von Chassigny, wo sich durchaus nichts daf"ur zu erkennen gibt, und bei deren Analyse auch Vauquelin keine Spur von Schwefel finden konnte.}} obgleich er gew"ohnlich in einem ungleich geringeren, oft "au"serst geringen, und offenbar gerade mit Zunahme des vorhergehenden in einem abnehmenden Verh"altnisse\footnote{\frakfamily{Dieses merkw"urdige Wechselverh"altnis spricht sich bei den meisten Meteor-Steinen sehr auffallend aus. So findet man bei den an Gediegeneisen sehr reichhaltigen Steinen von Eichst"adt, Timochin, Tabor, Charsonville, kaum ein deutliches Korn von Schwefeleisen, und es ist dasselbe "au"serst zart eingesprengt; dagegen erscheint es bei den an jenem minder reichhaltigen von Ensisheim, Salés, Lissa \emph{zc.} schon weit mehr und in gr"o"seren Massen; bei den eisenarmen Steinen vollends von Siena, Mauerkirchen, Benares, und besonders Parma, und den ganz eisenfreien von Stannern auffallend h"aufig und in ausgezeichnet gro"sen Massen.}} vorhanden, und oft, zumal auf rohen unpolierten Fl"achen, "au"serst schwer zu erkennen und vom vorigen zu unterscheiden ist, zeigt sich auf solchen Fl"achen mehr oder weniger h"aufig und deutlich, in "au"serst zarten, meistens mikroskopischen, teils einzeln eingestreuten, teils mehr oder weniger zusammen geh"auften Punkten und K"ornern, seltener in gr"o"seren, br"ocklich oder k"ornig zusammen geh"auften Massen von sehr verschiedener, ganz unregelm"a"siger Gestalt, und mehr oder minder dann verbrochen, zerrissen und zerkl"uftet, und bei diesem letzteren Vorkommen, von einem unebenen, feink"ornigen, bisweilen versteckt bl"atterigen, seltener unvollkommen und klein muschlichen Bruche, nicht selten mit kristallinischen Facetten, und unbestimmt eckige, ziemlich scharfkantige Bruchst"ucke gebend.

Es haben diese K"orner und Massen stets ein rein metallisches Ansehen, und auf rohen Fl"achen der Steinmasse, zumal die kleinsten derselben, gew"ohnlich einen starken, oft sehr starken, spiegelnden, metallischen Glanz, und eine mehr oder minder rein zinkgraue, oft auch beinahe zinn- oder silberwei"se, gew"ohnlich aber ins R"otliche --- beinahe wie der Kupfernickel --- meistens doch ins Gelbliche, Speis- oder Messing-Gelbe ziehende Farbe.\footnote{\frakfamily{Diese oft sehr auffallenden Abweichungen in der Farbe, auf welche schon Chladni aufmerksam machte, und deren nicht selten mehrere in einem und demselben Bruchstucke eines Steines vorkommen, scheinen wohl, zumal sie in einigem Einklange mit den "ubrigen Eigenschaften, als mit der Harte und der Retractibilit"at stehen, au fremdartige Beimischungen (Nickel, Chrom, Mangan, Silicium), oder doch auf ein verschiedenes Verh"altnis vom Schwefel zum Eisen, oder auf eine anderweitige Zustandsverschiedenheit dieses letzteren hinzudeuten.}} Gr"o"sere Massen erscheinen bisweilen, obgleich nur selten, matt oder doch minder gl"anzend, und dunkelgrau oder br"aunlich, auch tombakbraun, rostbraun oder kupferr"otlich, und bisweilen auch pfauenschweifig, dunkelblau, rot und Messinggelb angelaufen.

Geritzt geben diese Metallteilchen sogleich ihre Spr"odigkeit zu erkennen, wodurch sie sich von den vorigen sehr auffallend unterscheiden, und mit Gewalt aus der Steinmasse gebrochen, aus welcher sie sich mehr oder weniger leicht, st"uckweise aussprengen lassen, kann man sie auch mehr oder minder leicht zum feinsten Pulver zersto"sen und zerreiben, das dann eine mehr oder weniger matte und schw"arzliche Farbe annimmt. Jene Spr"odigkeit, so wie das ganze Ansehen und Verhalten dieser metallischen Massen sowohl im Ganzen als in ihren Massenteilchen, die Entwickelung von Schwefel-Wasserstoffgas bei Behandlung mit Salzs"aure, und vollends die Resultate der Analysen, geben die Natur dieses metallischen Gemengteiles als Eisen- oder Schwefelkies, und zwar, nach letzteren, wegen des geringen Verh"altnisses des Schwefels zum Eisen,\footnote{\frakfamily{Die meisten bisherigen Analysen von Meteor-Steinen lassen zwar nur durch einen bestimmt, oft auch ganz unbestimmt, und selbst nur als Spur angegebenen Gehalt an Schwefel auf die gewesene Gegenwart von geschwefeltem Eisen als Gemengteil derselben, keinesweges aber auf dessen quantitatives Verh"altnis zur Steinmasse, am wenigsten vollends auf dessen individuelle Zusammensetzung und auf das Verh"altnis des Schwefels zum Eisen in demselben schlie"sen. Inzwischen hat doch Howard schon das letztere naher bestimmt, indem er 14 Gran Kies aus einem Steine von Benares f"ur sich analysierte, und --- obgleich mit unberechenbarem Verluste an Schwefel --- 2 Gran desselben mit $\mathfrak{10\frac{1}{2}}$ Gran Eisen verbunden, demnach beil"aufig 20 Perzent Schwefel fand. Aus Stromeyers neuesten Analysen der Steine von Erxleben und K"ostritz ergibt sich (aber freilich nach st"ochiometrischem Kalk"ule, wobei es wohl in Frage stehen d"urfte, ob bei diesen r"atselhaften Produkten so ganz zuversichtlich darauf anzugehen sein mochte) bei ersteren ein Gehalt an Magnetkies von fast 8, bei letzteren von beinahe 7 Perzent, und bei beiden ein gleiches --- freilich pr"asumtives --- Mischungsverh"altnis von 58 Schwefel zu 100 Eisen (wie es Berzelius f"ur den terrestrischen Magnetkies statuiert hat). Schon Howard hat Nickel --- und zwar in einem auffallend gro"sen, unwahrscheinlichen Verh"altnis --- von beinahe 10 Perzent mit diesem Schwefeleisen in Verbindung gefunden, und da Vauquelin wenigstens (Moser und Klaproth nicht) eine Spur von jenem Metalle auch in der Masse der Steine von Stannern fand, die doch gar kein reines Gediegeneisen enthalten, so d"urfte es wohl einen best"andigen Bestandteil desselben ausmachen.}} und da er auch meistens mehr oder weniger auf den Magnet wirkt,\footnote{\frakfamily{Lange aber nicht aller, wie schon Graf Bournon ausdr"ucklich von jenem aus den Steinen von Benares, und Klaproth von dem, selbst speisgelben, aus jenem von Lissa und Erxleben bemerkt, und ich auch von jenem aus den Steinen von Siena und Mauerkirchen behaupten kann, von welchem auch nicht die kleinsten Atome von der Magnetnadel in Bewegung gesetzt werden; "ubrigens in sehr verschiedenen Graden. "Au"serst schwach z. B. wirkt jener aus den Steinen von Parma, und mehr \emph{en masse} als im Pulver, vielleicht blo"s in Folge der umgebenden oder anhangenden Atome von Gediegeneisen oder Eisen-Oxydul; hier und da einiger aus der Masse der Steine von Siena und Lissa, etwas starker; "au"serst stark dagegen und selbst in den kleinsten Atomen, jener aus den derben Gediegeneisen-Massen. Und ich glaube bemerkt zu haben, dass der verschiedene Grad von Retractibilit"at dieses Kieses "uberhaupt mit der Menge und Masse des vorhandenen Gediegeneisens in einem Verh"altnisse stehe. Ob derselbe "ubrigens von den oben erw"ahnten verschiedenen metallischen Beimischungen, oder von einer Zustandsverschiedenheit des Eisens, oder von dem Mischungsverh"altnisse des Schwefels zum Eisen abh"ange, will ich vor der Hand dahingestellt sein lassen, und nur die Analytiker darauf aufmerksam gemacht haben.}} als Schwefeleisen im Minimum oder als Magnetkies, zu erkennen.

Auf geschliffenen und polierten Fl"achen erscheint auch dieser Gemengteil ungleich deutlicher, da die kleinsten Teilchen mehr zur Ansicht kommen (obgleich viele w"ahrend des Schnittes ihrer Spr"odigkeit halber ausgesprengt werden m"ogen), und sich besser, ja oft ausschlie"slich nur hier, von jenen des Gediegeneisens unterscheiden lassen, indem sie immer einen etwas schw"acheren Glanz (wahrscheinlich als Folge des Anlaufens durch die angewendete Feuchtigkeit w"ahrend des Schnittes) und eine dunklere, stets ins Stahl- oder Zink-Graue fallende, und meistens ins R"otliche oder Gelbliche ziehende Farbe haben, und sich gew"ohnlich (zumal wenn in etwas gr"o"seren Massen), rissig, zersprungen und zerkl"uftet, oder "au"serst zartk"ornig angeh"auft zeigen. Sie sind "ubrigens mehr oder weniger h"aufig, sehr ungleichf"ormig durch die ganze Steinmasse zerstreut, und ebenso wie die Gediegeneisenteilchen in der Grundmasse sowohl, als, und zwar in einem "ahnlichen geringeren Verh"altnisse, im olivinartigen Gemengteile, und erscheinen als "au"serst zarte, oft mikroskopisch feine Punkte, entweder einzeln oder gruppiert, und in gr"o"seren oder kleineren, teils zart und vielfach ausgezackten und ausgeschlitzten, teils scharf begrenzten, dichten Flecken.

Von dem ganz mikroskopisch feinen Vorkommen dieser Kiesteilchen und deren innigen Verbindung mit den Gediegeneisenteilchen, ist bereits oben bei diesen Erw"ahnung geschehen; so wie auch, dass sie nur selten, wenn je, unmittelbar von Rostflecken begleitet sind.

Es ist dieser Gemengteil\footnote{\frakfamily{Dieser Gemengteil ist es vorz"uglich, der die Erkl"arung, selbst mancher Nebenerscheinungen und Ver"anderungen, welche mit diesen Massen offenbar in unserer Atmosph"are erst vorgehen, so schwierig macht, und zu den widersprechendsten Hypothesen Veranlassung gab. So lie"se sich z. B. --- wie es denn auch, jenes und manches andern Einspruches ungeachtet, ziemlich allgemein geschieht --- das Leuchten, Gl"uhen, Funkenspr"uhen und endliche Zerplatzen der Feuerkugeln, und vollends die Bildung der Rinde (anscheinend! das ausgesprochene Produkt eines gew"ohnlichen Schmelz-Prozesses) "uber die vereinzelten Bruchstucke derselben, durch --- unter mehr oder weniger annehmbaren Voraussetzungen zul"assliche --- Entwickelung oder Freimachung von Warmestoff am k"urzesten und leichtesten erkl"aren, wenn nicht das h"aufige Vorkommen dieses Gemengteiles in der ganzen Masse, und selbst an der Oberfl"ache, und ganz dicht unter der Rinde der Steine, und namentlich auch in den ganz reinen und derben Gediegeneisen-Massen, im ganz unver"anderten Zustande seiner oft aufs h"ochste ausgesprochenen metallischen Beschaffenheit bei dessen leichter Zerst"orbarkeit durch Hitze dagegen stritte, umso mehr als diese, wenigstens in unserer Atmosph"are, dem Einfl"usse des Oxygens, und bei der schweren Schmelzbarkeit der Stein- und vollends der Eisenmassen (welche letztere gerade durch ihr scheinbar geschmolzenes Ansehen manche Physiker verleiteten, sie geschmolzen fl"ussig bis zur Erde gelangen zu machen), einen Grad voraussetzen wurde, mit dem sich das Bestehen eines Schwefeleisens schlechterdings nicht vereinbaren lie"se.}} f"ur manche Meteor-Steine sehr charakteristisch (zu welchem Ende aber notwendig eine Fl"ache des Steines abgeschliffen werden muss), teils durch seine Menge (wie f"ur die Steine von Benares, Lissa, Parma \emph{zc.}), oder durch seine Seltenheit (wie f"ur jene von Eichst"adt, Timochin, Tabor, Charsonville \emph{zc.}), teils durch die Gr"o"se oder ausgezeichnete Farbe seiner Massen (wie f"ur die Steine von Parma, Stannern, Mauerkirchen, Benares \emph{zc.}).

Au"ser jenen vier, strengen Sinnes zur Wesenheit der Meteor-Steine, als gemengten Massen, geh"origen, dem freien Auge mehr oder weniger leicht unterscheidbaren Gemengteilen, findet sich, wenigstens bei vielen, wo nicht allen, noch ein f"unfter, der aber, auf rohen sowohl als auf geschliffenen Fl"achen der Steine, meistens nur mit H"ulfe eines Vergr"o"serungsglases, und selbst dann nur schwer und sparsam, am leichtesten noch und am h"aufigsten in der gr"oblich gepulverten Steinmasse unter dem Mikroskope aufgefunden werden kann, und der in Gestalt "au"serst zarter, unf"ormlicher, sehr ungleichf"ormig verteilter und einzeln eingestreuter, nur h"ochst selten in "au"serst kleinen Partien zusammen geh"aufter, von der Masse fest eingeschlossener Punkte oder K"orner von matter, schw"arzlich-brauner oder schwarzer Farbe erscheint. Es zeigen sich diese K"orner leicht zerreiblich, und geben ein gleichf"ormiges Pulver; sie werden mehr oder minder lebhaft von der Magnetnadel angezogen, und sind wohl ohne Zweifel f"ur ein Oxyd oder Oxydul von Eisen,\footnote{\frakfamily{Bekanntlich hat Klaproth, dem wir in Deutschland die fr"uhesten Analysen, und im Ganzen --- wo ich nicht irre --- die von sieben verschiedenen Meteor-Steinen verdanken, die Vermutung ge"au"sert: es k"ame das Eisen in allen Meteor-Steinen, ohne Ausnahme, selbst in jenen, wo sich durchaus keine Spur davon, weder physisch noch oryktognostisch, als rein und gediegen zu erkennen gibt (wie z. B. in jenen von Stannern --- wovon er doch selbst ein St"uck analysierte ---), stets nur im regulinischen Zustande vor, und dass selbst jenes --- wie auch der Nickel und das Mangan --- das sich in einem gr"o"seren oder geringeren Anteil, auch chemisch ausgesprochen, im offenbar oxydierten Zustande f"ande, nicht urspr"unglich so in denselben enthalten gewesen, sondern erst --- so wie die Rostflecke --- sp"ater als Folge der Oxydation des zuvor frei und gediegen vorhanden gewesenen, in der atmosph"arischen Luft entstanden sei; dass dagegen alles physisch und oryktognostisch unerkennbare und chemisch mit den erdigen Gemengteilen verbundene Eisen regulinisch in diesen (im oxygenfreien Zustande mit den einfachen Erden verbunden), in einer gegenseitig sich durchdringenden Mischung (wodurch auch dessen Wirksamkeit auf den Magnet aufgehoben werden kann) demnach blo"s verlarvt --- sich befinden m"ochte. Es ist nicht in Abrede zu stellen, dass die Motive, welche diese Mutma"sung veranlassten (die h"ochst wahrscheinliche Herstammung dieser Massen aus Regionen, wo, ebenso wahrscheinlich, kein Oxygen vorhanden sei; --- das h"aufige Vorkommen des so rein ausgesprochenen, ganz unver"anderten, und doch so leicht zerst"orbaren Schwefeleisens in denselben; --- die Ermangelung irgend einer Anzeige von Oxygen-Gehalt bei den wiederholten Analysen; --- und endlich die Resultate des Kalk"uls bei Bestimmung des quantitativen Verh"altnisses der verschiedenen Bestandteile der von ihm zerlegten Steine ---), dieselbe gerade nicht abn"otigten, im Gegenteil manche Einspruche gestatten (wovon gleich einen z. B. der Zustand der "ubrigen Gemengteile, jener der erdigen Bestandteile, als Oxyde metallischer Basen, machen d"urfte), und dass damit die bestimmt ausgesprochenen Befunde anderer Analytiker im offenbaren Widerspruche stehen, welche den Gehalt der Meteor-Steine an oxydiertem Eisen und andern Metallen (Mangan, Chrom, Nickel), und zwar nicht blo"s im Zustande von mechanischer Einmengung (in welchem Falle derselbe etwa nach Klaproth als Produkt sp"aterer Erzeugung angesehen werden k"onnte), sondern ganz verlarvt und chemisch mit den erdigen Bestandteilen verbunden, unwiderleglich dartun. (So erkl"arte Howard allen Gehalt an Eisen der von ihm analysierten Steine --- insofern sich dasselbe nicht als gediegen oder geschwefelt aussprach, --- so Vauquelin --- der, meines Besinnens, sogar an irgend einem Orte, alles, selbst das vollkommen regulinisch vorkommende Meteor-Eisen (wahrscheinlich der beobachteten partiellen Spr"odigkeit und in eben dem Grade schweren Schmiedbarkeit wegen) stets als etwas oxydiert erkl"art --- ebenso, und namentlich den ganzen, allem Ansehen nach durchaus verlarvten, doch 31 Perzent betragenden Eisengehalt des Steines von Chassigny; so Moser und derselbe jenen von 27 bis 29 Perzent --- wovon nur wenig auf den vorhandenen Kies f"allt, und eben so wenig sich als freies Oxyd ausspricht --- der Steine von Stannern, f"ur vollkommen oxydiertes Eisen; so gibt endlich Stromeyer den Gehalt an wahrhaft --- aber nur oxydulirten --- Eisen der Steine von Erxleben und K"ostritz auf 5 Perzent an.) Inzwischen verdient doch, meines Erachtens, Klaproths Vermutung noch alle Beachtung und besondere Aufmerksamkeit, und dies umso mehr, als dieselbe durch die --- oben in einer Note bei den Gediegeneisen --- bereits erw"ahnten Umst"ande (der innigen, wenn gleich nur mechanisch scheinenden Verbindung der Eisen- und Erdeteilchen, selbst in den mikroskopischen Massenteilchen, --- der selbst auf chemischem Wege erst m"oglichen, vollkommenen Scheidung beider, --- dem bei verschiedenen Meteor-Eisen so merklich abweichenden, und bei jenem aus Meteor-Steinen so auffallend geringen spezifischen Gewichte, und den anscheinend verschiedenen Graden von Retractibilit"at desselben ---) neue Bekr"aftigung zu erhalten scheint, und in der, dem chemisch ausgewiesenen oder sinnlich wahrnehmbaren Gehalte an Eisen, oft nicht entsprechenden Angabe des spezifischen Gewichts mancher Steine, und selbst, wie mir d"aucht, in obigen und manch andern, ziemlich sich widersprechenden Resultaten, insbesondere aber in jenen der, gewiss h"ochst verl"asslichen Analysen Stromeyers (nach welchen ein nur sehr geringer Teil --- und zwar bei anscheinend nur wenig Gediegeneisen und Kies f"uhrenden Steinen --- von Eisen, und dies nur im Minimum oxydiert, dagegen ein bedeutenderer Anteil an regulinischen ausgewiesen wird, als nach jenem Anscheine erwartet werden sollte, wovon demnach der "Uberschuss in den erdigen Gemengteilen verlarvt enthalten sein m"usste) einige Best"atigung finden m"ochte. Dass Silicium jene Verbindung wenigstens vermitteln m"usste, d"urfte wohl ebenso gut hier, als bei den von Daniell und Berzelius gefundenen "ahnlichen Verbindungen von Kieselerde mit metallischem Eisen, und von Laugier in sibirischen, vorauszusetzen sehn, worauf vielleicht schon der besondere Zustand, in welchem alle obige Analytiker die Kieselerde in den Meteor-Steinen "uberhaupt gefunden haben, hindeutet.}} von Mangan etwa zum Teil, und vielleicht auch von Chrom anzusehen. 

H"ochst selten, und nur bei einigen Meteor-Steinen (nach meiner "Uberzeugung und deutlich nur bei jenen von Chassigny und Lissa) erscheint dieser Gemengteil in etwas gr"o"seren, ebenso zerstreuten Massen von beinahe pechschwarzer Farbe, und ziemlich starkem, etwas fettigem Glanze, die wenig oder gar nicht auf den Magnet wirken.

F"ur Partikelchen von Rinde-Substanz, wof"ur sie, wenigstens zum Teil, Chladni anzusehen geneigt ist, kann man diese K"orner, zumal ersterer Art, nicht wohl erkennen, da sie nicht nur in ganzen Ansehen und durch ihre Retractibilit"at (vorz"uglich bei Steinen, wo es die Atome der Rinde gar nicht sind, wie z. B. bei jenen von Stannern, wo sie doch gerade am h"aufigsten vorkommen) sich davon unterscheiden, sondern auch die Art des Vorkommens und der Einmengung aller --- so sparsam und vereinzelnt, und "uberhaupt so selten, --- so mikroskopisch zart und isoliert, gar nicht in die Steinmasse "ubergehend (wie dies doch bei der oberfl"achlichen Rinde im Kontakte mit jener so auffallend Statt hat), und in einem gek"ornten Zustande --- mit jeder m"oglich denkbaren Art von Entstehung und Bildung von Rinde-Substanz mitten in der Steinmasse, namentlich aber mit jener durch Einknetung, im Widerspruche steht. Leichter k"onnte man diese K"orner, wenigstens letztere, dunklere, gl"anzende, mit dem olivinartigen Gemengteile oder mit der Substanz, die auch in Adern vorzukommen pflegt, und von welcher sogleich die Rede sein wird, verwechseln, mit welchen diese aber auch (zumal jene in den Steinen von Lissa) ziemlich gleicher Natur sein m"ochten. Am h"aufigsten und deutlichsten, und zwar gr"o"sten Teils von bedeutender, dem freien Auge wenigstens erkennbarer Gr"o"se, kommen derlei K"orner in der Masse der Steine von Chassigny vor. Diese scheinen aber eben so wenig oxydiertes Eisen als Rinde-Substanz zu sein. Dem ersteren widerspricht n"amlich die pechschwarze Farbe, der starke, etwas fettige Glanz, das kristallinische Ansehen und die g"anzliche Unwirksamkeit auf die Magnetnadel, ausgenommen in einzelnen wenigen, mikroskopisch kleinen Splittern, insofern auch die ganze Steinmasse einige Wirksamkeit "au"sert; dem letzteren aber --- nebst obigen Gr"unden in Betreff der Art des Vorkommens und der Einmengung --- Farbe, Glanz, und die ganze Beschaffenheit, verglichen mit der oberfl"achlichen, ganz eigenen Rinde dieser Steine, die "uberdies, obgleich schwach, doch merklich genug auf den Magnet wirkt.

Der ausgezeichnete Gehalt dieser Steine an Chrom, von 2 Perzent, welches Metall hier, nach Vauquelin, rein und regulinisch vorkommen soll, l"asst mit allem Grunde vermuten, dass es dieses Metall sei, welches hier auf solche Art erscheint, indes dasselbe in den "ubrigen Meteor-Steinen, wo es bisher, fast durchgehends zwar, aber nur als Spur, oder in der sehr unbedeutenden Menge von $\mathfrak{\frac{1}{4}}$ bis 1 Perzent gefunden worden ist, wahrscheinlich auf gleiche Art eingestreut, aber, nach Stromeyers Vermutung, immer als Oxyd und in Verbindung mit Eisen, als wahres Chromeisen, vorkommt.

Dass jene Atome von oxydiertem Eisen am h"aufigsten und mikroskopisch zart, in Begleitung und inniger, wenn gleich mechanischer Verbindung mit den eingemengten Gediegeneisenteilchen, selbst bei deren mikroskopischen Erscheinen in den Massenteilchen der erdigen Gemengteile, im Gefolge letzterer, und wahrscheinlich in Gesellschaft von "ahnlichen Kies-Atomen, vorkommen, dieselben gleichsam einh"ullen, und sich erst bei Fletschung derselben als schwarzes, mehr oder weniger retractiles Pulver zu erkennen geben, und dass es vorz"uglich diese Atome sein m"ochten, von welchen die Rostflecke in der Steinmasse vorzugsweise herr"uhren --- ist bereits bei Beschreibung des Gediegeneisens bemerkt worden, und dass dieselben einen wesentlichen Einfluss auf den Koh"asions-Zustand, den Magnetismus, auf das spezifische Gewicht, und mittelbar wenigstens, auf das "au"sere Ansehen der Steinmasse im Ganzen haben m"ussen, ergibt sich aus ihrer Natur und dem hier"uber Vorgebrachten.\footnote{\frakfamily{Das quantitative Verh"altnis dieses oxydierten Eisens im freien Zustande, als wahrer Gemengteil, kann "ubrigens --- dem Ansehen nach --- im Allgemeinen nur sehr gering, und, zumal bei Steinen, von welchen ein bedeutender Gehalt an Eisen im Ganzen, chemisch ausgewiesen, aber wenig, oder vollends gar keiner als regulinisch oder geschwefelt, oryktognostisch ausgesprochen ist, im Verh"altnis zum chemisch gebundenen oder verlarvten, --- nur h"ochst unbedeutend sein.}}

H"ochst merkw"urdig aber ist wohl das Wechselverh"altnis, welches --- insoweit aus dem "au"sern Ansehen und den Resultaten der, leider in dieser Beziehung nicht hinl"anglich befriedigenden, Analysen geschlossen werden kann --- zwischen dem Gehalte der verschiedenen Meteor-Steine an Eisen in diesem mehr oder weniger oxydierten Zustande (als Oxyd oder Oxydul --- wenn es ja bei diesen r"atselhaften Fossilien keine anderen Mischungsverh"altnisse zwischen Eisen und Oxygen --- so wie zwischen Eisen und Schwefel --- geben sollte --- als man bei den "ahnlichen Verbindungen in terrestrischen Fossilien als Norm annehmen zu d"urfen sich berechtigt glaubt), und jenem in ausgesprochen regulinischem zu bestehen scheint, indem ersterer --- offenbar oder verlarvt --- in dem Ma"se vorwaltet, als letzterer --- wenigstens offenbar --- in einem geringeren vorhanden ist.\footnote{\frakfamily{Der Total-Gehalt an Eisen in allen Zust"anden und Verbindungen, in welchen dasselbe vorzukommen pflegt (rein metallisch, und zwar frei, oryktognostisch ausgesprochen, oder als solches vielleicht auch verlarvt; mehr oder weniger mit Schwefel vererzt als Eisen- oder Magnetkies, und mehr oder weniger mit Oxygen verbunden, als Oxyd oder Oxydul, und als solches wieder mechanisch eingemengt oder chemisch gebunden), zusammen genommen, und alles auf Oxyd reduziert, wie es durch die Analyse der Steinmasse ohne mechanische Absonderung gewonnen wird, weicht bei allen bisher bekannten, dem Ansehen nach auch noch so verschiedenartigen Meteor-Steinen nicht sehr ab, schwankt gew"ohnlich nur zwischen 30 und 40, und steigt nur in h"ochst seltenen F"allen bis gegen 50 Perzent von der gesamten Steinmasse. Davon betr"agt das regulinische, sinnlich wahrnehmbare, wenn es nicht, was jedoch sehr selten der Fall ist (wie bei den Steinen von Chassigny, Stannern, Alais?), ganz fehlt: von 1 bis 19 Perzent --- wahrscheinlich wohl noch etwas mehr; --- das geschwefelte, wenn es nicht, was jedoch noch seltener der Fall ist (wie bei jenen von Chassigny, Alais??), ganz fehlt: von 1 bis etwa 12 oder 15; und das oxydierte endlich --- wovon jedoch in keinem Falle mehr als einige wenige Perzente mechanisch eingemengt sein d"urften --- das Ganze oder den Rest jenes Total-Gehaltes. Jene Steine, welche dem Ansehen und dem spezifischen Gewichte nach am reichhaltigsten an mechanisch eingemengtem, zumal gediegenem Eisen sind, enthalten im Ganzen eben nicht bedeutend mehr als jene, wo sich wenig oder selbst gar nichts oryktognostisch und physisch nachweisen l"asst. So steht von ersteren, deren spezifisches Gewicht = 3,7 ist (den Steinen von Eichst"adt, Timochin, Charsonville), der Total-Gehalt an erhaltenem Oxyd beil"aufig zwischen 36 und 43, bei letzteren, deren spezifisches Gewicht zwischen 1,9 und 3,3 ist (den Steinen von Alais, Stannern, Benares, Eggenfeld, Parma \emph{zc.}), zwischen 30 und 40 Perzent. (Merkw"urdig ist, dass das spezifische Gewicht der Steine von Chassigny, bei welchen sich doch keine Spur von mechanisch eingemengtem Eisen oder --- au"ser jenen sparsamen, schwarzen Atomen --- von einem andern Metalle findet, und deren Eisengehalt, nach Vauquelins Ausweis, selbst nur 31 Perzent an Oxyd betr"agt, beinahe das Mittelgewicht der Meteor-Steine "uberhaupt "ubersteigt, indem dasselbe nach eigener Wiegung 3,55 betr"agt.) Bei jenen an Gediegeneisen besonders reichen Steinen endlich, und namentlich bei jenen von Eichst"adt, verh"alt sich der Gehalt an Eisenoxyd zu dem an Gediegeneisen, nach Klaproths Angabe (die aber nicht befriedigend ist, indem er das gediegene Metall mit dem Magnete auszog, daher vieles, was in den erdigen Teilchen verh"ullt war, mit in die Aufl"osung von diesen brachte, und durch die Operation als Oxyd erhielt), wie 16,50 : 19, und bei jenen von Timochin (bei gleichem Verfahren) wie 25 : 17,60, oder nach N. A. Scherer, wie 17,50 : 17,75. (Von den Steinen von Charsonville gibt Vauquelin den Total-Gehalt an Eisen mit 25,8 regulinisch an, wie er ihn nach Kalk"ul des durch die Operation im Ganzen erhaltenen Oxydes herausbrachte). Bei den an Gediegeneisen besonders armen Steinen dagegen, und namentlich bei jenen von Benares, nach Howard, verh"alt sich der Gehalt an Eisenoxyd zu dem an ersterem, wie 34 : 2; bei jenen von Siena, nach Klaproth, wie 25 : 2,25; bei jenen von Mauerkirchen, nach Imhof, wie 40,24 : 2,33; und bei jenen von Eggenfeld, nach demselben, wie 32,54 : 1,8 (wobei freilich auch nach Klaproths Methode verfahren worden sein mochte). Nach Stromeyers ungleich genaueren und umsichtigern Analysen ergab sich f"ur die Steine von Erxleben und K"ostritz, die dem Ansehen nach (erstere mehr) zu den mittel reichhaltigen an Gediegeneisen geh"oren, und deren spezifisches Gewicht zwischen 3,6 und 3,5 steht, ein Verh"altnis von 5,57 und 4,89 an Oxydul zu 24,41 und 17,48 an metallischem Eisen, mit Inbegriff des Schwefeleisens.}}

Noch kommen bei Betrachtung der Steinmasse der Meteor-Steine im Allgemeinen zwei ebenso auffallende als merkw"urdige Beschaffenheiten zu erw"ahnen, die, wenn sie gleich nicht zu ihrer Wesenheit geh"oren, und sich gerade nicht bei allen Steinen finden, doch sehr h"aufig erscheinen, und als bedeutende Zustandsver"anderungen der Steinmasse, wo nicht als heterogene Gemengteile, anzusehen kommen, und deren h"ochst r"atselhafte Entstehung und Bildung einerseits mit mancher der gangbaren Theorien "uber die Herkunft und die urspr"ungliche Entstehung und Bildung der Massen selbst, sehr im Widerspruche stehen, andererseits in der Folgezeit, wenn sie bei vervielf"altigten Beobachtungen und weiteren Untersuchungen einst befriedigend sollten erkl"art werden k"onnen, manche Aufkl"arung in letzterer Beziehung erwarten lassen d"urften.

Die eine dieser Zustandsver"anderungen der Steinmasse ist das Vorkommen derselben als scharf begrenzte Adern oder G"ange von verschiedener M"achtigkeit und Dicke; die andere bezeichnet das verschiedene Aussehen derselben auf scheinbaren, zum Teil wirklichen Absonderungsfl"achen von verschiedener Ausdehnung, mitten im Innern der Steine.

Das erstere Vorkommen findet sich --- wie ich mich nun "uberzeugt habe --- bei sehr vielen, und h"ochst wahrscheinlich, mehr oder minder h"aufig und deutlich ausgesprochen, wohl bei den meisten Meteor-Steinen.\footnote{\frakfamily{Ich habe zuerst (1808) auf das r"atselhafte Vorkommen dieser Adern in der Masse der Meteor-Steine bei Gelegenheit der Beschreibung jener von Stannern, obgleich sie in diesen nur "au"serst selten und gewisser Ma"sen unvollkommen sich zeigen, aufmerksam gemacht. Beinahe gleichzeitig erw"ahnte ihrer Herr Bergrath Reu"s bei Beschreibung der bei Lissa gefallenen Steine, in welchen sie am h"aufigsten vorzukommen scheinen. Erst an den Steinen von Charsonville machten Bigot de Morogues, Hauy und Vauquelin dieselbe Beobachtung, und in ihrer Beschreibung (1811) als von etwas ganz Neuem und Merkw"urdigem, Erw"ahnung davon. In der Folge (1814) gaben die Steine von Agen Gelegenheit zur Erneuerung dieser Beobachtung, welche inzwischen Chladni und ich bereits an vielen, zum Teil lang bekannt gewesenen, "alteren Meteor-Seinen zur Gen"uge gemache hatten. Nach neuerlichster Untersuchung kann ich sie nun an, mitunter ziemlich kleinen, Bruchstucken von folgenden Meteor-Steinen nachweisen: von Lisa, Agen, Doroninsk, Charsonville, Chantonnay, Ensisheim, L'Aigle, Barbotan, Yorkshire, Laponas, Sigena, Toulouse, Salés, Apt, Tipperary, Weston, Stannern; und bei den meisten "ubrigen mir au"ser diesen noch bekannten Meteor-Steinen m"ochte es wohl nur an der individuellen Beschaffenheit des vorhandenen Bruchstucks oder seiner Bruchflache liegen, dass ich nicht dasselbe zu tun im Stande bin.}} Es zeigen sich n"amlich auf rohen, und noch deutlicher auf geschnittenen, zumal geschliffenen Fl"achen der Steinmasse von einigem Fl"acheninhalte, einzelne oder mehrere, oft sehr viele, k"urzere oder l"angere, gerade laufende oder bogenf"ormig gekr"ummte, auch mehrfach gebogene Adern, von sehr verschiedener, bald im ganzen Verlaufe gleichf"ormiger, bald allm"ahlich abnehmender, bald sehr und stellenweise j"ah und stark ab- oder zunehmender Breite und M"achtigkeit, und zwar vom Haarfeinen, kaum dem freien Auge sichtbaren, bis --- was jedoch h"ochst selten --- zu 3 Linien, welche nach allen Richtungen, und oft von einem Rande der Fl"ache bis zum andern entgegen gesetzten, und zwar an einem oder dem andern --- aber nicht immer mit dem breiteren Ende --- bisweilen auch an beiden R"andern, aber auch sehr oft an keinem, an die etwa da befindliche Rinde anstehend, oft aber auch ganz isoliert und frei im Mittel der Fl"ache oder der Steinmasse verlaufen. Es sind diese Adern teils, obgleich selten, ganz einfach, teils mehr oder weniger, oft sehr h"aufig ramifiziert, und es gehen die "Aste und Zweige von verschiedener St"arke und L"ange, "ubrigens von "ahnlicher Beschaffenheit, wie der Hauptstamm, unter oft sehr spitzigen Winkeln, nach allen Richtungen von demselben ab, und verlaufen auf "ahnliche Weise gegen die R"ander oder mitten in der Masse; sie sind nicht selten wieder zer"astelt, durchsetzen und durchschneiden sich, m"unden sich in einander ein, oder laufen zum Teil auch eine Strecke parallel --- wie dies alles nicht selten selbst die Hauptst"amme, wenn deren mehrere vorhanden sind, zu tun pflegen --- und bilden oft ein ziemlich enges, sehr ungleiches Netz oder Adergeflecht. Oft gehen diese Adern, als G"ange, in eine betr"achtliche Tiefe mit gleicher oder abnehmender, auch wohl ver"anderlicher M"achtigkeit, oft bei ansehnlicher Dicke des St"uckes, auf einige Zolle; aber dieselbe Ader nicht durchaus auf gleiche Tiefe. Manche scheinen wohl bis an die Oberfl"ache des Steines zu gehen, die bei weitem meisten aber nicht, und viele nur auf eine h"ochst unbedeutende Tiefe, so dass nach diesem und obigem manche --- und dies m"ochte wohl die meisten treffen --- ganz auf das Innerste der Steinmasse beschr"ankt sind, und mit der Oberfl"ache und der Au"senrinde in gar keiner Verbindung stehen; andere nach einer oder zweien, wieder andere vielleicht nach allen Richtungen ganz durchgehen. Nie scheinen diese G"ange auf irgendeine Tiefe ganz senkrecht, sondern immer mehr oder weniger schief durch die Steinmasse zu setzen.

Die Masse, welche diese Adern und G"ange bildet, ist im Wesentlichen, die Farbe abgerechnet, von der "ubrigen Steinmasse im Allgemeinen nicht verschieden, indem sie im Gegenteile eine in jeder Beziehung ganz ununterbrochene Fortsetzung von dieser ausmacht, und au"ser der scharfen Begrenzung durch die Farbe, durch gar nichts, das z. B. einem Salbande gliche, geschieden ist, sondern vielmehr unmittelbar in dieselbe "ubergeht. Sie zeigt dieselbe Textur, dieselbe Beschaffenheit der Oberfl"ache sowohl im Bruche als im Schnitte, dasselbe, meistens doch ein etwas feineres, Korn, nur mehr Dichtheit, Festigkeit und H"arte --- beil"aufig so wie der olivinartige Gemengteil in einem mittleren Grade von Ausbildung --- daher sie auch geschliffen --- so wie dieser --- eine h"ohere Politur und einen etwas fettigen Glanz annimmt, und sie enth"alt ebenso wie die "ubrige Steinmasse, Gediegeneisen eingesprengt; vom olivinartigen Gemengteile, nach der gew"ohnlichen Art seiner Ausscheidung und Begrenzung, konnte ich aber nie etwas in ihr bemerken. Nur, wiewohl h"ochst selten, und an einzelnen Stellen besonders breiter Adern, zeigt sie eine schwache Anlage zu einer schiefrigen Textur. Sie zeigt denselben, gew"ohnlich nur etwas, h"oheren Grad von Wirkung auf den Magnet wie die "ubrige Steinmasse, aber einen merklich geringeren als die Rinde desselben Steines.

Das beinahe einzige und wesentlichste Merkmal, wodurch sich die Masse dieser Adern und G"ange von der "ubrigen Steinmasse unterscheidet, ist die Farbe. Diese ist schw"arzlich, oft beinahe schwarz, gew"ohnlich aber graulich- oder bl"aulich-schwarz, oder bl"aulich- und mehr oder weniger dunkel schiefer-grau, nie so pech- oder kohlschwarz, wie die Rinde an manchem solcher Steine, am wenigsten braun, wie diese an den meisten, und ohne allem metallischen Ansehen; dagegen oft genauso, wie der olivinartige Gemengteil im ausgesprochneren Zustande in denselben oder in andern Meteor-Steinen zu erscheinen pflegt. Nur auf polierten Fl"achen zeigt diese Gangmasse einen mehr oder weniger ausgezeichneten, etwas fettigen, und dem olivinartigen Gemengteile im ausgesprochneren Zustande "ahnlichen Glanz, auch, wenigstens bei dem einen Steine von Stannern, wo auch die Farbe den dunkelsten Partien jenes Gemengteiles entspricht, ein "ahnliches, zerrissenes und zersprungenes, gleichsam gek"orntes Ansehen.

Es ist bemerkenswert, dass sich diese Adern und G"ange am h"aufigsten und ausgezeichnetsten in solchen Meteor-Steinen finden, die sich --- mit Ausnahme jener von Stannern, wo sie "ubrigens nur an einem unter so vielen gesehenen Bruchst"ucken, und auch hier nur in einem unvollkommenen Zustande beobachtet wurden --- durch eine betr"achtliche Dichtheit, Festigkeit und Innigkeit des Koh"asions-Zustandes sowohl als des Aggregats-Zustandes auszeichnen (wie die Steine von Lissa, Agen, Charsonville, Chantonnay, Ensisheim \emph{zc.}), und gerade in solchen, wo der olivinartige Gemengteil nur sehr wenig, oder doch nur als solcher unvollkommen ausgesprochen und nicht sehr mannigfaltig erscheint (wie dies ebenfalls bei den genannten Steinen der Fall ist).

Vauquelin und Chladni halten die Substanz dieser Adern und G"ange f"ur ganz homogen mit der Rindenmasse, inzwischen ergibt sich aus obigem, dass sie in der Farbe stark, in der Textur und "ubrigen Beschaffenheit aber ganz und gar von dieser abweicht (wie sie denn auch gar keine Porosit"at und nirgend einen "Ubergang in die Steinmasse zeigt, welches beides, wenigstens nach meiner Ansicht hinsichtlich ihrer Entstehung und Bildung, so gut wie bei der oberfl"achlichen Rinde der Fall sein m"usste), dagegen ungleich mehr "Ahnlichkeit mit der "ubrigen Steinmasse, zumal mit dem einen Gemengteile derselben erkennen l"a"st.\footnote{\frakfamily{Vauquelin erkl"arte die Entstehung dieser Adern und G"ange, nach obiger Voraussetzung und in Annahme eines wahren Schmelz-Prozesses zur Erzeugung der Rinde (durch Erhitzung in der Luft w"ahrend des Durchzuges und Niederfallens der Steine), wie jene der Au"senrinde: durch Verbrennung des Eisens und Verschlackung der Erden durch die atmosph"arische Luft, welche durch einen Riss, den der Stein im Gl"uhen bekam, und der nach der Hand wieder zusammengebacken wurde, in die Masse eingedrungen war. Allein gegen diese Mutma"sung streiten --- wenn man auch jene Annahme hinsichtlich der Bildung der Rinde im Allgemeinen zugeben k"onnte --- nicht nur die erw"ahnte Verschiedenheit der Substanz dieser Adern von jener der wahren Rinde, sondern die ganze Beschaffenheit und alle Eigenschaften jener, welche durchaus die Idee verbannen, dass sie, zumal sp"aterhin, durch Risse oder Spr"unge der Steinmasse entstanden sein k"onnen. So die Umst"ande: dass diese G"ange bisweilen nach allen Richtungen durch die ganze Masse durchsetzen, daher diese an solchen Stellen notwendig zerfallen sein m"usste; dagegen oft ganz in der Mitte mit gar keiner oder nur "au"serst schwacher Verbindung nach Au"sen erscheinen, wo demnach keine Luft eindringen konnte; oft nach Au"sen "au"serst d"unn, haarfein, im Verlaufe nach Innen aber bei einer Linie dick, was gerade bei einem Risse umgekehrt sein m"usste; bald im ganzen Verlaufe von gleicher, bald von sehr und wiederholt abweichender Dicke sich zeigen; dass sie "astig, verworren, beinahe ein Netz bildend, sich durchkreuzen, durchschneiden u. s. w., folglich einzelne St"ucke umschlie"sen, die sich h"atten lostrennen m"ussen; dass viele zu fein f"ur Risse, nach der Beschaffenheit der Steinmasse, auch oft zu grob --- bis 3 Linien dick --- als dass von Au"sen der Riss nicht sichtbar geblieben sein oder das St"uck sich nicht losgetrennt haben sollte, u. s. w.\\
\hspace*{6mm}Chladni meint dagegen (wie bereits oben bei Erkl"arung der f"unften Figur der vorhergehenden Tafel erw"ahnt worden ist), es w"aren diese G"ange oder (nach seiner Ansicht) Lagen durch das zuf"allige Zusammentreffen und Zusammenkleben bereits losgesprengter und schon "uberrindet gewesener Bruchst"ucke von Steinen, w"ahrend ihres Niederfallens, mit ihren Fl"achen aneinander, entstanden. Allein au"serdem, dass (wie an jenem Orte bemerkt worden ist) ein solches Zusammentreffen nicht wohl denkbar, ein solches Zusammenpassen, ein so festes, inniges Vereinigen und Zusammenkleben zweier, nach Ausdehnung, Bruch, Umriss u. s. w. gewiss oft ganz verschiedenartigen Fl"achen zweier Steine, oder --- wie es nach der netzartigen Durchkreuzung jener Adern anzunehmen n"otig w"are --- der Fl"achen gar vieler Bruchst"ucke gleichzeitig, gar nicht begreiflich ist; so stehen mit dieser Meinung nicht nur alle obigen Wahrnehmungen, am offenbarsten wohl jene, dass diese Lagen nur selten nach allen Dimensionen des Steines ganz durchsetzen, oft gar nicht nach Au"sen irgendwo anstehen, sondern ganz im Mittel der Masse eingeschlossen sind, --- sondern insbesondere noch folgende im Widerspruche: die Feinheit und oft haarscharfe Gleichf"ormigkeit dieser G"ange, da doch die Bruchfl"achen und selbst die "uberrindete Oberflache der Steine immer sehr uneben sind, und die d"unnste Rinde wenigstens drei Mahl dicker zu sein pflegt; dagegen oft wieder die Dicke derselben, welche jene der gew"ohnlichen Rinde bisweilen ums Sechsfache "ubersteigt; keine Spur von einer doppelten Schichte, die sich doch erkennen lassen m"usste, wo sie von zwei "uberrindeten Fl"achen zusammen traf; keine Spur von Porosit"at oder vom "Ubergange der Massenteilchen der Substanz derselben in jene der "ubrigen Steinmasse, die sich doch an der Au"senrinde so deutlich aussprechen u. s. w. "Ubrigens kommt gegen beide Meinungen zu bemerken: dass diese Adern und G"ange sich oft, selbst in einem kleinen Bruchst"ucke, in solcher Menge finden, dass sie sich schlechterdings nicht von so vielen Rissen und Spr"ungen, am wenigsten aber von ebenso vielen Absonderungen und Wiedervereinigungen herleiten lassen, und dass sie sich, wie bereits oben bemerkt worden ist, gerade am h"aufigsten und deutlichsten bei solchen Steinen zeigen, die den festesten Koh"asions-Zustand und das dichteste Gef"uge haben, bei welchen sich daher am wenigsten Risse und Zertr"ummerungen erwarten lie"sen, wie denn auch bei den meisten dieser Meteor-Massen gar keine oder nur eine geringe Vereinzelung Statt fand (so fielen die Massen von Ensisheim, Chantonnay, York --- und diese trotz ihrer bedeutenden Gro"se --- von Tipperary, Apt, Sigena, ganz und unvereinzelt, die von Laponas, Charsonville, Lissa, nur als zwei, drei oder vier St"ucke herab); endlich, dass sie bisweilen in solchen Steinen vorkommen, bei welchen selbst die Au"senrinde im Ganzen nur wenig oder unvollkommen gewesen zu sein scheint (wie bei den Steinen von Ensisheim und Chantonnay). Bigot de Morogues wollte gefunden haben, dass die Substanz dieser G"ange (die er "ubrigens f"ur ganz verschieden von der Rinde h"alt), wenigstens in den Steinen von Charsonville, ein bedeutend geringeres spezifisches Gewicht h"atte, als die "ubrige Steinmasse. Er fand n"amlich jenes dieser letzteren = 3,637, dagegen das eines St"uckes, worin eine Ader von jener Substanz vorkam, die, nach seiner Sch"atzung, etwa $\mathfrak{\frac{1}{15}}$ des Ganzen betrug, = 3,635, und berechnet nach diesem ("ubrigens h"ochst geringen Abstand und nach einem Kalk"ul, gegen den sich viel einwenden lie"se) das eigent"umliche Gewicht derselben auf 3,592, und (auf gleiche Weise nach einer --- wahrscheinlich des zuf"allig gr"o"seren Eisengehaltes wegen --- h"oheren Gewichtsangabe der Steinmasse von Hauy = 3,712) gar nur auf 2,457, und will daraus auf eine "Ahnlichkeit dieser Substanz mit der Masse der Steine von Alais schlie"sen. Die offenbar gr"o"sere Dichtheit dieser Ader-Substanz gegen die "ubrige Steinmasse, bei "ubrigens ganz gleicher Beschaffenheit, gleichem Eisengehalte und s. w. machte mir jenen, dem an sich h"ochst unverl"asslichen Kalk"ule zum Grunde liegenden, reellen Befund selbst h"ochst unwahrscheinlich, und ich wollte mich demnach durch eigene Wiegung "ahnlicher St"ucke von demselben Steine "uberzeugen. Ich fand das spezifische Gewicht eines $\mathfrak{27\frac{1}{2}}$ Gran wiegenden, von Rinde sowohl als Ader-Substanz ganz freien Stuckes der Masse eines Steines von Charsonville = 3,571; jenes dagegen eines $\mathfrak{26\frac{1}{4}}$ Gran schweren Stuckes von demselben Steine, welches zwar keine Rinde, aber eine, "uber eine Linie breite, ganz durchsetzende Ader von jener Substanz einschloss, die wenigstens $\mathfrak{\frac{1}{5}}$ des Ganzen betrug (was demnach ein drei Mal so auffallendes Resultat geben konnte, als das von Bigot de Morogues untersuchte), = 3,658.}}

Ich w"are vor der Hand geneigt, die Entstehung dieser Adern und G"ange, hinsichtlich des Momentes, f"ur ganz gleichzeitig mit der Bildung der "ubrigen Steinmasse und der Bildung und Ausscheidung ihrer Gemengteile, und, hinsichtlich der Art, f"ur ganz gleichartig mit jener der "ubrigen Gemengteile, und insbesondere des noch mehr und bezeichneter ausgeschiedenen und in der Wesenheit noch weit mehr abweichenden olivinartigen zu halten; die Substanz derselben aber f"ur homogen mit der Steinmasse, nur etwa mit einer kleinen Zustandsver"anderung oder Modifikation in der Art der Ausscheidung, und dieselbe "uberhaupt zum Teil mit dem olivinartigen Gemengteil, zum Teil mit jener Zustandsver"anderung der Steinmasse, von der sogleich die Rede sein wird, f"ur ein und dasselbe anzusehen.

Das andere Vorkommen der Steinmasse von ungew"ohnlichem Ansehen findet sich, wie es scheint, nicht minder h"aufig, und wo nicht immer, doch meistens auch bei jenen Steinen, wo obige Adern sich zeigen, so wie sich diese dagegen notwendig immer in irgendeiner Richtung zeigen m"ussen, wo jenes Vorkommen Statt hat. Es besteht dieses aber in einer mehr oder weniger dicken und massiven Schichte oder Lage, gew"ohnlich aber nur in einem "au"serst feinen und d"unnen H"autchen, oft nur zarten, durch die Steinmasse hie und da bisweilen selbst unterbrochenen Anfluge, von einer dichteren, scheinbar fremdartigen Masse, welche in Gestalt von gr"o"seren oder kleineren, ganz unregelm"a"sigen, gar nicht scharf begrenzten Flecken, oder mehr oder minder breiten, bandartigen, oft sehr scharf abgeschnittenen Streifen auf einer Bruchfl"ache erscheinen, und bisweilen dieselbe ganz bedecken, und die --- wie sich oft an derselben Bruchfl"ache, wenn sie gro"s und sehr uneben ist, noch mehr aber an entgegen gesetzten Bruchfl"achen eines gr"o"seren St"uckes zeigt --- ganz nach Art jener Adern und G"ange, und auf "ahnliche Weise hinsichtlich ihrer Ausdehnung in Bezug auf das Innere und die Oberfl"ache des Steines, in verschiedenen, nicht selten sich durchkreuzenden und schneidenden Richtungen durch die Steinmasse durchsetzen.

Es zeigen diese Flecke und Streifen, wenn sie sehr d"unn und zart, zumal anflugartig sind, die gew"ohnlichen Unebenheiten der nat"urlichen Bruchfl"ache des Steines, und ziehen sich gleichf"ormig "uber dieselbe hin"uber; wenn sie aber von einiger Dicke sind, erscheinen sie ebener und glatter, und unterscheiden sich solcher Gestalt auffallend von der "ubrigen Bruchfl"ache des Steines. Im ersteren Falle haben sie gew"ohnlich ein streifiges, bisweilen selbst ein, mehr oder weniger deutlich ausgesprochenes, obgleich unvollkommen schiefriges Ansehen, das selbst die Steinmasse angenommen zu haben scheint; im letzteren aber eine Anlage zu bl"atterigen Abl"osungen; in beiden F"allen endlich bilden sie mehr oder minder vollkommene, nat"urliche Absonderungsfl"achen, oder "ahnliche Stellen, nach welchen sich der Stein auch leicht zu spalten scheint. Letzteres doch nur dann, wenn ein bedeutender Metallgehalt ins Mittel tritt. Die Masse selbst hat im frischen Bruche, entweder, obgleich seltener, ein mattes erdiges, von der "ubrigen Steinmasse, zumal dem olivinartigen Gemengteil im unvollkommeneren Zustande, wenig verschiedenes Ansehen, und eine schiefergraue, ins Schw"arzliche ziehende Farbe, meistens aber, und wie es scheint, bei den an Gediegeneisen reichhaltigern Steinen, ein, wenigstens ganz oberfl"achlich mehr oder weniger metallisches, einiger Ma"sen dem Graphit "ahnliches Ansehen, eine lichter oder dunkler eisengraue Farbe, und einen ziemlich starken, metallischen, fleckweise schimmernden Glanz. Dieser Glanz r"uhrt von wirklich metallischem Eisen her, das an solchen Stellen in d"unnen, zarten Bl"attchen gleichsam angeflogen, indes dasselbe dort, wo diese Masse ein mehr erdiges, den "ubrigen Gemengteilen mehr "ahnliches Ansehen hat, ebenso wie in diesen, k"ornig eingesprengt erscheint; auch zeigen sich, besonders an ersteren Stellen, sehr h"aufige Rostflecke; vom olivinartigen Gemengteile im ausgesonderten, mehr oder weniger kugelicht begrenzten Zustande, konnte ich aber in keiner Art des Vorkommens dieser Masse eine deutliche Spur bisher finden.

Bisweilen erscheint diese Masse, zumal im erdigen Zustande, fleckweise und in kleineren und gr"o"seren, selbst in bedeutenden Partien von ansehnlicher Gr"o"se nach allen Dimensionen, ebenso wie der olivinartige Gemengteil, nur mehr unf"ormlich und nicht so scharf begrenzt, von der "ubrigen wie gew"ohnlich gemengten Steinmasse gleichsam abgeschieden, wie dies ganz besonders ausgezeichnet bei dem merkw"urdigen, noch wenig bekannten Meteor-Steine von Chantonnay der Fall ist. Hier zeigt sich diese Masse, welche beinahe den gr"o"seren Teil der --- jener der meisten Meteor-Steine (zumal jenen von Tabor, Barbotan u. a.) "ubrigens ganz "ahnlich gemengten --- Steinmasse ausmacht, von sehr dichtem, festem Gef"uge, und sehr feinem, nicht unterscheidbarem Korne, von schw"arzlich schiefergrauer Farbe, mattem, nur etwas schimmernden, erdigen, basalt"ahnlichen Ansehen, und ganz derb, im Ganzen von sehr festem Zusammenhange mit der "ubrigen Steinmasse, nur hie und da mit einer schwachen Anlage zur schiefrigen Textur, oder stellenweise zu schiefrigen Abl"osungen, und gleich der "ubrigen Steinmasse mit zarten, stark gl"anzenden Metallteilchen eingesprengt, sonst ganz gleichf"ormig. Und so erscheint dieselbe hier teils in gr"o"seren und kleineren Flecken, teils in schm"alern oder breiteren Adern (ganz und in jeder Beziehung jenen anderer Meteor-Steine, zumal jenen der Steine von Charsonville "ahnlich), teils aber in so gro"sen, frei anstehenden Partien, dass man ansehnliche Bruchst"ucke rein von dieser Masse erhalten kann, die dann der "ubrigen gemengten Steinmasse von gew"ohnlichem Aussehen, dem ersten Anblicke nach, so un"ahnlich sind, wie nur immer ein derbes Basaltst"uck einem feink"ornigen, eisensch"ussigen Sandsteine sein kann, und die wohl niemand, dem blo"sen "au"sern Ansehen nach, f"ur Bruchst"ucke eines Meteor-Steines, am wenigsten aber f"ur solche von diesem Steine erkennen m"ochte, wenn ihm nicht die "Ahnlichkeit dieser Masse mit jener der Adern und G"ange in andern Meteor-Steinen, und das fleck- und partienweise Vorkommen derselben in diesem vorhinein bekannt ist.\footnote{\frakfamily{Ich fand das spezifische Gewicht jenes Teiles der Steinmasse dieses Steines von gew"ohnlichem Aussehen = 3,440 bis 3,480 (das mir unerwartet gering vorkam); jenes des schwarzen Anteiles aber = 3,490 (ein unbedeutender Unterschied, der wohl auch nur, wie bei den Adern im Steine von Charsonville, in dem verschiedenen Grade von Dichtheit beider Massen seinen Grund haben d"urfte). Noch ist von diesem merkw"urdigen Steine keine Analyse bekannt. Vauquelin soll seit lange schon die Absicht gehabt haben, sie vorzunehmen; auch habe ich meinem geehrten Freunde, Hrn. Professor Stromeyer, ein kleines Stuck von beiden Massen dieses Steines zu diesem Ende zugesendet.}}

Wo der Art Schichten, Lagen, Flecke oder Streifen dieser Masse durch Bruch oder Schnitt eines St"uckes in irgendeiner Tiefe quer getroffen werden, m"ussen an der Oberfl"ache notwendig Adern erscheinen, welche die Ausg"ange derselben bezeichnen, und deren M"achtigkeit oder Breite demnach durch die Dicke, und deren Tiefe durch die Ausdehnung jener in die Breite bestimmt wird. Und aus dem, was vorhin von den Eigenheiten jener Adern und G"ange, und der Beschaffenheit ihrer Masse, und hier in denselben Beziehungen von diesem Vorkommen der Steinmasse bemerkt worden ist, und vollends aus den Modifikationen und "Uberg"angen, welche jener Stein von Chantonnay hinsichtlich beider zeigt: scheint sich wohl die Identit"at der Masse in beiden Arten des Vorkommens zu ergeben und die Schlussfolge ziehen zu lassen, dass demselben, so wie dem Hervortreten des olivinartigen Gemengteiles --- der in seiner Wesenheit wohl auch nicht sehr davon verschieden sein m"ochte --- ein und derselbe Bildungs- und Ausscheidungs-Prozess zum Grunde liege, der nur durch das, obgleich nur wenig abweichende quantitative Verh"altnis der entfernteren Bestandteile der Steinmasse im Ganzen, oder etwa durch ver"anderte Nebenumst"ande abge"andert werden, und darnach jene mannigfaltigen Modifikationen veranlassen d"urfte.\footnote{\frakfamily{Es war zur Zeit nicht m"oglich, von dieser merkw"urdigen Zustandsver"anderung der Steinmasse der Meteor-Steine eine befriedigende bildliche Darstellung f"ur gegenw"artige Bekanntmachung zu Stande zu bringen. Sie soll bei einer k"unftigen Veranlassung versucht werden.}}

\subsection{\frakfamily{Stannern.}}
\paragraph{}
Ein, bei 4 Loth wiegendes, auf drei Seiten (den Resten von drei verbrochenen urspr"unglichen Fl"achen) mit Rinde --- von der gew"ohnlichsten Art und Beschaffenheit --- bedecktes, frisches Bruchst"uck eines --- allem Ansehen nach --- urspr"unglich ziemlich gro"s gewesenen Steines von Stannern, welches mit unter denen war, die bei Gelegenheit der abgehaltenen Untersuchungs-Kommission von verschiedenen, gleich anf"anglich in viele St"ucke zerschlagenen Steinen, an Ort und Stelle erhalten, und welches, des ausgezeichneten Mengungszustandes der Steinmasse wegen, f"ur die Sammlung bestimmt wurde.\footnote{\frakfamily{Es sind n"amlich aus einem Vorrate von 93 St"ucken, zusammen an 46 Pfund wiegend, welcher teils unmittelbar bei Gelegenheit der Untersuchung an Ort und Stelle, teils nachtr"aglich durch Vermittlung des k. k. Kreisamtes zu Iglau, und auf andern Wegen von diesem Steinfalle zusammen gebracht wurde, 22 St"uck und mehrere kleine Fragmente, zusammen nahe an 25 Pfund, und zwar eilf ganze, mehr oder weniger vollkommen "uberrindete Steine, und ebenso viele gr"o"sere und mehrere kleine Bruchstucke, f"ur die kaiserliche Sammlung ausgew"ahlt worden, insofern sie bemerkenswerte Abweichungen in der Gro"se und Form, oder in der Beschaffenheit der Rinde und der Steinmasse zeigten.}}

Es zeigt dasselbe im Ganzen den, den Meteor-Steinen von Stannern eigent"umlichen, lockern, ziemlich leicht zerreiblichen Koh"asions-Zustand der Masse,\footnote{\frakfamily{So dass sie beim schw"achsten Versuche, Feuer zu schlagen, zerstiebt, und n"ahert sich hierin, in aufsteigender Progression --- mit Ausnahme der Steine von Alais und Chassigny, die im Ganzen noch lockerer sind --- jener der Steine von Eggenfeld, Mauerkirchen, Benares, Parma, Siena, welche letzteren unter diesen die dichtesten und festesten sind.}} und auf der einen, hier vorgestellten, mit zwei R"andern an die Rindendecke anstehenden, sehr roh und grob erzeugten, frischen Bruchfl"ache insbesondere --- auf etwa $\mathfrak{1\frac{1}{2}}$ Quadrat-Zoll Oberfl"ache --- bei einem sehr unebenen, unbestimmt eckige und ziemlich scharfkantige Bruchst"ucke, andeutenden Bruch --- den gew"ohnlichen, feinen, undeutlich ausgesprochenen und verworrenen, aber ziemlich gleichf"ormigen und innigen Aggregats-Zustand; ferner die eigene, teils br"ocklig-k"ornige, teils gleichsam filzig-faserige Textur von "au"serst feinem Korne, und endlich die, wie gew"ohnlich, im Ganzen ziemlich gleichf"ormig gemengte Steinmasse, von mattem, mehr oder weniger erdigen, rauen, magern, beinahe bimssteinartigen Ansehen, und teils kalkwei"ser, teils bl"aulich- oder perlgrauer Farbe, in welcher die Gemengteile zum Teil so undeutlich ausgesprochen und innig gemengt, wenigstens so gleichf"ormig verteilt sind, dass keiner derselben vorzugsweise als Grundmasse betrachtet werden kann.

Der eine, mehr erdige, lockere und raue Gemengteil, von meistens kalkwei"ser Farbe, welcher aller Analogie nach f"ur die Grundmasse angesprochen werden muss,\footnote{\frakfamily{Sowohl dem "au"sern Ansehen nach, das sich an einigen St"ucken --- wie selbst an diesem --- durch st"arkeres Hervortreten der andern Gemengteiles (welches sich am besten auf polierten Fl"achen zu erkennen gibt) schon deutlich genug ausspricht, als nach den physischen Eigenschaften und chemischen Bestandteilen, in welchen sich derselbe dem gleichnamigen und vollkommen als solchen ausgesprochenen in andern Meteor-Steinen, und zwar stufenweise und nachweisbar --- oft an einem und demselben Stucke --- (wie der n"achst zu beschreibende Stein zeigen wird) n"ahert. Vielleicht hat der gro"se Gehalt an Tonerde (7-9 bis 14 Perzent) und an Kalkerde (9 bis 12 Perzent), und der umso geringere an Talkerde (= 2) --- wodurch sich diese Meteor-Steine so sehr von allen "ubrigen auszeichnen --- die Ausbildung oder Ausscheidung dieses Gemengteiles --- welchem vorz"uglich Kiesel- und Talkerde zukommen --- verhindert.}} zeigt sich teils in pulverigen, fast staubartigen Punkten und sehr kleinen Flecken, teils in kurzen, schmalen, nach allen Richtungen laufenden, filzig-faserigen Streifchen\footnote{\frakfamily{Dem "au"sern Ansehen nach haben diese Streifchen einige "Ahnlichkeit mit einer Art des Vorkommens von Werners Schmelzstein, Dipyre. Auch in den Steinen von Siena, Benares, Parma, zumal aber in jenen von Mauerkirchen und Eggenfeld, zeigt sich die Grundmasse stellen- und partienweise von gleicher Beschaffenheit.}}; der andere, festere, dichtere und mehr glatte Gemengteil dagegen, von lichter und dunkler bl"aulich- oder perlgrauer Farbe (welcher ebenso dem mehr oder weniger kugelichten --- olivinartigen --- Gemengteile anderer Meteor-Steine entspricht), erscheint teils mehr oder weniger innig gemengt, teils mehr oder weniger scharf geschieden, und abwechselnd mit jenem, bald in "ahnlichen, aber festeren und dichteren Punkten, kleinen Flecken, K"ornern und kleinen Massen, bald, obgleich seltener, in "ahnlichen, ebenso beschaffenen Streifchen; und beide Gemengteile so, dass bald der eine, bald der andere von denselben, stellenweise mehr oder minder vorwaltet.\footnote{\frakfamily{Auf geschliffenen und polierten Fl"achen zeigt sich das Gemenge, nach dem verschiedenen Vorwalten des letzteren Gemengteiles, dessen mehr oder minder scharfen Ausscheidung und Begrenzung, Gestaltung und verschiedenen Intensit"at der Farbe, teils Granit- oder Porphyr-teils Marmorartig, und dieser Gemengteil f"allt hier durch seine gr"o"sere Dichte --- die mit der Scharfe der Begrenzung der Massen und mit der Intensit"at der Farbe im Verh"altnisse steht --- noch mehr auf, indem er, und zwar in denselben Graden, eine ziemlich gute Politur annimmt und einen etwas fettigen Glanz zeigt.}}

An der einen Seite der vorgestellten Fl"ache dieses Bruchst"uckes aber erscheint dieser letztere Gemengteil als eine bedeutend gro"se, dreieckige, gleichsam ausgeschiedene, isolierte Masse, obgleich nicht sehr scharf begrenzt, von beinahe lavendelblauer Farbe, und ein ganz "ahnlicher, nur ungleich kleinerer, aber mehr dreieckiger und sch"arfer begrenzter Fleck zeigt sich auf der andern Seite.\footnote{\frakfamily{Ich verweise auf das, was in Hinsicht der beiden erdigen Gemengteile und dieses letzteren olivinartigen insbesondere, in der Einleitung zur Erkl"arung dieser Tafel im Allgemeinen vorgebracht worden ist, und bemerke hier nur noch, dass sich dieser unvollkommene Grad von Ausscheidung und Figurierung desselben ganz genau so, auch bei andern Meteor-Steinen (z. B. bei jenen von Siena, Ensisheim, L'Aigle u. f. w.), und nicht selten in Verbindung mit vollkommeneren Graden von Ausbildung desselben finde. Auch ist bemerkenswert, dass an einem kleinen, etwas "uber 4 Loth wiegenden, beinahe ganzen, mit besonders d"unner, nur wenig und weitzellig-aderiger Rinde bedeckten Steine von Stannern, von welchem ein St"uck abgebrochen worden war --- die ganze Masse ausschlie"slich aus diesem letzteren Gemengteile zu bestehen scheint, indem die ganze, doch bei $\mathfrak{1\frac{1}{2}}$ Quadrat-Zoll betragende Oberfl"ache der quer "uber den ganzen Stein ausgedehnten Bruchfl"ache ein ganz gleichf"ormiges Ansehen hat, und einen festen Koh"asions-Zustand, eine ebenso dichte, "au"serst feink"ornige Textur, und eine licht lavendelblaue Farbe zeigt.}}

Zum Teil mit freiem Auge, mehr aber doch mit H"ulfe einer Lupe, entdeckt man in diesem Gemenge "au"serst zarte, einzelne, matte, schwarze K"orner\footnote{\frakfamily{Auch in dieser Beziehung verweise ich auf das, in Betreff dieses mikroskopischen und unbest"andigen Gemengteiles, oben in der Einleitung Gesagte, und bemerke nur, dass die Menge desselben auch hier nur h"ochst unbedeutend ist, und bei dem durch die Analysen ausgewiesenen Eisengehalte dieser Steine (mit Inbegriff des Schwefeleisens = 27 bis 32 Perzent) kaum in Anschlag gebracht werden kann; dass "ubrigens die Atome davon keinesweges mit Rinde-Substanz verwechselt werden k"onnen.}} (wohl gr"o"sten Teils Eisenoxyd, vielleicht auch Chromeisen), und ebenso zarte, aber hie und da zusammen geh"aufte, mehr oder weniger gl"anzende Metallteilchen von zinkgrauer, teils ins R"otliche, teils ins Gelbliche fallender Farbe (Schwefeleisen),\footnote{\frakfamily{Von diesem Gemengteile finden sich an andern St"ucken dieser Meteor-Steine nicht selten betr"achtliche Partien und Massen (h"aufiger und ungleich gr"o"sere als bei irgendeinem andern, mit Ausnahme jener von Parma, und etwa der von Benares, Mauerkirchen und Lissa) eingemengt, wie bei Beschreibung eines zweiten, auf dieser Tafel dargestellten, und in dieser Beziehung besonders ausgezeichneten Bruchst"uckes, gezeigt werden wird.}} ziemlich h"aufig eingestreut; von regulinischem Eisen findet sich aber an diesem St"ucke, so wie "uberhaupt in den Steinen von Stannern, keine Spur,\footnote{\frakfamily{Dieser Mangel an Gediegeneisen, wodurch sich die Steine von Stannern mit jenen von Chassigny bisher ausschlie"slich (denn von jenen von Alais ist es zweifelhaft, und von jenen von Agen erw"ahnter Ma"sen unrichtig) von allen bisher bekannten Meteor-Steinen auszeichnen, spricht sich auch durch das bedeutend geringere spezifische Gewicht aus (= 3,1 bis 3,2), welches nur bei jenen von Alais noch geringer ist (= 1,9); dagegen jenem der Steine von Benares, Eggenfeld, Parma, Siena, Mauerkirchen, als den, jenen von Stannern in jeder Beziehung n"achst verwandtesten Meteor-Steinen (wo dasselbe zwischen 3,3 und 3,4 schwankt) --- die auch nur einen geringen Gehalt an Gediegeneisen zeigen --- am n"achsten kommt. Bei den meisten "ubrigen Meteor-Steinen steht dasselbe zwischen 3,5 und 3,7. Vom Ensisheimer ist das spezifische Gewicht mit 3,23 zu gering angegeben worden, wie nach der ausgezeichneten Dichtheit der Masse dieses Steines und dem nicht so ganz unbedeutenden Gehalt an Gediegeneisen zu vermuten war, und betr"agt nach eigener Wiegung 3,480 bis 3,490. Eine merkw"urdige Abweichung in dieser Beziehung zeigt die Masse der Steine von Chassigny, deren spezifisches Gewicht --- bei g"anzlichem Mangel an mechanisch eingemengtem Gediegeneisen, und selbst an Schwefeleisen --- nach eigener "Uberzeugung, doch 3,550 betr"agt.) Noch bestimmter "au"sert sich "ubrigens der Mangel an Gediegeneisen bei diesen Steinen von Stannern durch die g"anzliche Unwirksamkeit der Masse sowohl als selbst der Rinde auf die empfindlichste Magnetnadel, die nur von letzterer an einzelnen seltenen Punkten kaum merklich in Bewegung gesetzt wird, und aus der fein gepulverten Masse und Rinde nur "au"serst wenige, einzelne mikroskopische K"ornchen anzieht, die allem Ansehen nach Eisenoxydul sind. Da "ubrigens der Total-Gehalt an Eisen der Steine von Stannern nach den Analysen Mosers, Klaproths und Vauquelins zwischen 27 und 32 Perzent betr"agt, das eingemengte Schwefeleisen im Durchschnitt nach einer oberfl"achlichen Sch"atzung kaum 5 Perzent der Masse, das ebenso vorhandene Oxyd aber kaum so viel betragen kann; so muss der gr"o"ste Anteil des Gehaltes in den erdigen Gemengteilen chemisch gebunden (als Oxyd nach Moser und Vauquelin), oder in irgendeinem Zustande verlarvt enthalten sein.}} und eben so wenig eine Andeutung von Rostflecken, die (wie bereits oben erw"ahnt worden ist), wo nicht ausschlie"slich, doch vorzugsweise das mechanisch eingemengte Gediegeneisen und dessen Umgebung zu begleiten pflegen.

Von Adern und G"angen, oder von einer andern Zustandsverschiedenheit der Steinmasse (von welchen oben in der Einleitung zur Erkl"arung dieser Tafel die Rede war), zeigt sich an diesem St"ucke ebenfalls keine Spur, und "uberhaupt zeigte, unter so vielen gesehenen Bruchst"ucken, nur eines das Vorkommen von ersteren in den Meteor-Steinen von Stannern.

\subsection{\frakfamily{Siena.}}
\paragraph{}
Dasselbe St"uck von dem Steinfalle bei Siena in Italien, welches der ausgezeichneten Form wegen bereits auf der zweiten Tafel von einer andern Ansicht gegeben worden ist, von einer polierten frischen Bruchfl"ache dargestellt, die mit zwei R"andern an die Au"senrinde st"o"st, und, auf etwa 1 Quadrat-Zoll Oberfl"ache, bei vollkommener Abgl"attung, aber etwas matter und ungleichf"ormiger eigentlicher Politur, die innere Beschaffenheit der Steinmasse zu erkennen gibt.

Es zeigt dieselbe einen ziemlich festen Koh"asions-Zustand, der jedoch --- wie eine zweite frische, aber rohe Bruchfl"ache noch besser erkennen l"asst --- ziemlich nahe ans Zerreibliche grenzt, und einen, zum Teil mehr oder weniger feinen, hie und da etwas undeutlich ausgesprochenen, verworrenen, zum Teil aber einen sehr grobbr"ockligen, und sehr auffallend ausgesprochenen, breccieartigen, im Ganzen daher sehr ungleichf"ormigen, aber ziemlich festen Aggregats-Zustand; eine --- abgesehen von dem breccieartigen Gemengteile --- k"ornige Textur von "au"serst feinem Korne, und im Ganzen eine merklich, obgleich nicht sehr stark und etwas ungleichf"ormig, vorwaltende Grundmasse von ganz mattem, erdigen Ansehen, und licht aschgrauer, aber mehr ins Schmutzig- und Gelblich-Graue als ins Bl"auliche ziehender Farbe, welche dem andern Gemengteile, anscheinend, zum Zemente dient.

Sie unterscheidet sich demnach, au"ser der kleinen Verschiedenheit im Koh"asions-Zustande und der Farbe, von jener des vorigen Steines durch das mehr offenbare Vorwalten der Grundmasse, und durch ein, wenigstens zum Teil, deutlicheres Hervortreten des andern (olivinartigen) Gemengteiles.

Dieser erscheint n"amlich hier, teils in eben so verschieden gestalteten und eckigen, mehr oder weniger scharf --- im Ganzen jedoch durchaus sch"arfer --- begrenzten, ganz "ahnlichen Flecken von verschiedener Gr"o"se, derselben Dichtheit und Festigkeit, gleichen, obgleich meistens mehr ins Dunkle bis ins Dunkelblaue und Br"aunlich- und Schw"arzlich-Graue ziehenden Farben-Tingirungen, und "ahnlichem fettigen Glanze, wie die ausgezeichneteren Massen dieses Gemengteiles in jenem Bruchst"ucke, und "uberhaupt in den Steinen von Stannern; teils aber auch schon, wie in den meisten andern Meteor-Steinen, in gr"o"seren oder kleineren, rundlichten oder ovalen Massen von bestimmterer Absonderung und noch gr"o"serer Dichtheit, die demnach auf der rohen Bruchfl"ache unverbrochen, als erhabene K"orner, zum Teil selbst als K"ugelchen erscheinen. Mitunter zeigen sich der Art Massen, selbst schon von einigem Grade von Durchscheinenheit und von gr"unlich-grauer ins Lauchgr"une fallender Farbe, und Graf Bournon und Klaproth bemerkten selbst in Bruchst"ucken von Steinen dieses Herkommens ganz durchscheinende, ja vollkommen durchsichtige K"orner von gelblicher und gr"unlich-gelber Farbe und fast vollkommenem Glasglanze.\footnote{\frakfamily{So dass demnach dieser Gemengteil hier in allen Graden von Ausbildung, Ausscheidung und Absonderung, von dem unvollkommensten, kaum von der Grundmasse unterscheidbaren Zustande, wie bei den Steinen von Stannern \emph{a potiori} (und zum Teil bei jenen von Parma, Ensisheim, L'Aigle u. a.), durch die vollkommeneren Mittelzust"ande, wie \emph{a potiori} bei den Steinen von Benares, Timochin (Tabor, Barbotan, Eichst"adt u. a.), bis zu dem vollkommensten, wie bei manchen andern Meteor-Steinen (\emph{a potiori} aber im sibirischen Eisen), in Steinen von einem und demselben Ereignisse, zum Teil selbst in einem und demselben Bruchst"ucke vorkommt.}}

Von den metallischen, dem bewaffneten, so wie selbst dem freien Auge zwar deutlich erkennbaren, aber nur sparsam erscheinenden Gemengteilen zeigt sich der eine --- das Gediegeneisen --- nur in einzelnen, zerstreuten, meistens "au"serst zarten Punkten oder K"ornern, von licht eisengrauer, ins Silberwei"se fallender Farbe, und starkem metallischen Glanze, und zwar auf der rohen Fl"ache als kleine Zacken, auf der polierten als Punkte oder kleine, "au"serst zart zackig gerandete Fleckchen; der andere --- das Schwefeleisen --- teils in ebenso zarten und zerstreuten einzelnen K"ornern, teils in kleinen Partien feink"ornig, und hie und da zu etwas gr"o"seren Massen br"ocklig angeh"auft, von zinkgrauer, bald ins R"otliche, bald ins Speisgelbe ziehender Farbe und ziemlich starkem metallischen Glanze.\footnote{\frakfamily{Der geringe Gehalt an eingemengtem, regulinischem sowohl als geschwefeltem, Eisen spricht sich "ubrigens sowohl durch das ziemlich niedere spezifische Gewicht (= 3,3 bis 3,4), als durch die "au"serst Wirkung der Steinmasse auf den Magnet aus; inzwischen ist der Total-Gehalt derselben an Eisen nicht unbedeutend, und betr"agt nach Howard bei 35, nach Klaproth etwa 28 Perzent (als durch die Operation erhaltenes Oxyd). Da nun, nach einer oberfl"achlichen Sch"atzung, das sichtlich eingemengte Gediegeneisen kaum 4 bis 6 Perzent, dass ebenso vorhandene Schwefeleisen aber nur wenig mehr betragen d"urfte, vom eingemengten Oxyde sich aber nur wenig Spur findet; so muss ein bedeutender Anteil jenes Gehaltes in den erdigen Gemengteilen chemisch gebunden oder verlarvt enthalten sein.}}

Von mechanisch eingemengtem Oxyde oder "ahnlichen Partikelchen findet sich nur "au"serst wenig, und nur sehr wenige kleine Stellen von schmutzig graulich-gelber, ins Br"aunlich- und R"otlich-Gelbe verlaufender Farbe, geben die Gegenwart von Rostflecken zu erkennen.

Von Adern, G"angen oder einer anderweitigen Zustandsver"anderung der Steinmasse findet sich aber, weder an diesem, noch an irgendeinem der mehreren von mir gesehenen Bruchst"ucke von Steinen dieses Herkommens, auch nur die entfernteste Andeutung.

\subsection{\frakfamily{Benares.}}
\paragraph{}
Ein ausgezeichnetes, $\mathfrak{4\frac{3}{4}}$ Loth schweres Bruchst"uck eines, wahrscheinlich urspr"unglich ziemlich gro"s gewesenen Steines von jenen, welche am 19. Dezember 1798, Abends, bei Krakhut in der N"ahe von Benares in Bengalen gefallen sind, und welches die kaiserl. Sammlung 1807 von dem j"ungst verstorbenen Charles Greville aus London zum Geschenke erhielt.\footnote{\frakfamily{Obgleich dieser Steinfall ziemlich bedeutend und ergiebig war, auch von ans"assigen Engl"andern das Factum gleich an Ort und Stelle untersucht, bekannt gemacht und viele Steine nach Europa versendet wurden; so finden sich doch nur wenige Bruchst"ucke im Besitze bekannter Anstalten oder Sammlungen. So meines Wissens nur im Pariser Museum, im Mus. brit. zu London, in De Drées, Blumenbachs und Klaproths Sammlung, wohin sie wohl s"amtlich durch Greville gekommen sind.}}

Es ist dasselbe von einer der gr"o"seren, rohen Bruchfl"achen dargestellt, welche das Innere der Steinmasse auf einer Ausdehnung von etwa 2 Quadrat-Zoll Oberfl"ache, und auf $\mathfrak{\frac{1}{2}}$, 1 bis $\mathfrak{1\frac{1}{2}}$ Zoll und mehr Entfernung von der "au"sersten mit Rinde bedeckten Oberfl"ache des Steines zeigt.

Der Koh"asions-Zustand der Masse im Ganzen ist nur wenig fester und dichter als bei den Steinen von Stannern, und merklich geringer als bei jenen von Siena. Die Grundmasse f"ur sich ist selbst ziemlich leicht zerreiblich, und zerstiebt beim Versuche, Feuer zu schlagen; "ubrigens ist sie sehr feink"ornig, doch minder so als jene der Steine von Siena. Der Aggregats-Zustand ist ziemlich locker, und bei weitem mehr als bei den Steinen von Stannern und Siena, da die Gemengteile gr"o"sten Teils sehr ungleichartig sind, und der eine sehr ausgeschieden und meistens scharf abgesondert ist; "ubrigens fein sandsteinartig, hinsichtlich des einen; grob k"ornig und kugelicht, hinsichtlich des andern Gemengteiles; und im Ganzen von mandelsteinartigem Ansehen.

Die Grundmasse, die sich, obgleich sie nicht sehr bedeutend "uber die "ubrigen Gemengteile vorwaltet, doch als solche wegen der Ausgeschiedenheit und scharfen Begrenzung dieser, sehr deutlich ausspricht, und gewisser Ma"sen als Zement derselben erscheint --- hat ein ganz mattes, erdiges, raues, mageres Ansehen, und eine sehr licht, nur etwas schmutzig aschgraue, stark ins Wei"se fallende Farbe.

Der olivinartige Gemengteil, der beinahe fast die H"alfte der Steinmasse betr"agt, erscheint hier auf der rohen Fl"ache in Gestalt vieler, mehr oder weniger "uber die Oberfl"ache hervorragender, zum Teil kleiner und sehr kleiner, zum Teil aber auch bedeutend gro"ser (von der Gr"o"se eines Hirsekornes oder kleinen Nadelkopfes bis zu der einer gro"sen Erbse von $\mathfrak{1\frac{1}{2}}$ bis $\mathfrak{2\frac{1}{2}}$ Linie im Durchmesser, und selbst noch mehr), selten stumpfeckiger und blo"s abgerundeter, gew"ohnlich ovaler oder rundlichter, meistens aber vollkommen kugelf"ormiger Massen, wovon die kleineren und die minder scharf begrenzten und weniger kugelicht ausgeschiedenen fester und inniger von der Grundmasse eingeschlossen sind, und gleichsam in dieselbe "ubergehen, die gr"o"seren und vollkommen kugelicht abgesonderten aber bisweilen so lose sitzen, dass sie leicht aus derselben heraus fallen oder ausgebrochen werden k"onnen. Erstere sind gew"ohnlich von Partikelchen der Grundmasse eingeh"ullt, und haben demnach wie diese ein mattes, raues, erdiges Ansehen, und eine gleiche, nur etwas dunklere Farbe; letztere, zumal die vollkommen kugelichten dagegen, haben meistens eine ganz glatte, schwach und etwas fettig gl"anzende Oberfl"ache, und eine schiefer- oder br"aunlich-graue, bisweilen schmutzig lauch- oder olivengr"une Farbe. Gebrochen zeigen erstere zwar ungleich mehr Festigkeit, Dichtheit und H"arte als die Grundmasse, doch bei weitem nicht so sehr wie letztere, welche ziemlich leicht Funken am Stahle geben, und deren scharfkantige Bruchst"ucke selbst etwas das Glas ritzen, oder dasselbe wenigstens matt machen; auch zeigen diese einen vollkommenen, flachmuschelichen Bruch, indes jener der ersteren sich in verschiedenen Abstufungen aus dem erdigen durch den dichten und ebenen nur allm"ahlich demselben n"ahert. Nur wenige, selbst von den ausgeschiedensten, zeigen einigen Grad von Durchscheinenheit an den scharfen Kanten ihrer Bruchst"ucke, alle aber im Bruche und auf einer geschnittenen und polierten Fl"ache --- wo sie mehr oder weniger rissig und zersprungen erscheinen --- nach den verschiedenen Graden ihrer Dichtheit, einen mehr oder weniger fettigen, oder doch schimmernden Glanz, und eine aus dem Grauen ins Lauch- oder schmutzig Olivengr"une und ins Br"aunliche ziehende Farbe. Dort, wo der Art vollkommen kugelichte und scharf ausgeschiedene Massen im Bruche ausgefallen sind, findet sich eine dem Volumen und der Form derselben entsprechende Grube in der Grundmasse, deren W"ande, von "ubrigens mattem, erdigem Ansehen und wei"slich-grauer Farbe, verdichtet und gleichsam gegl"attet erscheinen.\footnote{\frakfamily{So wie diese Steine einerseits durch die Beschaffenheit der Grundmasse --- und in vielen andern Beziehungen --- jenen von Siena (und noch mehr jenen von Mauerkirchen, Parma, Eggenfeld) gleichen; so n"ahern sich dieselben andererseits durch die Art der Ausscheidung sowohl, als durch die Beschaffenheit des olivinartigen Gemengteiles --- wenigstens in den hier einzeln vorkommenden niederen und mittleren Graden --- den meisten "ubrigen Meteor-Steinen, zumal jenen von Timochin (Eichst"adt, Tabor, Barbotan u. v. a.). Nur die besondere Gr"o"se einzelner Massen desselben, und die vollkommene Ausscheidung und Absonderung einiger derselben aus der Grundmasse, ist diesen Steinen ganz eigent"umlich, obgleich sich auch hierin jene von Weston denselben sehr n"ahern.}}

Die Gediegeneisenteilchen zeigen sich beinahe noch sparsamer, aber in etwas gr"oberen K"ornern und Zacken als an den Steinen von Siena, und ebenfalls von licht stahlgrauer, […] Silberwei"se fallender Farbe und metallischem Glanze; die Kies-Partikelchen dagegen zwar ebenso sparsam in zerstreuten, zarten, gl"anzenden, meistens gelblichen K"ornern, h"aufiger aber in gr"o"seren Partien feink"ornig, oder als gr"o"sere Massen br"ocklig (in etwas stumpskantigen, minder spr"oden und leicht zerreiblichen St"ucken) angeh"auft, und mehr von zinkgrauer, etwas ins R"otliche ziehender Farbe und schw"acherem Glanze.\footnote{\frakfamily{Auch hier spricht sich der geringe Gehalt an Gediegeneisen (das kaum 3 Perzent) und an Schwefeleisen (das h"ochstens das Doppelte von jenem betragen m"ochte) durch das geringe spezifische Gewicht (= 3,35) und durch den "au"serst schwachen Magnetismus der Steinmasse im Ganzen aus; doch betr"agt der Total-Gehalt an Eisen auch bei diesen Steinen nach Howard und Vauquelin 34 bis 38 Perzent.}} Von Rostflecken zeigt sich kaum eine Spur (obgleich doch, und zwar schon vor eilf Jahren, eine Fl"ache des St"uckes abgeschliffen und poliert worden war), und eben so wenig von deutlich eingemengtem Oxyde. Auch von Adern und G"angen, oder einer sonstigen Zustandsver"anderung der Steinmasse, findet sich durchaus keine Andeutung an diesem St"ucke.

\subsection{\frakfamily{Timochin.}}
\paragraph{}
Ein charakteristisches St"uck, 4 Loth 3 Qu"antchen wiegend, von dem am 13. M"arz 1807 bei Timochin (im Juchnow'schen Kreise, im Smolensk'schen Gouvernement) in Russland einzeln niedergefallenen, bei 140 Pfund wiegenden Steine,\footnote{\frakfamily{Au"ser Russland d"urften Bruchst"ucke von diesem Steine wohl sehr selten zu finden sein, und au"ser dem Klaproth'schen ist mir nur eines in Blumenbachs, und ein anderes in Chladnis Besitze bekannt.}} welches Klaproth von einem mir zur Ansicht mitgeteilten 18 Loth schweren Bruchst"ucke in seinem Besitze, abschneiden zu lassen gestattete, und der kaiserl. Sammlung gef"alligst "uberlie"s.

Es zeigt dasselbe das Innere der Steinmasse auf einer geschliffenen und polierten, nur an einer Seite an Rinde anstehenden Fl"ache, von $\mathfrak{2\frac{1}{2}}$ Quadrat-Zoll Oberfl"ache.

Der Koh"asions-Zustand der Masse im Ganzen ist nicht viel fester und dichter als bei den Steinen von Benares, aber inniger, wie es scheint, durch Vermittlung der so h"aufig eingemengten, rauen und zackigen Gediegeneisenteilchen, und vorz"uglich der vielen Rostflecke. Die Grundmasse f"ur sich w"are, abgesehen von letzteren, auch wohl etwas zerreiblich; in jenem Zusammenhange gibt sie aber, wahrscheinlich doch nur mittelst des h"aufig vorkommenden olivinartigen Gemengteiles, ziemlich leicht Funken am Stahle. "Ubrigens ist sie nicht sonderlich feink"ornig, weniger beinahe als die der Steine von Benares.

Der Aggregats-Zustand ist, obgleich der olivinartige Gemengteil so h"aufig, und zum Teil eben so scharf begrenzt (aber lange nicht so abgesondert) und kugelicht (aber viel kleiner) ausgeschieden erscheint --- und wahrscheinlich auch durch Vermittlung der Eisenteilchen und Rostflecke --- viel inniger und fester, obgleich lange nicht so, wie bei andern Meteor-Steinen (z. B. jenen von Charsonville, Salés, selbst jenen von Siena, zumal aber jenen von Ensisheim, L'Aigle u. a.), mehr sandsteinartig, von gr"oberem und ungleichf"ormigem Korne, und --- der geringen Menge und Kleinheit der weniger scharf ausgeschiedenen Massen des andern Gemengteiles wegen --- mehr von klein porphyrartigem als mandelsteinartigem Ansehen.

Die Grundmasse, welche hier sehr stark vorwaltet, --- obgleich sie sich, da sie sehr h"aufig und unmerklich in den andern Gemengteil "ubergeht, nur schwach ausspricht --- hat ein ganz mattes und erdiges, aber kein so raues und mageres Ansehen, und eine aschgraue, nur wenig ins Bl"auliche ziehende Farbe.

Der olivinartige Gemengteil, der, insofern er deutlich ausgesprochen erscheint, kaum $\mathfrak{\frac{1}{6}}$ der ganzen Steinmasse betragen m"ochte, zeigt sich auf dieser polierten Fl"ache sehr ungleichf"ormig zerstreut --- aber ziemlich gleichartig, und nicht sehr abweichend in Gr"o"se, Gestalt, Dichtheit, Farbe und Glanz --- in kleinen, sehr und ganz kleinen (selten von $\mathfrak{\frac{1}{2}}$, meistens nur von $\mathfrak{\frac{1}{4}}$ Linie im Durchmesser und noch weniger), meistens rundlichten, selbst auch vollkommen kugelichten K"ornern, von grauer, ins Lauch- und schmutzig Oliven-Gr"une, oder ins Braune ziehender Farbe, und schwachem, fettigem Glanze.

Es sind diese K"orner zwar scharf begrenzt und ausgeschieden, aber bei weitem nicht- so, wie wenigstens viele in den Steinen von Benares (selbst nicht wie manche in jenen von Weston; dagegen genauso wie die meisten in den Steinen von Eichst"adt, Tabor, Barbotan u. a.), aus der Grundmasse abgesondert, sondern innig von derselben eingeschlossen und festsitzend, so dass sie an rohen Bruchfl"achen nie ausgefallen oder ausgebrochen, aber auch nicht verbrochen, und mit rauer, erdiger Oberfl"ache, mehr oder weniger halbkugelicht, hervorragend erscheinen. Sie sind etwas schwer zersprengbar, zeigen einen dichten, ebenen Bruch, der sich mehr oder weniger dem flachmuschelichen n"ahert, und geben unbestimmt eckige, nur wenig scharfkantige, meistens vollkommen undurchsichtige, oder nur schwach an den Kanten durchscheinende Bruchst"ucke.\footnote{\frakfamily{Ihre Beschaffenheit ist in allen Beziehungen dieselbe, wie die der "ahnlichen in den Steinen von Siena, Benares, und vielen andern (und selbst im sibirischen Eisen), einzeln und selten, in vielen andern Meteor-Steinen aber, als jenen von Eichst"adt, Tabor, Barbotan u. a., h"aufig und vorwaltend in diesem Grade von Ausbildung vorkommenden Massen dieses Gemengteiles.}}

Au"ser diesen einzelnen, durch Farbe und Sch"arfe der Begrenzung mehr ausgesprochenen und auffallenden, findet sich aber noch eine Menge "ahnlicher, zum Teil noch weit kleinerer K"orner, die aber nur auf der polierten Fl"ache als Punkte oder kleine und "au"serst kleine Fleckchen zur Ansicht kommen, die sich von der Grundmasse --- mit der sie innig verbunden sind, und in welche sie zum Teil "uberzugehen scheinen --- blo"s durch eine bald etwas lichtere, bald etwas dunklere Farbe, etwas mehr Dichtheit, durch ein feineres Korn und durch ihre Figurierung -- die durch eine mehr oder weniger scharfe, oft kaum merkliche Ausscheidungslinie oder Begrenzung bestimmt wird --- unterscheiden.\footnote{\frakfamily{Von eben der Beschaffenheit, wie dieser Gemengteil wieder einzeln in den meisten Meteor-Steinen, h"aufig und beinahe ausschlie"slich aber in andern (z. B. in jenen von Charsonville, Salés, Berlanguillas, Apt, York, Lissa u. a.) vorzukommen pflegt.}}

Der Gehalt an mechanisch und sichtlich eingemengtem Gediegeneisen ist bei diesen Steinen ausgezeichnet stark, und betr"agt fast 20 Perzent, oder beinahe den f"unften Teil der Steinmasse.\footnote{\frakfamily{Dieser betr"achtliche Gehalt an Gediegeneisen, den Klaproth und N. A. Scherer, nach den Resultaten ihrer Analysen, auf beinahe 18 Perzent angeben, gibt sich auch durch das bedeutende spezifische Gewicht (= 3,700 --- worin diese Steine wohl nur von jenen von Eichst"adt "ubertroffen werden d"urften, und welchem sich jene von Tipperary, Tabor, Charsonville, Toulouse, Erxleben nur zu n"ahern scheinen ---), und durch eine sehr starke Wirkung auf den Magnet zu erkennen. Klaproth gibt "ubrigens noch 25, Scherer $\mathfrak{17\frac{1}{2}}$ Perzent als den Gehalt dieser Steine an oxydiertem Eisen an, dessen Vorhandensein ersterer den sp"ater, durch die Einwirkung unsrer Atmosph"are, entstandenen Rostflecken zuschreibt.}} Die Eisenteilchen erscheinen auf den rohen Bruchfl"achen als einzelne, mehr oder weniger hervorragende, ziemlich starke, raue Zacken und K"orner von eisengrauer Farbe und schwachem metallischen Glanze, insofern sie nicht von erdigen Massenteilchen bedeckt sind. Auf der polierten Fl"ache zeigen sie sich sehr h"aufig und ziemlich gleichf"ormig verteilt, als mehr oder weniger zarte Punkte, als gr"o"sere oder kleinere, meistens gezackte Flecke, und als mehr oder weniger gebogene, "astige und zum Teil zusammenh"angende Linien und Adern, von sehr licht stahlgrauer, ins Silberwei"se ziehender Farbe, und ziemlich starkem metallischen Glanze. Dagegen ist der Gehalt an Schwefeleisen h"ochst unbedeutend, und selbst auf der polierten Fl"ache kann man nur "au"serst zarte, mikroskopische Punkte, die hie und da zu kleinen Flecken angeh"auft sind, und sich durch eine Zinkwei"se, etwas ins Gelbliche oder R"otliche fallende Farbe, und einen etwas schw"acheren Glanz auszeichnen, daf"ur erkennen. Besonders h"aufig aber zeigen sich die Rostflecke, so dass man sie nach Chladni allerdings f"ur diese Steine (aber ebenso f"ur die Steine von Eichst"adt, Charsonville, Barbotan u. e. a.) als charakteristisch ansehen kann, indem sie beinahe die H"alfte der Steinmasse ausmachen, und derselben ein ganz eigent"umliches marmoriertes Ansehen geben. Sie sind "ubrigens hier sehr klein, zart, matt, erdig, und von besonders dunkler gelblich-brauner Farbe.

Von Oxyd oder "ahnlichen Partikelchen zeigt sich keine deutliche Spur; eben so wenig von Adern und G"angen oder einer andern Ver"anderung der Steinmasse.

\subsection{\frakfamily{Charsonville.}}
\paragraph{}
Ein gro"ses, 1 Pfund schweres St"uck von einem der am 23. November 1810 in der Gegend von Charsonville bei Orléans (Departement du Loiret) in Frankreich niedergefallenen Steine, welches w"ahrend meiner Anwesenheit in Paris (1815) auf mein Ansuchen und mit Genehmigung der k"oniglichen Administration des Museums der Naturgeschichte, von einem daselbst aufbewahrten Bruchst"ucke,\footnote{\frakfamily{In dem, dem Werke Chladnis angeschlossenen Verzeichnisse der Meteor-Massen der kaiserl. Sammlung, ist aus Versehen dieses Bruchst"uck als ein ganzer Stein angegeben worden. Aus Bigot de Morogues verl"asslichen Nachrichten "uber diesen Steinfall ergibt sich aber, dass dasselbe selbst nur ein Bruchst"uck, und zwar von dem einen gr"o"seren der niedergefallenen und aufgefundenen Steine war, dessen Gewicht bei 40 Pfund betrug, welches D. Pellieux zu Baugenci an den damaligen Minister des Innern (Grafen Montalivet) einsendete, von welchem dasselbe an das k"onigl. Museum abgegeben wurde.}} von 11 Pfund am Gewichte, abgeschnitten, und mir, nebst mehreren andern, f"ur die kaiserliche Sammlung gef"alligst mitgeteilt wurde.\footnote{\frakfamily{Obgleich dieser Steinfall hinsichtlich der Zahl der gefallenen Steine nicht sehr betr"achtlich war, indem deren nur drei im Falle beobachtet, und davon selbst nur zwei aufgefunden wurden; so geh"ort er doch der Masse nach zu den bedeutenderen, da der eine der aufgefundenen Steine bei 40, der andere 20 Pfund wog. Indessen ist mir au"ser obigem geteilten Bruchst"ucke nur noch eines in De Drées, und ein zweites in Chladnis Besitze bekannt.}}

Es ist dasselbe, auf der zum Teil mit Rinde bedeckten, zum Teil verbrochenen, gew"olbten Au"senseite liegend, von der durch den Schnitt erhaltenen, ganz ebenen, aber noch unpolierten Fl"ache dargestellt, die das Innere der Steinmasse auf einer Ausdehnung von ungef"ahr 4 Quadrat-Zoll, und, wo am dicksten, in einer Tiefe von beinahe $\mathfrak{1\frac{1}{2}}$ Zoll von der "au"sern Oberfl"ache des Steines, zur Ansicht bringt.

Der Koh"asions-Zustand der Masse ist sehr fest und dicht, so dass sie sich hierin den kompaktesten und h"artesten Meteor-Steinen (jenen von Ensisheim, Erxleben, Chantonnay) n"ahert, indes sie doch nur etwas schwer Funken gibt. Der Aggregats-Zustand ist ebenfalls sehr fest und innig, und dabei auch sehr gleichf"ormig --- da der olivinartige Gemengteil "au"serst wenig, nur h"ochst unvollkommen und schwach ausgeschieden, und von der Grundmasse in allen Beziehungen nur wenig abweichend, und selbst sehr gleichf"ormig erscheint --- und dicht sandsteinartig, von "au"serst feinem, sehr gleichf"ormigen Korne.

Die Grundmasse, welche hier besonders stark vorwaltet, und abgesehen von den eingemengten Metallteilchen, und ohne Lupe betrachtet, bis auf wenige Massen, in welchen sich der andere Gemengteil etwas deutlicher ausspricht, beinahe die ganze Steinmasse zu konstituieren scheint, indem sie gr"o"sten Teils allm"ahlich und sehr unmerklich in jenen "ubergeht --- hat ein ganz mattes, erdiges, aber, selbst auf rohen Bruchstellen, eben kein sehr raues noch mageres Ansehen, und eine aschgraue, nur wenig ins Bl"auliche ziehende Farbe.

Der olivinartige Gemengteil erscheint darin nur sehr schwach und undeutlich ausgesprochen, in sehr sparsamen, einzelnen, zerstreuten, sehr und "au"serst kleinen, oft kaum merklich ausgeschiedenen, oder doch nur sehr schwach begrenzten, meistens rundlichen oder ovalen, doch auch stumpfeckigen K"ornern, von mattem, erdigen Ansehen, und licht aschgrauer, gelblicher, bl"aulicher, nur selten br"aunlicher Farbe. Die meisten dieser Massen unterscheiden sich blo"s durch etwas gr"o"sere Dichtheit, Feinheit im Korne, und durch ihren Umriss von der Grundmasse, und gleichen zum Teil vollkommen jenen, welche in dem zuvor beschriebenen Steine von Timochin in ziemlicher Menge, einzeln aber in den meisten Meteor-Steinen, und zwar gemeinschaftlich mit andern vorkommen, die in verschiedenen und weit h"oheren Graden von Ausbildung und Ausscheidung sich befinden. Nur sehr wenige davon zeigen sich an den rohen Bruchstellen als vollkommen ausgeschieden oder abgesondert von der Grundmasse, in Kugel- oder K"ornerform, mit vorragender konvexer Oberfl"ache; die meisten sind mit der Grundmasse zugleich gebrochen, und zeigen nur einen dichteren, ebeneren Bruch.\footnote{\frakfamily{Wenn das quantitative Verh"altnis der n"achsten Bestandteile von mehreren Meteor-Steinen mit Verl"asslichkeit angegeben w"are; so lie"se sich vielleicht --- wie bereits oben erw"ahnt worden ist --- mit einiger Gewissheit nachweisen, dass in demselben und nicht in blo"sen Zustandsver"anderungen der Steinmasse, der n"achste Grund der ebenso auffallenden als mannigfaltigen Abweichungen in der Menge, Beschaffenheit und in der Art der Ausscheidung und Absonderung dieses Gemengteiles liege, wie dies zum Teil aus den vorhandenen Analysen hervor zu gehen scheint. Von den meisten Meteor-Steinen n"amlich, in welchen dieser Gemengteil nur schwach und unvollkommen ausgesprochen ist (wie z. B. in jenen von Stannern, Parma, Charsonville, Doroninsk, L'Aigle, Ensisheim), weisen jene einen verh"altnism"a"sig geringeren Gehalt an Talkerde (n"amlich zwischen 2 und 13 Perzent), und dabei einen nicht ganz unbedeutenden Gehalt an Thon- und Kalkerde (von ersterer 3 bis 9, von letzterer 4 bis 12 Perzent) aus; von jenen dagegen, wo derselbe h"aufiger, deutlich ausgesprochen, oder in einem besonders hohen Grade von Ausbildung, oder vollends vorwaltend erscheint (wie in jenen von Eichst"adt, Tabor, Benares, Eggenfeld, Erxleben, Chassigny), einen weit gr"o"seren Gehalt an Talkerde (17 bis 21; 23; 26 bis 32 Perzent), aber keine Spur, oder doch nur "au"serst wenig ($\mathfrak{1\frac{1}{4}}$ und $\mathfrak{\frac{1}{2}}$ Perzent von jenen von Erxleben), an Thon- und Kalkerde. Die Steine von Timochin hielten in beiden Beziehungen gerade das Mittel. (Klaproth gibt deren Gehalt an Talkerde auf $\mathfrak{14\frac{1}{4}}$, von Thon auf 1, und von Kalkerde auf $\mathfrak{\frac{3}{4}}$ Perzent an.)}}

Die Gediegeneisenteilchen werden durch ihre Menge und zum Teil durch ihre Beschaffenheit charakteristisch f"ur diese Steine. Sie erscheinen n"amlich "au"serst h"aufig --- so dass ihre Masse zusammen genommen, nach einer oberfl"achlichen Absch"atzung, gut den vierten Teil des Ganzen betragen m"ochte\footnote{\frakfamily{Bigot de Morogues sch"atzt den Gehalt auf 31 Perzent. Vauquelin gibt den Gehalt des von ihm analysierten St"uckes im Ganzen mit 25,8 als regulinisch an (nach Kalk"ul, denn er hatte nach seinen Verfahren alles Eisen daraus als Oxyd im \emph{maximum}, also etwa 36 Perzent erhalten). Es ergibt sich hieraus, dass der Total-Gehalt dieser Steine an Eisen eben nicht gr"o"ser ist, als bei den meisten Meteor-Steinen, und dass, da sich dieser Gehalt, dem "au"sern Ansehen nach, schon in dem mechanisch eingemengten Gediegeneisen ausspricht, in diesen Steinen wenig oder gar nichts oxydiert, vererzt oder sonst verlarvt enthalten sein k"onne. Der starke Gehalt an Gediegeneisen bew"ahrt sich "ubrigens nicht nur durch das spezifische Gewicht (das --- im Durchschnitt und mit Hinsicht auf Adern und Rinde --- zwischen 3,6 und 3,7 f"allt), sondern auch durch sehr starke Wirkung der Steinmasse auf den Magnet.}} --- h"ochst unregelm"a"sig zwar, aber doch ziemlich gleichf"ormig, und im Ganzen sehr dicht eingestreut, und auf dieser geschnittenen Fl"ache als etwas erhabene, "au"serst zarte Punkte von licht eisengrauer Farbe und etwas mattem metallischen Glanze, die hin und wieder zusammen geh"auft und gewisser Ma"sen zusammen geflossen, mehr oder weniger Adern gleichende, nur selten und wenig zusammen h"angende, gezackte, gek"ornte und gleichsam getr"aufte, kleine Flecke oder Massen bilden, welche sich mit einem st"ahlernen Instrumente sehr leicht breit und platt dr"ucken und ritzen lassen, und dann (wie an den rohen Bruchstellen) einen h"oheren metallischen Glanz und eine stark ins Silberwei"se fallende Farbe zeigen.

Von Kiesteilchen findet sich dagegen nur wenig Spur in "au"serst zarten Punkten, von etwas st"arkeren metallischem Glanze, und einer aus dem Wei"sen ins Messinggelbe ziehender Farbe, und noch weniger von Oxyden oder "ahnlichen Partikelchen; umso h"aufiger erscheinen aber die Rostflecke, die durch ihre Menge sowohl --- da sie der ganzen Oberfl"ache ein zart marmoriertes Ansehen geben --- als durch ihre Zartheit und Farbe --- indem sie meistens als einzelne, "au"serst feine Punkte, die nur stellenweise in Flecke zusammen geflossen sind, und von einer eigenen graulich-gelblichen Farbe erscheinen --- ebenfalls als charakteristisch f"ur diese Steine angesehen werden k"onnten, insofern sie nicht sp"aterhin und zuf"allig entstanden sind.\footnote{\frakfamily{Es ist bemerkenswert, dass die Rostflecke an diesem St"ucke in einem Zeitraume von f"unf Jahren, w"ahrend welchem dasselbe der Luft, dem Lichte und selbst h"aufiger Betastung ausgesetzt war, sich gar nicht merklich vergr"o"sert, vermehrt, noch in irgendeiner Beziehung ver"andert haben.}}

Das Merkw"urdigste an diesem Steine, und weshalb auch dessen bildliche Darstellung versucht wurde, sind die Adern und G"ange von einer, scheinbar, fremdartigen Substanz, welche auf dessen Oberfl"ache erscheinen und die Steinmasse durchziehen, von welchen bereits oben in der Einleitung zur Erkl"arung dieser Tafel im Allgemeinen gesprochen wurde, und die bei diesen Steinen, zwar gerade nicht am h"aufigsten (denn ungleich h"aufiger zeigen sie sich bei jenen von Agen und Lissa), aber durch St"arke und Ausdehnung am ausgezeichnetsten vorkommen.

Es zeigen sich auf der geschnittenen Fl"ache dieses St"uckes zwei solche Adern.\footnote{\frakfamily{Bigot de Morogues, welcher Gelegenheit hatte, Bruchst"ucke von beiden aufgefundenen Steinen zu untersuchen, bemerkte in dem einen zwar viele, aber "au"serst zarte, dem freien Auge kaum sichtbare Adern, in dem andern mehrere, aber durchaus st"arkere, und darunter eine von 1 bis 3 Linien in der Breite oder M"achtigkeit, und von sehr abweichender Dicke oder Tiefe. Hauy und Vauquelin haben an dem gro"sen Bruchst"ucke des Museums, welches mit letzterem St"ucke Bigots von demselben Steine herstammt, nur eine Ader bemerkt, indes an dem hier beschriebenen, unmittelbar von ersterem abgeschnittenen St"ucke, deren zwei vorkommen.}} Die eine davon geht von einem Rande des St"uckes etwas schief quer "uber die Fl"ache zum andern, die aber beide nicht die Grenzen des urspr"unglichen Steines und dessen Oberfl"ache bezeichnen, indem sie verbrochen und rindenlos sind. Sie ist an einem Ende bei $\mathfrak{\frac{5}{4}}$ Linien breit, verschm"alert sich allm"ahlich, und l"auft gegen das andere beinahe haarfein aus. Im Verlaufe macht sie nur einige schwache und kleine Biegungen, und erscheint bald breiter, bald schm"aler, so dass sie an einigen Stellen $\mathfrak{\frac{1}{2}}$, gleich unmittelbar darauf schnell abnehmend, kaum $\mathfrak{\frac{1}{4}}$ Linie breit ist, zeigt aber nur einen einzigen, zarten Seitenzweig im ersten Drittel ihres Laufes, der unter einem ziemlich spitzen Winkel von ihr ausgeht, schief vor- und aufw"arts steigt, und sich sehr bald haarfein in die Steinmasse verl"auft. In derselben Gegend zeigt sich ein ebenso zarter, aber unausgef"ullter, leerer Riss oder Sprung in der Steinmasse, der quer vom Rande herkommt, und sich nahe an der Hauptader verliert, ohne mit ihr in Ber"uhrung zu kommen; ein zweiter "ahnlicher zeit sich am andern Ende derselben, der eine Strecke weit schief einw"arts geht. An beiden R"andern des St"uckes, wo diese Ader ausgeht, und wo absichtlich ein kleines St"uck abgeschlagen wurde, um den Verlauf in die Tiefe zu verfolgen, zeigt sich, dass diese Ader eine an Breite oder M"achtigkeit den beiden Ausg"angen entsprechende Lage bezeichnet, die in schiefer Richtung (unter einem Winkel von etwa 60$^{\circ}$ gegen die Oberfl"ache) die Steinmasse auf eine Tiefe von einem halben Zoll durchsetzt.

Die zweite Ader geht von demselben Rande aus, weicht aber im Verlaufe von jener ab, und zieht ebenfalls etwas schief und quer "uber die Fl"ache gegen einen andern Rand hin, wo wirklich von Au"sen Rinde ansteht, in welche sie sich verl"auft. Sie ist beinahe durchaus im ganzen Verlaufe haarfein, nur in ihrer Mitte bildet sie gleichsam einen ovalen Wulst oder Botzen (2 Linien lang, 1 Linie breit), der durch einen quer aus der Mitte der Fl"ache herkommenden schwachen Riss etwas zerkl"uftet ist --- und erscheint zwei Mahl etwas bogenf"ormig in entgegen gesetzten Richtungen geschwungen. Sie zeigt wohl hin und wieder eine Spur von Seitenzweigen, die von ihr unter verschiedenen Winkeln und in verschiedenen Richtungen ausgehen, und gegen einen Rand hin oder in die Steinmasse verlaufen --- sie sind aber mikroskopisch fein, so wie eine "ahnliche Ader, die in geringer Entfernung von dieser, und fast in paralleler Richtung mit ihr, frei mitten auf der Fl"ache eine Strecke fortl"auft; dagegen findet sich ein Netz von "ahnlichen Adern, gegen den einen Rand des St"uckes, die teils von diesem, teils von jener Hauptader ausgehen, und ebenso wechselseitig gegen einander sich verlaufen, unter sich verzweigen, einm"unden, und verschiedentlich sich durchschneiden und kreuzen.

Alle diese Adern zeigen, sowohl auf der geschnittenen Fl"ache als an den, dieser entgegen gesetzten, frischen Bruchstellen, eine matte, schw"arzlich- bl"aulich- oder dunkel schiefer-graue Farbe, durch welche allein sie sich von der "ubrigen Steinmasse unterscheiden. Die Substanz selbst ist gar nicht fremdartig, durch gar nichts von jener getrennt, sondern blo"s durch die Farbe, durch diese aber scharf von ihr geschieden; im Gegenteil ist die Verbindung und der Zusammenhang mit derselben sehr fest und innig, so zwar, dass die Steinmasse beinahe leichter quer "uber als an und in der Richtung dieser Adern bricht, zumal wenn sie von einiger Dicke sind. Die Unebenheiten jener setzen sich ununterbrochen und in derselben Richtung "uber diese fort; der Bruch ist ganz derselbe, nur etwas dichter, und an einer, obgleich nur kleinen Stelle der breiteren Ader, zeigt sich eine Spur von unvollkommen schiefriger Textur, in perpendikul"arer, aber etwas schiefer und gekr"ummter Richtung. Es wirkt diese Ader-Substanz "ubrigens etwas st"arker als die "ubrige Steinmasse, aber doch schw"acher als die Au"senrinde, auf die Magnetnadel, auch ist dicht an ihr und mitten in ihr, ebenso wie in der ganzen Masse, Gediegeneisen eingesprengt. Mit der Rinde des Steines hat sie weder der Farbe, noch weniger der Textur und "ubrigen Beschaffenheit nach, die geringste "Ahnlichkeit. Von einer anderweitigen, dieser Substanz mehr oder weniger "ahnlichen Beschaffenheit der Steinmasse, von Absonderungsfl"achen oder metallischem Anfluge zeigt sich an diesem St"ucke keine deutliche Spur.\footnote{\frakfamily{An einem kleinen St"ucke, dass ich selbst besitze, findet sich eine Absonderungsfl"ache mit metallischem, graphit"ahnlichen Anfluge, ganz von der Art, wie an den Steinen von York, Sigena, Laponas \emph{zc.}}}

\subsection{\frakfamily{Salés.}}
\paragraph{}
Ein charakteristisches St"uck, $\mathfrak{2\frac{1}{2}}$ Loth schwer, von dem am 12. M"arz 1798 bei Salés (nicht weit von Ville Franche, Departement du Rhone) in Frankreich\footnote{\frakfamily{Der verz"ogerten Bekanntwerdung des Factums, die wir den sp"ateren, eifrigen Nachforschungen des Marquis De Drée verdanken, und der Unbedeutendheit der niedergefallenen Masse ist es zuzuschreiben, dass nur mehr wenige Fragmente davon nachweisbar vorhanden sind, wovon sich eines im Mus. brit. zu London, aus Grevilles Verm"achtnis, und "ahnliche in De Drées, Blumnenbachs und Chladnis Besitze sich befinden.}} einzeln gefallenen Steine, der ungef"ahr 20 bis 25 Pfund wog, welches die kaiserl. Sammlung der gef"alligen Mitteilung des Marquis De Drée verdankt.

Es ist dasselbe von einer der gr"o"seren, abgeschliffenen Fl"achen dargestellt, die das Innere der Steinmasse auf einem Fl"achenraume von etwa $\mathfrak{1\frac{1}{4}}$ Quadrat-Zoll, und auf wenigstens $\mathfrak{1\frac{1}{2}}$ Zoll Entfernung von der "au"sersten Oberfl"ache des Steines zeigt, wo n"amlich an einer Seite Rinde ansteht.

Der Koh"asions-Zustand ist beinahe eben so dicht und fest, wie am Steine von Charsonville; die H"arte der Steinmasse im Ganzen doch bedeutend geringer, da sie nur schwer und schwach Funken gibt. Der Aggregats-Zustand ist zwar (des schon etwas h"aufiger und zum Teil mehr ausgesprochenen olivinartigen Gemengteiles wegen) im Ganzen gr"ober, doch beinahe eben so dicht und innig; die Textur von ebenso feinem und gleichf"ormigen Korne, beinahe noch in einem h"oheren Grade, und die ziemlich stark vorwaltende, aber im Ganzen nur wenig durch die Gemengteile herausgehobene Grundmassen von mattem, erdigem Ansehen, und von licht aschgrauer, beinahe gar nicht ins Bl"auliche fallender Farbe.

Der olivinartige Gemengteil erscheint darin weit h"aufiger als im Steine von Charsonville, und teils, und zwar gr"o"sten Teils, in ganz "ahnlichen, ebenfalls nur schwach und undeutlich ausgesprochenen, sehr kleinen, schwach begrenzten und innig mit der Grundmasse verbundenen, runden, ovalen, mitunter auch stumpfeckigen K"ornern und Mandeln von mattem, erdigem Ansehen, und licht und dunkler aschgrauer, mehr oder weniger ins Bl"auliche ziehender Farbe, die dem Ganzen ein schwach porphyrartiges Ansehen geben; teils aber auch, obgleich in einem nur geringen Verh"altnisse, in einzelnen, kleinen und gr"o"seren, scharf ausgeschiedenen und begrenzten (zum Teil selbst durch eine zarte, vertiefte Linie von der Grundmasse abgesonderten), meistens vollkommen kugelichten (ganz jenen ausgesprochenern im Steine von Timochin und vielen von jenen im Steine von Benares "ahnlichen) K"ornern, von dunkel bl"aulichgrauer, ins Lauchgr"une ziehender Farbe, etwas fettigem Glanze, gr"o"serer Dichtheit, H"arte, rissiger Oberfl"ache u. s. w., die auch auf den rohen Bruchfl"achen als insitzende K"ugelchen mit hervorragender konvexer Oberfl"ache, auch wohl schon ausgebrochen, erscheinen.

Der Gehalt an Gediegeneisen zeigt sich dagegen ungleich geringer als am Steine von Charsonville\footnote{\frakfamily{Ich fand das spezifische Gewicht eines kleinen, rindelosen, und, nach m"oglichst genauer Pr"ufung, von gr"o"seren Gediegeneisenteilchen ganz freien St"ucke = 3,434; da nun aber das in gr"o"seren Massen zerstreut eingemengte Gediegeneisen im Ganzen bald mehr betragen d"urfte, als das zart eingesprengte zusammen genommen, und ersteres demnach auf die ganze Steinmasse verteilt werden m"usste; so m"ochte das das spezifische Gewicht wohl zwischen 3,5 und 3,6 anzusetzen sein, welchem auch der wahre Total-Gehalt an Gediegeneisen, nach oberfl"achlicher Absch"atzung (= etwa 0,08 bis 0,10) entspr"ache. (Vauquelin erhielt bei der Analyse 38 Perzent als Oxyd.) Abgesehen von den gr"o"seren Eisenteilchen ist die Wirkung der Steinmasse auf den Magnet auch nur schwach, ebenso wie bei den Steinen von Lissa, st"arker jedoch als bei jenen von Siena und Benares.}} (Timochin u. v. a.), und die Eisenteilchen erscheinen gr"o"sten Teils --- au"ser in eben nicht sehr h"aufig eingestreuten, zarten Punkten und K"ornern --- von seltenerer Art des Vorkommens, n"amlich in betr"achtlicheren Massen, die auf der polierten Fl"ache als unregelm"a"sig gestaltete, eckige, zum Teil gezackte und klein"astige, scharf begrenzte, aber fest eingeschlossene Flecke von licht eisengrauer, stark ins Silberwei"se fallender Farbe, und mit starkem metallischen Glanze sich zeigen, und wovon einer der gr"o"seren hier, von ovaler, etwas keilf"ormiger Gestalt, 2 Linien in der L"ange, und $\mathfrak{1\frac{1}{2}}$ in der gr"o"sten Breite mi"st.\footnote{\frakfamily{De Drée fand in einem St"ucke dieses Steines ein 24 Gran wiegendes Korn von Gediegeneisen.}}

Kiesteilchen lassen sich nur "au"serst wenige, h"ochst zart eingesprengt und feink"ornig angeh"auft, auf der polierten Fl"ache durch eine mattere, aus dem Zinkgrauen etwas ins R"otliche stechende, auf den rohen Bruchfl"achen aber durch eine gl"anzendere, und mehr ins Gelbe ziehende Farbe von jenen unterscheiden.\footnote{\frakfamily{Es ist dieser Kies sehr spr"ode, leicht zersprengbar, und l"asst sich sehr leicht zum feinsten Pulver zerreiben, zeigt sich aber auch als solches ganz ohne Wirkung auf die Magnetnadel.}} Von Oxydk"ornern zeigt sich keine Spur, und von Rostflecken nur "au"serst wenig. Zarte, mikroskopisch feine, schw"arzliche Adern durchziehen die Masse nach allen Richtungen, ohne doch die R"ander, selbst dieser kleinen Fl"ache, zu ber"uhren; von Absonderungsfl"achen oder einem metallischen Anfluge findet sich aber an diesem St"ucke sonst keine weitere Andeutung.

\subsection{\frakfamily{Stannern.}}
\paragraph{}
Ein $\mathfrak{13\frac{1}{2}}$ Loth schweres Bruchst"uck von demselben gro"sen, urspr"unglich bei 4 Pfund schwer gewesenen Steine von Stannern, von welchem, durch Zerschlagen der davon erhaltenen H"alfte, auch das oben beschriebene und Fig. 5 der vorigen Tafel abgebildete St"uck erhalten worden war.

Dieses Bruchst"uck --- von welchem hier des Raumes wegen nur ein Teil vorgestellt ist --- zeigt auf seiner ganzen, bedeutend gro"sen, rohen Bruchfl"ache von 5 Quadrat-Zoll Ausdehnung, an allen R"andern an Rinde ansto"send, das gew"ohnliche, sehr zarte und feine, und hier ganz besonders gleichf"ormige Gemenge der beiden erdigen Gemengteile von ganz gleicher Textur und Beschaffenheit, nur dass sich der olivinartige etwas durch Farbe und gr"o"sere Dichtheit unterscheidet, ohne sich jedoch durch eine bestimmtere Form oder sch"arfere Begrenzung auszuzeichnen.

Das Merkw"urdige an diesem St"ucke ist der ausgezeichnete Gehalt an Schwefeleisen. Es ist dasselbe hier nur wenig in zarten Punkten und K"ornern eingestreut, dagegen an mehreren Stellen in betr"achtlichen Massen eingemengt. Eine solche fast viereckige von $\mathfrak{\frac{1}{4}}$ Zoll Ausdehnung zeigt sich, und zwar ganz dicht, kaum auf 1 Linie Entfernung von der anstehenden Rinde an dem einen Rande, zerkl"uftet und in unregelm"a"sige, unbestimmt eckige, ziemlich scharfkantige Bruchst"ucke zersprungen und br"ocklig angeh"auft, von k"orniger Textur, ziemlich dunkelgrauer, wei"s schimmernder, ins R"otliche stechender Farbe, und mit schwachem metallischen Glanze. An einer andern Stelle, ganz dicht an der Rinde, findet sich eine kleinere Masse, die zum Teil wie geschmolzen aussieht, von pfauenschweifigem Farbenspiele und etwas st"arkerem Glanze.
\clearpage
\section{\frakfamily{Achte und neunte Tafel.}}
\paragraph{}
Der Zweck der bildlichen Darstellungen dieser Tafeln ist die Versinnlichung des merkw"urdigen kristallinischen Gef"uges der vorz"uglichsten Gediegeneisen-Massen, deren meteorischer Ursprung teils faktisch erwiesen, teils h"ochst wahrscheinlich, ja unbezweifelbar ist, und deren Untersuchung in jener Beziehung mir bisher m"oglich war.\footnote{\frakfamily{Es ist die Entdeckung dieser Eigent"umlichkeit des Gediegeneisens, wahrhaft meteorischen Ursprunges, schon seit mehreren Jahren ziemlich bekannt; denn Herr Direktor v. Widmannst"atten machte sie bereits im Jahre 1808 bei Gelegenheit der ersten physisch-technischen Versuche, die er mit der Agramer Eisenmasse vornahm, und wir waren weit entfernt sie geheim zu halten, im Gegenteile ward dieselbe allen Wissenschaftsfreunden gelegenheitlich mitgeteilt, und jene Masse, an welcher (wie bereits oben erw"ahnt wurde) eine bedeutende Fl"ache ge"atzt worden war, um das Gef"uge darzustellen, bleib nach wie vor, und zwar seit 1809, mit den "ubrigen vorhandenen Meteor-Massen und der zahlreichen Suite von ausgew"ahlten St"ucken vom Steinfalle zu Stannern vereinigt, und als eine f"ur sich bestehende Sammlung abgeschlossen, am kaiserl. Mineralien-Kabinette zur "offentlichen Ansicht ausgestellt. Noch in demselben Jahre hatte Herr v. Widmannst"atten Gelegenheit, an einem ausgezeichnet sch"onen Ladenst"ucke vom sibirischen Eisen aus der Von der Null'schen Sammlung --- deren sachverst"andiger Besitzer sich sehr bereitwillig fand den Schnitt wund Schliff dieses kostbaren St"uckes zu gestatten, da es damit von der andern Seite ein h"oheres Interesse gewann; --- im Jahre 1810 aber an dem St"ucke vom Mexikaner Eisen, welches die kaiserl. Sammlung eben durch Klaproth erhalten hatte; dann im Jahre 1812 an der gro"sen Gediegeneisen-Masse, welche vom Magistrate zu Elbogen in B"ohmen an das kaiserl. Naturalien-Kabinett abgegeben wurde; endlich 1815 an dem St"ucke vom karpatischen Eisen, welches Herr Baron v. Brudern dem kaiserl. Kabinette zum Geschenke machte --- jene interessante Entdeckung zu bew"ahren. Da sich jenes Gef"uge auf ebenen und polierten Fl"achen bei der Behandlung durch "Atzung in tastbaren, und zwar nach Ma"sgabe der Dauer des Prozesses, in mehr oder weniger erhabenen und vertieften Figuren (\emph{en basrelief}) ausspricht; so kam Herr v. Widmannst"atten gleich Anfangs, bei der Agramer Masse schon, auf die gl"uckliche Idee, durch unmittelbare Abdr"ucke solcher Fl"achen mittelst Druckerschw"arze --- die Masse selbst gleich als nat"urliche Form oder Stereotyp ben"utzend --- eine vollkommen getreue und leicht vervielfachbare Darstellung zu bewirken, und der gute Erfolg dieses Verfahrens veranlasste uns 1813, von der gro"sen ge"atzten Fl"ache der Elbogner Masse, welche das Gef"uge besonders sch"on und deutlich zeigte, solche unmittelbare Abdr"ucke in hinl"anglicher Menge abziehen zu machen, um sie als Belege zu einer Abhandlung zu gebrauchen, die wir damals schon "uber diesen Gegenstand auszuarbeiten und bekannt zu machen dachten. Allein Zeitumst"ande und Verh"altnisse erschwerten unsere Arbeiten, die eine Reihe von m"uhsamen und ununterbrochenen Versuchen und Untersuchungen notwendig machten, und brachten uns zuletzt --- wie mirs 1809 mit meinen fr"uheren "ahnlichen Unternehmungen ergangen war --- ganz davon ab, so dass jene Autografe bis zu dieser Stunde, als sie endlich eine neue Veranlassung --- leider nur zu unvorbereitet und peremtorisch --- ans Tageslicht ruft, unbenutzt liegen blieben. Inzwischen wurde der Gegenstand durch m"undliche Mitteilungen, zumal durch Fremde und Reisende, immer mehr und mehr bekannter, und endlich, vorz"uglich teils durch Chladni selbst --- der w"ahrend seines Aufenthalts in Wien, im Fr"uhjahr 1812, Zeuge unsrer fr"uheren und damaligen Versuche war --- teils auf dessen Anregung "offentlich zur Sprache gebracht; so "au"serten Herr Gubernialrat Neumann in Prag, auf dessen Veranlassung, bei Gelegenheit seiner Nachricht von der Elbogner Masse (1812, Hesperus, Heft 9), und nach dieses letzteren Mitteilung, Schweigger (1813, Journal f"ur Chemie und Physik, Bd. 7) ihre, und Chladni selbst (1815, in Gilberts Annalen, Bd. 50) seine Meinung und Erfahrung dar"uber, und auch unser Herr v. Hammer erw"ahnte desselben bei Gelegenheit einer Mutma"sung "uber die orientalischen damaszierten Klingen (1815, in den Fundgruben des Orients, Bd. 4, daraus im Hesperus Heft 9). Sp"aterhin ward der Gegenstand vollends durch mich selbst in Gespr"achen mit wissenschaftlichen Freunden, auf meiner Gesch"aftsreise nach Paris, 1815, in Deutschland und Frankreich verbreitet, und in der Folge durch Mitteilung von einzelnen Bl"attern jener autographischen Abdr"ucke an einige meiner Korrespondenten, dort und auch England noch genauer bekannt, und veranlasste die "Au"serungen Gillet de Laumonts (Jour. des Mines, Vol. 38, Sept. 1815), und S"ommerrings (in einer Vorlesung an der k"onigl. Bayerischen Akademie der Wissenschaften im Februar 1816, abgedruckt in der Bibl. univers. T. 7, und in Schweiggers Journal f"ur Chemie und Physik, Bd. 20), Schweiggers (in dessen Journal, Bd. 19), und Leonhards (in dessen Taschenbuche f"ur Mineralogie, Bd. 12).}} Es zeigt sich dasselbe am sch"onsten und deutlichsten auf ganz ebenen, rein abgeschliffenen und fein polierten Fl"achen solcher Massen --- insofern diese nicht etwa durch k"unstliche Hitze oder durch mechanische Gewalt vorher eine Ver"anderung erlitten haben\footnote{\frakfamily{Wird n"amlich ein St"uck einer solchen Masse, und zwar blo"s kalt und nur nach einer Richtung mehr oder weniger platt geh"ammert, dann erst abgeschliffen, poliert und ge"atzt; so zeigen sich auf licht stahlgrauem matten Grunde nur wellenf"ormige und verschiedentlich gebogene und gekr"ummte, nach verschiedenen Richtungen, und nur zum Teil parallel verlaufende, im Verlaufe sehr ungleich begrenzte, oft fleckartig ausgebreitete, erhabene Linien, und unregelm"a"sige, mehr oder weniger zusammenhangende Winkelz"uge von licht stahlgrauer, stark ins Silberwei"se fallender Farbe und einigem Glanze. Wird ein solches St"uck aber vollends hei"s und nach verschiedenen Richtungen geh"ammert; so erscheint eine h"ochst unvollkommene und verworrene Zeichnung, von der sich zuletzt, bei fortgesetzter "ahnlicher Behandlung, [...] Spur verliert, und die licht stahlgraue Oberfl"ache durch die Einwirkung der S"aure nicht ver"andert, sondern nur etwas, und zwar im Ganzen und gleichf"ormig, dunkler gef"arbt und matt erscheint.}} --- wenn dieselben mit Salpeters"aure\footnote{\frakfamily{Schwefel- und Salzs"aure bewirken zwar dieselbe Erscheinung, aber nicht so vollkommen, und langsamer. Sehr konzentrierte rauchende Salpeters"aure wirkt zwar schneller, aber oft zu tumultuarisch; man tut am besten, dieselbe, wenn man gerade nicht schnell und tief "atzen will, mit etwa zwei auch drei Teil Wasser zu verd"unnen. Die zu "atzende Fl"ache muss in eine feste, vollkommen horizontale Lage gebracht, und mit einem, etwa eine Linie hohen Saum oder Rand von Wachs umgeben werden, damit die S"aure nicht abflie"se, die doch $\mathfrak{\frac{1}{4}}$ oder $\mathfrak{\frac{1}{2}}$ Linie hoch die Fl"ache gleichf"ormig bedecken soll. Wenn die "Atzung etwas tief zu geschehen hat, so ist notwendig die S"aure zu wiederholten Mahlen zu erneuern, und dabei ist es gut, wenn man unter einem die Fl"ache jedes Mal mit reinem Wasser absp"ult, auch wohl mittelst eines Pinsels oder einer feinen B"urste abstreift, um sie von dem erzeugten Eisenoxyde, und dem, bei Verd"unstung des Fluidums, darauf niedergeschlagenen salpetersauren Eisen zu reinigen, welche die Einwirkung der frisch aufgegossenen S"aure verhindern w"urden. Soll die "Atzung sehr tief (z. B. $\mathfrak{\frac{1}{4}}$ bis $\mathfrak{\frac{1}{2}}$ Linie tief) eindringen; so fordert dies, auch bei jenem Verfahren, mehrere Tage Zeit, und wenn man den Prozess beschleunigen will, muss die Wirkung der S"aure au"serdem noch durch W"arme, auch wohl durch Zusatz von etwas Salzs"aure, verst"arkt werden.}} "ubergossen werden, und diese eine Zeitlang auf die Oberfl"ache eingewirkt hat.\footnote{\frakfamily{Eine Spur von dem Gef"uge zeigt sich zwar schon, aber nur wie ein Hauch, und nur bei gewissen Wendungen gegen das Licht, auf einer Fl"ache die vorl"aufig aus dem Rohen geschliffen und adoucirt worden ist; sie verliert sich aber ganz wieder w"ahrend des weitern Polierens, so dass eine vollends fein polierte Fl"ache, abgesehen von den durch Farbe, Glanz und Textur sich auszeichnenden, zerstreut eingemengten Massen der heterogenen br"ocklig-k"ornigen Substanz, ein vollkommen gleichf"ormiges Ansehen von licht stahlgrauer, mehr oder weniger ins Silberwei"se fallender Farbe, und von ziemlich starkem, metallisch spiegelnden Glanze zeigt. Auffallend und ausgezeichnet sch"on aber spricht sich das Gef"uge auf solchen fein polierten Fl"achen aus, wenn man dieselben, wie Stahl, auf die gew"ohnliche Art durch Erhitzung blau anlaufen l"asst. Anstatt n"amlich, dass dieselben mit den bekannten Farben, aus dem Goldgelben ins Veilchenblaue bis ins Dunkelblaue in allm"ahlicher Progression nach der Dauer des Prozesses, gleichf"ormig anlaufen, zeigen sie vielmehr diese Farben, wenn der Prozess bis zum Erscheinen des Blauen gekommen ist, alle zugleich, und zwar nach den verschiedenen Teilen des Gef"uges, eine "ahnliche Zeichnung wie die "Atzung hervorbringend. Die Streifen n"amlich erscheinen purpurrot ins Blaue, die Zwischenfelder oder Figuren bald aus dem Blauen, bald aus dem Rothen ins Goldgelbe (nach Glattheit oder Streifung derselben) verlaufend, die R"ander oder Einfassungslinien aber, so wie selbst die zartesten Schraffierungslinien, rein Goldgelb, jene Massen der k"ornig-br"ockligen Substanz endlich von etwas matter und ins Messinggelbe fallender Farbe.}} Die Einwirkung geht gew"ohnlich auf der Stelle vor sich, und nach wenigen Minuten schon, oft augenblicklich, zeigt sich das Gef"uge in den gleich n"aher zu beschreibenden geraden Streifen und winkeligen Figuren, die sich aber noch gar nicht durch Erhabenheit und Vertiefung, sondern blo"s, gleichsam als ein oberfl"achlicher Anflug, oder vielmehr wie angehaucht, durch Farbe und Glanz aussprechen; die Streifen n"amlich erscheinen matt und von sehr licht stahlgrauer, die Figuren oder Zwischenfelder dagegen, welche von jenen begrenzt oder eingeschlossen werden, zwar ebenfalls matt, aber doch --- bei schiefer Richtung der Fl"ache --- mit einigem Schimmer von ihrem Rande her, und von ziemlich dunkler, eisengrauer Farbe; die R"ander von beiden endlich sind von einer gemeinschaftlichen, zarten Linie eingefasst, die aber ebenfalls nur bei schr"ager Richtung und bei Wendungen deutlich sichtbar wird, und sich dann durch eine silberwei"se Farbe, und durch einen starken, spiegelnden Glanz auszeichnet. In gr"o"seren oder kleineren Kl"uften, und in zarten, oft sehr feinen Rissen --- welche sich urspr"unglich schon und vor der "Atzung auf der Oberfl"ache zeigten --- aber auch h"aufig zerstreut eingemengt und fest eingeschlossen, in einzelnen kleinen und "au"serst kleinen Partien br"ocklig ober feink"ornig angeh"auft, oft auch nur als einzelne zarte K"orner eingesprengt in die "ubrige Metallmasse, erscheint eine andere metallische Substanz --- insofern sie nicht hier und da durch Schnitt und Schliff der Fl"ache ausgesprengt worden ist --- von ziemlich starkem Glanze und silberwei"ser oder zinkgrauer, bisweilen etwas ins Gelbliche oder R"otliche ziehender Farbe, auf welche die S"aure schon etwas weniger als auf die "ubrige Oberfl"ache eingewirkt zu haben scheint.

Wird die "Atzung l"angere Zeit fortgesetzt, so erscheinen die einzelnen Teile des Gef"uges nicht nur immer deutlicher, sondern allm"ahlich und immer mehr und mehr, und zwar in verschiedenen Graden vertieft, und es zeigen sich jene Streifen nun am tiefsten, die Zwischenfelder oder Figuren dagegen etwas weniger tief, deren Einfassungslinien aber und die Massen jener br"ocklig-k"ornigen Substanz am erhabensten. Hat man demnach die "Atzung bis auf einen gewissen Grad\footnote{\frakfamily{Auf etwa $\mathfrak{\frac{1}{12}}$ Linie der tiefsten Stellen. Es darf nat"urlich dieser Grad nicht um gar viel "uberschritten werden, weil sonst die minder erhabenen Stellen im Verh"altnis zu den erhabensten zu tief zu liegen kommen, und sich nur schwach oder gar nicht ausdrucken.}} fortgesetzt; so ist die ganze Zeichnung eines unmittelbaren Abdruckes von der Fl"ache mittelst Druckerschw"arze f"ahig, indem die erhabensten Stellen sich stark, die minder erhabenen schw"acher, die tieferen dagegen sich gar nicht ausdrucken, und da sie alle regelm"a"sig abwechseln und unter einander verbunden sind, so erh"alt man solcher Gestalt nicht nur eine ganz vollkommene und genaue Darstellung der ge"atzten Fl"ache, sondern auch ein treues Bild des nat"urlichen Gef"uges der Masse, wie sich dasselbe durch die "Atzung ausspricht.\footnote{\frakfamily{Obgleich die M"oglichkeit des Vorkommens von wahrhaft meteorischem Gediegeneisen ohne solchem Gef"uge nicht geradezu in Abrede gestellt werden kann, zumal wenn dasselbe --- was jedoch nicht wahrscheinlich ist --- von einer blo"sen Zustands-Modifikation des reinen Metalle, und blo"s von einer regelm"a"sigen mechanischen Lagerung und F"ugung der Grundteilchen, nicht aber von einer besonderen und eigent"umlichen, chemischen oder mechanischen Verbindung mit andern Stoffen, einem eigenen Mischungs- und regelm"a"sigen Mengungs- und Absonderungsverh"altnisse abh"angen sollte; so ist doch merkw"urdig, dass dasselbe noch bei allen Gediegeneisen-Massen gefunden wurde, deren meteorischer Ursprung, wenn gleich nicht --- so wie von der Agramer --- faktisch erwiesen, aber doch der vollkommensten "Ahnlichkeit wegen mit dieser und nach allen physischen und chemischen Kriterien unbezweifelbar ist, und selbst bei den kleinen, mechanisch eingemengten Massen von Gediegeneisen in Meteor-Steinen --- insofern dieselben nur Gr"o"se genug hatten, um darauf ohne Ver"anderung ihrer Struktur (durch allzustarke Fletschung z. B.) untersucht werden zu k"onnen --- dagegen keine Spur davon bei solchen, die jenen Forderungen, eine "ahnliche Herkunft zu bew"ahren, nicht vollkommen entsprechen, und die auch nur insofern noch ihres Ursprunges wegen mehr oder weniger f"ur problematisch angesehen werden, als sie zum Teil an Orten gefunden worden sind, wo man keinen Grund hat nat"urliche Eisenlager in der N"ahe, oder die fr"uhere Existenz von Eisenh"utten zu vermuten, und es sich zur Zeit nicht wohl begreifen l"asst, wie sie dahin gekommen, oder durch welchen irdischen Prozess sie dort gebildet worden sein konnten: wie jene Massen von Aachen, Mailand, Cilly, Kamsdorf, Florac, u. m. a., die "ubrigens aber auch des als eigent"umlich und charakteristisch (obgleich wohl nicht minder unter gewissen Restriktionen) f"ur jenen Ursprung angesehenen Gehaltes an Nickel ermangeln, und daher umso billiger bezweifelt werden. Indes waren wir, trotz [...] wiederholten Versuchen, doch auch nicht im Stande, eine Spur jenes Gef"uges an den uns zu Gebote stebend- [...] St"ucken vom Kap'schen und dem Peruanischen Eisen zum Vorschein zu bringen, obgleich dieselben [...] ganz verl"asslichen H"anden erhalten worden sind --- so dass "uber deren Echtheit hinsichtlich ihrer Herstammu- [...] -ein Zweifel Statt finden kann --- und da doch "uber deren unbezweifelbar meteorischen Ursprung --- f"ur welchen selbst das andere als entscheidend betrachtete Kriterium, n"amlich der Gehalt an Nickel, und zwar in einem ganz "ahnlichen quantitativen Verh"altnisse, und die meisten "ubrigen physischen und chemischen Eigenschaften, B"urgschaft zu leisten scheinen --- vorl"angst abgesprochen ist. Es fr"agt sich demnach noch, ob das Erscheinen dieses Gef"uges als ein unbedingtes und best"andiges Merkmal des meteorischen Gediegeneisens zu betrachten sei; und beinahe ebenso sehr steht es in Frage, ob es denselben, wenigstens strenggenommen, ausschlie"send zukomme. Denn einerseits l"asst sich die M"oglichkeit einer "ahnlichen Zustands-Modifikation und einer gleichen Tendenz zur Kristallisation, sowie eines "ahnlichen Mischungs- und Mengungsverh"altnisses mit "ahnlichen Stoffen (mit Schwefel zu Eisen- und Magnetkies; mit Kohle zu Stahl und Graphit; mit Silicium, Magnesium, und vielleicht selbst mit Nickel), je nachdem dieses oder jenes als n"achste Ursache jener Erscheinung zu Grunde l"age, bei terrestrischem und k"unstlich erzeugtem regulinischen Eisen nicht l"augnen, in Gegenteile beweisen ersteres deutliche Anzeigen eines und zwar ganz "ahnlichen kristallinischen Gef"uges, im Bruche mancher Roheisen-St"ucke, letzteres (nur wie es scheint, mit Ausnahme des Nickels zur Zeit noch) die Resultate mehrerer Analysen verschiedener Arten von Roh- und Frischeisen-Massen (man sehe was hier"uber Herr Professor Hausmann in dem gehaltreichen Aufs"atze --- \emph{Specimen Crystallographiae metallurgicae} --- vorgelesen im Mai 1818 in der k"onigl. Gesellschaft der Wissenschaften zu G"ottingen, und abgedruckt in den neuern Schriften derselben, Bd. 4, 1820, in beiden Beziehungen vorgebracht hat), andererseits zeigt beinahe jedes k"unstliche Roheisen (so wie namentlich auch das Cillier, des Fundortes wegen f"ur problematisch angesehene, metallische Eisen) eine, obgleich nur entfernt "ahnliche, und keineswegs so regelm"a"sige Figurierung, und zwar stets und in mannigfaltig abweichenden Modifikationen, die sich auch nur schwach, blo"s oberfl"achlich und gew"ohnlich sowohl nach dem Schliffe als nach der feinen Politur, durch "Atzung aber (unsern Erfahrungen nach) keinesweges vollkommener und \emph{en basrelief} (wie auch Daniells Versuche lehren --- mit deren Resultaten man "ubrigens die unsrer "Atzungsversuche mit dem Meteor-Eisen verwechselt zu haben scheint --- wohin wohl auch das, durch eine "ahnliche Prozedur bewirkte und auf gleichem Prinzipe beruhende, Moirieren des verzinnten Bleches zu z"ahlen sein d"urfte) ausspricht: inzwischen hat doch Gillet de Laumont, seiner Versicherung nach, an einem St"ucke durch Kunst geschmolzenen, reinen, regulinischen Eisen, von besonders deutlich bl"atterigem Gef"uge (\emph{en grand lames}), tiefe, gl"anzende Streifen (\emph{des stries profondes}), die sich sogar ebenso (?) und zwar unter gleichen Winkeln, wie am Elbogner Eisen, durchkreuzten, durch "Atzung erhalten.\\
\hspace*{6mm}Nichts desto weniger d"urfte denn doch das Erscheinen jenes Gef"uges von der Art und Beschaffenheit, wie es sich am Agramer Eisen, als Prototyp, und diesem ganz "ahnlich, und mit nur sehr unbedeutenden Abweichungen bei der B"ohmischen, Karpatischen und Mexikanischen derben Eisenmasse zeigt, f"ur das Meteor-Eisen charakteristisch, und demselben ausschlie"slich eigent"umlich sein, so wie dasselbe auf ein Mischungs- und Mengungsverh"altnis, auf eine Vereinigung und Absonderung von Bestand- und Gemengteilen nach einem bestimmten Affinit"ats- und Kristallisationsgesetze, und auf einen Prozess hinzudeuten scheint, auf welche wir von nichts ganz "ahnlichem, auf unsern Planeten vorkommenden, nach Analogie schlie"sen k"onnen.}}

Die neunte Tafel zeigt nun einen solchen unmittelbaren Abdruck von einer gro"sen, auf den geh"origen Grad ge"atzten Fl"ache an der Eisenmasse von Elbogen, die ich ihres autographischen Vorzuges wegen, und da sie das zusammen gesetzteste Gef"uge zeigt, nach welchem sich jenes der "ubrigen Massen am besten vergleichend beschreiben l"asst, als Norm w"ahle, obgleich dieses Vorrecht, an sich und der Folgerungen wegen, der Agramer Masse, als Prototyp, geb"uhrte.\footnote{\frakfamily{Es war nicht m"oglich, von dieser und den "ubrigen Gediegeneisen-Massen, "ahnliche, zur Bekanntmachung geeignete autographische Darstellungen ihres Gef"uges auf der Stelle zu bewerkstelligen, indem die Zustandebringung viele Zeit raubend mechanische Vorarbeiten und Vorkehrungen notwendig gemacht h"atte. Sie sollen f"ur eine k"unftige Veranlassung vorbereitet werden. Vorl"aufig finden sich von denselben auf der achten Tafel mit m"oglichster Genauigkeit aus freier Hand lithographisch nach der Natur gefertigte Kopien.}}

Bei Betrachtung dieses Abdruckes fallen nun auf den ersten Blick oben erw"ahnte Streifen auf, welche, da sie auf der ge"atzten Fl"ache die tiefsten Stellen ausmachen, hier unabgedruckt und wei"s, und nur durch ihre Begrenzung --- durch jene erhabenen Einfassungslinien --- bezeichnet erscheinen, insofern nicht einige zart erhaben punktiert, gestrichelt oder gestreift vorkommen. Da sich diese Streifen h"aufig durchschneiden, durchkreuzen, und folglich sich wechselseitig und hinsichtlich ihrer Verteilung sehr unregelm"a"sig unterbrechen, so erscheinen sie von sehr verschiedener Ausdehnung in der L"ange, und zwar hier von einer halben bis zu sieben Linien, und beinahe in allen denkbaren Zwischenma"sen; dagegen zeigen sie nur wenig Verschiedenheit in der Breite, die nur zwischen $\mathfrak{\frac{1}{4}}$ und $\mathfrak{\frac{1}{2}}$ Linie abweicht, und nur bei einzelnen wenigen $\mathfrak{\frac{3}{4}}$ oder eine ganze Linie betr"agt. Bey etwas genauerer Betrachtung findet man bald, dass diese Streifen regelm"a"sig und genau, aber ungleich an Menge und ganz unordentlich in der Aufeinanderfolge, einer drei- und zum Teil einer vierfachen Richtung folgen; dass die nach einer Richtung gehenden unter sich einen vollkommenen Parallelismus beobachten, und dass sie sich nach diesen verschiedenen Richtungen regelm"a"sig und unter bestimmten Winkeln durchschneiden oder unterbrechen. Die eine dieser Richtungen geht (nach der Lage der Fl"ache, in welcher dieselbe hier vorgestellt ist --- mit dem schm"alern Teile nach oben ---) vollkommen senkrecht. Die Streifen welche ihr folgen, scheinen von allen "ubrigen am h"aufigsten und am gleichf"ormigsten verteilt vorzukommen, sind auch unter sich am gleichf"ormigsten, die schm"alsten, zartesten, und am sch"arfsten gerandet oder begrenzt. Die andere Richtung geht schief von der Rechten zur Linken abw"arts, so dass die Streifen --- welche im Ganzen minder zahlreich, ziemlich gleichf"ormig verteilt, aber ungleichf"ormiger unter sich, meistens l"anger und etwas breiter (so dass an Masse im Ganzen das ersetzt wird, was etwa an Menge gegen erstere gebrechen m"ochte), und nicht so schnurscharf gerandet sind --- die ersteren meistens unter einem Winkel von 60$^{\circ}$ (nur selten unter einem merklich davon abweichenden und dann doch immer zwischen 56 und 65 fallenden Winkel) durchschneiden. Die dritte Richtung geht jener entgegen gesetzt, schief von der Linken zur Rechten abw"arts, und die derselben folgenden Streifen sind noch weniger zahlreich selbst als letztere, dagegen meistens bedeutend l"anger, und im Durchschnitt auffallend breiter (so dass sich das Verh"altnis der Masse gegen jene wieder auszugleichen scheint), viel ungleichf"ormiger verteilt, noch weit ungleichf"ormiger unter sich, weniger scharf und sehr ungleich begrenzt --- so dass sie in ihrem Verlaufe nicht selten [...] breit, hie und da bauchig und geschweift erscheinen --- und sie durchschneiden die Streifen der se- [...] Richtung sowohl, als die der andern schiefen, unter ganz "ahnlichen Winkeln wie diese jene, so dass durch ihre wechselseitige Durchkreuzung Dreiecke gebildet werden, die teils, und zwar meistens, vollkommen gleichseitig, teils gleichschenklich (wo zwei Winkel gleich sind, z. B. = 62 zum dritten = 56$^{\circ}$), teils, obgleich nur selten, ganz ungleichseitig sind (z. B. mit Winkeln = 56, 60 und 64$^{\circ}$). Au"ser diesen zeigen sich "ahnliche Streifen, aber in ungleich geringerer Menge, meistens partienweise von 3, 4 bis 8 und 9 zusammen gereiht, dicht aneinander, und sehr ungleichf"ormig verteilt. Diese sind h"ochst ungleichf"ormig unter sich, bald kurz, bald lang, von 1 bis 6, und selbst von 9 Linien L"ange, aber bedeutend breiter als alle vorigen, von $\mathfrak{\frac{1}{4}}$ bis zu einer vollen Linie, im Verlaufe "ubrigens oft sehr abweichender Breite, und meistens sehr ungleichf"ormig begrenzt, so dass ihre R"ander oft sehr ausgeschweift und gebogen erscheinen. Ihre Richtung geht (bei obiger Lage der Fl"ache) schief von der Linken zur Rechten abw"arts, also gleich jener der Streifen der dritten Richtung, aber nicht parallel mit dieser, sondern unter einem Winkel von beil"aufig 27$^{\circ}$ mit derselben sich kreuzend, und demnach die Streifen der beiden "ubrigen Richtungen unter andern Winkeln als diese durchschneidend, woraus nun wieder mehr oder weniger ungleichseitige Dreiecke, und zwar von dreierlei Art erwachsen, die aber nicht zahlreich vorkommen, da der Streifen dieser Richtung verh"altnism"a"sig nur wenige, und diese meistens partienweise zusammen geh"auft sind.\footnote{\frakfamily{Um sich eine deutliche Ansicht und eine leichte Unterscheidung dieser verschiedenen Streifen nach ihrem meist schnurgeraden, aber oft unterbrochenen Laufe, von den verschiedenen Richtungen welche sie verfolgen, von dem Parallelismus den sie hierin halten, und von ihren h"aufigen Durchkreuzungen, zu verschaffen; tut man am besten, wenn man alle Streifen einer jeden Richtung, ihrem ganzen Verlaufe nach, mittelst eines Lineals mit verschieden gef"arbten Zeichenstiften (Pastel- oder Wachs-Crayons) "uberf"ahrt; so wie, um sich eine m"oglichst genaue Vorstellung von der Form der Dreiecke und der Beschaffenheit der Winkel zu verschaffen, wenn man einige dieser solcher Gestalt gef"arbten Streifen "uber den Abdruck hinauszieht, und so weit verl"angert, bis sich alle, ihrer Richtung nach entgegen gesetzten, au"serhalb des Abdruckes wechselseitig durchkreuzen. Man erh"alt solcher Gestalt, und zwar nach einem beliebig gro"sen Ma"sstabe, viererlei Dreiecke; n"amlich: aus der Durchkreuzung der drei ersteren, regelm"a"sigen und fast ganz best"andigen Richtungen, ein meistens mehr oder weniger vollkommen gleichseitiges Dreieck mit Winkeln von 60$^{\circ}$ (und wenn man will und mit Pr"azision verf"ahrt, auch alle kleinen Abweichungen davon, die sich jedoch ziemlich auf Dreiecke mit Winkeln von 62, 62 und 56$^{\circ}$, oder 60, 64 und 56$^{\circ}$ beschr"anken), und dann aus der Durchkreuzung der Streifen der vierten unregelm"a"sigern Richtung mit je zwei und zwei der vorher gehenden, dreierlei mehr oder weniger ungleichseitige und ungleichschenkliche Dreiecke (meistens mit Winkeln = 95, 60, 25 oder 98, 55, 27; ferner = 25, 120, 35 oder 30, 115, 35; endlich 60, 85, 35 oder 65, 76, 39$^{\circ}$. --- Abweichungen, die "ubrigens bei oft mangelhafter Sch"arfe der Streifen und unm"oglich zu erreichender Pr"azision in der Darstellung und Messung, wohl mehr von der Unvollkommenheit der Bestimmung, als von der Unregelm"a"sigkeit des Gef"uges herr"uhren m"ochten). Die Rhomben und Trapezen, die durch einzelne Streifen entstehen, welche, einem der Schenkel jener Dreiecke parallel, diese durchschneiden und Segmente derselben bilden, zeigen dem urspr"unglichen Dreiecke entsprechende Winkel und Winkel-Supplemente; demnach bei solcher Durchschneidung vollkommen gleichseitiger Dreiecke --- die hier am h"aufigsten vorkommen --- ein Winkel-Supplement von 60, folglich Winkeln von 120$^{\circ}$, wie sie Gillet de Laumont, Leonhard, Schweigger u. a. bemerkt haben.}}

Bei weiterer Betrachtung des Abdruckes bemerkt man ferner h"aufige, gr"o"sere und kleinere, sehr ungleichf"ormig verteilte und unregelm"a"sig zerstreute, meistens dreieckige, bisweilen aber auch rhomboidale oder trapezoidale (keineswegs aber vollkommen viereckige --- wie zum Teil behauptet wurde --- als welche bei dieser Struktur nicht wohl vorkommen k"onnen) Figuren, Felder oder Zwischenr"aume, welche durch die Durchkreuzung von 3 oder 4 jener Streifen verschiedener Richtungen, oder durch das Zusammensto"sen zweier Dreiecke, gebildet werden, und notwendig gebildet werden m"ussen, insofern nicht jene Streifen --- was bisweilen der Fall ist --- dicht an einander sto"sen, und solcher Gestalt gar keinen, wenigstens keinen dem freien Auge auffallenden, Zwischenraum lassen.

Die Form der Dreiecke und die Beschaffenheit ihrer Winkel entspricht jenen regelm"a"sigen Richtungen und den oben angegebenen Durchkreuzungspunkten der Streifen, und die der Rhomben und Trapezen jenen Dreiecken, insofern diese durch einzelne, irgendeiner jener Richtungen parallellaufende Streifen wieder durchschnitten, oder in Abschnitte geteilt worden sind. Es erscheinen diese Figuren oder Felder hier nicht nur im Umrisse, indem sie von jenen, ihnen sowohl als den Streifen als gemeinschaftliche Scheidewand dienenden, erhabenen, und folglich im Abdruck erscheinenden Linien begrenzt werden, sondern selbst ihrer Oberfl"ache nach, obgleich etwas schw"acher ausgedruckt, und zwar glatt und gleichf"ormig, oder mehr oder wenig --- und in diesem Falle etwas st"arker ausgedruckt --- mikroskopisch zart punktiert, gestrichelt oder gestreift, und dies zwar in verschiedenen, oft sich durchkreuzenden, aber stets ihren R"andern oder den Einfassungslinien und den angrenzenden Streifen parallel laufenden Richtungen.

Ferner bemerkt man, hie und da zerstreut, zwischen und auch oft mitten in den Streifen, mehr oder minder stark abgedruckte, gr"o"sere oder kleinere, ganz unregelm"a"sig und verschieden gestaltete Flecke und Punkte, welche "ahnlichen Erhabenheiten der Metall-Masse auf der ge"atzten Fl"ache, und jenen bereits erw"ahnten, mechanisch eingemengten Massen der heterogenen br"ocklig-k"ornigen Substanz entsprechen.

Endlich zeigen sich in diesem Abdrucke ziemlich h"aufige (wohl zwischen 50 und 60) und dem Anscheine nach ganz unregelm"a"sig zerstreute, mehr oder weniger fleckartige, oft ziemlich gro"se, 2, 4, 6, 8 bis 12 und 16 Linien lange, und $\mathfrak{\frac{(2-3)}{12}}$ und $\mathfrak{\frac{1}{2}}$ bis 2 Linien breite, meistens gegen beide Enden spitz zulaufende Striche, welche die Oberfl"ache in sehr verschiedenen Richtungen, doch, wie es scheint, nicht ganz und gar unabh"angig von jenem regelm"a"sigen Gef"uge (indem doch wenigstens drei Richtungen vorherrschen, nach welchen auch diese Striche einen Parallelismus zeigen, obgleich kaum eine davon mit einer der Streifen koinzidiert), [...] Es erscheinen diese Striche hier gr"o"sten Teils oder ganz unabgedruckt, und nur im Umrisse durch die begrenzende, abgedruck [...] Umgebung angedeutet --- indem sie betr"achtlich tiefen, leeren Rissen entsprechen, die sich, wie bereits oben erw"ahnt wurde, in [...] Metall-Masse selbst, schon vor der "Atzung der Fl"ache vorfanden --- und nur zum Teil fleckig oder punktiert, insofern diese noch mit Br"ockeln und K"ornern obiger heterogener Substanz, die durch den Schnitt und Schliff nicht vollends ausgesprengt wurden, stellenweise ausgef"ullt sind.\footnote{\frakfamily{Ein besonderer Abdruck von der geschnittenen und polierten Fl"ache vor der "Atzung, gab ein reines und deutliches Bild dieser, die Gleichf"ormigkeit und Homogenit"at der Metall-Masse unterbrechenden Striche, und von deren Beschaffenheit, Verteilung und Richtung.}}

Diese verschiedenen Teile in welchen sich das Gef"uge durch den Abdruck ausspricht, zeigen sich nun auf der ge"atzten Metall-Fl"ache selbst, von folgender Beschaffenheit.

Die nach den vier Richtungen gehenden Streifen erscheinen bei diesem Grade von "Atzung als die tiefsten Stellen (jene Risse ausgenommen, die aber nicht durch die "Atzung zum Vorschein gebracht worden sind), und zwar alle von ganz gleicher Tiefe; die R"ander aber, die im Abdrucke deren Kontur gaben, am erhabensten, als Leisten oder d"unne Zwischenw"ande, durch welche jene unter sich sowohl als von den Figuren oder Feldern geschieden erden, und die deren, nun zum Teil ausge"atzte, Substanz begrenzen und gleichsam einfassen, daher wir sie Einfassungsleisten nennen wollen.

Die vertiefte Oberfl"ache, oder die r"uckst"andige Substanz dieser Streifen, hat ein etwas raues, unter dem Mikroskope gleichsam flachnarbiges oder platt runzlicht-faltiges Ansehen, eine Zinkwei"se Farbe, und einen schwachen metallischen, etwas seidenartig schimmernden Glanz; die Leisten dagegen sind vollkommen glatt, und haben eine licht stahlgraue, stark ins Silberwei"se ziehende Farbe, und einen sehr starken, spiegelicht metallischen Glanz.

Einige (obgleich hier nur wenige) dieser Streifen erscheinen teils durch einzelne wenige, und dann ziemlich starke, teils aber auch durch sehr viele, dicht an einander gereihete, und dann mehr oder weniger zarte, oft mikroskopisch feine, bisweilen blo"s aus zusammen gereiheten Punkten oder kurzen Strichelchen zusammen gesetzte, oft im Verlaufe aussetzende, abgebrochene, erhabene Linien --- die unter sich sowohl als den Einfassungsleisten parallel, aber nicht vollkommen geradlinig, sondern meistens etwas gebogen oder fast wellenf"ormig verlaufen --- der L"ange nach gestreift. Es haben diese Linien, die wir zum Unterschiede Streifungs- --- oder besser, zumal sie eine entsprechende Wirkung hervor bringen --- Schraffierungsleisten nennen wollen, gleiche H"ohe mit den Einfassungsleisten (daher sie auch im Abdrucke erscheinen), mit welchen sie selbst ihrer Substanz nach von ganz gleicher Beschaffenheit zu sein scheinen, wie sie denn auch dieselbe Bestimmung haben, indem sie "ahnliche Streifen begrenzen, nur dass diese oft so mikroskopisch zart sind, dass jene Leisten sich fast ber"uhren.

Die Felder oder Figuren, welche zwischen jenen Streifen liegen --- durch deren Zusammensto"sen und Durchkreuzen sie gebildet werden -- erscheinen zwar ebenfalls tiefer als die Einfassungsleisten --- die zwischen ihnen und den Streifen gleichsam die gemeinschaftliche Scheidewand bilden, und daher im Abdrucke auch zugleich die Form und Begrenzung jener bezeichnen --- aber bei weitem nicht so tief ge"atzt wie die Streifen, wie sich denn auch ihre Oberflache, zumal wenn diese rau oder gestreift ist, bei einem gewissen Grade von "Atzung, obgleich schw"acher als die Einfassungsleisten, abdruckt.

Es haben diese Felder eine eisengraue Farbe, ein ganz mattes metallisches Ansehen, und teils eine glatte, teils aber, und zwar durchaus oder nur zum Teil, meistens gegen die Winkel zu, eine raue, mikroskopisch fein gek"ornte Oberfl"ache; sehr viele aber haben dieselbe ganz, oder zum Teil, zart erhaben gestreift. Diese Streifung (Schraffierung) wird, so wie vorhin bei den Streifen bemerkt wurde, durch ganz "ahnliche, aber gew"ohnlich "au"serst zarte und mikroskopisch feine, mehr oder weniger, doch meistens sehr dicht aneinander gereihete, erhabene Linien oder Leisten hervorgebracht, die, bei ihrer Menge und Zartheit, mittelst ihres Glanzes diesen Feldern oft einen seidenartigen Schimmer geben. Es laufen diese Schraffierungsleisten aber auf den einzelnen Feldern nur h"ochst selten blo"s nach einer Richtung (wie dies bei den Streifen der Fall ist), sondern gew"ohnlich erscheinen sie partienweise, und zwar parallel unter sich sowohl als mit ebenso vielen Seitenr"andern, nach zwei oder drei Richtungen, die sich im Kleinen ebenso und unter "ahnlichen Winkeln durchschneiden und durchkreuzen wie die Streifen im Gro"sen (daher eine wahre Schraffierung bewirken). Sehr oft sind diese Leisten nicht nur einzeln oder partienweise solcher Gestalt unterbrochen, sondern sie selbst setzen oft aus, und lassen einen glatten Zwischenraum, oder erscheinen blo"s als in eine Linie gereihte Punkte oder Strichelchen. Beinahe jedes Feld hat seine eigent"umliche Schraffierung, ohne Bezug auf die n"achstliegenden. Jene vertieften Streifen scheinen eine vollkommene Trennung oder Isolierung zwischen denselben zu bewirken. Es scheint dieselbe "ubrigens von den R"andern der Felder oder von den Einfassungsleisten her ausgegangen zu sein, wenigstens zeigen sich hier immer die meisten Leisten, auch wenn sich im Mittel oft gar keine finden und sie selbst nicht weit hinein reichen, sondern als abgebrochene Strichelchen an einem der R"ander erscheinen; inzwischen zeigt sich doch auch oft im Mittel eines Feldes die Streifung fleckweise unterbrochen; so dass z. B. mitten in einer Partie senkrecht laufender Leisten ein Fleck von ganz unregelm"a"siger Form von solchen einer schiefen Richtung vorkommt. In manchen Feldern erscheint die Streifung nur in Gestalt zarter, mikroskopisch feiner, mehr oder weniger dicht und anscheinend ganz unordentlich zerstreuter, noch gar nicht in parallele Linien und nach einer bestimmten Richtung gereiheter, erhabener Punkte.\footnote{\frakfamily{Um eine deutliche Vorstellung von der merkw"urdigen Beschaffenheit der Oberfl"ache dieser Felder zu verschaffen, ist eine stark vergr"o"serte Darstellung mehrerer derselben durchaus notwendig, welche nebenher in dieser Zwischenzeit mit der geh"origen Genauigkeit zu Stande zu bringen ich nicht vermochte.}} Die glatten Felder erscheinen etwas tiefer ge"atzt, zumal aber ist ihr Mittel bisweilen grubenartig vertieft, gleichsam eingesunken, indes sich der Rand allm"ahlich gegen die Einfassungsleisten zu erhebt.

Die im Abdrucke bemerkten gr"o"seren und kleineren, unregelm"a"sig gestalteten und zerstreut in und zwischen den Streifen erscheinenden Flecke und Punkte, zeigen sich hier als erhabene Massen, und zwar gr"o"sten Teils von gleicher H"ohe mit den Einfassungsleisten, mitunter aber auch etwas tiefer, und daher und "uberhaupt bei n"aherer Betrachtung der Oberfl"ache noch ungleich h"aufiger als im Abdrucke, so dass die Masse ganz damit durchs"aet erscheint, aber in allzu zarten K"ornern, als dass sie, oft ihrer Erhabenheit ungeachtet, durch den Abdruck bemerkbar werden konnten. Die Substanz derselben zeichnet sich von der "ubrigen Metall-Masse durch ein br"ocklig-k"orniges, oder doch rissiges Aussehen, eine matte, dunkeleisengraue, im Schliffe aber hier stark und beinahe ganz rein ins Silberwei"se fallende Farbe und starkem spiegelnden Glanze aus.

"Ahnliche, aber meistens mehr vertiefte, und daher im Abdrucke nur im Umrisse und undeutlich erscheinende, und gr"o"sten Teils rundliche oder ovale Flecke von verschiedener, zum Teil bedeutender Gr"o"se (von $\mathfrak{\frac{1}{4}}$ bis "uber 2 Linien im st"arksten Durchmesser), zeigen sich ziemlich h"aufig und ganz unordentlich zerstreut, aber scharf begrenzt, zwischen den Streifen und Feldern gleichsam wie eingeknetete oder eingekeilte Massen oder K"orner von matter, schw"arzlich eisengrauer, durch den Schliff nur wenig ver"anderter Farbe, glatter Oberfl"ache und einem Ansehen, das zwischen jenem der Substanz der Felder und jener br"ocklig-k"ornigen gleichsam das Mittel h"alt.

Die beim Abdrucke erw"ahnten fleckartigen Striche erscheinen hier als wahre Risse und enge Kl"ufte, die zum Teil ziemlich tief (oft "uber eine Linie), teils senkrecht, teils schief in die Masse eindringen, und die schon urspr"unglich vorhanden waren und nicht erst durch die "Atzung hervor gebracht worden sind; dagegen ist wohl durch den Schnitt und Schliff der Fl"ache die urspr"unglich in denselben enthalten gewesene, br"ocklig-k"ornige Substanz --- die mit jener in einzelnen K"ornern zerstreut eingesprengten von ganz gleicher Beschaffenheit ist --- verm"oge ihrer Spr"odigkeit und br"ockligen Anh"aufung, mehr oder weniger ausgesprengt worden, und die Risse erscheinen daher stellenweise leer und im Abdrucke demnach blo"s nach ihrem, von den angrenzenden erhabenen Teilen bestimmten Umrisse, oder nur fleckweise ausgedruckt.

Eine auf der achten Tafel gegebene, mit m"oglichster Genauigkeit aus freier Hand lithographisch nach der Natur kopierte Darstellung eines auf "ahnliche Art und in einem gleichen --- zum Abdrucke geeigneten --- Grade ge"atzten Pl"attchens von der Agramer Eisenmasse, zeigt ein ganz "ahnliches Gef"uge, nur mit folgenden kleinen Abweichungen.\footnote{\frakfamily{Die Beschreiung ist teils von diesem Pl"attchen, teils von einer auf der Masse selbst ge"atzten Fl"ache (deren oben bei Beschreibung der Masse Erw"ahnung gemacht wurde), von 6 Quadrat-Zoll Ausdehnung, genommen.}}

Die Streifen zeigen sich n"amlich hier nur nach drei Richtungen, und zwar in den drei regelm"a"sigeren, nach welchen sie vollkommen parallel verlaufen, und zwar so, dass sie sich unter Winkeln von beil"aufig 56, 50 und 74$^{\circ}$ kreuzen; die der vierten Richtung fehlen ganz und gar, und es finden sich demnach, als durch sie gebildete Zwischenfelder oder Figuren, nur einerlei, und zwar mit "au"serst wenig Abweichung, ungleichschenkliche Dreiecke, und, aus deren Verbindung und Durchschneidung, Rhomben und Trapezen, ebenfalls von wenig Abweichung und mit leicht zu bestimmenden, jenen obiger Dreiecke entsprechenden Winkeln. Die Zeichnung erscheint solcher Gestalt viel einfacher, gleichf"ormiger, und zum Teil regelm"a"siger, als bei der Elbogner Masse.

Die Streifen selbst, die im Ganzen jedoch merklich minder zahlreich, dagegen aber etwas st"arker und breiter als an jener Masse vorkommen --- daher das ganze Gef"uge ein etwas gr"oberes Ansehen hat --- sind "ubrigens ebenso ungleichf"ormig verteilt, und die einer Richtung auf "ahnliche Art partienweise zusammengeh"auft, und nach diesen Richtungen, mit auffallender "Ubereinstimmung, ebenso an Menge und Masse abweichend, wie an jener; auch durchschneiden und unterbrechen sie sich in einem "ahnlichen Grade, und erscheinen demnach im Ganzen von "ahnlicher L"ange, nur, wie bemerkt, im Durchschnitte von etwas st"arkerer Breite --- doch so, dass die breitesten kaum $\mathfrak{\frac{1}{2}}$ Linie erreichen --- und mit einer "ahnlichen und "ubereinstimmenden Abweichung in derselben nach der verschiedenen Richtung, zeigen aber nach beiden Dimensionen etwas mehr Gleichf"ormigkeit.

Auf der ge"atzten Fl"ache selbst zeigen diese Streifen eine etwas minder raue und narbige oder faltige, bisweilen sogar eine ganz glatte Oberfl"ache, eine mehr ins Silberwei"se fallende Farbe, dagegen etwas weniger Glanz als die der Elbogner Masse, erscheinen aber h"aufiger "ubrigens ganz auf "ahnliche Art schraffiert, und die erhabenen R"ander oder Einfassungsleisten weniger silberwei"s, mehr stahlgrau, und etwas schw"acher gl"anzend.

Die Zwischenfelder oder Figuren haben hier eine etwas dunklere, mehr schw"arzlich-graue Farbe, sonst dasselbe Ansehen und dieselbe Beschaffenheit wie jene der Elbogner Masse, nur dass sie im Durchschnitte seltener und meistens nur teilweise, gew"ohnlich auch blo"s nach einer Richtung --- einer Einfassungslinie parallel --- gestreift, dagegen h"aufiger rau und zart gek"ornt und nur selten ganz glatt vorkommen, daher auch die meisten nicht blo"s im Umrisse, sondern mit ihrer ganzen Oberfl"ache im Abdrucke ausgedruckt erscheinen. Merkw"urdig ist, dass einige, zumal kleinere, solche Felder ebenso erhaben, glatt und gl"anzend wie die Einfassungsleisten, von ganz gleichem Ansehen und gleicher Beschaffenheit, und gleichsam mit denselben zusammen geflossen erscheinen, als wenn ihre Substanz in diese "ubergegangen w"are.

Flecke und Punkte von der br"ocklig-k"ornigen Substanz in den Streifen zeigen sich, sowohl im Abdrucke als auf der ge"atzten Fl"ache, im Ganzen nur sehr wenige, und ebenso finden sich auch wenigere eigentliche Risse, dagegen mehr fleckartige, sehr unregelm"a"sig und unordentlich zerstreute, zum Teil ziemlich gro"se, mehr oder minder mit solcher Substanz --- die aber hier eine mehr Zinkwei"se und etwas, teils ins Messinggelbe, teils ins R"otliche fallende Farbe hat --- ausgef"ullte Kl"ufte.

Von der besonderen, in rundlichen Massen gleichsam eingekeilten metallischen Substanz, findet sich hier keine deutliche Anzeige.

Auf derselben Tafel findet sich eine auf "ahnliche Art versuchte Darstellung einer ebenso ge"atzten Platte von der Eisenmasse von Lénarto, welche in Vergleichung mit beiden vorigen folgende Abweichungen im Einzelnen des Gef"uges zeigt.\footnote{\frakfamily{Auch diese Beschreibung ist nicht blo"s nach der vorgestellten Platte, sondern nach noch zwei, in verschiedenem Grade ge"atzten Fl"achen, von 12 Quadrat-Zoll Ausdehnung, an gro"sen St"ucken von dieser Masse abgefasst.}}

Die Streifen erscheinen hier ebenfalls nur nach drei Richtungen, die sich aber unter ganz andern Winkeln, n"amlich meistens und mit kaum merklichen Abweichungen von beil"aufig 77, 77 und 26$^{\circ}$ kreuzen, und daher gleichschenkliche, aber lang gezogene und scharf zugespitzte Dreiecke, und diesen entsprechende rhomboidale und trapezoidale Segmente zu Zwischenfeldern haben. Die Zeichnung ist demnach ebenfalls einfacher und gleichf"ormiger, und selbst noch mehr als an der Agramer Masse, da die Anzahl der Streifen im Ganzen noch bedeutend geringer ist und diese noch weit seltener durch Risse und Kl"ufte unterbrochen werden.

Die Streifen selbst, da sie im Ganzen ungleich weniger zahlreich sind, durchschneiden sich weit seltener, sind demnach um so l"anger, so dass die meisten von 6 bis 7, viele selbst von 12 bis 15 Linien L"ange erscheinen; inzwischen finden sich doch auch viele $\mathfrak{\frac{3}{4}}$, 2 bis 4 Linien lang. Sie haben dabei eine ungleich st"arkere Breite als an den beiden vorigen Massen, die meisten zwischen $\mathfrak{\frac{(7 und 9)}{12}}$ bis zu $\mathfrak{1\frac{3}{12}}$ Linie, daher das Gef"uge im Ganzen noch ein ungleich gr"oberes Ansehen hat, als das der Agramer Masse. Sie sind "ubrigens etwas gleichf"ormiger verteilt, oder, wenigstens den verschiedenen Richtungen nach, weniger partienweise zusammen geh"auft, dagegen bei weitem weniger scharf begrenzt, und selten geradlinig, sondern meistens bauchig und geschweift und oft wie ausgeflossen; so dass viele der k"urzeren, bei ihrer Breite, oft als Flecke erscheinen und dadurch die Regelm"a"sigkeit des Gef"uges st"oren.

Auf der ge"atzten Fl"ache haben diese Streifen ein beinahe durchaus ganz glattes, gar nicht narbiges oder faltiges, sondern nur bisweilen ein etwas streifiges Ansehen, eine zinkgrauliche, mehr ins Bl"auliche als Wei"se ziehende Farbe, und einen etwas st"arkeren, und zwar schimmernd seiden-fast atlasartigen, metallischen Glanz. Nur wenige erscheinen gestreift, und diese nur zum Teil und durch einzelne, weit abstehende und abgebrochene Schraffierungsleisten; dagegen finden sich in denselben einzelne K"orner und Massen jener br"ocklig-k"ornigen Substanz, von allen Gr"o"sen und Gestalten, als erhabene Punkte, Flecke, Winkelz"uge, Linien, eingewachsen und fest eingeschlossen "au"serst h"aufig, und von licht stahlgrauer, ins Silberwei"se fallender Farbe, mit starkem, bei schiefer Richtung, metallisch spiegelndem Glanze. Die Einfassungsleisten haben hier eine etwas matte, stahlgraue Farbe.

Die Zwischenfelder oder Figuren, welche hier ungeachtet der geringeren Anzahl der Streifen, wegen gleichf"ormigerer Verteilung derselben, verh"altnism"a"sig h"aufiger und aus denselben Gr"unden bei weitem gr"o"ser, eben deshalb aber auch seltener als Dreiecke, mit oben angegebenen Winkelma"sen, sondern meistens in rhomboidalen oder trapezoidalen, oft sehr kleinen, Segmenten derselben erscheinen --- sind beinahe durchgehends, und zwar "au"serst zart und dicht, gew"ohnlich nach zwei auch drei, den Seiten parallelen Richtungen, und mit all den, oben bei der Elbogner Masse bereits erw"ahnten, Modifikationen, teilweise, zumal an den R"andern, oder durchaus schraffiert, oder doch durch ebenso zarte mikroskopische Punkte rau. Da jene Schraffierungsleisten und diese Punkte erhaben sind, so erscheinen auch alle diese Felder --- und daher weit mehrere als an beiden vorigen Massen --- nicht blo"s im Umrisse (durch die Einfassungsleisten), sondern mehr oder weniger, ihrer ganzen Oberfl"ache nach, im Abdrucke ausgedruckt, und da jene Leisten und Punkte eine gl"anzende, ins Silberwei"se fallende Farbe haben, so geben sie ihrer Menge, Zartheit und Dichtheit wegen, der Oberfl"ache dieser Felder, die an sich matt und dunkel eisengrau w"are, ein "ahnliches Ansehen und einen seidenartigen Schimmer, wodurch selbst die ganze Fl"ache ein lichteres und gl"anzenderes Aussehen bekommt. Nur einzelne wenige und meist sehr kleine Felder zeigen sich, auch unter dem Mikroskope, ganz glatt, und dann etwas vertieft, wenigstens im Mittel, und von matter, dunkler, selbst schw"arzlich-grauer, oft ganz schwarzer Farbe. Gr"o"sere Kl"ufte oder Risse, welche mehr oder weniger mit jener br"ocklig-k"ornigen Substanz ausgef"ullt w"aren, finden sich hier beinahe gar nicht; dagegen --- obgleich nicht so h"aufig wie im Elbogner Eisen, daf"ur aber in gr"o"seren Partien (von 4 bis 5 Linien im Durchmesser) --- jene dichte, harte, schw"arzlich-eisengraue metallische Substanz in rundlichten oder ovalen (hier bisweilen l"anglichten und linienf"ormigen) Massen fest eingeknetet, und gleichsam eingekeilt. Merkw"urdig ist, dass diese f"ur sich scharf begrenzten Massen (hier wenigstens besonders deutlich) fast durchaus und rings um ihren Rand von einem schmalen, aber ungleich breiten Saume von jener k"ornig-br"ockligen Substanz, von gew"ohnlicher Beschaffenheit, Farbe und Glanz, umgeben, eingefasst und durch denselben von der "ubrigen Metall-Masse fast vollkommen geschieden sind.\footnote{\frakfamily{An dem gro"sen, bei 37 Pfund wiegenden St"ucke, welches Herr Baron von Brudern von diesem Meteor-Eisen besitzt, scheint eine Masse der Art, gleichsam wie ein an Dicke etwas abnehmender, langer, rundlichter Zapfen, durch die ganze H"ohe des St"uckes durchzugehen, wenigstens zeigt sich dieselbe auf der einen Abschnittsfl"ache als ein unvollkommen rundlichter Fleck, von 4 Linien in Durchmesser und vollkommen senkrecht unter demselben auf der entgegen gesetzten Abschnittsfl"ache, auf mehr als 6 Zoll Tiefe, zeigt sich ein "ahnlicher (und hier einziger), etwas ovaler (von $\mathfrak{2\frac{1}{4}}$ : $\mathfrak{3\frac{3}{4}}$ Linien in beiden Durchmessern), der jenem vollkommen entspricht, und denselben aufs Haar zentriert. Es w"are denn doch ein ganz besonderer Zufall, wenn sich zwei blo"s oberfl"achliche Flecke oder nicht tief eindringende Massen von derselben Substanz, Beschaffenheit und Form, auf zwei entgegen gesetzten und doch so weit voneinander abstehenden Fl"achen von betr"achtlicher Ausdehnung, und wo sie, wenigstens hinsichtlich ihrer Gr"o"se, einzeln stehen, so haarscharf begegnen sollten, ohne miteinander in wirklicher Verbindung zu stehen. De Gegenfl"ache von jener ersteren Abschnittsfl"ache befindet sich an dem, $\mathfrak{5\frac{3}{4}}$ Pfund wiegenden St"ucke der kaiserl. Sammlung, das von jenem abgeschnitten worden war, und hier fand sich auch die Fortsetzung jenes pr"asumierten Zapfens als ein ganz "ahnlicher Fleck. In der Hoffnung, dass die Masse auch hier noch wenigstens auf einige Tiefe gehen w"urde, lie"s ich eine 3 Linien dicke Platte, der Fl"ache horizontal und dicht an einem Rande dieses Fleckes abschneiden, in der Absicht, diese Masse dann aus der Platte herausbrechen und f"ur sich chemisch untersuchen zu machen. Leider ward ich aber in meiner Hoffnung get"auscht, denn die Masse fand sich kaum auf $\mathfrak{\frac{1}{4}}$ Linie tief eingedrungen. Da die andere Abschnittsfl"ache jenes St"uckes von dem in Nation-Museum zu Pesth aufbewahrten, bei 134 Pfund wiegenden Hauptst"ucke genommen ist, so muss sich dort auf der diesem Abschnitte entsprechenden Fl"ache die weitere Fortsetzung oder das andere Ende jenes Zapfens finden.}}

Dieselbe Tafel gibt ferner eine "ahnliche Darstellung einer ebenso ge"atzten Fl"ache an dem St"ucke vom mexikanischen Gediegeneisen, welches die kaiserl. Sammlung der Mitteilung Klaproths verdankt.

Es zeigt dieselbe ziemlich wesentliche Abweichungen im Einzelnen des Gef"uges von den vorhergehenden, und es scheint beinahe als w"are dieses durch irgendeine mechanische Gewalt, etwa beim Lostrennen dieses St"uckes von der Stamm-Masse, oder einem gr"o"seren St"ucke, durch, vielleicht nach einer Richtung fortgesetztes, Mei"seln, H"ammern oder Schlagen in etwas ver"andert worden. Die Streifen erscheinen n"amlich beinahe ausschlie"slich nur nach zwei, und zwar oft ziemlich rechtwinkelig sich durchschneidenden, Richtungen und in diesen selbst nicht immer vollkommen parallel und sogar gekr"ummt und gebogen; so dass die Zwischenfelder zum Teil sehr ungleichartige und selbst verzogene, vierseitige Figuren, Parallelepipeden, Rhomben, Rhomboiden, Trapezen, aber nie Dreiecke bilden.

Die Streifen sind "ubrigens eben so zart und scharf begrenzt, wie bei der Elbogner und Agramer Masse, und da sie ziemlich zahlreich und dabei gleichf"ormiger als bei jenen verteilt und nicht so partienweise nach einer Richtung zusammen geh"auft sind; so durchschneiden sie sich umso h"aufiger, erscheinen demnach im Ganzen k"urzer, und bilden im Verh"altnis h"aufige, aber kleine Zwischenfelder. Das Gef"uge erh"alt dadurch ein viel feineres und zarteres Ansehen, so wie es auch, da eine Richtung von Streifen beinahe ganz fehlt (denn es zeigen sich nur einzelne wenige, und diese nur undeutlich in einer dritten schiefen Richtung), viel einfacher und gleichf"ormiger erscheint.

Auf der ge"atzten Fl"ache zeigen diese Streifen eine sehr unebene, narbige Oberfl"ache, "au"serst selten eine Spur von Schraffierung, und eine ganz matte, schw"arzlich eisengraue, nur hie und da etwas ins Zinkwei"se fallende Farbe, so dass sie von den nur etwas weniger vertieften Zwischenfeldern kaum zu unterscheiden sind, die ein ganz "ahnliches Ansehen, aber, insofern sie nicht schraffiert sind --- was jedoch ebenfalls nicht h"aufig und meistens nur zum Teil und nach einer Richtung der Fall ist --- eine glatte Oberfl"ache haben.

Nur die erhabenen Einfassungsleisten, die Schraffierungsleisten aber nur zum Teil, zeigen, und selbst dieses nur bei einer schiefen Wendung, eine licht stahlgraue Farbe und einen starken metallischen Glanz.

Au"ser einigen K"ornern und kleinen Massen in den Streifen, findet sich von der br"ocklig-k"ornigen Substanz in einzelnen kleinen Rissen und Kl"uften die Spur, am meisten aber in einer gro"sen rissartigen, ganz damit angef"ullten Kluft, die das St"uck der Quere nach in einer etwas gebogenen Richtung, aber hier von keiner betr"achtlichen Tiefe mehr, beinahe ganz durchzieht.\footnote{\frakfamily{Ein diesem am meisten "ahnliches Gef"uge zeigen die gr"o"seren, zu einer "Atzung geeigneten Massen oder K"orner von Gediegeneisen, welche sich bisweilen als Gemengteile in der Steinmasse von Meteor-Steinen isoliert eingeschlossen finden, aber dieses nur fleck- oder stellenweise. Es zeigen sich n"amlich unter der Lupe auf der gleichf"ormigen, glatten, matt eisengrauen Oberfl"ache Stellen welche gestreift erscheinen, und zwar durch erhabene, mikroskopisch zarte, lichtere und etwas gl"anzende Linien, die gr"o"sten Teils nach zwei sich durchkreuzenden Richtungen parallel laufen und ein enges Netz mit rhomboidalen und trapezoidalen, vertieften und etwas dunkler gef"arbten, matten Zwischenfeldern bilden, wie dies z. B. jenes gro"se Korn in dem auf der siebenten Tafel von der abgeschliffenen Fl"ache vorgestellten Meteor-Steine von Salés sehr deutlich zeigt, aber der mikroskopisch zarten Beschaffenheit wegen nicht dargestellt werden konnte. (Bemerkenswert ist, dass in diesem abgeschliffenen und ge"atzten Korne derben, gediegenen Metalle --- nebst Atomen von der br"ocklig-k"ornigen Substanz von ins R"otliche ziehender Farbe --- zwei kleine unf"ormlich eckige K"orner von unver"anderter und von der S"aure unangegriffener Steinmasse eingekeilt erscheinen.) An jenen kleinen Massen Gediegeneisen, welche in Gestalt wahrer Zacken in der Steinmasse der Meteor-Steine vorkommen, habe ich bisher durch "Atzung keine Spur eines Gef"uges oder irgendeiner Heterogenit"at des Metalle erhalten k"onnen, und die Oberfl"ache derselben zeigte sich stets gleichf"ormig an Farbe und Glanz, jene war aber lichter und dieser st"arker als an jenen gr"o"seren, derberen Massen.\\
\hspace*{6mm}Es d"urfte wohl voreilig scheinen entscheiden zu wollen, welcher von jenen vier Metall-Massen, dem Gef"uge nach --- dessen Darstellung und Beschreibung vergleichend gegeneinander zu stellen hier versucht worden ist --- in Hinsicht auf Vollkommenheit oder Vollendung in der Ausbildung, der Vorzug geb"uhre; inzwischen will ich mir doch erlauben eine Vermutung zu "au"sern. Das Gef"uge der Elbogner Masse zeigt von allen unstreitig den h"ochsten Grad von Ausscheidung und regelm"a"siger Absonderung der einzelnen, mehr oder weniger verschiedenartig erscheinenden Teile desselben, n"amlich: die h"aufigsten, zartesten, gleichf"ormigsten und am sch"arfsten begrenzten Streifen; die meiste, und zwar der vorauszusetzenden Grund-Kristallisation --- dem regelm"a"sigen Oktaeder --- am vollkommensten entsprechende Regelm"a"sigkeit und Gleichf"ormigkeit der Zwischenfelder und in deren Schraffierung; die vollkommenste, h"aufigste und zum Teil selbst etwas regelm"a"sige Ausscheidung der br"ocklig-k"ornigen, und die nicht minder h"aufige der "ahnlichen, h"arteren Substanz, so wie die Ausgesprochenheit aller dieser Teile und der Substanzen aus welchen sie gebildet sind, in "au"sern Ansehen sowohl als in den physischen Eigenschaften, auf deren Eigent"umlichkeit und Reinheit hinzudeuten scheint. Das Gef"uge aller "ubrigen zeigt dagegen, und zwar in derselben Reihenfolge in welcher sie dargestellt und beschreiben worden sind, ungleich mehr Einfachheit; aber eben diese Einfachheit hat offenbar ihren Grund in einer minder h"aufigen und weniger scharfen Absonderung der homogenen Teile, und in einer mangelhaften Ausscheidung der heterogenen Substanzen und Stoffe --- die doch in der Total-Masse vorhanden zu sein scheinen --- welches einerseits in der geringeren Menge, minder scharfen Begrenzung und weniger regelm"a"sigen --- wenigstens der vorauszusetzenden Grundform im Allgemeinen minder entsprechenden Absonderung jener, und in der Mangelhaftigkeit oder doch geringeren Menge dieser, und in der weniger ausgesprochenen Eigent"umlichkeit und Heterogenit"at der vorhandenen, hinl"angliche Bekr"aftigung finden d"urfte. Insofern demnach die Vollkommenheit oder ein h"oherer Grad von Ausbildung dieser Massen "uberhaupt, in der h"aufigeren und vollkommeneren Ausscheidung der heterogenen Bestandteile, und in der sch"arfern Absonderung und regelm"a"sigeren Lagerung der aus ihnen einzeln bestehenden, oder aus der neuen Verbindung einiger derselben gebildeten Substanzen zu suchen ist; insofern m"ochte wohl die Elbogner Masse unter den hier abgehandelten den ersten Anspruch darauf machen d"urfen.\\
\hspace*{6mm}Lange nachdem diese Note schon niedergeschrieben war, und eben als dieser Bogen der Presse "ubergeben werden sollte, erhalte ich durch Herrn v. Widmannst"atten die Resultate einiger physisch-technischer Versuche, welche derselbe auf meine Bitte mit dem uns so sehr problematisch scheinenden Kap’schen Gediegeneisen, soweit es der Drang der Zeit und der Umst"ande gestattete, zum Behufe dieser Ausarbeitung noch vorzunehmen die G"ute hatte. Es zeigte sich nach denselben an dieser Masse weder im Schliffe, noch beim Anlaufen, noch durch "Atzung, auch nur die entfernteste Spur eines Gef"uges.\\
Blank polierte Fl"achen zeigten denselben metallisch spiegelnden Glanz, dieselbe, das Meteor-Eisen auszeichnende, licht stahlgraue, stark in Silberwei"se fallende Farbe, einen hohen Grad von Dichtheit und eine Gleichf"ormigkeit in dieser, die selbst nicht im Geringsten durch eine heterogene, eingesprengt oder in Rissen enthaltene Substanz unterbrochen erschien, und die sich auch im Schnitte bewh"arte, bei welchem jene h"aufigen, harten, spr"oden, die Sage verw"ustenden Stellen nicht beobachtet wurden.\\
\hspace*{6mm}Salpeters"aure brachte auf solchen Fl"achen, und zwar ohne merkliche Entwickelung von Schwefelwasserstoffgas, nur einige gr"o"sere und kleinere, meistens geflammte und allm"ahlich sich verlaufende, selten etwas sch"arfer begrenzte, eisen- oder mehr oder weniger schwarzgraue, matte Flecke zum Vorschein, welche auf eine Ungleichartigkeit der Substanz und auf eine unvollkommene Ausscheidung des einen Anteiles schlie"sen lie"sen. Unter der Feile und S"age zeigte sich die Masse im Ganzen vollkommen und ziemlich gleichf"ormig geschmeidig, wie gew"ohnliches, sehr dichtes und weiches Eisen, aber nicht so weich wie die Streifen-Substanz des Gef"uges der beschriebenen Gediegeneisen-Massen. Jene bemerkte Ungleichartigkeit der Substanz sprach sich aber bei Untersuchung einzelner St"ucke f"ur sich, die, so viel als bei der unvollkommenen Absonderung jener m"oglich war, durch mechanische Trennung erhalten wurden, sehr auffallend aus. M"oglichst reine St"ucke des gl"anzenden, lichtern Anteiles zeigten einen sehr dichten, gl"anzenden, wei"sen Bruch und einen hohen Grad von Geschmeidigkeit, so dass sich ein etwa 45 Gran wiegendes St"uckchen sehr gut zu einem beinahe 3 Zoll langen St"abchen hei"s strecken lie"s; St"ucke vom grauen Anteile dagegen zeigten einen feink"ornigen, matten, schwarzgrauen, sehr schnell br"aunlich sich beschlagenden Bruch, und gaben im Feilen einen zwar metallischen, aber grauen Strich, und nur sehr wenige Sp"ane, sondern gr"o"sten Teils ein schwarzes Pulver. Einzelne St"ucken davon hielten in der Rothitze nur einige schwache Hammerschl"age aus, und lie"sen sich damit etwas weniges zusammendr"ucken, zerbr"ockelten aber beim dritten, vierten Schlage; andere zersprangen selbst beim ersten Schlage schon. Beide Anteile lie"sen sich durchaus nicht h"arten, ersterer schien sich aber --- so viel ein Versuch im Kleinen lehren konnte --- leicht schwei"sen zu lassen. Beide zeigten starke Wirkung auf den Magnet, aber schwache Polarit"at, nahmen diese aber durch Streichen bald mehr an, und der erstere erhielt dadurch eine betr"achtliche magnetische Kraft.\\
\hspace*{6mm}Das spezifische Gewicht der Masse im Ganzen fand Herr v. Widmannst"atten = 7,318 (also betr"achtlich unter jenem der "ubrigen von mir darauf untersuchten Gediegeneisen-Massen, bei welchen ich dasselbe, wie ich schon in einer fr"uheren Note bemerkte, zwischen 7,600 und 7,830 fand; n"amlich: von der Elbogner = 7,800-7,830; der Agramer = 7,730 bis 7,800; der Lénartoer = 7,720-7,800; der Mexikaner = 7,600-7,670; der Peruaner = 7,600-7,650; und selbst noch unter jenem des sibirischen = 7,540-7,570; aber h"oher als jenes der in Meteor-Steinen eingemengten Gediegeneisen-K"orner = 6,000-6,600; dagegen dem angenommenen Mittelgewichte des Roheisens = 7,200-7,500 am n"achsten kommend); jenes des wei"sen Anteiles f"ur sich, nach dem verschiedenen Grade der Reinheit der St"ucke, zwischen 7,633 bis 7,877 (also zum Teil weit "uber dem angenommenen Mittelgewichte des gew"ohnlichen, weichen und geschmeidigen Eisens = 7,700); jenes des grauen dagegen zwischen 6,655 und 6,926 (demnach weit unter jenem des Roheisens), von welchen beiden nun das arithmetische Mittel eine der obigen ganz "ahnliche Zahl gibt, und in welchen der Grund der Differenzen des Befundes Anderer zu suchen ist (so gab v. Dankelmann das spezifische Gewicht dieser Masse mit 7,708; Van Marum mit 7,654 an, indes ich es einst = 7,260 gefunden hatte).\\
\hspace*{6mm}Unter einem ward die uns nicht minder zweifelhafte peruanische Eisen-Masse nochmals gepr"uft, soweit es die Kleinheit des zu Gebote stehenden St"uckes erlaubte. Auch diese zeigte keine Spur von jenem eigentlichen Gef"uge; unter der Lupe erschienen aber doch auf der stark ge"atzten, kleinnarbigen Oberfl"ache viele, "au"serst zarte, mikroskopisch feine, erhabene Linien, die nach mehreren, offenbar vorherrschend aber nach drei, zumal zwei, Richtungen meist gerade, nur selten etwas gebogen, und stets parallel verlaufend, sich durchkreuzen, und hie und da ein sehr enges Netz bilden, ganz "ahnlich jenem auf der ge"atzten Oberfl"ache des gro"sen Metall-Kornes in dem oben beschriebenen und auf der siebenten Tafel abgebildeten Meteor-Steine von Salés, und, stellenweise, jenem auf der ge"atzten Fl"ache des Mexikaner Eisens. Eine polierte Fl"ache zeigte eine besonders stark ins Wei"se fallende, die ge"atzte aber eine beinahe zinnwei"se Farbe. Unter der Feile gab sich die Masse merklich h"arter als die Kap'sche zu erkennen. Das spezifische Gewicht fand Herr v. Widmannst"atten = 7,646.\\
\hspace*{6mm}So sehr nun auch diese Resultate in vielen Beziehungen von jenen abweichen, welche sich bei "ahnlicher Untersuchung der "ubrigen, oben beschriebenen, derben Gediegeneisen-Massen ergaben, und die Zweifel "uber den pr"asumtiven meteorischen Ursprung dieser beiden vermehren (so dass selbst Herr v. Widmannst"atten mehr geneigt w"are, zumal die Kap'sche, f"ur das Produkt eines k"unstlichen Schmelz- und unvollkommenen, unvollendeten Verfrischungs-Prozesses anzusehen, welche Mutma"sung durch v. Dankelmanns Nachrichten von der geognostischen Beschaffenheit jener Gegend, wo diese Masse urspr"unglich gefunden worden war, und wo Eisenerze aller Art und in gro"ser Menge zu Tage stehen --- welche vielleicht einst von den Bewohnern zu Gute gemacht wurden --- auch von dieser Seite einige Wahrscheinlichkeit erh"alt); so finde ich doch darin keinen Bestimmungsgrund, meine sowohl in einer fr"uheren Note "uber das Eigent"umliche und Charakteristische des Gef"uges am Meteor-Eisen, als in dieser "uber den verschiedenen Grad von Vollkommenheit desselben und den, diesem wahrscheinlich zum Grunde liegenden Ursachen, ausgesprochenen Ideen und Mutma"sungen abzu"andern, auch selbst dann nicht, wenn auch jene Zweifel (gegen welche die ausgezeichnete Farbe, die offenbare und ganz eigent"umliche Mengung, und das betr"achtliche spezifische Gewicht der Masse, im Ganzen sowohl als insbesondere des einen Anteiles, vorz"uglich aber der erwiesene, selbst quantitativ entsprechende Gehalt an Nickel in Ber"ucksichtigung kommen) f"ur nicht hinl"anglich begr"undet erachtet werden sollten; im Gegenteile d"urfte ich darin vielmehr in jedem Falle einige Bekr"aftigung f"ur dieselben zu finden glauben.}}

Wird nun die "Atzung solcher Fl"achen noch l"angere Zeit (z. B. bis auf eine Tiefe von $\mathfrak{\frac{1}{4}}$ Linie) fortgesetzt; so sprechen sich die erhabenen und vertieften Stellen gegenseitig noch immer mehr aus, und es ver"andert sich zum Teil das Ansehen des Gef"uges oder der Zeichnung im Ganzen, indem zuletzt manche der tiefen --- namentlich die Streifen --- ganz verschwinden, und bisweilen andere --- gew"ohnlich ein Teil eines Zwischenfeldes --- an ihre Stelle treten.

Die Streifen erscheinen nun als seichtere und tiefere, mehr oder weniger leere Kan"ale oder Rinnen (Geleise), indem oft nur die erhabenen Begrenzungslinien oder die Einfassungsleisten als ihre gemeinschaftlichen Scheidew"ande, als Kontur, und ihr Boden, der mit diesen von einerlei Beschaffenheit ist und mit denselben ein Ganzes, gleichsam eine Rinne bildet, welche die Substanz der Streifen selbst einschloss, noch "ubrig sind, letztere aber von manchen ganz ausge"atzt und von der S"aure aufgel"ost worden ist. Hat man demnach ein Pl"attchen von einer bestimmten Dicke (z. B. von $\mathfrak{\frac{1}{4}}$ oder $\mathfrak{\frac{1}{2}}$ Linie) einem solchen Versuche unterzogen, so erscheinen manche dieser Streifen --- insofern sie tiefer gingen als die Platte dick war, und ihre Substanz solcher Gestalt zuf"allig, und zwar noch "uber den Boden des Canals getroffen worden ist --- ganz ausge"atzt, ohne Boden, und die Einfassungsleisten stehen als die ehemaligen Scheidew"ande, wie Lamellen, frei da, und h"angen nur mittelst ihrer Enden mit den "ubrigen minder angegriffenen Teilen (den angrenzenden, quer gehenden, "ahnlichen Leisten, oder mit solchen von Feldern) der Masse zusammen.\footnote{\frakfamily{Wenn zuf"allig --- was jedoch selten der Fall ist --- aller Zusammenhang fehlt, so fallen solche Lamellen, einzeln oder mehrere unter sich oder mit einem Zwischenfelde verbunden, ab und finden sich als solche --- unver"andert in der Aufl"osung --- am Boden des Gef"a"ses worin die "Atzung geschah.}} Manche, und zwar ungleich mehrere (zumal wenn die Platte $\mathfrak{\frac{1}{2}}$ Linie dick war, da nur "au"serst wenige so tief gehen), die seichter lagen, erscheinen als leere Rinnen von verschiedener Tiefe, und auf der entgegen gesetzten Seite des Pl"attchens (die, falls dasselbe von beiden Seiten gleichzeitig ge"atzt wurde, eine ganz andere Zeichnung und Verteilung der Streifen und Felder zeigt) finden sich unter ihnen Felder, oder zum Teil Streifen in einer andern Richtung, auf welchen sie mit ihrem Boden auflagen, der nun --- falls er nicht etwa wegen allzu seichter Lage der enthaltenen Substanz und der langen Dauer des Prozesses ebenfalls wegge"atzt wurde -- mit den Einfassungsleisten auf denselben aufsitzt.\footnote{\frakfamily{Es ergibt sich hieraus, dass sowohl die Streifen als Felder, so wie auch ihre gemeinschaftlichen Scheidew"ande --- die Einfassungs- und Schraffierungsleisten --- und "uberhaupt alle einzelnen, mehr oder weniger verschiedenartigen Teile des Gef"uges die sich durch die "Atzung aussprechen, nur auf eine gewisse, und zwar nicht sehr betr"achtliche (wie es scheint, selten "uber $\mathfrak{\frac{1}{2}}$ Linie reichende), "ubrigens aber sehr unbestimmte und ungleichf"ormige Tiefe gehen und unordentlich "uber einander geh"auft, nur mit einiger Regelm"a"sigkeit in der Ausscheidung, Absonderung und Lagerung unter sich, die Total-Masse konstruieren. Am deutlichsten zeigt sich dies an zwei W"urfeln (von beil"aufig 4 Linien Seite), die ich aus einem St"ucke von der Elbogner Masse ausschneiden und woran ich an dem einen noch eine Ecke abstumpfen lie"s, und dann beide ringsherum, auf allen Fl"achen und Kanten, gleichzeitig und gleichf"ormig "atzte. Jede Fl"ache auf denselben zeigt nun eine andere Zeichnung oder Gruppierung der Streifen und Felder, und manche von diesen oder jenen, oft von beiden, so auch die mit der br"ocklig-k"ornigen Substanz mehr oder weniger ausgef"ullten Risse und Kl"ufte, setzen sich "uber die gemeinschaftliche Kante auf die n"achstansto"sende Fl"ache mehr oder weniger weit fort, je nachdem sie gerade an diesen Stellen tiefer oder seichter in die Masse eingedrungen waren.}}

Die Einfassungsleisten, die nun mehr oder weniger frei da stehen, zeigen --- was zum Teil bei einer minder tiefen "Atzung schon beobachtet werden kann --- ("ubrigens aber auch vom Schnitte abh"angen mag, je nachdem dieser die Richtung derselben traf) eine, gegen die Ebene der Fl"ache, etwas schiefe --- doch immer unter sich parallele --- Lage, und gleichen papierblattd"unnen Lamellen von der L"ange der vormaligen Streifen, einem gegenseitigen Abstande welcher der Breite dieser entspricht, und von sehr ungleicher H"ohe oder Tiefe, je nachdem die enthaltene Substanz mehr oder weniger in die Tiefe ging und ausge"atzt worden ist. Kurz, sie bilden paarweise die Seitenw"ande eines schr"agen, aber gleich weiten Canals, in welchem die Streifen-Substanz eingeschlossen war, und sind nach unten durch eine "ahnliche Lamelle verbunden und geschlossen, welche solcher Gestalt den Boden des Canals vorstellt. Boden und W"ande haben ein etwas unebenes, gebogenes und gleichsam faltiges, oder vielmehr breit und flach gefaltetes Ansehen, eine stahlgraue, stark ins Silberwei"se fallende aber meistens eisengrau angelaufene Farbe, und einen schwachen metallischen Glanz, indes der obere Rand der W"ande (Einfassungsleisten) lichter und gl"anzender ist.

Wo zwei oder mehrere Streifen einer Richtung dicht an einander liegen und durch solche Lamellen dem Anscheine nach nur einfach getrennt sind, scheinen diese doch alle Mahl, wenigstens hie und da im Verlaufe, doppelt oder doch dicker und gleichsam aus zwei zusammen geschmolzen zu sein, schlie"sen, in eben diesem Grade mehr oder weniger deutlich, etwas von der den Feldern oder Figuren eigent"umlichen Substanz ein, und bilden kleine, oft nur linienf"ormige (doch immer Segmenten der vorkommenden Dreiecke entsprechende) "ahnliche Zwischenfelder; so dass demnach Streifen und Felder und die verschiedenartige Substanz beider stets und regelm"a"sig abwechseln und jene Lamellen oder Einfassungsleisten gleichsam als Trennungsmittel dienen und die gemeinschaftliche Scheidewand bilden.

Ebenso, wie diese Lamellen, zeigen sich auch die Schraffierungsleisten, in den Streifen sowohl als auf den Figuren --- wie sie denn auch in der Tat und in jeder Beziehung "ahnliche Einfassungsleisten, obgleich im mikroskopisch Kleinen, vorstellen --- nur etwas weniges minder erhaben --- zumal die auf den Figuren als die feinsten --- indem sie, verm"oge ihrer Zartheit, doch etwas mehr von der S"aure angegriffen worden zu sein scheinen.

Auf der Oberfl"ache der noch r"uckst"andigen Streifen-Substanz, oder wo diese ganz ausge"atzt ist, auf dem Boden der Streifen-Kan"ale, finden sich hie und da K"orner oder kleine Massen von jener, bereits fr"uher bemerkten, br"ockligk"ornigen Substanz, fest aufsitzend oder gleichsam eingewachsen, und zwar teils ebenso erhaben wie die Leisten, teils tiefer, teils ganz tief, je nachdem sie urspr"unglich seichter oder tiefer in der Streifen-Substanz eingeschlossen waren, und je nachdem diese mehr oder weniger ausge"atzt wurde.\footnote{\frakfamily{Nach Beendigung eines solchen fortgesetzten "Atzungsversuches finden sich demnach auch auf dem Boden des Gef"a"ses viele einzelne K"orner und Massen von dieser Substanz, die frei wurden, als jene der Streifen, in welche sie eingemengt gewesen, von der S"aure aufgel"ost worden war, und die --- so wie oben von den unzusammenh"angenden Einfassungsleisten bemerkt wurde --- unver"andert in der Aufl"osung zu Boden fielen, da sie wie diese, von der S"aure (zumal von Salpeters"aure im diluierten Zustande und bei langsamen Gange des Prozesses) wenig oder gar nicht angegriffen werden.}}

Die Zwischenfelder oder Figuren erscheinen auch nach dieser starken "Atzung nur wenig minder erhaben als die Leisten, haben aber, abgesehen von der Schraffierung, einiger Ma"sen ihr Ansehen ver"andert, und ihre Oberfl"ache zeigt sich nun eisenschwarz, rau wie bestaubt, und zum Teil "au"serst zart und verworren, unvollkommen faserig (beinahe wie feine ausgebrannte Steinkohlen, \emph{cox}) und matt, mit einem licht eisengrauen Schimmer.

Die Beschaffenheit des solcher Gestalt durch "Atzung zum Vorschein gebrachten Gef"uges dieser Massen, die verschiedene Art der Absonderung, Lagerung und Gestaltung, und das verschiedenartige "au"sere Ansehen der dasselbe konstruierenden Teile, lassen demnach eine vier- und zum Teil f"unffache Verschiedenheit der Substanzen erkennen und unterscheiden; n"amlich: jene der am meisten von der S"aure angegriffenen Streifen, die der etwas minder angegriffenen Zwischenfelder, und die der am wenigsten angegriffenen Einfassungs- und Schraffierungsleisten; ferner jene der mechanisch eingemengten, br"ocklig-k"ornigen, ebenfalls nur sehr wenig angegriffenen, und endlich die --- der letzteren am n"achsten verwandt scheinenden --- der auf "ahnliche Art und gleichsam eingekeilt vorkommenden, auch nur wenig angegriffenen, rundlichen Massen. Und diese Verschiedenheit spricht sich auch noch durch andere physische Merkmahle, insbesondere durch die H"arte aus.\footnote{\frakfamily{Diesem verschiedenen Grade von H"arte und Geschmeidigkeit der verschiedenen, das Gef"uge konstituierenden Teile und Substanzen, ist wohl das oben erw"ahnte oberfl"achliche Erscheinen desselben nach Schliff und erster Politur einer Fl"ache zuzuschreiben, so wie der Umstand des Wiederverschwindens bei der feinen Politur dadurch erkl"arlich wird, dass jene Verschiedenheit der Wirkung des zur letzteren gebrauchten sch"arferen und feineren Polierpulvers nicht im Wege steht.}} Die Substanz der Streifen zeigt sich n"amlich (mit einer Stahlnadel geritzt, selbst nur stark gedr"uckt) am weichsten ( beinahe so weich wie Bley) und am geschmeidigsten, und gibt einen noch lichtern und etwas gl"anzenderen Strich; jene der glatten (nicht schraffierten) Felder zeigt sich, obgleich nur wenig, doch hinl"anglich merklich h"arter, und gibt einen "ahnlichen, zinkgrauen, mehr oder weniger ins Silberwei"se fallenden, doch aber minder lichten und minder gl"anzenden Strich, der ein dunkelgraues Pulver abscheidet, welches wie ein Anflug die Oberfl"ache bedeckte; die Substanz der Einfassungsleisten endlich ist betr"achtlich h"arter, wenigstens bedeutend z"aher, nimmt aber doch den Eindruck an, der Farbe und Glanz unver"andert zeigt, und ebenso verh"alt sich die der Schraffierungsleisten, insofern deren Zartheit den Versuch nicht beg"unstiget, was inzwischen, zumal auf der Oberfl"ache der Felder, wo sie oft als mikroskopisch feine F"aden erscheinen, so weit geht, dass sich eine ganze Partie derselben mit einem Striche oder Drucke zerst"oren l"asst. Die br"ocklig-k"ornige Substanz dagegen verh"alt sich betr"achtlich hart und vollkommen spr"ode, ist etwas schwer zersprengbar, l"asst sich aber doch zum feinsten Pulver zersto"sen und zerreiben, das dann schw"arzlichgrau erscheint, und durchaus, ohne Ausnahme eines St"aubchens beinahe, dem Magnete folgt; die schw"arzlichgrauen rundlichen Massen aber zeigen sich nicht nur derb und dicht, sondern noch h"arter (doch nicht Funken gebend) und schwerer zersprengbar, und zwar auch spr"ode, aber mit einiger Z"ahigkeit, so dass sich die Substanz gr"o"sten Teils schwer zersto"sen und noch schwerer zerreiben l"asst. Geritzt gibt ihre Oberfl"ache einen ziemlich scharfen, stahlgrauen, etwas ins R"otliche fallenden, schwach metallisch gl"anzenden Strich,\footnote{\frakfamily{Von diesem, zumal in gr"o"seren Massen so merklich werdenden, verschiedenen Grade von H"arte und Geschmeidigkeit der einzelnen, zum Teil so ungleich verteilten und hie und da partienweise so ungleichf"ormig angeh"auften Gemengteile, insbesondere aber von der so h"aufig und unregelm"a"sig durch die ganze Metall-Masse zerstreuten br"ocklig-k"ornigen und der ihr n"achst verwandten Substanz, r"uhrt denn auch die Schwierigkeit der technischen Bearbeitung des Meteor-Eisens im Gro"sen und Ganzen, die von mehreren Physikern bereits bemerkte Ungleichf"ormigkeit in der Schmiede- und Schwei"sbarkeit desselben (welche Vauquelin verleitete einen geringen Grad von Oxydation dieses Eisens im Allgemeinen anzunehmen), und vorz"uglich der Umstand her, dass sich solche Massen "au"serst schwer mit der S"age behandeln lassen, so dass man bei m"uhsam fortgesetzter Arbeit vieler Tage Zeit bedarf um auch nur ein Handgro"ses St"uck vollends abzus"agen, und dabei viele S"agen (aus Uhrfederbl"attern) zu Schanden arbeitet, indem man h"aufig auf Stellen st"o"st die einen gro"sen Widerstand zeigen, die S"agen stumpf machen und die Z"ahne ausbrechen, indes andere sich beinahe wie Blei schneiden.}} der ein schwarzes Pulver abscheidet. Das durch Zersto"sen und Zerreiben erhaltene Pulver ist nicht ganz gleichf"ormig, ballt sich zusammen und klebt etwas; einzelne St"uckchen widerstehen dem St"o"sel mehr, und zeigen sich minder retractorisch als die "ubrigen Atome, die fast durchgehends ziemlich stark, aber doch weniger als jene der br"ocklig-k"ornigen Substanz, auf die Nadel wirken.

Und diese Verschiedenheit, im "Au"sern sowohl als in den physischen Eigenschaften --- insofern letztere ohne mechanische Absonderung und partielle Untersuchung aller dieser Substanzen f"ur sich, was nicht wohl m"oglich ist, ausgemittelt werden konnten --- vollends aber der so auffallend sich aussprechende verschiedene Grad von Einwirkung des "Atzmittels auf dieselben --- der denn doch von einem verschiedenen Grade von Verwandtschaft derselben zur S"aure oder dem Oxygen abh"angt,\footnote{\frakfamily{Diesen verschiedenen Grad von Verwandtschaft zum Oxygen scheint wohl auch schon das oben erw"ahnte Resultat des Versuches mit dem Blau-Anlaufen polierter Fl"achen zu best"atigen. Und von demselben r"uhrt ohne Zweifel jene von selbst entstandene, einer schwachen "Atzung gleichende Andeutung des Gef"uges her, die sich durch vertiefte, den Streifen entsprechende Linien und erhabene, in jeder Beziehung mit den Figuren "ubereinstimmende Zwischenr"aume, zumal an einer der von dem unvollkommenen rindenartigen "Uberzuge entbl"o"sten Fl"ache der Masse von Elbogen ausspricht, indem diese --- so viel aus der dunkelen, mit Volkssagen und M"archen verwebten Geschichte "uber die Herstammung derselben bekannt ist --- durch mehrere Jahrhunderte den Injurien der Zeit und dem Einfl"usse der atmosph"arischen Luft und selbst des Wassers ausgesetzt war.}} scheinen wohl auf eine wesentliche, und zwar auf eine chemische Verschiedenheit hinzudeuten.\footnote{\frakfamily{Manche Physiker, welchen die Erscheinung des Gef"uges bei dieser Behandlung bekannt wurde, und die eine Mutma"sung dar"uber "au"serten (namentlich Neumann, Schweigger u. a.), glaubten den verschiedenen Grad von Einwirkung der S"auren, wodurch dasselbe zum Vorschein gebracht wird, von dem bekannten Gehalte dieses Eisens an Nickel, und von einer ungleichen Verteilung desselben in jenem --- oder vielmehr von der ungleichen Verteilung aber einer gewissen regelm"a"sigen Absonderung des damit legierten Anteiles von Gediegeneisen --- herleiten zu k"onnen, obgleich der von den Chemikern --- wenigstens anf"anglich --- als so h"ochst unbedeutend angegebene Gehalt der verschiedenen Meteor-Eisen-Massen an jenem Metalle (im Durchschnitt von $\mathfrak{1\frac{1}{2}}$ bis $\mathfrak{3\frac{1}{2}}$ Perzent, nach Klaproth wenigstens) eine so auffallende Wirkung kaum erwarten lie"s. Jene Vermutung h"atte daher durch die Resultate sp"aterer Untersuchungen Stromeyers, nach welchen jener Gehalt betr"achtlich h"oher (n"amlich zwischen 10 und 11 Perzent) gestellt wurde, einerseits sehr an Wahrscheinlichkeit gewonnen; allein mit diesem Befunde ergab sich auch, dass dieser Gehalt nicht nur unver"anderlich und bei allen Meteor-Eisen-Massen gleich gro"s, sondern dass derselbe auch stets durch die ganze Masse vollkommen gleichf"ormig verteilt sei, und solcher Gestalt lie"se sich, falls dies im strengsten Sinne zu nehmen w"are, jene Erscheinung durchaus nicht von demselben herleiten --- wie denn auch der --- von Stromeyer selbst ausgewiesene "ahnliche --- Gehalt an Nickel im Kap’schen Eisen (es mag dasselbe nun terrestrischen oder wirklich meteorischen Ursprungs sein), bei "ahnlicher Behandlung der Masse, sich wirklich gar nicht ausspricht ---; man m"usste denn zugeben --- was jenen Resultaten unbeschadet wohl auch der Fall sein kann --- dass bei jenen Massen doch eine, beziehungsweise wenigstens, ungleichf"ormige Verteilung des Nickels Statt f"ande. Sollte n"amlich der Gehalt an diesem Metalle, wenigstens zum Teil, die verschiedene Einwirkung der S"auren und somit die Erscheinung des Gef"uges wirklich veranlassen; so m"usste eine teilweise Verbindung desselben mit dem Eisen (folglich eine ungleichf"ormige Verteilung gegen dieses), und eine --- dem Gef"uge entsprechend regelm"a"sige Absonderung dieses Nickel-Eisens von dem "ubrigen (womit die Gleichf"ormigkeit der Verteilung des Nickels gegen die Masse im Ganzen ziemlich wieder hergestellt w"urde) angenommen werden, und in dieser Voraussetzung k"onnte man von allen Teilen des regelm"a"sigen, eigentlichen Gef"uges, die Einfassungs- und Schraffierungsleisten am passendsten daf"ur erkennen, als welche sich als die erhabensten, folglich als die von der S"aure am wenigsten angegriffenen, durch eigent"umliche H"arte und Z"ahigkeit, Farbe und Glanz (Eigenschaften, welche, \emph{a priori} zu schlie"sen, das Eisen durch eine solche Legierung h"ochst wahrscheinlich in einem solchen Grade erlangen d"urfte) ausgezeichneten Teile, und umso mehr daf"ur aussprechen, als auch ihre Masse zusammen genommen dem chemisch ausgewiesenen \emph{maximo} des Total-Gehaltes an Nickel, und ihre ziemlich gleichf"ormige Verteilung in der Gesamtmasse, der Forderung in dieser Beziehung, am meisten entspr"ache. Die am st"arksten angegriffene Substanz der Streifen k"onnte dann, allen ihren Eigenschaften gem"a"s, f"ur reines Gediegeneisen gelten; es bliebe demnach nur noch die dritte Substanz "ubrig, die sich durch das Gef"uge und durch ihre Eigenschaften als heterogen von jenen beiden ausspricht, n"amlich die der Zwischenfelder oder Figuren. Gillet de Laumont scheint geneigt, diese f"ur gekohltes Eisen anzusehen; und wirklich hat diese Vermutung vieles f"ur sich, zumal da Kohlenstoff als Bestandteil mehrerer Meteor-Steine, und selbst --- nach Tennant -- des Kap’schen Eisens (in welchem derselbe vielleicht, so wie der Nickel und die "ubrigen Bestandteile, ganz gleichf"ormig in der Masse verteilt, und nicht blo"s mit einem bestimmten Anteile des Gediegeneisens chemisch verbunden und mit diesem ausgeschieden, sich befindet), bereits dargetan ist.\\
\hspace*{6mm}Das Gef"uge w"are demnach die Folge einer mehr oder weniger vollkommenen Ausscheidung der fr"uher gleichf"ormig im Ganzen vermischt und gebunden gewesenen Bestandteile des Meteor-Eisens, des Nickels und Kohlenstoffes, einer neuen chemischen Verbindung derselben mit einem Teile des Gediegeneisens und einer mehr oder weniger vollkommen und regelm"a"sigen Absonderung und Gruppierung der solcher Gestalt verschiedenartigen, neu gebildeten Gemengteile von dem reinen Eisen.\\
\hspace*{6mm}Was die beiden andern metallischen Substanzen betrifft, die keine integrierenden Teile jenes Gef"uges ausmachen, und die ihre Heterogenit"at schon durch die Art ihrer Einmengung in die "ubrige Masse, und durch ihr "Au"seres zu erkennen geben; so spricht sich die eine, br"ocklig-k"ornige, durch alle oryktognostischen und physischen Merkmahle deutlich genug als Schwefeleisen, und zwar als Magnetkies aus, und es ist unbegreiflich, wie bei dessen gro"ser Menge --- indem die Massen ganz damit durchs"aet sind --- der eine Bestandteil desselben, n"amlich der Schwefel, bei den bisherigen Analysen solcher Massen (bis neuerlichst der sibirischen durch Laugier) so ganz aller Wahrnehmung entgangen sein konnte.\\
\hspace*{6mm}Die andere, derbere, dichtere, in rundlichen Massen bisweilen eingemengt vorkommende Substanz hat wohl viele "Ahnlichkeit mit jener, scheint aber doch wesentlich von ihr verschieden und vielleicht eine unvollkommene Ausscheidung oder ein R"uckstand des urspr"unglichen Total-Gemisches aller Bestandteile der metallischen Meteor-Masse, also gekohltes und geschwefeltes, und vielleicht auch noch mit Nickel verbundenes Gediegeneisen, und demnach, hier gleichsam als Gemengteil, in demselben Zustande zu sein, in welchem die Masse des Kap’schen Eisens noch im Ganzen sich befindet.\\
\hspace*{6mm}Auch diese Note war bereits niedergeschrieben und zum Drucke bereitet --- der nicht l"anger mehr aufgeschoben werden konnte --- als ich durch Herrn Apotheker Moser die Resultate einer, wie mir d"aucht, sehr entscheidenden chemischen Untersuchung, welche derselbe auf mein Ansuchen und nach meinen W"unschen in dieser Zwischenzeit vorzunehmen die G"ute hatte, mitgeteilt erhielt. Nach einer vorl"aufigen Analyse eines St"uckes von der Elbogner Eisen-Masse im Ganzen, wobei sich --- nach Wollastons Verfahren vorgegangen --- in hundert Teilen ein Gehalt an Nickel von 7,29 ergab, wurden drei Pl"attchen, welche aus einem "ahnlichen, aber gr"o"seren St"ucke dieser Masse geschnitten worden waren --- jedes beil"aufig von einem Zoll im Gevierte, und etwa $\mathfrak{\frac{3}{12}}$ Linie dick, am Gewichte zusammen [...] Gran [...] --- so lange in Salpeters"aure gebeitzt, bis die am leichtesten aufl"osliche Substanz, n"amlich die der Streifen, ganz ausge"atzt und aufgel"oset war. Es wurde nun das r"uckst"andige Gerippe oder Netz von Lamellen (den Einfassungsleisten) und den zum Teil ausgef"ullten Zwischenr"aumen (der Substanz der Figuren oder Zwischenfelder) sowohl als die Fl"ussigkeit, welche das Ausge"atzte aufgel"ost enthielt, beide einzeln f"ur sich, nach gleichem Verfahren untersucht, und es ergab sich bei ersterem ein Gehalt an Nickel von 9,83, in letzterer nur von 4,18 in hundert Teilen. Das arithmetische Mittel von diesen beiden Zahlen gibt nun beinahe ganz genau obige Summe des Gehaltes der Masse im Ganzen. Es ist demnach wohl nicht zu zweifeln, dass wo nicht aller, doch der bei weitem gr"o"sere Teil des Nickels in dem unaufl"oslicheren Teile der Masse, und zwar h"ochst wahrscheinlich in den Lamellen oder Einfassungsleisten enthalten sei; denn da bei diesem Versuche die Beitze "uber die Geb"uhr und so lange fortgesetzt wurde, bis selbst ein gro"ser, ja der gr"o"ste Teil der Figuren oder Zwischenfelder ganz durchge"atzt, und auch deren Substanz aufgel"ost worden war, so konnte wohl nur wenig von jenem Nickelgehalte des Gerippes von dieser letzteren herr"uhren, im Gegenteil ist es weit wahrscheinlicher, dass der in der Fl"ussigkeit aufgefundene Nickelgehalt von derselben, oder vielmehr von der Substanz der Einfassungs- zumal der Schraffierungsleisten herzuleiten sei, deren gleichzeitige Aufl"osung, wenn gleich in einem geringeren Grade, schlechterdings immer unvermeidlich ist, bei diesem Versuche aber, der langen Dauer des Prozesses wegen, bedeutend gewesen sein muss. Alle im Obigen ge"au"serten Vermutungen, hinsichtlich des Nickels und seines Anteiles an der Bildung und Erscheinung des Gef"uges, f"anden sich somit bew"ahrt, so wie wohl auch jene von der Substanz der Streifen, da sich au"ser Eisen und Nickel kein anderweitiger Stoff in der Aufl"osung ausmitteln lie"s. Von Kohle oder Grafit wollte sich dagegen bei diesen Haupt- so wie bei mehreren, zum Teil absichtlich darauf vorgenommenen Nebenversuchen durchaus keine Spur finden, und da sich auch von keinem anderweitigen Stoffe, mit Ausnahme des Schwefels, weder von Silicium noch selbst von Chrom (auch nicht von Kobalt), worauf Bedacht genommen wurde, eine Anzeige ergab; so bleibt die Natur der Substanz, welche die Figuren oder Zwischenfelder des Gef"uges bildet, zur Zeit noch zweifelhaft. Auf dem Boden des Gef"a"ses, in welchem die Beitze vorgenommen wurde, fanden sich --- nebst mehreren teils einzelnen, teils zu zwei und drei zusammenhangenden Lamellen, welche von den "Atzungspl"attchen wegen Mangel an Verbindung abgefallen waren und die mit zur Untersuchung des Gerippes verwendet wurden --- mehrere unf"ormliche St"ucke und K"orner (wovon doch eines eine vollkommene W"urfelform zeigte), zusammen von 4,40 Gran am Gewichte, von jener br"ocklig-k"ornigen Substanz, die wir bereits f"ur Schwefeleisen erkannten, als welches sie sich auch durch die Analyse bew"ahrte, und zwar in einem Verh"altnisse des Schwefels zum metallischen Eisen wie 0,30 : 4,10; ein Verh"altnis das demnach weit unter jenem steht, welches f"ur das terrestrische Schwefeleisen im \emph{minimum} als konstant angenommen wird. Auch jene dichte, h"artere Substanz, welche in Gestalt rundlicher Massen eingemengt vorkommt, und namentlich jene oben erw"ahnte aus dem Lénartoer Eisen, erwies sich als reines Schwefeleisen, in welchem jedoch offenbar das Verh"altnis des Schwefels zum Eisen --- das wegen der allzu geringen Menge, die davon zu Gebote stand, nicht genau ausgemittelt werden konnte --- ein ganz anderes ist. Es geht hieraus die Richtigkeit der schon fr"uher gemachten Bemerkung hervor, dass das Schwefeleisen in den Meteor-Massen von ganz eigener und von sehr mannigfaltiger Art sei, und dass man bei dessen Beurteilung nicht von dem f"ur das terrestrische Schwefeleisen festgesetzten Prinzip ausgehen, und vollends bei Bestimmung des quantitativen Verh"altnisses desselben nicht st"ochiometrisch vorgehen d"urfe.\\
\hspace*{6mm}Noch muss ich bei dieser Gelegenheit des Resultates eines Versuches erw"ahnen, welches einerseits die zu vermuten gewesene Zerst"orbarkeit dieses Schwefeleisens durch Hitze, andererseits die nicht minder \emph{a priori} wahrscheinlich gewesene, h"ochst schwere Schmelzbarkeit des Meteor-Eisens best"atiget, und somit meine hin und wieder ge"au"serten Zweifel gegen die herrschende Meinung, als w"aren die Meteor-Massen mehr oder weniger das Produkt eines Schmelz-Prozesses, und als k"amen die Metall-Massen wohl gar im geschmolzenen Zustande selbst zu Erde, zu bekr"aftigen scheint. Es wurde n"amlich ein drei Qu"antchen schweres St"uck gew"ohnliches, wei"ses Roheisen, und gleichzeitig ein 1 Linie dickes, 40 Gran wiegendes Pl"attchen vom Elbogner Eisen, welches von einem, ganz mit solchem Schwefeleisen ausgef"ullten Risse durchzogen war, zu schmelzen versucht. Jenes St"uck Roheisen schmolz bei ungef"ahr 130$^{\circ}$ Wedgd. vollkommen; das Pl"attchen Meteor-Eisen dagegen blieb ganz unver"andert, selbst an den scharfen Kanten und Ecken; aber das im Risse enthalten gewesene Schwefeleisen war ganz zerst"ort. Und diese Zerst"orung nahm selbst schon bei sehr m"a"siger Rotgl"uhhitze ihren Anfang. Wie k"onnten sich demnach die feinen Atome von Schwefeleisen, mit welchen die lockere, por"ose Steinmasse der Meteor-Steine, und vollends die Metall-Massen vom Mittelpunkte bis zur "au"sersten Oberfl"ache ganz durchs"aet sind, so unver"andert im metallischen und zum Teil selbst im kristallinischen Zustande erhalten haben, wenn erstere auch nur eine solche Hitze, welche zur Erzeugung der Rinde auf diesem Wege n"otig ist, und letztere eine solche --- durchdringende und anhaltende --- welche etwa notwendig sein d"urfte, ihre Masse --- oft von mehreren Zentnern --- in Fluss zu bringen, ausgestanden h"atten.}}

Auf der achten Tafel befindet sich endlich noch eine getreue Darstellung der abgeschliffenen Fl"ache an dem bereits erw"ahnten, in der Von der Null'schen Sammlung befindlichen, sch"onen Ladenst"ucke\footnote{\frakfamily{Es zeichnet sich dieses (28 Loth wiegende) St"uck durch ein besonders frisches Ansehen, reine und gute Erhaltung, durch Gr"o"se der Metall-Zacken, und vorz"uglich durch einen auffallend reichen Gehalt am olivinartigen Gemengteile aus, so dass dieser im Ganzen, dem Volum nach, wohl mehr als der Anteil am Metalle betragen m"ochte. Obgleich dieser Gemengteil hier --- was sonst nur selten und im Einzelnen der Fall ist --- gr"o"sten Teils in einem besonders reinen Zustande und hohen Grade von Ausbildung vorkommt, so zeigt er doch eine Menge von Abstufungen darin, und geht --- wie bereits in der Beschreibung desselben bei den Meteor-Steinen bemerkt worden ist, nur in umgekehrter Progression, in entgegen gesetztem Quantit"ats-Verh"altnisse und gew"ohnlich mit Abnahme an Volum der Massen --- aus den lichtesten, blass gelblich-wei"sen und gr"unlichen Farben, einerseits durch wachs- und honiggelbe Tinten ins Dunkelbr"aunliche, Zimtbraune und selbst ins Hyacinthrote, andererseits durch spargel- und pistaziengr"une ins Schmutzig- und Olivengr"une "uber, und in eben dem Ma"se nehmen die Grade der Durchscheinenheit, vom vollkommen Durchsichtigen bis zum Undurchsichtigen ab; der Glasglanz n"ahert sich immer mehr und mehr dem Fettglanze; der Bruch verl"auft sich aus dem flachmuschlichten, versteckt-bl"atterigen, in den ebenen, nicht selten mit deutlich bl"atterigen, oft selbst schaligen Absonderungen; die scharfkantigen Bruchst"ucke erscheinen stumpfer; und die H"arte sinkt vom Glasritzen bis beinahe zum Weichen herab. Nur h"ochst selten findet sich, selbst an diesem St"ucke, ein einzelnes Korn, wenn nur von einiger Gr"o"se, das, zumal im h"oheren Grade von Reinheit, aus jener Suite von Eigenschaften durchaus nur ein Glied zeigte; gew"ohnlich finden sich deren zwei auch drei, oft sehr entfernte, meistens aber ineinander verlaufende, an einem und demselben. Sehr h"aufig aber, obgleich an diesem St"ucke nur wenig und nur stellenweise, dagegen an den meisten St"ucken die ich kenne, namentlich dem gro"sen ($\mathfrak{5\frac{1}{2}}$ Pfund schweren), und noch mehr an dem --- angeblich aus Norwegen herstammenden ("uber 2 Pfund schweren) St"ucke der kaiserl. Sammlung; so auch an der, im Museum zu Gotha aufbewahrten, in Sachsen aufgefundenen "ahnlichen, "astig-zelligen Eisenmasse, und wie auch Graf Bournon von dem einen gr"o"seren, mehrere Pfund wiegenden St"ucke der Howard'schen Sammlung bemerkt --- bei weitem vorwaltend, findet sich dieser Gemengteil in "ahnlichen, doch meistens in den unvollkommensten oder doch minder vollkommenen Graden von Ausbildung in ganz unf"ormlichen, gr"o"seren und kleineren K"ornern und Bruchst"ucken, zum Teil in, dem Eisenspate in verschiedenen Abstufungen, ungemein "ahnlichen Partien von bl"atterigem Gef"uge zusammen gemengt und durch Eisenoxyd zu einer festen, kompakten Masse gleichsam zusammen gekittet, und bildet gewisser Ma"sen eine Grundmasse, welche von Zacken des Gediegeneisens durchwachsen ist, die hie und da als Spitzen "uber die Oberfl"ache hervorragen, aber nur h"ochst selten, und dann nur unvollkommene, kleine Zellen bilden. Hie und da findet sich an allen gr"o"seren St"ucken der Art, und namentlich auch an einem gro"sen (3 Pfund 19 Loth schweren) St"ucke im Besitze Sr. kaiserl. Hoheit des Erzherzogs Johann (im Johanneo zu Gr"atz) --- das in Hinsicht auf Frischheit im Ansehen, der guten Erhaltung, der Gr"o"se der Metall-Zacken und der Ausgeschiedenheit und Reinheit des Olivines im Einzelnen, jenem aus der Von der Null'schen Sammlung keineswegs nachsteht --- in gr"o"seren oder kleineren Partien, eine ganz erdige, trockne, zum Teil ganz zerreibliche, matte, graulichwei"se, der Grundmasse der Meteor-Steine vollkommen "ahnliche Substanz (wie auch Graf Bournon an jenem gr"o"seren Howard'schen St"ucke bemerkte), die vielleicht f"ur verwitterte Olivin-Masse angesehen werden k"onnte, mancher R"ucksichten wegen aber wohl richtiger f"ur urspr"unglich minder ausgebildeten Olivin, oder f"ur wirkliche, jener der Meteor-Steine ganz "ahnliche, Grundmasse zu halten sein m"ochte. Die abgeschliffene Fl"ache eines k"urzlich erhaltenen kleinen St"uckes der Art, welches diese erdige Substanz mit jener unf"ormlich gemengten, verschieden gef"arbten Olivin-Masse in bedeutender Menge und nur mit einzelnen wenigen und zarten Metall-Zacken durchwachsen zeigt, w"urde jedermann eher f"ur die eines Meteor-Steines, als eines St"uckes vom sibirischen Eisen erkennen.}} von der sibirischen Eisen-Masse, um das Gef"uge zu versinnlichen, welches eine "ahnliche "Atzung auf der polierten Oberfl"ache der durchschnittenen Metall-Zacken zum Vorschein gebracht hat. Es zeigt dasselbe zwar einige Abweichung von jenem obiger derber Eisen-Massen, im Wesentlichen aber doch dasselbe; n"amlich: eine wo nicht so regelm"a"sige und vielfach vereinzelnte, doch eine "ahnliche und ebenso scharfe Absonderung von wenigstens zwei heterogen scheinenden metallischen Substanzen. Die Oberfl"ache eines jeden solchen ge"atzten Zackens zeigt n"amlich, gleichsam als Kern desselben, ein Feld von matter, eisengrauer Farbe, welches von einem zwar nicht immer gleich breiten, aber scharf abgeschnittenen, und selbst durch eine mikroskopisch feine Linie getrennten Saume von spiegelicht gl"anzender, stark ins Silberwei"se fallender Farbe, eingefasst ist, der, indem er die Kante der Fl"ache oder den Rand des Umrisses vom Zacken selbst bildet, jenes Feld ringsum begrenzt. Die Form dieser Felder ist keineswegs gleichf"ormig und regelm"a"sig --- so wie jene der Zwischenfelder oder Figuren an den derben Eisen-Massen zu sein pflegt --- sondern vielmehr h"ochst verschieden und unbestimmt, indem sie ziemlich genau dem Umrisse des Zackens entspricht. Die Grenzlinie jener Felder folgt n"amlich allen Ecken, Kr"ummungen und Ausbuchten des Zackenrandes, nur mit der Abweichung, dass sie nicht immer gleich weit vom "au"sersten Rande sich entfernt, so z. B. in den Kr"ummungen einen verh"altnism"a"sig kleineren Bogen, in den Ecken meistens einen weit spitzeren und mehr gedehnten Winkel bildet; bisweilen macht aber doch der Umriss eines Feldes eine Kr"ummung oder Ecke, die jenem des Zackens nicht entspricht. Da nun der Saum den Rand des Zackens oder die Kante der Fl"ache desselben bildet und den Zwischenraum zwischen dieser und dem Felde ausf"ullt; so folgt, dass derselbe ungleich breit sein m"usse. Im Durchschnitte hat er eine Breite von $\mathfrak{\frac{1}{3}}$ oder $\mathfrak{\frac{1}{4}}$ Linie, oft jedoch kaum von $\mathfrak{\frac{1}{12}}$ Linie; dagegen nicht selten, zumal in den Kr"ummungen, von einer halben, und in den Ecken bisweilen selbst von einer ganzen Linie.

Wo die Fl"ache eines Zackens sehr schmal ist, wo n"amlich der Schnitt einen Seiten- oder Verbindungsast, oder die Schneide eines liegenden Zackens traf, da zeigt sich kein Feld oder Kern, sondern die S"aume von beiden R"andern sto"sen zusammen und sind blo"s durch eine zarte Linie getrennt; wie sich aber diese Fl"ache erweitert (was sehr oft bei Seiten- und Verbindungs"asten der Fall ist, indem sie sich gegen die Hauptst"amme hin verdicken), so trennen sich die beiden S"aume und der Kern erscheint als ein grauer Strich, der nach Ma"sgabe der zunehmenden Breite der Fl"ache immer breiter und endlich zu einem Felde wird, dessen Umriss wieder jenem der Fl"ache entspricht.

Da mir die "Atzung an diesem St"ucke zu schwach schien, so ersuchte ich Herrn v. Widmannst"atten, zum Behufe dieser Ausarbeitung an einem kleinen St"ucke von diesem Eisen in meinem Besitze einige abgeschnittene Zacken st"arker und bis auf jenen Grad zu "atzen, bis zu welchem jene Fl"achen obiger derber Eisen-Massen, um eines Abdruckes f"ahig zu sein, fr"uher von ihm selbst ge"atzt worden waren. Es zeigte sich nun, dass die Substanz des Kernes, der nun dunkler eisengrau erschien, ganz jener der Figuren oder Zwischenfelder, die des Saumes oder Au"senrandes der Zacken aber vollkommen jener der Streifen entspreche, indem sie nun nicht nur in eben dem Grade gegen erstere vertieft, sondern auch von ganz "ahnlicher, zinkwei"ser Farbe und mit gleicher und zwar ziemlich grobnarbiger Oberfl"ache erschien; und dass endlich jene zarten Linien, welche zuvor zwischen Kern und Saum bemerkt wurden, vollkommen mit den Einfassungsleisten "uberein kommen, indem sie nun ebenso erhaben und ganz von gleicher Beschaffenheit sich zeigten. Es finden sich demnach auch an dieser Metall-Masse jene drei verschiedenartigen Substanzen, welche bei den derben Meteor-Eisenmassen das beschriebene Gef"uge bilden, und zwar ebenso deutlich ausgesprochen und scharf begrenzt und ganz von derselben Beschaffenheit, nur mit dem Unterschiede, dass sie hier nicht mit jener kristallinischen Regelm"a"sigkeit ausgeschieden und gegenseitig gelagert sind.\footnote{\frakfamily{Ein mit einem zweiten "ahnlichen St"ucke von dieser Masse vorgenommener Versuch zum Blau-Anlaufen durch Erhitzung gab nicht nur ein vollkommen entsprechendes Resultat, sondern brachte auch eine Menge h"ochst zarter Linien --- Einfassungs- und Schraffierungsleisten --- zum Vorschein, die sich auf dem teils violetten teils dunkelblauen Grunde durch eine sch"on goldgelbe Farbe auszeichneten, und die, wahrscheinlich ihrer Zartheit wegen, durch die S"aure zerst"ort wurden, daher sich an dem ge"atzten St"ucke nur hie und da Spuren davon finden. Und genau dasselbe zeigte ein St"uckchen von jener s"achsischen Masse.}}

Die Oberfl"ache jener Zacken, welche nur fein poliert, aber nicht ge"atzt wurde (welches letztere am Von der Null'schen St"ucke --- wie auch aus der Darstellung zu ersehen ist --- nur auf der einen H"alfte der abgeschliffenen Fl"ache geschah), zeigt von dieser Trennung der Substanzen, in Kern und Saum, Feld und Einfassung, so wie "ahnlich behandelte Fl"achen an den derben Massen, noch keine Spur, sondern es hat dieselbe ein ganz gleichf"ormiges Ansehen, gleichen spiegelichten Glanz, und eine durchaus gleiche, sehr licht stahlgraue, stark ins Silberwei"se fallende Farbe.

Die zerstreut und mechanisch eingemengte, br"ocklig-k"ornige Substanz (das Schwefeleisen) zeigt sich aber hier wie dort und so wie bei jenen derben Massen, sehr deutlich und h"aufig, so dass sie hier wenigstens den sechsten Teil des gesamten Metall-Anteiles dieser Masse ausmachen d"urfte, von welchem sie sich durch ihr k"orniges oder doch rissiges Ansehen, durch eine Zinkwei"se, schwach ins R"otliche ziehende Farbe und durch einen schw"acheren Glanz auszeichnet. Sie findet sich teils in einzelnen kleinen und "au"serst kleinen K"ornern, teils in gr"o"seren br"ocklig zusammen geh"auften Partien, teils in dichteren, zart rissigen Massen, und zwar meistens am Rande der Zellen, welche durch die Metall-Zacken gebildet werden und den Olivin einschlie"sen, und die sie oft, entweder ganz oder stellenweise und abwechselnd mit dem Eisen und zwischen dieses gleichsam eingekeilt, gleich einer, obgleich ungleichf"ormigen Einfassung umgibt. Bisweilen bildet sie selbst ganze Nebenzacken, Seiten- oder Verbindungs"aste von den Hauptzacken oder St"ammen des Eisens; in jedem Falle ist sie aber immer durch eine zarte Furche von diesem geschieden.
\clearpage
\section{\frakfamily{Zehnte Tafel.}}
\subsection[\frakfamily{Plan der Gegend um Stannern in M"ahren.}]{\frakfamily{Plan der Gegend um Stannern in M"ahren,}}
\paragraph{}
in der sich, am 22. Mai 1808, jener merkw"urdige Steinfall ereignete,\footnote{\frakfamily{Umst"andliche Nachrichten davon --- die Resultate einer schon am sechsten Tage nach dem Ereignisse an Ort und Stelle gemeinschaftlich mit Herrn Direktor v. Widmannst"atten und unter Mitwirkung des k. k. Kreisamtes zu Iglau von mir vorgenommenen f"ormlichen und wissenschaftlichen Untersuchung --- finden sich in Gilberts Annalen der Physik, Bd. 29, Jahrg. 1808. Leider wurde die Fortsetzung derselben --- zu welcher die nicht genug anzur"uhmende Betriebsamkeit jener Landesbeh"orde, und insbesondere die, bei dieser Gelegenheit gar sehr in Anspruch genommene, zum Gl"uck durch das eigene Interesse den anziehenden Gegenstand lebhaft angeregte pers"onliche Aufmerksamkeit von Seite des Herrn Kreishauptmannes, Gubernialrates v. Hu"s, Materialien zu Gen"uge geliefert hatten (indem noch im Laufe desselben Jahres eine zwei Mahl wiederholte Durchsuchung des Fl"achenraumes nach den etwa verborgen liegenden und die sorgf"altigste Nachforschung "uber die bereits aufgefundenen Steine, eine ebenso oftmalige amtliche Einberufung und Vernehmung aller Finder und Beobachter von solchen, und endlich, gemeinschaftlich mit den angrenzenden Kreis"amtern, sehr umst"andliche Nachforschungen "uber die Ausdehnung und Grenzen einiger, die Begebenheit begleitender, merkw"urdiger Nebenerscheinungen vorgenommen und die Resultate davon bereits eingesendet worden waren) --- so wie die Bekanntmachung vieler dahin Bezug habender Untersuchungen, Arbeiten und Versuche (als Fortsetzung jener, welche bereits im 31. Bande desselben Werkes angefangen wurde), durch die ung"unstigen Zeitumst"ande --- den Ausbruch des Krieges von 1809 --- unterbrochen, und durch deren lange Fortdauer und Folgen zuletzt ganz unterbleiben gemacht.}} von welchem viele der ausgezeichnetsten Steine hier beschrieben und dargestellt worden sind.

Es erstreckt sich dieser Plan\footnote{\frakfamily{Es wurde dieser Plan, auf Anordnung der betreffenden hohen Landesstelle, nach den mitgeteilten Anforderungen durch den Landes-Ingenieur Herrn v. Berniere in Fr"uhjahre 1809 vor Bestellung der Gr"unde und nachdem alle oben erw"ahnten Untersuchungen und Nachforschungen bereits vollendet waren, unter Leitung des k. k. Kreisamtes und mit Zuziehung der Ortsobrigkeiten und aller jener Individuen, welche Steine aufgefunden oder im Niederfallen beobachtet hatten, an Ort und Stelle aufgenommen, und nach einem willk"urlichen aber bestimmten Ma"sstabe --- welcher zur gegenw"artigen Kopie genau auf die H"alfte reduziert wurde --- ausgefertiget.}} "uber eine Gegend von 4 Meilen in der L"ange (von dem Marktflecken Schelletau in S. bis zur Kreisstadt Iglau in N.), und auf 2 Meilen in der gr"o"sten Breite (von den Landst"adtchen Telsch und Trisch in W. bis zum Dorfe Pirnitz in O.), durch welche die m"ahrisch-b"ohmische Poststra"se, beinahe in gerader Richtung von S. nach N., durch den Marktflecken Stannern zieht, der ziemlich im Mittelpunkte dieses Fl"achenraumes (in einer Entfernung von 20 Meilen N. W. von Wien, 22 S. O. von Prag, und 13 N. W. von Br"unn) liegt.

Es sind in demselben nicht nur alle innerhalb dieses Umkreises befindlichen Ortschaften in ihrer geh"origen Lage aufgef"uhrt, H"ugel und T"aler, Geh"olze, Waldungen, "Acker und Wiesen, B"ache und Teiche, Wege und Fu"ssteige nach deren verh"altnism"a"siger Ausdehnung angedeutet, sondern auch die einzelnen Stellen, wo Steine aufgefunden oder im Niederfallen mit Verl"asslichkeit beobachtet,\footnote{\frakfamily{Solcher, blo"s im Niederfallen beobachteter und nicht wirklich aufgefundener Steine, sind in diese Tabelle eigentlich nur zwei aufgenommen worden; n"amlich die beiden unter Nr. 30 und 42 im Plane angedeuteten, welche in den einen bei Stannern gelegenen Teich fallen gesehen und geh"ort wurden.}} und wo sie von den Findern oder Beobachtern auf dem Platze selbst angegeben worden waren, mit m"oglichster Genauigkeit durch Punkte und fortlaufende Zahlen bezeichnet. Letztere beziehen sich auf eine dem Plane beigef"ugte Tabelle, welche die Nahmen der Deponenten nach ihren Wohnorten und in der Ordnung, nach welcher die amtliche Verhandlung ihrer Vernehmung gepflogen wurde, und das Gewicht der einzelnen Steine, welches teils nach wirklicher Abwiegung, teils nach einer beil"aufigen Absch"atzung bestimmt worden war, angibt.

Bezeichnet man die Grenzen des von Steinen wirklich befallenen Fl"achenraumes nach den "au"sersten Punkten oder den entferntesten Fallsstellen (wie dies auf der Karte durch eine punktierte Linie geschehen ist); so erh"alt man ein elliptisches Feld,\footnote{\frakfamily{In der Voraussetzung --- die "ubrigens alle Wahrscheinlichkeit f"ur sich hat --- dass die niederfallenden Steine Tr"ummer oder Bruchst"ucke einer Masse (des Meteors oder einer, bei solchen Ereignissen gew"ohnlich --- wie auch bei diesem --- beobachteten, so genannten Feuerkugel) sind, welche, in Folge wiederholter Zersprengung oder Zerplatzung letzterer w"ahrend ihres mehr oder weniger horizontalen oder vielmehr (weil sie selbst im Niederfallen ist) parabolischen Zuges durch unsere Atmosph"are, von ihr losgetrennt und nach allen denkbaren Richtungen hinweg geschleudert werden, ist die elliptische (und oft selbst --- wie gerade hier --- vollkommen und zugespitzt eif"ormige) Form des Fl"achenraumes, auf welchen dieselben niederfallen und dessen Grenzen ihre entferntesten Fallstellen bestimmen, sehr begreiflich, und die nat"urliche Folge teils der Vorw"artsbewegung der Masse selbst, w"ahrend jener sukzessiven Zersprengung, teils der zusammengesetzten Wurfs- und Fallsbewegung dieser von ihr weggeschleuderten einzelnen Bruchst"ucke, welche letztere, selbst bei ganz gleichgesetzter Wurfkraft, nach der Richtung, in welcher die Wegschleuderung geschieht, verschieden gedacht werden muss. Ein anderes ist es n"amlich, wenn diese Wurfsrichtung mit der eigent"umlichen Bewegung der Masse koinzidiert und solcher Gestalt die erhaltene Wurfkraft verst"arkt wird, als wenn sie nach seitw"arts, nach hinten oder vollends nach abw"arts Statt findet, in welchen F"allen der, der Wurfkraft entgegenwirkenden, Schwerkraft der einzelnen Steine weniger Widerstand geboten, oder diese wohl gar selbst verst"arkt wird. Im ersteren Falle m"ussen die Steine ungleich weiter vom Mittelpunkte der Explosion und weit langsamer, n"amlich nach Ma"sgabe der H"ohe, auf welcher diese vor sich ging, in einer mehr oder weniger schiefen oder parabolischen Richtung zur Erde kommen; in letzteren F"allen dagegen weit n"aher jenem Centro oder selbst in demselben, schneller und mehr oder weniger senkrecht niederfallen. Und in dieser so mannigfaltig modifizierten und komplizierten Fallsewegung, vollends aber in der weitern Zersprengung einzelner solcher Bruchst"ucke w"ahrend derselben (wof"ur nicht nur mehrere bei solchen Ereignissen gew"ohnlich beobachtete Nebenerscheinungen, sondern --- wie aus obigen Beschreibungen erhellet --- manche Beobachtungen an den Steinen selbst --- zumal r"ucksichtlich der verschiedenen Beschaffenheit der Oberfl"ache und der Rinde an einem und demselben St"ucke --- zu sprechen scheinen), wodurch sie wieder abge"andert und in eine neue, "ahnliche, auf eben die Art und noch mehr komplizierte aufgel"oset wird, m"ochte wohl die Erkl"arung jenes r"atselhaften Umstandes zu suchen sein, dass viele Steine, trotz ihres bedeutenden spezifischen Gewichtes und der betr"achtlichen H"ohe, in welcher deren Lostrennung von der Masse in den meisten F"allen vorzugehen scheint, so "au"serst sanft auffallen, dass sie kaum die Erde aufsch"urfen, eine Strecke weit fortrollen oder auf weichem, lockern Boden oberfl"achlich liegen bleiben.}} das ziemlich das Mittel jener Gegend einnimmt, den Marktflecken Stannern beinahe zum Mittelpunkte hat, bei 7000 Klafter in der L"ange und "uber 2600 in der gr"o"sten Breite misst und einen Fl"acheninhalt von mehr als zw"olf Millionen Quadrat-Klafter begreift.\footnote{\frakfamily{Da, wie aus dem Folgenden erhellen wird, sowohl der Zahl und Masse als dem Gewichte nach, doch wenigstens zwei Drittel der bei diesem Ereignisse niedergefallenen Steine mit hinl"anglicher Verl"asslichkeit ausgemittelt und die Fallsstellen derselben angegeben werden konnten, und da vorz"uglich auf die Grenzpunkte alle Aufmerksamkeit gerichtet worden war; so d"urfte die angegebene Lage, Richtung und Ausdehnung dieses Feldes als ziemlich richtig angenommen werden k"onnen.}}

Eine innerhalb dieses Feldes von der "au"sersten Fallsstelle in N. (Nr. 60) bis zur "au"sersten in S. (Nr. 1) gezogene Linie --- welche eine der Richtung des magnetischen Meridians parallel laufende (vorausgesetzt dass bei "Ubertragung der Stellung der Magnetnadel auf den Plan die damals Statt gehabte Abweichung geh"orig ber"ucksichtigt wurde) unter einem Winkel von etwa 7$^{\circ}$ durchschneiden m"ochte --- w"urde dasselbe der L"ange nach in zwei sehr ungleiche H"alften teilen,\footnote{\frakfamily{Obgleich, auch in Annahme obiger Voraussetzung, eine solche, die gr"o"sere Achse der Ellipse, und somit wohl auch beil"aufig den Zug des Meteors bezeichnende Linie, "uberhaupt nur h"ochst unsicher auf die wahre Bahn des Meteors schlie"sen lie"se, indem dies voraussetzen w"urde, dass die "au"sersten Punkte derselben durch Steine bestimmt worden w"aren, die in einer ihr vollkommen entsprechenden Richtung von der Masse abgeschleudert wurden --- was wohl bei einem "ahnlichen Vorfalle je erweislich sein m"ochte --- so w"are dies hier umso weniger zul"asslich, da ein Drittel der wahrscheinlich gefallenen Steine, wenigstens ihren Fallsstellen nach, nicht ausgemittelt werden konnten, wovon doch leicht einige --- welches hier, wie aus dem Folgenden erhellen wird, wirklich h"ochst wahrscheinlich der Fall war --- wenn gleich noch innerhalb des Feldes, doch so zu liegen gekommen sein konnten, dass sie die Richtung jener Linie ab"andern w"urden, wenn auch keiner davon, als der Bahn vollkommen entsprechend, den wahren Endpunkt derselben bezeichnet haben sollte.}} und eine Linie, welche man quer durch dasselbe, und zwar von der "au"sersten Fallsstelle in O. (Nr. 51) zur "au"sersten in W. (Nr. 63) z"oge, w"urde jene etwas "uber dem Mittel ihrer L"ange, dem Nordende etwas n"aher, durchkreuzen.

Bei einiger Aufmerksamkeit auf die Punkte, welche die Fallsstellen der Steine bezeichnen, bemerkt man bald, dass sie nicht durchaus und gleichf"ormig "uber das Feld verbreitet, sondern vielmehr deutlich in drei Gruppen verteilt sind, die durch betr"achtliche, ganz freie Zwischenr"aume voneinander getrennt werden und in deren Mittel sie zum Teil --- wenigstens auf der Karte --- ziemlich dicht erscheinen, indes sie au"serhalb desselben sehr weitschichtig und nach allen Richtungen um selbes herum zerstreut vorkommen.\footnote{\frakfamily{Diese Gruppen oder partienweisen Steinniederf"alle entspr"achen nun wirklich den angenommenen sukzessiven Zerplatzungen des Meteors umso mehr, als diese selbst durch ebenso viele Haupt-Detonationen w"ahrend des Ereignisses, die gleich starken Kanonensch"ussen oder gewaltigen Donnerschl"agen selbst auf sehr weite Entfernung --- nach gewissen Richtungen auf 10 bis 14 Meilen weit --- ziemlich allgemein vernommen worden waren, bezeichnet wurden; so wie wohl auch die gedr"angtere Lage der Fallsstellen unmittelbar und gleichsam im Centro dieser Gruppen, dagegen die weite Zerstreuung vieler anderer, offenbar dazu geh"origer, um dasselbe in sehr verschiedenen Abst"anden, obige Schlussfolgerung in Betreff der so mannigfaltigen und komplizierten Wurfs- und Fallsbewegung der Bruchst"ucke, und vollends das eigene, nach einstimmiger Aussage, einem Peloton- oder kleinem Gewehrfeuer "ahnliche, fortgesetzte Get"ose, die Annahme einer wiederholten Zersprengung vieler einzelner Steine w"ahrend ihres Falles zu bekr"aftigen scheinen.}}

Die eine Gruppe findet sich am n"ordlichen Ende des Feldes, bei dem Orte Neustift und zwischen diesem und dem Orte Roschitz, und begreift vier Fallsstellen, die sich ziemlich nahe sind, so dass die einzelnen Steine kaum 300 bis 400 Klafter weit voneinander entfernt zu liegen kamen, und einen Fl"achenraum von etwa 200,000 Quadrat-Klafter einschlie"sen. Die zweite Gruppe zeigt sich ziemlich genau im Mittel des elliptischen Feldes und ungleich betr"achtlicher an Zahl der Fallsstellen sowohl als an Ausdehnung, in und um Stannern, bei Sorez, Falkenau und bis "uber Mitteldorf und Otten in W. und gegen Hasliz in O. hinaus. Sie begreift 36 Fallsstellen, wovon 16 gewisser Ma"sen die Hauptgruppe oder das Mittel derselben bilden, die sich zum Teil besonders nahe sind, n"amlich auf 100, 200 bis 300 Klafter, und zusammen einen Fl"achenraum von kaum 600,000 Quadrat-Klafter einschlie"sen; die "ubrigen liegen mehr zerstreut und in weit gr"o"seren Entfernungen, so dass manche 400 bis 600 Klafter voneinander und die "au"sersten in O. und W. (Nr. 51 und 63) von einem als wahrscheinlich anzunehmenden Mittelpunkte der Gruppe, "uber 1000 und bei 1600 Klafter abstehen, und so dass alle 36 Fallsstellen einen Fl"achenraum von nahe an 5 bis 6 Millionen Quadrat-Klafter einnehmen.\footnote{\frakfamily{Diese betr"achtlichen Fallsentfernungen setzen eine, den einzelnen Bruchst"ucken mitgeteilte, horizontale Bewegung und eine Wurfkraft voraus, die sich, bei der H"ohe in der jene Explosionen, welche selbe bewirken sollen, vorzugehen scheinen, einerseits mit dem spezifischen Gewichte dieser Massen und der daraus resultierenden und jener entgegen wirkenden Schwerkraft, andererseits mit der leichten Zersprengbarkeit, dem lockern Koh"asions-Zustande, in welchem sie wenigstens zur Erde kommen, nicht wohl zusammen reimen lassen. Und noch mehr als diese betr"achtlichen Fallsentfernungen der Steine sprechen f"ur die Gewalt, welche die Explosionen der Masse bewirkt, die ausnehmende St"arke und die weite Ausdehnung des Get"oses, das dieselben bezeichnet, und welches bei allen "ahnlichen Ereignissen in einem ziemlich gleichen Grade und von auffallend gleichf"ormiger Art beobachtet wurde. So verbreitete sich bei diesem Ereignisse --- nach den Resultaten einer im Laufe desselben Jahres noch von dem Kreisamte zu Iglau einvernehmlich mit den angrenzenden Kreis"amtern von Znaim in M"ahren, Czaslau und Tabor in B"ohmen, und von Korneuburg und Krems in "Osterreich (deren sowie aller untergeordneten Beh"orden t"atige Mitwirkung "uber hundert, mit Protokollen und andern Dokumenten belegte, Amtsberichte bew"ahrten) in dieser Beziehung gepflogene Untersuchungsverhandlung --- das Get"ose --- wenigstens jenes der Haupt-Detonationen --- von Stannern aus --- den Ort selbst als Mittelpunkt angenommen --- in N. gegen Czaslau auf 4, in O. gegen Br"unn auf 8, in S. gegen Stockerau und in W. gegen Tabor selbst bis auf 14 Meilen weit, und zwar mit solcher St"arke noch, dass mit demselben, wenigstens nach jenen weitern Richtungen hin, auf eine Entfernung von 8 bis 12 Meilen von jenem Mittelpunkte, eine Ersch"utterung der Geb"aude und ein Klirren der Fenster bemerkt wurde. (Merkw"urdig ist, dass die Grenzen des Fl"achenraumes, "uber welchen sich dieses Get"ose ausgebreitet hatte, die ich nach den, mit den Berichten erhaltenen, sehr genauen Angaben und nach den bezeichneten Ortschaften auf eine Karte "ubertrug, eine "ahnliche und jener des von den Steinen befallenen Fl"achenraumes entsprechende Ellipse gaben, deren gr"o"sere Achse ebenfalls wie die von jener von N. N. W. gegen S. S. O. und derselben sehr parallel lief, und dass damit auch ganz auffallend die Richtung und Ausdehnung des in Begleitung des Ph"anomenes beobachteten und unbezweifelbar mit demselben zusammenhangenden Nebels "uberein kam, der nur auf engere Grenzen als das Get"ose beschr"ankt war, indem sich derselbe in S. auf 8, in N. kaum auf 4, in W. nur wenig weiter, in O. nicht einmal so weit erstreckte. Dass der Nebel sowohl als vorz"uglich das Get"ose sich bedeutend weiter gegen S. und W. als gegen N. und O. ausgedehnt haben, mag wohl Nebenumst"anden zuzuschreiben sein, die leicht darauf Einfluss gehabt haben konnten, z. B. dem Luftstrome --- obgleich w"ahrend der Dauer des Ereignisses, so wie selbst den ganzen Tag "uber, wenigstens in der niedereren Region, die Atmosph"are vollkommen ruhig war --- zum Teil auch dem Niveau des Terrains, das sich gegen O. und vorz"uglich gegen N. betr"achtlich erhebt --- obgleich diese Erhebung bei der H"ohe, in welcher die Explosionen Statt gefunden zu haben scheinen, geradezu keinen gro"sen Einfluss auf die Verbreitung des Schalles gehabt haben kann. ---)\\
\hspace*{6mm}Wenn man nun erw"aget, dass der Umfang der bei diesem Ereignisse niedergefallenen Masse im Ganzen (als Feuerkugel) --- deren Form als sph"aroidisch sich gedacht und die physische Beschaffenheit ihrer Substanz in dem Zustande angenommen, in welchem die einzelnen Steine als Bruchst"ucke derselben zur Erde kommen --- nach dem wahrscheinlichen absoluten Gewichte von 150 Pfund im Vergleich mit dem spezifischen von 3 : 1 des Wassers, kaum mehr als einen Schuh im Durchmesser (in Dampfgestalt --- bei gew"ohnlicher Kompression --- etwa 6000 Kubik-Schuh k"orperlichen Inhalt) gehabt haben konnte (und bei den meisten "ahnlichen Ereignissen muss dieser noch ungleich kleiner gewesen sein, indem die teils im Ganzen teils in nur wenigen einzelnen St"ucken herabgefallene Masse oft nur wenige Pfund betrug); so m"ochte man sich wohl bestimmt finden von der Idee, diese Gewalt als eine blo"s mechanische zu betrachten, abzugehen und dieselbe vielmehr als die Wirkung eines uns ganz fremden, gro"sen chemischen Prozesses anzusehen, dessen Resultat vielleicht die Bildung der n"achsten Bestandteile, in welchen sich uns die meteorischen Massen, wenn sie einmal zur Erde gekommen sind, zu erkennen geben, aus den uns zur Zeit noch unbekannten Urstoffen (Chladnis Ur-Materie) sein d"urfte, und wobei die Explosion und Zertr"ummerung der Masse nur Nebenwirkung w"are.}} Die "au"serste, h"ochst wahrscheinlich zu dieser Gruppe geh"orige Fallsstelle gegen N. (Nr. 55) steht von der "au"sersten der vorigen Gruppe gegen S. (Nr. 61) "uber 1000 Klafter ab, so dass im elliptischen Felde zwischen diesen beiden Gruppen ein steinfreier Zwischenraum von wenigstens 2 Millionen Quadrat-Klafter auff"allt. Die dritte Gruppe endlich findet sich gegen das s"udliche Ende des Feldes, zwischen und "uber den Orten Hungerleiden, Lang- und Klein-Pirnitz, und zeigt sich ebenfalls sehr betr"achtlich an Zahl der Punkte und an Ausdehnung. Erstere bel"auft sich auf 26, wovon wieder mehrere, zumal 10, sich ziemlich nahe, nur auf 100, 200 Klafter Entfernung voneinander liegen, so dass sie einen Fl"achenraum von kaum 2 bis 300,000 Quadrat-Klafter einschlie"sen. Die "ubrigen liegen wieder mehr zerstreut und entfernter voneinander, so dass alle zusammen einen Fl"achenraum von etwa 2 bis 3 Millionen Quadrat-Klafter einnehmen m"ochten. Diese Gruppe hat sich "ubrigens mehr in die L"ange als in die Breite ausgedehnt,\footnote{\frakfamily{Wahrscheinlich weil der Stein Nr. 1 ziemlich horizontal und der urspr"unglichen Bewegung des Meteors entsprechend, folglich mit verst"arkter Wurfkraft, vorw"arts geschleudert wurde, daher die schiefste Richtung oder l"angste Parabel im Falle beschrieb und folglich am weitesten flog, wie dessen Fallsstelle denn auch in gerader Linie bei 1600 Klafter vom annehmbaren Centro dieser Gruppe entfernt liegt.}} denn der entfernteste Fallspunkt gegen S. (Nr. 1) --- der "uberhaupt auch sehr weit, bei 1600 Klafter, vom wahrscheinlichen Mittelpunkte derselben sich befindet --- ist vom "au"sersten dieser Gruppe gegen N. (Nr. 11. b.) auf 2200 Klafter entfernt, indes der "au"serste gegen W. (Nr. 25) vom "au"sersten gegen O. (Nr. 18) nur 1300 Klafter absteht. Jener "au"serste dieser Gruppe gegen N. (Nr. 11. b.) ist von dem "au"sersten der vorigen, mittleren Gruppe gegen S. (Nr. 62. b.) ebenfalls auf beinahe 1000 Klafter entfernt, so dass demnach auch hier, wie zwischen letzterer und der ersten Gruppe am Nordende, ein "ahnlicher steinfreier Zwischenraum von beil"aufig 2 Millionen Quadrat-Klafter bemerkbar wird.\footnote{\frakfamily{Die Gleichheit dieser steinfreien R"aume zwischen den Gruppen, so wie die der Abst"ande dieser voneinander, sowohl ihrem wahrscheinlichen Mittelpunkte als ihren Endfallsstellen nach, ist meines Erachtens sehr merkw"urdig, indem sie auf gleiche Intervalle zwischen den Explosionen schlie"sen l"asst, welche "ubrigens auch die Aussagen "uber das Vernehmen der drei Haupt-Detonationen der Dauer der Zwischenzeit nach, best"atigten.}}

Jene durch die "au"sersten Fallsstellen in N. und S. --- nach Angabe des Planes --- der L"ange nach durch das Feld gezogene Linie durchschneidet eben so wenig den als wahrscheinlich anzunehmenden Mittelpunkt dieser Gruppen als jenen des Feldes im Ganzen; um diesen Forderungen zu entsprechen, m"usste eine andere angenommen werden, welche in N. um einige Grade mehr westlich fiele, welches den, in mehrfacher Beziehung auch wirklich sehr wahrscheinlichen Umstand voraussetzen w"urde, dass am Nordende des Feldes noch einige Steine, gegen Willenz und Porenz zu, gefallen w"aren, die nicht zur Notiz kamen oder deren Fallsstellen wenigstens nicht ausgemittelt werden konnten.\footnote{\frakfamily{Obgleich nach alle dem, was bereits "uber die Explosionen des Meteors und "uber die so mannigfach modifizierte und komplizierte Wurfs- und Fallsbewegung, mit welcher die von demselben weggeschleuderten Bruchst"ucke zur Erde kommen, als wahrscheinlich vorgebrach worden ist, es wohl unm"oglich sein d"urfte, den wahren Mittelpunkt dieser Gruppen von Fallsstellen, und noch mehr jenen, diesen in senkrechter H"ohe zentrierenden der Explosionen selbst, mit voller Zuversicht zu bestimmen; so kann man doch mit aller Wahrscheinlichkeit ersteren dort annehmen, wo die meisten Fallsstellen und diese im Durchschnitte am dichtesten beisammen liegen, letzteren aber etwas hinter diesem Punkte, da wohl vorausgesetzt werden kann, dass die eigent"umliche Bewegung der Masse auf alle von ihr getrennten Bruchst"ucke mehr oder weniger Einfluss gehabt habe. Der Mittelpunkt der mittleren Gruppe m"ochte demnach etwa 600 Klafter O. N. O. von der Kirche von Stannern und etwa 1000 Klafter N. von Sorez und 600 S. S. W. von Falkenau zu setzen sein; so dass der "au"serste Stein dieser Gruppe in W. 15 bis 1600, der "au"serste in O. 1000 bis 1100, der s"udlichste 12 bis 1600, der n"ordlichste etwa 800 Klafter davon entfernt zu liegen k"ame; Entfernungen die den denkbaren Wurfs- und Fallsbewegungen der dieser Gruppe angeh"origen Bruchst"ucke ganz gut entsprechen m"ochten. Der Mittelpunkt der dritten Gruppe k"onnte sich um die Fallsstelle Nr. 11 a. gedacht werden, etwa 400 Klafter N. von Lang-Pirnitz, 2800 Klafter vom Mittelpunkte der vorigen; so dass der "au"serste Stein in W. etwa 800, der in O. 500, der s"udlichste 1600, der n"ordlichste 600 Klafter davon zu liegen k"ame. Zieht man nun eine gerade Linie durch diese beiden Punkte und verl"angert sie bis ans Nordende des elliptischen Feldes; so w"urde ihr Endpunkt hier gegen Willenz zu, etwa 600 Klafter "ostlich von diesem Orte, und etwa 200 Klafter W. N. W. von der Fallsstelle Nr. 60 fallen. Diese Linie --- welche etwa um 3 oder 4 Grade von der hier angegebenen Richtung des magnetischen Meridianes abwiche --- w"urde nun nicht nur die beiden in Hinsicht der gefallenen Steine am besten beobachteten und den Fallsstellen nach am genauesten ausgemittelten Gruppen in ihrem wahrscheinlichen Mittelpunkte durchschneiden, sondern auch die Zahl der Fallsstellen und selbst den befallenen Fl"achenraum --- obgleich letzteres von weniger Belang ist -- in zwei ziemlich gleiche H"alften teilen, und somit mit vieler Wahrscheinlichkeit als die wahre Bahn des Meteors bezeichnend angesehen werden k"onnen. Es w"urde dieselbe nur voraussetzen, dass auf jenen durch sie bezeichneten Punkt, oder vielmehr noch mehr westlich, gegen Willenz oder Porenz zu, einige Steine mit der ersten Explosion innerhalb der Grenzen des elliptischen Feldes gefallen seien. Und dies war h"ochst wahrscheinlich wirklich der Fall; denn nicht nur, dass von dieser Gruppe nur vier Steine ausgemittelt werden konnten, deren doch, im Verh"altnis zur Zahl und Masse der "ubrigen, nicht gar so wenige gefallen sein k"onnen, und dass deren Fallsstellen so nahe beisammen und alle nach einer Seite hin liegen, so dass kaum ein Mittelpunkt oder eine Durchschnittslinie, am wenigsten eine solche denkbar w"are, welche jener der "ubrigen Gruppen nur einiger Ma"sen entspr"ache; so ist es auch sehr m"oglich, dass in dieser Gegend mehrere Steine unbeobachtet niederfielen oder nicht aufgefunden wurden, da diese Gegend weit weniger bev"olkert ist und w"ahrend der Momente des Ereignisses beinahe ganz von den Einwohnern verlassen war, die sich eben auf dem weiten Wege zur Kirche nach Stannern befanden; auch hat sich im Verfolg der Nachforschungen ergeben, dass hier nach der Hand wirklich noch einige Steine aufgefunden wurden, von welchen aber keine n"ahere Notiz erhalten werden k"onnte, so wie auch gleich Anfangs am Tage der Begebenheit selbst, mehrere Steine und Bruchst"ucke von Fuhrleuten, die gerade dieses Weges und namentlich von Willenz kamen und weiter zogen, von daher nach Stannern gebracht, daselbst gezeigt und weiter mitgenommen wurden, daher auch diese einer sp"ateren Notiznehmung entgingen. Sowohl in diesem pr"asumtiven als in dem bestehenden Falle --- wie ihn inzwischen der Plan ausweiset --- w"urde der Mittelpunkt dieser Gruppe von jenem der zweiten oder mittleren in einem "ahnlichen Abstande, d. i. von beil"aufig 2600 bis 2800 Klafter, wie der von dieser zu jenem der dritten Gruppe zu liegen kommen, was auch die oben bemerkte Gleichheit der steinfreien R"aume zwischen denselben und des Abstandes der Gruppen \emph{en masse}, so wie die Gleichheit des Zeit-Momentes im Vernehmen der, die Explosionen bezeichnenden Detonationen vermuten lie"sen. Der von dem Meteore w"ahrend diesen Explosionen, die jene Steingruppen als Produkt gaben, auf seinem Zuge zur"uckgelegte Raum, w"urde demnach eine Strecke von 5 bis 6000 Klafter in gerader Linie betreffen, und danach einstimmigen Aussagen so vieler Augen- und Ohrenzeugen des Ph"anomenes, das begleitende Get"ose im Ganzen 6 bis 8 Minuten dauerte, so bestimmt diese Dauer beil"aufig den Zeitraum, welchen das Meteor brauchte, jene Strecke zur"uckzulegen; die Schnelligkeit der Bewegung scheint demnach nicht ausnehmend gro"s gewesen zu sein, man mag die H"ohe auch als noch so betr"achtlich und die Richtung des Zuges auch noch so schief oder parabolisch annehmen, auch wohl voraussetzen, dass die Zeitsch"atzung, wie kaum zu bezweifeln ist, um vieles zu hoch ausgefallen sein m"ochte.}}

Aus einer "Ubersicht der dem Plane beigef"ugten Tabelle ergibt sich, dass die vier am n"ordlichen Ende des elliptischen Feldes niedergefallenen und die erste Gruppe bildenden Steine alle ansehnlich gro"s und gewichtig waren (der gr"o"ste von 13 Pfund, der kleinste --- der wohl nur ein Bruchst"uck eines sp"ater im Falle zersprungenen Steines gewesen sein d"urfte --- von 28 Loth), und zusammen bei 27 Pfund wogen. Die 36 Steine der mittleren oder zweiten Gruppe betragen dagegen am Gewichte zusammen nur etwas "uber 55 Pfund, und es waren meistens kleinere oder doch nur mittelgro"se Steine, im Durchschnitte von 1 bis 3 Pfund (nur 8 von 2 Pfund und dar"uber, 3 von 3 und 2 von 4 Pfund, der gr"o"ste von $\mathfrak{4\frac{1}{2}}$ Pfund; dagegen aber auch keiner unter 8 Loth, nur 8 unter 16 Loth, 13 unter einem Pfund). Jene, die dritte, s"udliche Gruppe bildenden 26 Steine endlich geben ein Gesamtgewicht von kaum mehr als 11 Pfund und waren fast durchgehends kleine und sehr kleine Steine, im Durchschnitte von 7 bis 12 Loth (12 davon unter 8, 7 unter 16 Loth, nur einer von $\mathfrak{1\frac{3}{4}}$, der gr"o"ste etwas "uber 2 Pfund) der kleinste hier aufgezeichnete wog $\mathfrak{3\frac{1}{2}}$ Loth, und ohne Zweifel sind hier noch weit kleinere gefallen, die aber entweder nicht aufgefunden oder der Aufzeichnung nicht wert befunden wurden, wie dies die beiden auf der f"unften Tafel, Fig. 3 und 4 abgebildeten, der Anzeige nach aus dieser Gegend herstammenden und folglich zu dieser Gruppe geh"origen Steine bew"ahren, wovon der eine kaum $\mathfrak{2\frac{1}{2}}$ Qu"antchen, der andere kaum 56 Gran wiegt.\footnote{\frakfamily{Den sprechendsten Beleg f"ur die Richtigkeit dieses merkw"urdigen Umstandes, der sich uns bei der Untersuchung des Ereignisses an Ort und Stelle sogleich bemerkbar machte (so wie er bereits von dem franz"osischen Physiker Biot --- bei Gelegenheit der wissenschaftlichen Untersuchung des Steinfalles bei L'Aigle --- bemerkt worden war) --- dass n"amlich an dem einen Ende der gro"sen Achse der Ellipse, und zwar --- nach dem bei diesen beiden Ereignissen mit aller Verl"asslichkeit beobachteten Zuge des Meteors und der ganzen Erscheinung --- mit der ersten Explosion, meistens gro"se und darunter die gr"o"sten Steine, am entgegen gesetzten dagegen, mit der letzten Explosion, meistens kleine und die kleinsten, im Mittel und als Produkt der zweiten Explosion, aber meistens mittelgro"se Steine fielen (woraus allenfalls zu schlie"sen w"are, dass die Masse Anfangs z"aher und schwerer zersprengbar gewesen sein m"ochte) --- und zugleich f"ur die Genauigkeit des Planes und der Tabelle (deren mittel- und unmittelbare Zustandebringer denn doch keine entfernte Ahndung dieses Umstandes hegen konnten), geben die wirklich vorhandenen, gr"o"sten Teils lange vor der Zustandebringung jener und unmittelbar aus erster Hand erhaltenen, oben beschriebenen unverbrochenen Steine von diesem Ereignisse, deren angegebene Fallsstellen genau den Erwartungen (so wie auch vollkommen den "uber manche pers"onlich eingeholten Privat-Notizen) entsprachen, zu welchen das respektive Volumen und Gewicht derselben in diesen Beziehungen berechtigten. So ist der auf der vierten Tafel abgebildete gr"o"ste Stein zun"achst der "au"sersten Fallsstelle am n"ordlichen Ende des Fl"achenraumes, als Glied der ersten Gruppe unter Nr. 59 angezeigt, so finden sich die Fig. 5 auf der f"unften und Fig. 1 bis 4 auf der sechsten Tafel abgebildeten gr"o"seren Steine s"amtlich im Mittel des elliptischen Feldes, als Produkt der zweiten Explosion unter Nr. 45, 26, 35, 43 und 40; dagegen die kleineren Fig. 1. 2. der f"unften Tafel, im s"udlichen Ende und unter der letzten Gruppe des Feldes unter Nr. 19 und 16 angedeutet, und die beiden kleinsten, Fig. 3. 4. derselben Tafel, wenigstens als in dieser Gegend, n"amlich in der N"ahe von Lang-Pirnitz aufgefunden, angegeben.}}

Die Tabelle weiset "ubrigens 63 Finder und 66 Steine, und von letzteren ein Gesamtgewicht von 93 Pfund aus. Ich hatte bereits in der ersten von diesem Ereignisse gegebenen Nachricht, nach den Resultaten der an Ort und Stelle gepflogenen Untersuchung und nach Erw"agung aller Umst"ande und Verh"altnisse, die Total-Zahl der gefallenen Steine auf 100 St"uck und das Gesamtgewicht derselben auf 150 Pfund gesch"atzt, obgleich ich damals nur von 40 aufgefundenen verl"assliche Notiz, und trotz des angelegentlichsten Einsammelns der bereits aufgefunden gewesenen und der eifrigsten Betreibung des Aufsuchens der liegen gebliebenen Steine durch zwei Tage, nur 61 St"uck (wovon die meisten nur Fragmente waren), am Gewichte zusammen bei 27 Pfund, aufbringen konnte,\footnote{\frakfamily{Die Umst"ande und Verh"altnisse, welche damals --- als noch von Seite des, weder durch besondere Neugierde, noch weniger durch Eigennutz gereitzten Landvolkes jener Gegend keine absichtliche Verheimlichung oder Zur"uckhaltung der aufgefundenen Steine zu besorgen war, indem man vielmehr das F"ormliche der Verhandlung, das Angelegentliche des Aufsuchens und Eintreibens solcher an sich ganz wertloser (der vorherrschenden Meinung nach f"ur angebrannte Mauerst"ucke eines in die Luft gesprengten Pulver-Magazines angesehene) Steine und die Verg"utung f"ur die dabei bezeigte Willf"ahrigkeit und Bem"uhung h"ochst sonderlich fand --- bei jener Absch"atzung Ber"ucksichtigung heischten, waren: die physische und agronomische Beschaffenheit des Fl"achenraumes --- der zum Teil mit Geh"olze und Waldungen bedeckt, gr"o"sten Teils aber bebauet und in dieser Jahreszeit von der bereits [...] Saat bedeckt, das Auffinden der Steine schwierig machte --- ferner der Umstand, dass gerade w"ahrend des Verlaufes des Begebenheit das gesamte Landvolk aus den umliegenden und gerade aus den, den befallenen Fl"achenraum begrenzenden Ortschaften auf den Gange zur Pfarrkirche nach Stannern begriffen, gr"o"sten Teils schon in dieser Gegend versammelt, und von ersteren, zumal den entlegeneren in N. und S. schon ziemlich entfernt und mit dem R"ucken dahin gekehrt war --- so dass folglich in jenem beschr"ankteren Bezirke die meisten der gefallenen Steine im Nieder- oder Auffallen (insofern dies an sich --- wie wirklich mehrenteils --- der Fall war) beobachtet und daher gleich aufgehoben oder auch sp"ater noch bald aufgefunden, in ersteren Gegenden dagegen nur wenige, von den zur"uck gebliebenen und zuf"allig gerade im Freien sich befindenden Bewohnern, im Falle bemerkt und daher mehrenteils nach der Hand nur zuf"allig oder durch absichtliches Aufsuchen, gefunden werden konnten. (Diess bew"ahrte sich auch in der Folge durch die sp"ater aufgefundenen und eingelieferten Steine, die alle aus diesen Gegenden herstammten, so wie durch manche andere, die mir in dieser Zwischenzeit mittel- oder unmittelbar zu Gesicht und Kenntnis kamen, welche --- in Folge der allm"ahlichen Aufkl"arung und des gereitzten Eigennutzes sp"aterer Finder --- auf Nebenwegen in fremden Besitz geraten waren; und ohne Zweifel kommen, wo nicht alle, doch die meisten Steine jenes Drittels der wahrscheinlichen Total-Zahl, die, wenigstens den Findern und den Fallstellen nach, nicht mit Verl"asslichkeit ausgemittelt werden konnten, dahin zu versetzen.) Ein dritter zu ber"ucksichtigender Umstand war endlich jener, dass eben w"ahrend und unmittelbar nach der Begebenheit --- die durch das Wunderbane und L"armende die ganze Gegend auf ziemlich weite Entfernung, wenigstens f"ur den ersten Tag, in Angst und Staunen versetzte --- mehrere Reisende auf der Poststra"se ab und zu, und mehrere Fuhrleute des Weges von Willenz her, durch diese Gegend kamen, welche teils selbst aufgefundene, teils mitgeteilt erhaltene Steine, die ihnen denn doch mehr oder weniger sonderbar (und vielleicht nicht ganz so wie angebrannte Mauerst"ucke) vorgekommen sein mochten, der Merkw"urdigkeit wegen und als Beleg der von ihnen ganz oder zum Teil beobachteten Erscheinung, mit sich fortnahmen. Einem dieser Reisenden --- dessen Aufmerksamkeit gl"ucklicherweise lebhafter und von einer richtigeren Ansicht aus angeregt wurde, obgleich derselbe weder Augen- noch Ohrenzeuge des Ph"anomenes, sondern nur Zeuge des ersten Eindruckes war, den dasselbe einige Stunden fr"uher unter den Bewohnern einer so betr"achtlichen Strecke allgemein verbreitet hatte, und ein Bruchst"uck eines der gefallenen Steine mitgeteilt erhielt --- hatte man auch die erste, verl"assliche und hinl"anglich fr"uhzeitige Nachricht von diesem Ereignisse zu verdanken, ohne welche die Notiz davon --- wie von so vielen "ahnlichen --- h"ochst wahrscheinlich auf jene Gegend und den schnell verl"oschenden Eindruck beschr"ankt, f"ur die Geschichte und f"ur die Wissenschaft unbenutzt geblieben w"are.}} und ich fand auch sp"aterhin, nach den nachtr"aglich erhaltenen Notizen, keine Ursache davon abzugehen. Durch dieses Verzeichnis, das beinahe ein Jahr sp"ater zu Stande gebracht wurde --- nachdem zu zwei verschiedenen und den g"unstigsten Perioden (zur Schnitt- und Herbstbestellungszeit der Gr"unde) das sorgf"altigste Aufsuchen der etwa verborgen liegenden Steine in der ganzen Gegend veranlasst und alle Individuen, welche seit dem Momente des Ereignisses Steine aufgefunden hatten, oder auch nur um die Auffindung von welchen wussten, zu wiederholten Mahlen amtlich vernommen worden waren --- erhielt ich vollends in jeder Beziehung die vollkommenste Best"atigung, und finde auch jetzt, nach einer Zwischenzeit von 12 Jahren, w"ahrend welcher ich nicht vers"aumte, mittel- und unmittelbar meine Nachforschungen "uber die Besitzverbreitung der Steine von diesem Ereignisse fortzusetzen, keinen Grund, jene Annahme auch nur im geringsten abzu"andern.\footnote{\frakfamily{Unter mehr denn 40 "ahnlichen Ereignissen, die sich in der neuesten Zeit, den letzten 30 Jahren, zutrugen und wovon wir n"ahere und verl"assliche Nachrichten haben, war dieses --- sowohl der Masse im Ganzen als der Zahl der gefallenen Steine nach --- eines der bedeutendsten und ergiebigsten; nur jenes, das 1790 in der Gegend von Barbotan, und vollends jenes, welches 1803 bei L'Aigle in Frankreich sich ergab, haben dasselbe in beiden R"ucksichten "ubertroffen, und jene die sich, 1794 zu Siena im Toskanischen, 1807 zu Weston in Nord-Amerika, 1812 zu Toulouse in Frankreich, 1813 zu Limerick in Irland und 1814 zu Agen in Frankreich zutrugen, m"ochten demselben gleich gestellt werden d"urfen.}}

Wenn man nun bedenkt, dass jene 100 Steine einen Fl"achenraum von mehr denn zw"olf --- oder wenn man streng sein und die steinfreien R"aume zwischen den Gruppen in Abrechnung bringen will --- doch wenigstens von acht Millionen Quadrat-Klafter trafen, und dass, selbst da wo sie am dichtesten fielen, die einzelnen doch 100 bis 300 Klafter voneinander zu liegen kamen\footnote{\frakfamily{Diese weitschichtige Zerstreuung der Steine "uber den befallenen Fl"achenraum bei Ereignissen der Art, wo deren mehrere und wo sie selbst in gro"ser Menge fielen, wie z. B. bei jenen von L'Aigle, wo zwischen 2- und 3000 "uber einen Fl"achenraum von $\mathfrak{2\frac{1}{2}}$ franz. Meilen in der L"ange und von einer in der Breite verteilt waren; scheint einerseits auf eine betr"achtliche H"ohe, in welcher die Explosionen vor sich gehen, hinzudeuten, andererseits aber wohl auch die im Obigen vorausgesetzte, h"ochst mannigfaltig und vielseitig wirkende Wurfkraft zu best"atigen.}}; so wird man es wohl nicht gar so sonderbar finden, dass bei solchen Ereignissen, wenn sich dieselben selbst bei Tage und in bewohnten Gegenden zutragen, so selten Menschen oder Vieh von den Steinen getroffen werden, so wie man wohl auch in dieser Hinsicht den Ausdruck Steinregen f"ur nicht ganz passend erachten m"ochte.
\clearpage
\section{\frakfamily{Erkl"arung der Titel-Vignette.}}
\paragraph{}
Es stellt dieselbe einen massiven, rein aus einem St"ucke von der Elbogner Meteor-Eisen-Masse geschnittenen, und ohne alle H"ammerung, blo"s durch Feilen zu Stande gebrachten Siegel-Ring von antiker Form vor, dessen Oberteil oder Kranz eingedreht worden war, und der dann im Ganzen fein poliert und auf der Siegelfl"ache mit Salpeters"aure auf eine solche Tiefe ge"atzt wurde, dass sich das bei solcher Behandlung zum Vorschein kommende kristallinische Gef"uge der Metall-Masse nicht nur deutlich und vollkommen erkennen, sondern sich auch als Zeichnung im geschmolzenen Siegellacke gut abdrucken l"asst. Als Devise ist auf jene Fl"ache ein Pfeil --- das in der chemischen Bildersprache zur Bezeichnung des Eisens (dem Mars geweiht) "ubliche Symbol --- in Verbindung mit einem Sterne, in der Richtung des Falles --- als ein die Natur und zugleich den Ursprung der Masse bezeichnendes Sinnbild --- graviert worden.

Diesem Ringe zur Seite sind zwei W"urfel --- jeder von beil"aufig 4 Linien Seite, und woran an dem einen noch eine der Ecken abgestumpft worden war --- dargestellt, welche ebenfalls rein aus einem St"ucke jener Masse winkelrecht geschnitten, eben und scharfkantig zugefeilt, fein poliert, und dann im Ganzen, auf allen Fl"achen und Kanten zugleich, auf einen "ahnlichen Grad ge"atzt wurden, um die verh"altnism"a"sige Tiefe zu zeigen, auf welche die verschiedenen, das Gef"uge der Masse konstituierenden Teile in dieselbe eindringen, und wie diese, auf einer bestimmten und im Ganzen gleichf"ormigen Tiefe von der Oberfl"ache, gegen einander gelagert, unter einander verbunden und voneinander geschieden sind. (Siehe Seite 81, Note 2.)
\clearpage
\setlength\intextsep{0pt}
\pagestyle{fancy}
\fancyhf{}
\rhead{\frakfamily{Titel-Vignette.}}
\cfoot{\frakfamily{\thepage}}
\begin{figure}[p]
\frakfamily{\tiny Elbogen.}
\includegraphics[width=\textwidth,keepaspectratio]{Figures/Elbogen.jpeg}
\end{figure}
\clearpage
\rhead{\frakfamily{Tafel 1.}}
\cfoot{\frakfamily{\thepage}}
\begin{figure}[p]
\frakfamily{\tiny Agram.}
\includegraphics[width=\textwidth,keepaspectratio]{Figures/Table1-Agram.png}
\end{figure}
\clearpage
\rhead{\frakfamily{Tafel 2.}}
\cfoot{\frakfamily{\thepage}}
\begin{figure}[p]
\includegraphics[scale=0.55,keepaspectratio]{Figures/Table2-Tabor.png}\frakfamily{\tiny Tabor.}
\includegraphics[scale=0.55,keepaspectratio]{Figures/Table2-Eichstädt.png}\frakfamily{\tiny Eichst"adt.}
\includegraphics[scale=0.55,keepaspectratio]{Figures/Table2-LAigle.png}\frakfamily{\tiny L'Aigle.}
\includegraphics[scale=0.55,keepaspectratio]{Figures/Table2-Siena.png}\frakfamily{\tiny Siena.}
\end{figure}
\clearpage
\rhead{\frakfamily{Tafel 3.}}
\cfoot{\frakfamily{\thepage}}
\begin{figure}[p]
\frakfamily{\tiny Lissa.}
\includegraphics[width=\textwidth,keepaspectratio]{Figures/Table3-Lissa.png}
\end{figure}
\clearpage
\rhead{\frakfamily{Tafel 4.}}
\cfoot{\frakfamily{\thepage}}
\begin{figure}[p]
\frakfamily{\tiny Stannern.}
\includegraphics[width=\textwidth,keepaspectratio]{Figures/Table4-Stannern.png}
\end{figure}
\clearpage
\rhead{\frakfamily{Tafel 5.}}
\cfoot{\frakfamily{\thepage}}
\begin{figure}[p]
\includegraphics[scale=0.6,keepaspectratio]{Figures/Table5-Stannern-1.png}\frakfamily{\tiny 1.}
\includegraphics[scale=0.6,keepaspectratio]{Figures/Table5-Stannern-3.png}\frakfamily{\tiny 3.}
\includegraphics[scale=0.6,keepaspectratio]{Figures/Table5-Stannern-2a.png}\frakfamily{\tiny 2a.}
\includegraphics[scale=0.6,keepaspectratio]{Figures/Table5-Stannern-2b.png}\frakfamily{\tiny 2b.}
\includegraphics[scale=0.6,keepaspectratio]{Figures/Table5-Stannern-4.png}\frakfamily{\tiny 4.}
\includegraphics[scale=0.6,keepaspectratio]{Figures/Table5-Stannern-5.png}\frakfamily{\tiny 5.}
\end{figure}
\clearpage
\rhead{\frakfamily{Tafel 6.}}
\cfoot{\frakfamily{\thepage}}
\begin{figure}[p]
\includegraphics[scale=0.45,keepaspectratio]{Figures/Table6-Stannern-1.png}\frakfamily{\tiny 1.}
\includegraphics[scale=0.45,keepaspectratio]{Figures/Table6-Stannern-2.png}\frakfamily{\tiny 2.}
\includegraphics[scale=0.45,keepaspectratio]{Figures/Table6-Stannern-3.png}\frakfamily{\tiny 3.}
\includegraphics[scale=0.45,keepaspectratio]{Figures/Table6-Stannern-4.png}\frakfamily{\tiny 4.}
\includegraphics[scale=0.45,keepaspectratio]{Figures/Table6-Stannern-5.png}\frakfamily{\tiny 5.}
\end{figure}
\clearpage
\rhead{\frakfamily{Tafel 7.}}
\cfoot{\frakfamily{\thepage}}
\begin{figure}[p]
\includegraphics[scale=0.7,keepaspectratio]{Figures/Table7-Charsonville.png}\frakfamily{\tiny Charsonville.}
\includegraphics[scale=0.7,keepaspectratio]{Figures/Table7-Stannern-2.png}\frakfamily{\tiny Stannern.}
\includegraphics[scale=0.7,keepaspectratio]{Figures/Table7-Stannern.png}\frakfamily{\tiny Stannern.}
\includegraphics[scale=0.7,keepaspectratio]{Figures/Table7-Sales.png}\frakfamily{\tiny Salés.}
\includegraphics[scale=0.7,keepaspectratio]{Figures/Table7-Benares.png}\frakfamily{\tiny Benares.}
\includegraphics[scale=0.7,keepaspectratio]{Figures/Table7-Timochin.png}\frakfamily{\tiny Timochin.}
\includegraphics[scale=0.7,keepaspectratio]{Figures/Table7-Siena.png}\frakfamily{\tiny Siena.}
\end{figure}
\clearpage
\rhead{\frakfamily{Tafel 8.}}
\cfoot{\frakfamily{\thepage}}
\begin{figure}[p]
\includegraphics[scale=0.9,keepaspectratio]{Figures/Table8-pallas.jpeg}\frakfamily{\tiny Sibirien.}
\includegraphics[scale=0.9,keepaspectratio]{Figures/Table8-2.png}\frakfamily{\tiny Mexiko.}
\includegraphics[scale=0.9,keepaspectratio]{Figures/Table8-4.png}\frakfamily{\tiny Agram.}
\includegraphics[scale=0.9,keepaspectratio]{Figures/Table8-3.png}\frakfamily{\tiny Lénarto.}
\end{figure}
\clearpage
\rhead{\frakfamily{Tafel 10.}}
\cfoot{\frakfamily{\thepage}}
\begin{figure}[p]
\includegraphics[scale=0.54,angle=90,origin=c,keepaspectratio]{Figures/Table10.png}
\end{figure}
\clearpage
\end{document}
