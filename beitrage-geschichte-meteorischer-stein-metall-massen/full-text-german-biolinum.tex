\documentclass[a4paper, 11pt, oneside, german]{article}
\usepackage[ngerman]{babel}
\usepackage[sfdefault]{biolinum}
\usepackage[LGR,T1]{fontenc}
\usepackage{booktabs}
\setlength{\emergencystretch}{15pt}
\usepackage{fancyhdr}
\usepackage{graphicx}
\usepackage{microtype}
\graphicspath{ {./} }
\begin{document}
\begin{titlepage} % Suppresses headers and footers on the title page
	\centering % Centre everything on the title page
	%\scshape % Use small caps for all text on the title page

	%------------------------------------------------
	%	Title
	%------------------------------------------------
	
	\rule{\textwidth}{1.6pt}\vspace*{-\baselineskip}\vspace*{2pt} % Thick horizontal rule
	\rule{\textwidth}{0.4pt} % Thin horizontal rule
	
	\vspace{1.5\baselineskip} % Whitespace above the title
	
	{\scshape\LARGE Beiträge zur Geschichte und Kenntnis }
	
	\vspace{1\baselineskip} % Whitespace after the title block

	{\scshape\LARGE meteorischer Stein- und Metall-Massen, }

	\vspace{1\baselineskip} % Whitespace after the title block

 	{\scshape\LARGE und der Erscheinungen, welche deren }

	\vspace{1\baselineskip} % Whitespace after the title block

	{\scshape\LARGE Niederfallen zu begleiten pflegen.}

	\vspace{1.5\baselineskip} % Whitespace above the title

	\rule{\textwidth}{0.4pt}\vspace*{-\baselineskip}\vspace{3.2pt} % Thin horizontal rule
	\rule{\textwidth}{1.6pt} % Thick horizontal rule
	
	\vspace{1\baselineskip} % Whitespace after the title block
	
	%------------------------------------------------
	%	Subtitle
	%------------------------------------------------
	
	{\scshape Von D. Karl von Schreibers,} % Subtitle or further description
	
	\vspace*{1\baselineskip} % Whitespace under the subtitle
	
    {\scshape\footnotesize der österreichischen Erblande Ritter und Landstande in Nieder-Österreich, k. k. Rate und Direktor der Hof-Naturalien-Kabinette, Mitgliede der medizinischen Fakultät und der k. k. Landwirtschafts-Gesellschaft in Wien; der königl. Akademie der Wissenschaften zu München; der königl. Gesellschaft der Wissenschaften zu Göttingen; der ehemals kaiserl. Leopoldinisch-Karolinischen Akademie der Naturforscher zu Bonn; der königl. Akademie nützlicher Wissenschaften zu Erfurt; der Sozietät für National-Industrie und der philomatsichen Gesellschaft zu Paris; der Gesellschaft für Künste und Wissenschaften zu Lille; der kaiserl. Gesellschaft der Naturforscher zu Moscow; der Gesellschaft naturforschender Freunde zu Berlin, und der naturforschenden Gesellschaften zu Jena, Leipzig, Hanau, Marburg; der mineralogischen Gesellschaften zu Jena, Petersburg, Dresden; der Werner'schen Sozietät für Naturkunde zu Edinburgh; der physisch-medizinischen zu Erlangen und der pharmazeutischen zu St. Petersburg; die niederrheinischen Gesellschaft für Natur- und Heilkunde zu Bonn; der Sozietät für Forst- und Jagdkunde zu Dreissigacker, u. s. w. Mitgliede, und der mineralogischen Sozietät zu Jena ordentlichem Assessor.} % Subtitle or further description
    
	%------------------------------------------------
	%	Editor(s)
	%------------------------------------------------
    \vspace*{\fill}

	\vspace{1\baselineskip}

	{\small\scshape Wien. 1820.}
	
	{\small\scshape{Im Verlage von J. G. Heubner.}}
	
	\vspace{0.5\baselineskip} % Whitespace after the title block

    \scshape Internet Archive Online Edition  % Publication year
	
	{\scshape\small Namensnennung Nicht-kommerziell Weitergabe unter gleichen Bedingungen 4.0 International} % Publisher
\end{titlepage}
\setlength{\parskip}{1mm plus1mm minus1mm}
\clearpage
\tableofcontents
\clearpage
Segnius irritant animos demissa per aures,
Quam quae sunt oculis subjecta fidelibus.
Horat.
\clearpage
\section*{Vorrede.}
\paragraph{}
Die seltene Gelegenheit, sich von der Realität und den nähern Umständen eines ebenso wunderbaren als lange und vielfach bestrittenen Naturereignisses --- eines so genannten Steinregens persönlich überzeugen zu können, bot sich mir, ebenso unerwartet als höchst erwünscht, im Jahre 1808 bei dem Steinfalle um Stannern in Mähren dar. Wenn gleich nicht als Augenzeuge bei dem Vorfalle selbst zugegen, machten es mir doch die günstigen Umstände einer geringen Entfernung des Schauplatzes von Wien, und einer sehr frühzeitigen und verlässlichen Kunde davon, besonders aber die höheren Ortes erhaltene, in solchen Fällen höchst nötige Vollmacht und die kräftige Unterstützung von Seite der Behörden, möglich, eine allen Wünschen und Forderungen entsprechende Untersuchung an Ort und Stelle, und wenige Tage unmittelbar nach dem Vorfalle selbst, vorzunehmen. Es konnte demnach umso weniger fehlen, dass ein in so vielfacher Beziehung höchst anziehender Gegenstand der Physik meine ganze Aufmerksamkeit, die bisher immer nur schwach und bloß durch von Zeit zu Zeit bekannt gewordene, mehr oder weniger befriedigende Nachrichten von, in der Ferne vorgefallenen, ähnlichen Begebenheiten angeregt wurde, auf sich zog, als derselbe vor Kurzem eben lebhaft und nachdrücklich wieder zur Sprache gebracht und das Interesse dafür durch die mannigfaltigen, von vielen angesehenen Physikern darüber vorgebrachten, ebenso seltsamen als widersprechenden Meinungen und Hypothesen, so allgemein und mächtig in Anspruch genommen worden war.

Diese kräftige Anregung und vollends die erfolgreiche Benutzung jener Gelegenheit, welche so mannigfaltigen Stoff und so zahlreiche Materialien zu eigenen Beobachtungen, Erfahrungen und Reflexionen darbot, hatten nicht nur eine fortgesetzte, ernstliche Beschäftigung mit diesem Gegenstande und eine Reihe von Untersuchungen, Arbeiten und Versuchen zur Folge; sondern veranlassten auch den Entschluss, Alles bis auf diese Zeit an Beobachtungen und Erfahrungen, an Erklärungen und Meinungen hierüber bekannt gewordene, zu sammeln, zusammen zu stellen und einer Vergleichung und kritischen Beurteilung zu unterziehen, und alles aufzubieten, von den etwa noch vorhandenen Produkten früherer, und den künftig vorkommenden, zeitweiliger Ereignisse der Art, so viele als möglich aufzubringen und die kaiserl. Sammlung hieran so vollständig als möglich zu machen.

Diese weit aussehenden Pläne und Vorsätze und jene, Zeit und Ruhe heischenden, mannigfaltigen Unternehmungen, wurden leider nur zu früh und gewaltsam, durch die bald nach dieser Periode eingetretenen ungünstigen Zeitverhältnisse, die den literarischen Verkehr erschwerten und mir ganz andere Beschäftigungen aufdrangen, unterbrochen, und zuletzt, durch die lange Fortdauer und Folgen derselben, zum Teil ganz in Vergessenheit gebracht. Inzwischen war doch bereits nicht nur eine erschöpfende Benutzung jener Gelegenheit erzielt, die umständlichste und befriedigendste Untersuchung jenes Ereignisses zu Stande gebracht, und selbst das Wesentlichste der hierbei erhaltenen Resultate bekannt gemacht, sondern auch eine Fülle neuer Ansichten und Aufklärungen gewonnen und eine Menge belehrender Versuche und erfolgreicher Untersuchungen als Vorarbeiten angestellt, welche zu interessanten Beobachtungen und Erfahrungen führten, die aber, nach einmal so gewaltsam abgerissenem Faden, zu dessen Wiederauffassung sich in langer Zwischenzeit weder Muße noch Veranlassung finden wollte, nur zum Teil und außer Zusammenhang, Bruchstückweise und auf indirekten Wegen, zur öffentlichen Kenntnis gebracht werden konnten.

Glücklicheren Erfolg, als jene ungünstigen Zeitumstände erwarten ließen, hatte mein Bestreben in Auftreibung und Erhaltung der materiellen Belege früherer und in der Zwischenzeit vorgefallener Ereignisse; denn im Laufe von 10 Jahren war es mir doch gelungen, von 29 derselben, die noch vorhanden und irgendwo aufbewahrt oder eben zur Kenntnis gekommen waren,\footnote{Es mochten deren damals, und ungefähr von der Mitte des 15ten Jahrhunderts her, nebst sechs Eisenmassen, bei vierzig derlei Steinmassen gewesen sein, von welchen sich, notorisch, ein oder das andere Bruchstück als sprechender Beleg, nach gen und Ort verschiedener, solcher Ereignisse, ursprünglich im Besitze irgend eines bekannten Privat-Liebhabers befand, der es, seinen individuellen Ansichten gemäß, der Merkwürdigkeit des mehr oder weniger beglaubigten und für ein Wunder angesehenen Factums wegen, oder als ein Dokument der Leichtgläubigkeit der Menschen, als ein Kuriositäts-Stück aufzubewahren für gut fand; von welchen aber in der Zwischenzeit leider viele, ja die Mehrzahl, wie es mit Privat-Besitzungen, zumal solcher Art, zu gehen pflegt, vollends in Verlust geraten sind. und so kam es denn auch, das von beinahe hundert und zwanzig bedeutenden, zu ihrer Zeit ziemliches Aufsehen erregenden und hinlänglich beurkundeten Steinfällen, die demnach von gleichzeitigen Schriftstellern, Historikern und Chronikschreibern der Mit- und Nachwelt bekannt gemacht wurden, und die seit dem Anfange unserer Zeitrechnung bis zum Jahre 1806 sich ereignet hatten, nun kaum mehr neunzehn durch derlei authentische Belege sich bekräftigen lassen, und zwar außer jenem von Ensisheim, von 1492, wovon wir die lange Erhaltung des Beleges der kräftigen Fürsorge Maximilians, der ihn zu einem Kirchenschatz machte, zu verdanken haben, und dem höchst zufällig (in Laugiers Händen in Paris) in einem kleinen Fragmente noch erhaltenen, von dem 1668 bei Verona Statt gehabten Steinfalle --- keiner von einem früheren Datum als aus der zweiten Hälfte des 18ten Jahrhunderts; und die in der kaiserlichen Sammlung seit ihrem Niederfalle aufbewahrte Eisenmasse von Agram, 1751, und der Stein von Tabor, 1753 (außer welchem nur noch wenige Fragmente von diesem, doch sehr bedeutend gewesenen Steinfalle in anderweitigen Besitz zu sein scheinen), sind nebst jenen beiden, so viel bekannt, bereits die ältesten noch vorhandenen Belege der Art.} charakteristische und zur Aufstellung geeignete Stücke zu erhalten. Die kaiserl. Sammlung erwuchs somit zur ansehnlichsten und vollständigsten von der Art kostbarer und merkwürdiger Natur-Produkte, indem dieselbe nun --- mit den bereits früher schon vorhanden gewesenen\footnote{Schon vor fünfzehn Jahren, als ich die Direktion der k. k. Hof-Naturalien-Kabinette antrat, fanden sich deren bereits sieben --- und manche davon schon seit lange --- zwar gerade nicht als Belege der immer noch bezweifelten Ereignisse, die sich, mehr oder weniger befriedigenden, damals hier, so wie überhaupt noch ziemlich allgemein, wenig beglaubigten Nachrichten zu Folge, zu früheren Perioden in verschiedenen Ländern zugetragen hatten, und für deren Produkte sie ausgegeben waren, sondern vielmehr nur als seltsame Fossilien eines rätselhaften Ursprunges und Herkommens, unter den Schätzen des Mineralreiches daselbst aufbewahrt. Namentlich waren es Musterstücke von jenen 1753 bei Tabor in Böhmen, 1768 bei Mauerkirchen in Bayern, 1785 bei Eichstädt in Franken und 1803 um L'Aigle in Frankreich gefallenen Steinen, und nebst der 1751 bei Agram in Kroatien niedergefallenen Metall-Masse, ein Bruchstück von der durch Pallas aus Sibirien bekannt gewordenen und von einer dieser sehr ähnlichen, angeblich aus Norwegen herstammenden Eisenmasse. Und unstreitig war dieser Vorrat damals schon, als wohl kaum jemand an das Zusammensammeln dieser rätselhaften Natur-Produkte noch dachte, der reichhaltigste und in Hinsicht der Größe und Vollkommenheit der Stücke bereits der kostbarste in seiner Art, wie er denn auch, und zwar schon viel früher --- 1798 -- Hrn. D. Chladni, der damals nur die sibirische Masse und den bei Mauerkirchen gefallenen Stein kannte, Gelegenheit verschaffte, sich in seinen bereits bekannt gemachten Mutmaßungen über die Natur und den Ursprung dieser Massen, durch die Wahrnehmung ihrer übereinstimmenden Abweichung von allen terrestrischen Fossilien und der auffallenden Ähnlichkeit derselben unter sich, zu bestärken, und einige Jahre später --- 1801 --- des Hrn. v. Buchs Aufmerksamkeit erregte, und, auf dessen Mitteilung des Gesehenen, einer ähnlichen, entscheidenden und zu jener Zeit noch sehr gewagten, auch lange nach der Hand noch lebhaft bestrittenen Äußerung des Hrn. Pictet, in einer Versammlung des National Institutes zu Paris, zur Bekräftigung diente; so wie auch ich demselben die Kenntnis zu danken hatte, die mich ein ganz unerwartet vorgelegtes Bruchstück von jenen um Stannern gefallenen Steinen, auf der Stelle als identisch und folglich gleichen Ursprunges mit jenen Massen erkennen machte, und die mir Muth und Zuversicht gab, diese vorteilhafte Gelegenheit zur vollsten Selbstüberzeugung und zur möglichsten Überzeugung Anderer zu benutzen und ohne Furcht mich zu kompromittieren, die Schritte zu machen, welche nötig waren, um eine amtliche und förmliche Untersuchung des Factums, so schnell wie möglich, einzuleiten. Die sorgfältige Aufbewahrung und Aufstellung dieser, teils zufällig (der Eisenmassen aus Sibirien und Norwegen und des Stein-Fragments von Eichstädt) oder bei irgend einer Gelegenheit (der Metall-Masse von Agram und des Steines von Tabor) erhaltenen, teils selbst absichtlich und relativ um sehr hohe Preise beigeschafften (des Stein-Fragments von Mauerkirchen und des Steines von L'Aigle) zweideutigen Fossilien, zeugen übrigens von der Aufmerksamkeit und Werthschatzung, welche die Wiener Naturforscher diesen Natur-Produkten zu jener Zeit schon zollten, indes so manche von jenen oben erwähnten vierzig ähnlichen, materiellen Belegen solcher Ereignisse, von welchen, notorisch, teils ein Fragment, meistens aber ein ganzer Stein, teils selbst die ganze niedergefallene Masse und zwar gewöhnlich mit authentischen Nachrichten von glaubwürdigen Männern, oft selbst mit formlich abgefassten Urkunden, einem wissenschaftlichen Vereine zur Beurteilung, oder Kabinetten und öffentlichen Anstalten zur Aufbewahrung eingesendet worden waren, in Verlust gerieten; so dass nicht nur an diesen vermeintlich sichern Bestimmungsplätzen sich gegenwärtig keine Spur mehr von denselben findet, sondern selbst nur von drei derselben kleine Fragmente in Privat-Besitz nachweisbar noch vorhanden sind. So kam einer von den bei Roa in Spanien 1438 gefallenen Steinen, in das königl. Museum zu Madrid; einer von jenen aus der Gegend von Schleusingen 1552, in das herzogl. Museum zu Rudolstadt; der 39 Pfund schwere, 1581 in Thüringen gefallene Stein (nebst der im Archive zu Dresden noch aufbewahrten Urkunde) und der 1/2 Zent. schwere, 1647 bei Zwickau gefallene Stein, in die Kunstkammer nach Dresden; einer von jenen 1654 auf der Insel Fünen gefallenen, in das königl. Naturalien-Kabinett zu Kopenhagen; der in demselben Jahrhunderte in Mailand gefallene Stein mit dem Settalianischen Kabinette, in welchem derselbe ursprünglich aufbewahrt gewesen, in die Ambrosianische Bibliothek daselbst; der im Kanton Bern 1698 gefallene, in die dortige Stadt-Bibliothek; der Stein von Terranova in Kalabrien 1755, in die königliche Bibliothek zu Neapel; jener von Sigena in Aragonien 1773, in das königl. Museum zu Madrid (ein kleines Fragment davon befindet sich im königl. Museum zu Paris; und der 6 1/2 Pfund schwere, 1775 bei Rodach gefallene Stein, in das herzogl Naturalien-Kabinett zu Coburg. So wurden mehrere von den vielen und großen, 1668 im Veronesischen gefallenen Steinen, der damaligen Akademie zu Verona vorgelegt, und gegenwärtig scheint, wie bereits erwähnt, nur ein kleines Fragment mehr davon vorhanden zu sein, und so wurden Bruchstücke von den bei Nicorps in der Normandie 1750 und von jenen bei Lucé 1768 in Frankreich gefallenen Steinen, der Pariser Akademie eingeschickt, und nur von letzteren finden sich derzeit noch einige kleine Fragmente im Besitze von Privaten.} --- 36\footnote{Nämlich 27 Stein- und 9 Metall-Massen. Von ersteren möchten derzeit, 1820, im Ganzen etwa 40 --- notorisch und nachweisbar in Händen bekannter Besitzer indes, wohl kaum mehr als 34 --- als materielle Belege von Ereignissen der Art --- deren doch dermal beinahe 150 seit unserer Zeitrechnung zur nähern Kenntnis kamen, und hinlänglich beurkundet sind --- von letzteren etwa 12, gleichen, obgleich nicht faktisch erwiesenen Ursprunges, als Stücke oder in Fragmenten, vorhanden und noch irgendwo aufbewahrt sein. Von ersteren besaß das Pariser Museum 1815, nur 13; das britische Museum in London 1818, 11; und von den vorzüglichsten Privat-Sammlern (zu welchen insbesondere auch Heuland und Sowerby in London gehören, deren Sammlungstand mir inzwischen zur Zeit nicht speziell genug bekannt ist) Klaproth 1810, 10; Lavater in Zürich 1811, 10; Blumenbach 1812, 11; De Drée in Paris 1818, 26; Chladni 1819, 27.} aufzuweisen hat, die in verschiedenen Ländern und zu verschiedenen Perioden, nach ganz verlässlichen Nachrichten entweder sichtbar niedergefallen oder zwar bloß zufällig aufgefunden, aber, nach aller Wahrscheinlichkeit und Analogie, allgemein auch als solche anerkannt sind; und zwar in so bedeutenden Massen, dass deren Gesamtgewicht beinahe drei Zentner erreicht.

Viele von diesen neu akquirierten, frühzeitig erhaltenen und jene, früher schon im kaiserl. Kabinette vorhanden gewesenen, so wie manche einzelne in hiesigen Privat-Sammlungen befindliche und mehrere von entfernten Besitzern gefälligst mir zur Ansicht mitgeteilte, ausgezeichnete Stücke, insbesondere aber die reiche Ausbeute von dem Steinfalle um Stannern und die vielen, besonders ausgezeichneten, frischen und vollkommenen Exemplare von daher, gaben gleich Anfangs zur Anfertigung von Abbildungen Veranlassung. Eine genaue und vorzüglich in oryktognostischer Beziehung vorgenommene Untersuchung und Vergleichung dieser rätselhaften Fossilien, wie sie bei diesem Vorrate möglich war, machte nämlich auf so Manches aufmerksam, was ebenso wesentlich zu deren Erkenntnis als merkwürdig an sich und dabei einer Versinnlichung bedürftig und einer solchen auch fähig schien, dass naturgetreue Darstellungen umso zweckmäßiger und erwünschter erachtet wurden, als die Objekte selbst, ihrer Seltenheit und Kostbarkeit wegen, und gewisser Maßen bloß als Einzelheiten existierend, nur von Wenigen besessen, von Vielen nicht einmal je gesehen werden können.

Mehr als siebzig derlei Original-Abbildungen waren bereits schon zu Anfang des Jahres 1809 von der Hand eines geschickten Künstlers zu Stande gebracht, die, trotz der oft erprobten Schwierigkeit bei Darstellung anorganischer Natur-Produkte, allgemeinen Beifall fanden und den Wunsch erregten, dass eine preiswürdige Vervielfältigung derselben möglich sein möchte; allein die im gewöhnlichen Wege auf Kupfer veranstalteten Proben zeigten nur zu bald die Schwierigkeiten der Ausführung und die Kostspieligkeit einer solchen Unternehmung; so dass der Zweck nur unvollkommen und einseitig zu erreichen gewesen wäre. Die Fortschritte, welche in dieser Zwischenzeit im Steindrucke gemacht wurden, die Vorteile, welche dieser gewährt und der gute Erfolg, mit welchem man denselben bereits verschiedentlich zur Darstellung naturhistorischer Gegenstände anwendete, bestimmten mich, auch dieses Mittel zur Vervielfältigung versuchen zu machen, und da der Versuch, wo nicht meinen Wünschen, doch den Erwartungen entsprach, viele Sachverständige befriedigte und das Wesentlichste erzielen zu lassen verhieß; so fand ich mich umso bereitwilliger, der erneuerten Aufforderung mehrerer Wissenschaftsfreunde, und namentlich des Herrn D. Chladni, bei Gelegenheit der eben hier veranstalteten Herausgabe seines neuesten Werkes über diesen Gegenstand, zu entsprechen, und wenigstens eine Auswahl aus jener Sammlung von Abbildungen auf diesem Wege vervielfältigen zu lassen und bekannt zu machen, als meine Verhältnisse und Berufsgeschäfte bereits lange schon alle Hoffnung mir benommen hatten, den früheren Plan zu einer umfassenderen Bearbeitung des Gegenstandes, je realisieren und selbe demnach ihrer ursprünglichen Bestimmung gemäß benutzen zu können, dagegen eine so günstige Gelegenheit, wie die Erscheinung jenes Werkes war --- die eben sowohl zu meiner Beruhigung, als zum unbezweifelbaren Gewinn der guten Sache, jener Realisierung zuvor kam und sie nun vollends ganz entbehrlich machte - mir die Versicherung gab, sie einer vorteilhafteren Bestimmung widmen und, in solch empfehlender Begleitung, für selbe eine willkommenere Aufnahme gewärtigen zu können.

Während einer Reihe von tumultuarischen und geschäftsvollen Jahren durch mannigfaltige, zum Teil sehr heterogene Berufs- und Wissenschafts-Anforderungen, ganz von diesem Gegenstande abgelenkt, mehrerer schriftlicher Aufsätze verlustiget, des chaotischen Vorrates zahlloser Notaten kaum Meister, und all des Vergangenen im Einzelnen nur schwach mich besinnend war es anfänglich meine Absicht nur, diese Abbildungen durch kurze Beschreibungen zu erläutern, und dies umso mehr, als einerseits die gründliche und so vielseitig vollständige Bearbeitung des Gegenstandes in jenem Werke jeden weitern Kommentar entbehrlich, andererseits der Drang der Zeit, um der nun einmal gemachten Verheißung zu entsprechen, so wie der Mangel an erforderlicher Muße, Geschäftsfreiheit und Geistesruhe, um jene vorhandenen Gedächtnisbehelfe benutzen und die volle Erinnerung wieder gewinnen zu können, der Zustandebringung eines solchen sehr entgegen waren.

Da inzwischen selbst diese beschränkte Behandlung des Gegenstandes nicht nur ein aufmerksames Studium jenes Werkes und eine Zurateziehung mehrerer anderer, sondern insbesondere auch, der häufigen in dieser Zwischenzeit neu erhaltenen, erst noch zu bearbeitenden Materialien wegen, eine erneuerte Durchsicht und Prüfung eigener früherer Ausarbeitungen, eine weitere Verfolgung derselben und selbst eine Fortsetzung und Wiederholung von abgebrochenen und unbefriedigend gebliebenen einstmaligen Versuchen und Untersuchungen notwendig machte; so wurden bald wieder alle Berücksichtigungspunkte, welche die Vielseitigkeit des Gegenstandes in physischer und philosophischer Hinsicht darbietet und die, jetzt noch wie vor, den Physikern so reichhaltigen Stoff zu eigenen Mutmaßungen und Ansichten, und so vielfachen Anlass zu Debatten und Kontroversen geben --- und wohl noch lange geben möchten --- mittel- oder unmittelbar angeregt, und, samt den einst im Verfolge jener umfassenderen, früheren Bearbeitung des Gegenstandes erhaltenen Resultaten, Bruchstückweise wenigstens, ziemlich lebhaft wieder ins Gedächtnis zurück gerufen.

Und die mit Erweckung des Erinnerungsvermögens wieder erwachte alte Vorliebe für den Gegenstand und ein bei jener vergleichenden Rekapitulation und Nachholung des im Laufe eines vollen Dezenniums, zumal auf den soliden Wegen der Erfahrung, Beobachtung und Untersuchung, Geschehenen, in etwas geschmeicheltes Selbstgefühl, reitzten mich umso mehr, manche Resultate früherer Forschungen und Versuche, und einige dadurch motivierte Reflexionen und Folgerungen bei dieser Gelegenheit unter einem bekannt zu machen, als ich, nach eigenem Gefühle im Verfolg der Ausarbeitung, besorgen zu müssen glaubte, dass einerseits die Trockenheit einer so einseitigen Behandlung des, gerade von der spekulativen und vernünftelnden Seite am meisten anziehenden, Gegenstandes, rein deskriptiv, wie sie anfänglich beabsichtigt war, zumal durch die, bei solchen Objekten doch unerlässliche physiographische Kleinigkeitskrämerei, den Leser anekeln, andererseits Manches, hie und da mit Nachdruck Angedeutete, unverständlich oder unerheblich, wenigstens deutungs- und beziehungslos erscheinen möchte.

Die Vielseitigkeit des Gegenstandes und die häufigen Berührungs- und Beziehungspunkte, welche das Materielle der Objekte in obigen Rücksichten darbot, motivierten nun eine bedeutende Menge solcher Einstreuungs-Artikel, die, dem einmal angenommenen Plane der Bearbeitung und ihrer individuellen Bestimmung gemäß --- als Erläuterung oder Deutung irgend eines in der Physiographie berührten Punktes zu dienen, oder um, auf Veranlassung eines solchen, irgend eine, das Objekt oder den Gegenstand im Allgemeinen betreffende, physische, chemische, philosophische oder historische Tat- und Erfahrungssache zur Kenntnis, oder endlich irgend eine vorgefasste Meinung oder gangbare Hypothese zur Berichtigung, oder eine neue Mutmaßung und Ansicht in Anregung zu bringen --- als Noten zum Text angebracht wurden.

Obgleich diese Zugabe solcher Gestalt weder den Plan noch den eigentlichen Zweck der Behandlung des Gegenstandes abänderte, sondern nur eine Veränderung des Titels und in der Art der Ankündigung veranlasste: so ist damit doch, da dieselbe den Hauptgehalt bedeutend überwiegt, das Volumen des Werkes beträchtlich über meine anfängliche Absicht, und weit über die ursprüngliche Berechnung des Verlegers herangewachsen, und ich fände mich über die Folgen davon --- die Verzögerung des Erscheinungs-Termines und die Erhöhung des Preises --- verantwortlich, wenn ich mich nicht für erstere, durch meine Verhältnisse und die Anforderungen der Aufgabe unter oben geschilderten Umständen, entschuldiget, und gegen letztere überhaupt, durch jede Aufopferung von meiner Seite, vorhinein schon verwahrt zu haben, glauben könnte. Dagegen muss ich über den Wert des Gehaltes, der hierzu Veranlassung gab, sowie über jenen des Ganzen, das Urteil kompetenter Richter gewärtigen, hoffe aber hierbei auf jene Nachsicht rechnen zu dürfen, auf welche die Natur des Gegenstandes und die vielseitigen und schwierigen Anforderungen desselben, dem regen Eifer seiner kühnen Verfechter bei so sehr beschränkten Kräften, den vollsten Anspruch geben:

Quod si deficiant vires, audacia certe  
Laus erit; in magnis et voluisse sat est.  
Propert.

Wien, im Julius 1820.
\clearpage
\section{Erste Tafel.}
\begin{center}
Die Gediegeneisen-Masse

von 71 Pfund Wiener Kommerziell-Gewicht,\footnote{Bekanntlich ist nebst dieser nur noch eine zweite, kleinere Masse von 16 Pfund als Produkt des vorausgegangenen Feuer-Meteors, der beobachteten Feuerkugel, niedergefallen, welche nicht nur im Niederfallen, und selbst bei der Lostrennung von jener gesehen, sondern auch gleichzeitig mit jener, und auf 2000 Schritt Entfernung von derselben, aufgefunden und aus der Versenkung gehoben wurde; über deren Aufbewahrung oder Verwendung aber ursprünglich keine Nachricht gegeben ward, und von deren Nochvorhandensein auch bis jetzt keine weitere Kenntnis erlangt werden konnte.}
\end{center}
\paragraph{}
welche am 26. Mai 1751 gegen 6 Uhr Abends bei dem Dorfe Hraschina in der Agramer Gespanschaft (etwa drei Meilen N. O. von Agram) in Kroatien, unter den gewöhnlichen meteorischen Erscheinungen und im Angesichte mehrerer Augenzeugen aus der Luft gefallen, und drei Klafter tief in einen kurz zuvor gepflügten Feldgrund eingedrungen war.

Es wurde diese Masse\footnote{Es ist dieselbe umso interessanter und schätzbarer, als sie von den ohne dies sehr wenigen ähnlichen Eisen-Massen, deren Niederfallen historisch und faktisch erwiesen ist (wie die, ihrer Beschaffenheit nach, zwar zweifelhaften, und wie es scheint, ganz in Verlust geratenen Miscolz in Ungarn 1559, und von Torgau 1561; die zwar noch --- in Gotha --- vorhandene, aber dem Fundorte nach zweifelhafte --- aus Sachsen --- von 1540 oder 1550 ? und nebst einigen, die seit unserer Zeitrechnung im Orient --- China, Japan, Persien --- gefallen sein mögen; jene, am zuverlässigsten bekannte, 1621 zu Lahore in Indien gefallene, welche aber der mogolische Kaiser Dschehan-gir ganz verschmieden ließ), die einzige noch vorhandene zu sein scheint; so wie sie die einzige von dieser Art ist, welche physisch und chemisch untersucht wurde, und durch den Befund ihres Gehaltes und ihrer physischen Eigenschaften, als Prototyp auf einen gleichen meteorischen Ursprung jener ähnlichen Eisen-Massen, nach Analogie zu schließen berechtigte, welche zufällig zu verschiedenen Zeiten und an verschiedenen, sehr entfernten Orten aufgefunden worden, bekannt und noch vorhanden sind, aber bei welchen es, ihre Herkunft zu erweisen, an historischen und faktischen Belegen fehlte (wie dies bei den, in dieser Beziehung problematischen Eisen-Massen aus Süd- und Nord-Amerika, Brasilien, Afrika, Sibirien, Böhmen, Ungarn u. s. w. der Fall ist). Auch war sie von den derben Gediegeneisen-Massen die erste, und überhaupt mit von den ersten Meteorolithen (mit dem Eisen aus Sibirien, dem Eichstädter und Sieneser Meteor-Steine), welche auf Veranlassung der kaum bekannt geworden Untersuchungen Howards (1802) in Deutschland analytisch untersucht wurden, und zwar von Klaproth (der die Resultate seiner Untersuchungen zuerst in einer Vorlesung in der königl. Akademie der Wissenschaften zu Berlin, und dann im neuen allgemeinen Journal der Chemie, B. 1, zu Anfang des Jahres 1803 bekannt machte), welchem zu diesem Ende ein kleines Stück von dieser Masse (gleichzeitig mit einem Stücke vom Eichstädter Meteor-Steine) schon im Jahre 1802 von hier aus mitgeteilt worden war. Im Jahre 1808 wurde, soweit es ohne Beeinträchtigung der Form und Ansicht der Masse geschehen konnte, ein größeres Stück von etwa 20 Loth abgesägt, um zu technischen Versuchen zu dienen, die Hr. Direktor von Widmanstätten auf meine Veranlassung vornehmen wollte, und welche zu merkwürdigen Resultaten, und insbesondere zur höchst interessanten Entdeckung des kristallinischen Gefüges, welches diesen Massen, wo nicht ausschließlich, doch vorzugsweise eigentümlich und für dieselben charakteristisch zu sein scheint, führten. Die durch Absägung jenes Stückes an der Masse erhaltene Fläche wurde mit Salpetersäure geätzt, um jenes Gefüge oberflächlich darzustellen und die Entdeckung zu bewähren; von dem Überreste des abgesägten Stuckes wurden kleine Abschnitte nach London, Paris und Harlem mitgeteilt.} ihrer Merkwürdigkeit wegen, und als Beleg des wunderbaren Naturereignisses, von dem bischöflichen Konsistorium zu Agram, welches, aus eigenem Antriebe, durch Abgeordnete das Factum sogleich (am 2. Julius desselben Jahres) an Ort und Stelle amtlich und förmlich untersuchen ließ,\footnote{Es war diese eine der frühesten Begebenheiten der Art (die erste, mit Ausnahme jener von Thüringen 1581 und von Bern 1698, welche ebenfalls von den Lokal-Behörden legal untersucht, und durch eine ausgefertigte Urkunde dokumentiert wurden, wovon sich jene von der ersteren Begebenheit, nach Chladnis Versicherung, noch zur Zeit im Archive zu Dresden aufbewahrt befindet), welche einer amtlichen Untersuchung von einer Behörde wert geachtet und durch eine ausgefertigte formliche Urkunde der Nachwelt aufbewahrt, und die erste, von welcher diese selbst, wenn gleich gerade nicht mit der Absicht, das Factum beglaubigen zu machen, zur Publizität gebracht wurde (Stutz, Bergbaukunde B. 2, 1790); und es wäre in der Tat unbegreiflich, wie eine so unbefangene und reine, deutungs- und beziehungslose Darstellung von einer so achtbaren Behörde so wenig Aufmerksamkeit erregen, so wenig auf die Überzeugung wirken konnte, wenn nicht zu vermuten stände, dass sie durch jene Publizierung nur wenigen eigentlichen Physikern zur Kenntnis kam. Sie verdient umso mehr an einem schicklicheren Orte, wie bei einer andern Veranlassung geschehen soll, und im Original bekannt gemacht zu werden, als es die ausdrückliche Absicht der Aussteller und Einsender dieser, mit allen Förmlichkeiten ausgestatteten, Urkunde war, nicht nur die Mitwelt von der Realität des Factums zu überzeugen, sondern auch diese Überzeugung durch ein authentisches Dokument auf die Nachwelt zu bringen.} samt einer schriftlichen Urkunde, welche das Untersuchungs-Protokoll enthielt, noch in demselben Jahre an den kaiserl. Hof eingesendet, wo sie in der k. k. Schatzkammer zu Wien aufbewahrt, und in der Folge, bei Übertragung der naturwissenschaftlichen Gegenstände aus derselben, an das k. k. Hof-Naturalienkabinett abgegeben wurde.

Es hat dieselbe eine platt gedrückte, etwas verschobene, dreiseitige Gestalt, und zeigt demnach zwei Flächen und drei Ränder. Die eine dieser Flächen ist, schief von den Rändern aufsteigend, mäßig gewölbt, nach oben sich verebnend, und durch mehr oder weniger unterbrochene, gebogene und wellenförmige, rippenartige, abgerundete Erhabenheiten, und durch größere und kleinere, seichtere und tiefere, meistens rundliche oder ovale Vertiefungen und Eindrücke, welche von jenen begrenzt werden, sehr uneben; die andere entgegen gesetzte Fläche ist dagegen beinahe flach und eben, und zeigt nebst einigen kleineren und tieferen Eindrücken gegen die Ränder hin, nur drei große, sehr seichte und breit verlaufende Vertiefungen, welche, idem sie durch flache Zwischenräume in einander übergehen, und gewisser Maßen zusammen hängen, diese Fläche im Ganzen etwas ausgehöhlt erscheinen machen.

Die Ränder, unter welchen diese beiden Flächen zusammenstoßen, sind von der konvexen Fläche her schief nach Außen abgerundet, und nicht nur durch die rippenartigen Erhabenheiten, welche sich von daher über dieselben bis an die entgegen gesetzte Fläche fortsetzen, und durch ähnliche Eindrücke, sehr uneben, sondern auch, zumal gegen die Mitte, sehr stark ausgeschweift und gewisser Maßen unterbrochen, so dass man ihre Richtung nur schwer bestimmen kann. Zieht man inzwischen nach den hervorragendsten Punkten eines jeden Randes eine, demselben parallel laufende, gerade Linie, und schließt man das solcher Gestalt erhaltene Dreieck durch Verlängerung dieser Linien über die abgerundeten Ecken hinaus, bis sie sich berühren; so fallen die Linien, welche den beiden Seitenrändern oder den beiden längeren Schenkeln der dreieckigen Form der Masse entsprechen, auf die Grundlinie, welche --- die Masse in dieser Richtung betrachtet -- dem untern Rande entspricht, unter einem Winkel von beiläufig 80° auf. Die dritte oder obere, dem untern Rande gegenüberstehende Ecke der dreiseitigen Masse, fällt außer das Mittel derselben, und --- die Masse von der konvexen Fläche betrachtet -- stark gegen den rechten Seitenrand hin, indem der linke Seitenrand bogenförmig sich gegen jenen hinüberzieht, und sich mit demselben in eine gegen ihn gerichtete, etwas stumpfe Spitze vereinigt. Die ganze Masse verflacht sich mehr gegen die linke Seite hin, zumal nach oben an der Krümmung des Seitenrandes, der hier am dünnsten, an einer Stelle beinahe schneidend, und da von der entgegen gesetzten Fläche etwas übergebogen ist; dagegen erhebt sich die rechte Seite hier mit dem Außenrande und der Spitze, indem sie von der entgegen gesetzten Fläche gleichsam herüber gedrückt erscheint, so dass dort, abgesehen von den an dieser Stelle befindlichen ziemlich großen und tiefen Eindrücken, welche den äußersten Rand auch ziemlich dünn machen, eine starke Abweichung von der horizontalen Ebene dieser Fläche bewirkt wird, und die Spitze des Dreiecks, oder vielmehr beinahe die ganze obere Hälfte der Masse, solcher Gestalt etwas verdreht erscheint. An dieser Fläche dagegen laufen die Ränder, abgesehen von den genannten Abweichungen und von den zufälligen Eindrücken, größten Teils horizontal mit der Ebene derselben; nur gegen die eine untere Ecke, welche der Richtung der verdrehten Spitze entspricht, ist der Seitenrand schief abgerundet, und ebenfalls gegen die konvexe Fläche gedrückt, so dass es scheinen möchte, als wenn die Masse, in noch weichem Zustande! auf dieser ganzen Seite, im Auffallen einen größeren Widerstand gefunden hätte.\footnote{Es findet sich leider in der Urkunde nicht bemerkt, in welcher Lage diese Masse in ihrer Versenkung gefunden wurde, sondern es wird nur erwähnt, dass die Spalte (nicht Grube) in der Erde drei Klafter tief und eine Elle weit gewesen sei, nach welchen Ausdrücken zu mutmaßen käme, als wäre sie mit einen der Ränder eingedrungen und auf keine der Flächen aufgefallen, wie dies auch nach dem Schwerpunkte der Masse, der auf deren unteren Rand fällt, der Fall gewesen sein musste, da eine rotierende Bewegung, zumal flächenwärts, nach Form und Beschaffenheit derselben nicht wohl angenommen werden kann. Umso merkwürdiger ist die auffallende Verschiedenheit der Oberflache der beiden Flächen. Es wird zwar in der Urkunde bemerkt, dass in den Vertiefungen der konvexen Fläche (also gerade der entgegen gesetzten) etwas Erde eingedrückt war; daraus kann aber noch nicht gefolgert werden, dass gerade die Masse auf diese Fläche auffiel, indem beide Flächen wohl in ziemlich gleich stark drückenden Contact mit der Erde kamen, wenn die Masse mit einem Rande vorwärts in dieselbe eindrang; dass sich aber nur an der einen Fläche Erde eingedrückt fand, mag von der starken Unebenheit ihrer Oberfläche hergerührt haben. Dass sich übrigens gegenwärtig keine Spur von Erde an der ganzen Masse mehr findet, mag wohl mit als Beweis dienen können, dass die Masse nicht im geschmolzenen oder gar flüssigen Zustande zur Erde gekommen sei, in welchem Falle die Erde wohl etwas mehr fixiert worden wäre.}

Die größte Länge der Masse, von den hervorragendsten Punkten des rechten Seitenrandes, von der oberen Ecke oder Spitze bis zur hervorragendsten Erhabenheit am untern Rande dieser Seite gemessen, beträgt 15 1/2 Zoll; am linken Rande nur 13 Zoll.

Die größte Breite, von den hervorragendsten Erhabenheiten an beiden Seitenrändern, etwa 3 Zoll ober dem untern Rande, beträgt 12; im Mittel der Masse ist sie 8; am oberen Ende, etwa 3 Zoll unter der Spitze, von ähnlichen Punkten gemessen, 6 1/2 Zoll.

Die größte Dicke, von den erhabensten Stellen an beiden Flächen zusammen gemessen, beträgt 3 3/4 Zoll; an Stellen, wo zufällig von beiden Flächen Vertiefungen zusammenfallen, übereinander zu liegen kommen, beträgt sie kaum 2, hie und da selbst kaum 1 Zoll; wo dies nicht der Fall ist, kann man sie im Durchschnitt auf 3 Zoll annehmen. An den äußersten Rändern ist die Masse hie und da sehr dünn, kaum 1/2, selbst nur 1/4 Zoll dick; an einer Stelle beinahe sogar schneidend scharf.

Die Vertiefungen und Eindrücke, welche sich auf der konvexen Fläche zeigen, haben zwar viele Ähnlichkeit mit jenen, welche sich auf der Oberfläche der meisten Meteor-Steine finden, sind aber hier ungleich größer, tiefer, häufiger und zusammen hangender, so dass die rippenartigen Erhabenheiten, welche sie begrenzen, gewisser Maßen ein unregelmäßiges und verworrenes Netz bilden, und der Oberfläche ein zellenförmiges Ansehen geben. Manche dieser Vertiefungen haben im Mittelpunkte 5 bis 7, und wenn man das Niveau von den zunächst liegenden höchsten Erhabenheiten nehmen will, 9 bis 15 Linien Tiefe bei einer Ausdehnung von 1 1/2 bis 2 1/2 Zoll im Durchmesser. In diesen größeren Vertiefungen, welche meistens einen mehr oder weniger rundlichen, aber mehrfach ausgeschweiften Umriss, und bald eine Grube, bald eine Erhabenheit zum unregelmäßigen Mittelpunkte haben, liegen die seichteren, 1/2 bis 2, 3 Linien tiefen, daum- oder fingerartigen Eindrücke von verschiedener Größe, zu 3, 4 bis 5 unregelmäßig, bisweilen aber auch kreisförmig beisammen; inzwischen kommen solche Eindrücke auch einzeln oder isoliert außer den Vertiefungen vor. Die rippenartigen Erhabenheiten, welche durch diese Vertiefungen und Eindrücke gebildet werden, entsprechen der Stärke, Höhe und Dicke nach, der Tiefe derselben und ihrer wechselseitigen Entfernung voneinander; und ihrer Ausdehnung und Richtung nach, nach welchen sie bald länger, bald kürzer, bald wellenförmig, bald unter verschiedenen Winkeln gebogen erscheinen, der Lage und Form derselben, und ihrer wechselseitigen Verbindung unter sich. Demnach haben die Erhabenheiten zwischen aneinander grenzenden Hauptvertiefungen oft mehrere Linien Höhe, und eine nicht minder beträchtliche Dicke, zumal an ihrer Basis, und nicht selten ein paar Zoll Länge, insofern ihr zusammen gedrückter, abgerundeter Rücken nicht durch isolierte Eindrücke unterbrochen, breit gedrückt und gewisser Maßen gedoppelt wird; die Erhabenheiten dagegen, welche die seichteren, in den größeren Vertiefungen liegenden, Eindrücke begrenzen, sind nur sehr schwach, oft kaum merklich, und verflachen sich mit ihrer Basis nicht selten, ohne einen bedeutenden Rücken oder eine Kante gebildet zu haben. Es finden sich jedoch einige Erhabenheiten auf dieser Fläche, welche nicht, wenigstens nicht unmittelbar, durch Eindrücke entstanden zu sein scheinen, da sie solche nicht geradezu begrenzen, und zapfen- oder zitzenförmig vorragen; und andere, welche zum Teil zwar durch Vertiefungen veranlasst worden zu sein scheinen, indem sie zwischen solchen liegen, auch rippenartig, wie die meisten, gestaltet, aber höher und stärker sind, als sie, vermöge der Ausdehnung und Tiefe jener, gerade zu sein hätten.\footnote{Wenn gleich im strengen Sinne der Kunstsprache diese Beschaffenheit der Oberfläche keineswegs zellig, ästig und zackig genannt werden kann, so ist sie doch, wenigstens dem Ansehen nach, im Ganzen einer solchen sich sehr annähernd, und obgleich sie auch als solche hier nur auf die Oberfläche beschränkt ist, und ihr noch ein wesentlicher Umstand, nämlich die Ausfüllung der Zellen durch eine anscheinend fremdartige Substanz, ermangelt; so ist doch gerade durch sie eine Ähnlichkeit dieser mit der sibirischen Eisen-Masse und eine Annäherung an dieselbe unverkennbar. Und so wie auf der andern Seite eine ähnliche, und, wie mir deucht, ganz unbestreitbare Annäherung der eigentlichen Meteor-Steine an dieselbe, ja, wie ich zu behaupten wage, durch die stark eisenhaltigen (wie jene von Eichstädt, Timochin, Tabor, bei welchen das Gediegeneisen nicht bloß in zerstreuten Körnern eingesprengt, sondern schon in mehr oder weniger zusammen hängenden Zacken, und nur von noch vorwaltender erdiger Masse eingehüllt erscheint) ein wahrer Übergang in dieselbe (zumal, wenn man die dichteren, mehr erdigen Partien, die sich an manchem größeren Stücke von der sibirischen Masse finden, oder die ungleich weniger ästigen und zelligen, vorgeblich aus Sachsen und Norwegen herstammenden, der sibirischen übrigens höchst ähnlichen Eisen-Massen als Zwischenglieder betrachten will) Statt findet; so fehlt es vielleicht nur noch an ein paar Zwischengliedern (welche sich wohl noch finden möchten, und wozu sich z. B. gleich die Brasilianer Eisen-Masse eignen dürfte, welche, obgleich im Ganzen dicht und derb, nach den neuesten Reiseberichten der Bayer'schen Naturforscher, die selbe an Ort und Stelle sahen, voll Gruben, Löcher und oberflächlicher Eindrucke ist, die zum Teil mit eingekeilten Quarz?-Stucken erfüllt sein sollen), um diesen auch hier sinnlich nachweisen zu können. Es findet eine ungleich größere Verschiedenheit im äußern Ansehen sowohl, als im Aggregats- und Kohäsions-Zustande ja selbst im qualitativen und quantitativen Verhältnisse der Gemeng- und Bestandteile zwischen manchen Meteor-Steinen Statt, als zwischen jenen Massen. Ein in der mineralogischen Diagnostik geübtes Auge dürfte zwischen einem etwas grobkörnigen, eisenschüssigen Sandsteine, und einem etwas dichten, porphyrartigen Bimssteine wohl kaum mehr Verschiedenheit auffinden können, als z. B. zwischen den Meteor-Steinen von Eichstädt und von Stannern. Und doch lässt sich zwischen diesen letzteren durch eine Reihe von Zwischengliedern, welche die allmähliche Abänderung des Aggregats- und Kohäsions-Zustandes, und die graduelle Zustandsveränderung mancher einzelnen Gemengteile und deren allmähliches Hervortreten versinnlichen, ein augenscheinlicher Übergang nachweisen, welches zum Teil bei Erklärung der siebenten Tafel geschehen wird, und bei einer künftigen Veranlassung umständlicher geschehen soll.\\
Keine Verwandtschaft von Gattungen terrestrischer Fossilien versinnlicht wohl den Begriff einer Sippschaft (wie ich mich sehr bald überzeugte, und daher dieses Ausdruckes schon bei Gelegenheit meiner Beschreibung der mährischen Aerolithen in Gilberts Annalen 1808 bediente, als ich zuerst auf die viel zu wenig beachtete Verschiedenheit der Meteor-Stein überhaupt, und auf die doch zwischen ihnen bestehende Verwandtschaft vorläufig aufmerksam machte), selbst ganz rein oryktognostisch genommen, deutlicher, und bei weitem keine zeigt so ausgedehnte Grenzen und so heterogen scheinende Extreme bei so allmählichen Übergangen, als die Meteor-Massen, und bei keiner Verwandtschaftsstufe terrestrischer Fossilien ist die Konstruierung einer so genannten Suite, in Werners Sinne, zu ihrer vollständigen Erkenntnis notwendiger und an sich interessanter und lehrreicher.\\
Die Betrachtung der Meteor-Massen von dieser Seite, nämlich von Seite ihrer so wesentlichen Verschiedenheit voneinander, welche bisher, wie nun auch Chladni bemerkt, so wenig berücksichtigt wurde, obgleich noch weit auffallendere Beispiele, als das oben angeführte (z. B. die unter sich sowohl als von allen übrigen noch weit mehr als jene, und in vielfachen Beziehungen abweichenden Meteor-Steine von Alais, Chantonnay, Erxleben, Langres), Aufmerksamkeit hätten erregen sollen, --- und nach dieser ihrer Versippung unter einander: möchte wohl, wo nicht über den Ort, doch über die Art ihrer ursprünglichen Entstehung und Bildung, und über manche, noch lange nicht befriedigend erklärte Erscheinungen bei ihrem Niederfalle, einiges Licht geben, und vielleicht selbst manche unserer geognostischen und oryktognostischen Ansichten berichtigen.}

Die Vertiefungen und Erhabenheiten, welche an der entgegen gesetzten ebenen Fläche gegen den Rand zu liegen, zumal an der linken Seite (die Masse von dieser Fläche betrachtet) der oberen Hälfte, gleichen ziemlich jenen der vorigen Fläche, nur sind erstere seichter, minder ausgeschweift in ihrem Umrisse, und haben wenigere und breitere Eindrücke, oder gleichen vielmehr selbst bloß aneinander stoßenden größeren Eindrücken, und die zwischen ihnen liegenden Erhabenheiten sind auch nur wenig erhaben und rippenartig, und verflachen sich mehr nach Art jener, welche einzelne seichte Eindrücke zu begrenzen pflegen. Die drei großen ausgezeichneten Vertiefungen aber, welche in und gegen die Mitte, zumal der untern Hälfte, dieser Fläche liegen, unterscheiden sich sehr von allen übrigen, und zwar nicht nur durch ihre Größe, indem die größte über 4 Zoll im Durchmesser misst, und durch ihre geringe Tiefe, indem eben diese Vertiefung an der tiefsten Stelle kaum 6 Linien unter die horizontale Ebene der Fläche reicht, sondern vorzüglich dadurch, dass sie keinen runden, sondern einen unregelmäßigen, obgleich wenig ausgeschweiften Umriss, und sehr seichte, kaum merkliche, aber große und breit verlaufende, gleichsam in einander fließende Eindrücke haben, und dass sie, einzelne Stellen ausgenommen, wo sie an tiefere Randeindrücke grenzen, von keinen rippenartigen Erhabenheiten begrenzt sind, sondern schief aufsteigend, allmählich in die ziemlich horizontalen Ebenen, die zwischen und an ihnen liegen, und die an den meisten Stellen selbst etwas weniges ausgehöhlt sind, übergehen.\footnote{Die auffallende Verschiedenheit dieser Fläche von der entgegen gesetzten, welche offenbar zeigt, dass auch solche Massen während ihres Niederfallens noch eine wesentliche --- sei es auch nur eine oberflächliche --- Veränderung erleiden, wovon bei den Meteor-Steinen, wenn sie auch noch so kleine Bruchstücke der zerplatzten Feuerkugel sind, die um und um sie umgebende Rinde den Beweis liefert, wäre hier schlechterdings nicht zu erklären, zumal sie nur einen Teil, wenn gleich den größeren, derselben betrifft, wenn man nicht annähme, was auch höchst wahrscheinlich ist, dass diese Fläche, oder vielmehr bloß jener Teil derselben, erst später gebildet worden, und zwar durch Lostrennung jenes zweiten zugleich herabgefallenen kleineren Stückes, während dem Niederfallen, entstanden sei. Da jedoch dieses Stück nur 16 Pfund, demnach kaum den vierten Teil dieser vorhandenen Masse, gewogen haben soll, jener Teil dieser Fläche derselben aber, welchen sie nach obigem bedeckt haben müsste, eine Ausdehnung von 10 bis 12 Zoll an Länge und 4 bis 7 Zoll an Breite hat; so müsste jenes Stück sehr flach, und kaum einen Zou dick gewesen sein. Es heißt nun zwar in der Urkunde, dasselbe sei viel kleiner als die eingesendete Hauptmasse gewesen, doch wird auch darin erwähnt, dass dasselbe eine bei 2 Ellen weite, also eine selbst noch größere Spalte als jene, in die Erde gemacht habe, folglich wenigstens nach einer Richtung eine beträchtliche Ausdehnung gehabt haben müsse; auch erhellet aus der Urkunde, dass dasselbe zerstückelt worden sei, indem die Untersuchungs-Kommission nur einen Teil davon erhielt: das Stück muss demnach wirklich sehr dünn gewesen sein, sonst wäre eine Zerstückelung oder auch nur Teilung desselben, bei der bekannten außerordentlichen Zähigkeit solcher Massen, nicht leicht möglich gewesen. Dass aber außer diesem Stücke noch mehrere sich losgetrennt haben und unbeobachtet niedergefallen sein sollten, ist wohl nicht wahrscheinlich, da so viele Augenzeugen auf dem Platze waren, die Feuerkugel im Zerplatzen, und die beiden Stücke, in welche sie sich teilte, im Niederfallen gesehen, und das eine Stück selbst auf 2000 Schritt Entfernung (und eine noch größere Entfernung vom Punkte des Niederfalls der Hauptmasse, folglich eine noch mehr parabolische Richtung in Falle eines Stückes ist, bei der geringen Höhe, auf der die Feuerkugel, wenigstens im Momente des Zerplatzens, gestanden zu haben scheint, und sei der allenfalls voraussetzbaren Projektions-Kraft gegen die Zentripetal-Kraft eines Körpers von solchem spezifischen Gewichte, nicht wohl denkbar) sogleich aufgefunden wurde. Dass aber vollends die Hauptmasse im Momente des Auffallens auf festen Grund ihren Umfang verändert, sich abgeplattet, und demnach in die Breite und Lange ausgedehnt haben sollte, so dass jenes Stuck vor seiner ursprünglichen Lostrennung einen weit kleineren Fleck zu bedecken gehabt hätte, und folglich beträchtlich kleiner gewesen sein dürfte; diesem, an und für sich, widerspricht nicht nur bei gegenwärtigem individuellen Falle die ganze Beschaffenheit der Masse in allen Beziehungen, sondern es streitet überhaupt eine Menge von Gründen gegen die, einer solchen Annahme zur Basis dienende, Voraussetzung, und, wie es scheint, ziemlich allgemein angenommene Meinung, als kämen die Meteor-Massen jeder Art, Metalle wie Steine, in einem solchen Grade von Weichheit, ja selbst von Flüssigkeit zur Erde, wenn gleich andererseits nicht in Abrede gestellt werden kann, dass, wenigstens bei letzteren, die Gemengeteile, und vielleicht selbst die entfernteren Bestandteile, sich vor und in den Momente des Niederfallens in einem ganz andern Kohäsions-Zustande befinden müssen, als die Steine im Ganzen kurze Zeit nachher erkennen lassen.\\
Um obige Mutmaßung vollkommen zu bewähren, müsste das in Frage stehende Stück ebenso kontrastierende, und den Flächen unserer Hauptmasse respektiv entsprechende Oberflächen zeigen; und es wäre demnach sehr zu wünschen, dass wenigstens ein Teil davon noch aufgefunden werden möchte, zu welchen Ende neuerdings Nachforschungen eingeleitet worden sind.}

Auf der konvexen Fläche zeigen sich nebst ein paar zarten, engen, wahrhaften Nissen oder Sprüngen, welche sich über Erhabenheiten und Vertiefungen eine bedeutende Strecke von mehreren Zollen fortsetzen, teils gerade, teils gebogen und gezackt verlaufen, und, soweit es sich durch ein hinein gestecktes Blatt Papier messen lässt, wenigstens einige Linien tief sind, --- merkwürdige Einschnitte, die das Ansehen haben, als wären sie absichtlich durch ein meißel- oder hakenartiges Werkzeug hervor gebracht worden, aber keineswegs daher rühren können, da sie viel zu häufig und regelmäßig vorkommen, keinen der Schärfe eines schneidenden Instrumentes, sondern vielmehr einen der Beschaffenheit der Oberfläche der Eindrücke entsprechenden Grund, und abgerundete, der Beschaffenheit der Oberfläche der Erhabenheiten entsprechende, klaffende Ränder erkennen lassen. Es zeigen sich diese Einschnitte vorzüglich queer über dem Rücken, seltener nach der Länge der rippenartigen Erhabenheiten, nur wenige finden sich in der Tiefe der seichteren Eindrücke, und setzen sich über diese, wenn sie von Erhabenheiten kommen, auch nicht weit und seichter fort. Nur wenige haben die Länge von 1 bis 2 Zollen, die meisten nur von 3 bis 6 Linien, bei einer ziemlich gleichförmigen Tiefe von etwa 2/12 bis 3/12 Linie, und einer ähnlichen Weite, die nach der Tiefe, und meistens nach den beiden Enden hin, etwas abnimmt. Wenn man mit einem feinen spitzigen Instrumente die Tiefe verfolgt, so gelangt man nach Hinwegschaffung des den Naum ausfüllenden, gelben ockrigen Pulvers, oft erst in einer Tiefe von einer halben, ja beinahe ganzen Linie, auf den Grund, der entweder von dem Rindenhäutchen der Oberfläche bedeckt ist, das sich in jedem Falle an den Wänden bis hinab zieht, oder bisweilen Farbe und Glanz des Metalle zeigt. Sie laufen meistens schnurgerade, nur äußerst wenige sind nach den Krümmungen der Erhabenheiten etwas gebogen, und stehen meist einzeln, weit voneinander entfernt, inzwischen doch auch bisweilen paarweise genähert, sich parallel oder in einen Winkel zusammenstoßend. Höchst merkwürdig ist, dass diese Einschnitte, so abgebrochen in ihrem Verlaufe und so zerstreut auf der Oberfläche der Masse sie auch erscheine, doch beinahe ohne Ausnahme in drei bestimmten, leicht erkennbaren Richtungen streichen, und einen Parallelismus und eine Winkeldurchkreuzung zeigen, welche, insoweit sie bestimmt und verglichen werden können, dem kristallinischen Gefüge der Masse, von dem in der Folge die Rede sein wird, zu entsprechen, oder wenigstens mit demselben in einigem Zusammenhange zu stehen scheinen.

Die Masse hat im Ganzen eine schwärzlich-braune Farbe, welche kaum im geringsten die metallische Natur derselben verrät. Alle Vertiefungen und Eindrücke, sowohl die auf der konvexen als auch die gegen den Rand der ebenen Fläche und auf allen Rändern liegenden, so wie auch die Einschnitte auf der konvexen Fläche, sind matt, und ihre Farbe zieht sich aus dem schwärzlich-braunen ins graue, erd- und rostbräunlich-gelbe, hie und da ins roströtlich-gelbe; die großen Vertiefungen aber mit ihren Eindrücken, und die zwischen und an ihnen liegenden flachen, nur etwas ausgehöhlten Stellen auf der ebenen Fläche, haben eine sehr matte, rostbraune Farbe. Alle Erhabenheiten dagegen, welche diese Vertiefungen und Eindrücke an beiden Flächen und an allen Rändern begrenzen, die klaffenden Ränder der Einschnitte an der konvexen Fläche, dann der äußerste Rand, mit welchem die großen Vertiefungen auf der ebenen Fläche in die angrenzenden flacheren Stellen übergehen, endlich der ganze linke Seitenrand dieser Fläche mit der oberen Spitze, welcher gleichsam gegen die konvexe Fläche hinüber gedrückt erscheint, haben eine fettige, etwas wachsähnliche, glänzende, bräunlich-schwarze Farbe, welche an den äußersten Kanten, zumal auf dem Rücken der rippen- und zapfenartigen Erhabenheiten, hie und da in eine rein metallisch eisengraue gleichsam übergeht. Der Art Stellen von rein metallischem Ansehen und Glanze, deren Farbe aus dem eisengrauen bald mehr ins Zink-, bald mehr ins Silberweiße sich zieht, finden sich von verschiedener Größe und Ausdehnung, doch meistens nur sehr klein, hin und wieder auch selbst in den Eindrücken der konvexen Fläche, am meisten aber und von ausgezeichnet silberweißer Farbe am übergebogenen untern Seitenrande der ebenen Fläche, der übrigens mit äußerst dünner, glatter, schwarzbrauner Rinde bedeckt ist. Die ganze Oberfläche der Masse, jene der großen Vertiefungen und der angrenzenden flachen Stellen der ebenen Fläche ausgenommen, erscheint dem bloßen Auge beinahe glatt, bei näherer Betrachtung mit einer Lupe aber erscheint sie, und zwar in allen Vertiefungen und Eindrücken, äußerst fein gekörnt, chagrinartig rau; an allen, dunkleren und glänzenderen, Erhabenheiten und Stellen dagegen mehr glatt und nur zart aderig, metallische Ramifikationen bildend, die sich ziemlich weitschichtig, und meistens von dem Rücken der Erhabenheiten über die Verflachung zu beiden Seiten gegen die Eindrücke, welche sie begrenzen, hin verlaufen; an den rein metallischen, glänzenden Stellen erscheint sie aber vollkommen glatt und spiegelich. Betrachtet man diese letzteren Stellen genauer, so ersieht man bald, dass sie von einer äußerst zarten Decke oder Rinde entblößt sind, welche wie ein dünnes Oberhäutchen die ganze Masse umkleidet, sich über alle Vertiefungen und Erhabenheiten ziemlich gleichförmig ausdehnt, und jenes geaderte oder chagrinartig raue Ansehen der übrigen Oberfläche hervor bringt, und die hier an diesen Stellen, wie ihre Ausrandung zeigt, welche einen offenbaren gewaltsamen Bruch, bisweilen aber auch eine natürliche Begrenzung erkennen lässt, entweder zufällig oder absichtlich abgerieben oder abgeschlagen, bisweilen aber auch in ihrer ursprünglichen Bildung unterbrochen worden ist.\footnote{Unverkennbar ist die Ähnlichkeit dieser Rinde, und überhaupt der Beschaffenheit der Oberfläche in dieser Beziehung, mit jener Meteor-Steine, zumal aus der Suite der stark eisenhaltigen, wie z. B. der Steine von Eichstädt, Timochin, Tabor, Barbotan, L'Aigle \emph{zc.}, und in gewisser Beziehung der von Chantonnay, Erxleben und Ensisheim.}

So wenig auffallend jene verschiedenartige Beschaffenheit der Oberfläche, und insbesondere ihre Rauigkeit, und die Existenz dieser zarten Rinde, an der konvexen Fläche sowohl, als auch an den, in den übrigen Beziehungen derselben entsprechenden und gleichartigen Stellen der entgegen gesetzten ebenen Fläche erscheinen, um so auffallender und in die Augen springender zeigen sie sich hier auf jenem Teile dieser Fläche, der auch in den übrigen Rücksichten so wesentlich von der Beschaffenheit der ganzen übrigen Oberfläche abweicht, und hier erscheint alles gleichsam nach einem vergrößerten Maßstabe.

Die körnig-raue Oberfläche der drei großen, und selbst einiger an dieselben grenzender kleinerer Vertiefungen zum Teil, so wie auch der zwischen jenen und an und um dieselben liegenden ebeneren Stellen, spricht sich hier dem unbewaffneten Auge, so wie dem Gefühle, sehr deutlich aus, und ebenso auffallend erscheinen die glatten, rein metallischen, eisengrauen, nur durch neu entstandenen Eisenrost hie und da matt angeflogenen Flecken von beträchtlichem Umfange, die sich vorzüglich auf den ebeneren Stellen finden, und welche die ursprüngliche Bedeckung durch eine ähnliche körnig-raue (hier ganz unverkennbare, meist zufällig, und wohl noch mehr absichtlich abgeschlagene, oder vielmehr abgeschälte) Rinde umso deutlicher erkennen machen, da sie an allen diesen Stellen durchgehend von ansehnlicher Dicke ist, die selten weniger als eine halbe Linie, gewöhnlich 3/4 Linien beträgt.

Schon mit freiem Auge kann man hier erkennen, dass die Rauigkeit dieser Rinde durch kleine und äußerst kleine rundliche Erhabenheiten oder Wärzchen hervorgebracht wird, welche unordentlich dicht aneinander gehäuft, bisweilen in kurze Schnüre einzeln aneinander gereihet, oder hie und da zu feinen Adern, und, wie wohl selten, zu größeren Tropfen oder Flecken von verschiedener Form zusammengeflossen sind. Mit der Lupe betrachtet, erscheinen diese Erhabenheiten als einzelne, gleichsam aus der Masse ausgeschwitzte, aufsitzende Tröpfchen mit konvexer, etwas rauer, gewisser Maßen geträufter Oberfläche, von schwarzer Farbe und pechartigem Glanze, die an einander gereiheten oder mehr oder weniger zu Adern zusammen geflossenen aber etwas abgeplattet, und die Zwischenräume sind mit einem erd- oder ockerbräunlichen Zement ausgefüllt, welches, da diese sowohl an sich als zusammen genommen mehr Raum ausfüllen als jene Tröpfchen, eine solche raue Oberfläche im Ganzen rostbraun erscheinen machen.

Ein mittelst eines Meißels losgetrenntes Blättchen solcher Rinde, das sich, wenn die Kontinuität einmal unterbrochen ist, an solchen Stellen sehr leicht von der glatten, selbst spiegelichen Oberfläche der Masse abschälen lässt, zeigt an den Bruchstellen gar keine schlackige oder poröse Beschaffenheit, sondern vielmehr, und zwar an den Bruchrändern, eine zart- und ziemlich geradfaserige Textur nach der Dicke des Blattes. Die Fasern scheinen durch ein ähnliches ockerartiges, bräunlich und rötlich-gelbes Zement verbunden, oder vielmehr selbst (durch Einwirkung der Luft, welche zwischen die Rauigkeiten der Oberfläche eingedrungen sein konnte) in eine solche ockerartige Substanz verwandelt worden zu sein, und gehen, in zarte Bündel zusammen gehäuft, kolbenförmig und meistens etwas nach einer Richtung gebogen, in jene schwarze Tröpfchen an der Oberfläche über. Ein von den Randerhebungen jener rauen Vertiefungen und Stellen, und von dem Rücken der sie begrenzenden Erhabenheiten (wo, wie bereits oben erwähnt worden, die Rinde immer dünner, obgleich hier nie so papier- oder vielmehr schneidig dünn, wie an jenen der konvexen Fläche, schwärzer, etwas glänzend und beinahe glatt, auch dichter und fester erscheint) auf gleiche Art abgenommenes Rindenblättchen (was jedoch wegen der geringen Dicke und des stärkeren Zusammenhanges mit der Oberfläche hier schwerer und nur in kleinen Fragmenten bewirkt werden kann), zeigt an den Bruchrändern eine ähnliche, aber etwas zartere und mehr eine körnig-faserige Textur, und eine beinahe Zinkweiße Farbe mit starkem, rein metallischem Glanze (wahrscheinlich weil hier wegen Dichtheit der Rinde an der Oberfläche die Luft nicht einwirken konnte), und die Fasern gehen unmittelbar in die eigentliche schwarze Rinde, die hier keine Wärzchen mehr erkennen lässt, indem diese bereits in ein Häutchen zusammengeflossen zu sein scheinen, über. An einigen Stellen, zumal an solchen, wo die raue Rinde der Vertiefungen in die glatte der Erhebungen übergeht, und hier noch eine beträchtliche Dicke hat, erscheint sie an ihren Bruchrändern gewisser Maßen stratifiziert, und zwar in drei, obgleich nur sehr schwach angedeuteten, ziemlich gleich dicken, horizontalen Schichten, wovon die oberste die eigentliche Rinde, und die unterste das Blatt, welches unmittelbar auf der Masse auflag, bildet, und welche beide in die mittlere, etwas dickere, ohne merkliche Unterbrechung der Richtung der Fasern, übergehen, nur durch eine äußerst zarte horizontale Linie von derselben getrennt scheinen, und sich bloß durch etwas schwächeren metallischen Glanz und etwas veränderte Farbe von ihr unterscheiden, indem erstere mit schwarzen Rindeteilchen, die andere mehr oder weniger mit gelben oder braunen Ocherteilchen gemengt ist. Hie und da, sowohl an dem einen als dem andern Rindenblättchen, erscheinen die Fasern an den Bruchrändern bisweilen Speis- oder Messinggelb angelaufen, und diese Stellen zeigen gleichsam den Übergang vom metallischen Zustande derselben in den ockrigen, nach dem verschiedenen Grade der Einwirkung der Luft; wie sie sich denn auch meistens dort finden, wo die raue Rinde in die glatte übergeht, folglich der Luft ein geringerer Zutritt gestattet wurde. Die Fläche, mit welcher diese Rindenblättchen überhaupt auf der Oberfläche der Masse aufsitzen, ist ganz dicht und ziemlich glatt, nur etwas unregelmäßig streifig, von mattem, metallischem Ansehen, und eisengrau, mit einem schwachen oberflächlichen Farbenanfluge von Blau, Roth und Messinggelb, wie Eisen, das längere Zeit an der Luft gelegen hat. Nur da, wo ein solches Blättchen sehr dünn eine erhabene Stelle bedeckte, und sehr dicht und fest aufsaß, erschien jene Fläche mehr oder weniger Zinkweiß und metallisch glänzend.\footnote{So sehr das ganze äußere Ansehen dieser, so wie aller ähnlichen Massen meteorischen Ursprunges (selbst der Meteor-Steine), und insbesondere das kristallinische Gefüge des Eisens, aus dem sie bestehen, oder das sie enthalten, unwiderlegbar einen ursprünglich flüssigen Zustand derselben voraus setzen; so widersprechen doch eben dieselben, insbesondere aber die dem Meteor-Eisen ganz eigentümlichen, und von jenen der, durch die bekannten Schmelz-Prozesse erhaltenen Produkte der Kunst, so sehr abweichenden physischen Eigenschaften (der hohe Grad von Dehnbarkeit und Zähigkeit, der sich bei großer Hitze verliert, indem bekanntlich alles Meteor-Eisen gerade dann erst brüchig wird), und vor Allem der Umstand, dass diese Massen von mechanisch eingemengtem, ganz unveränderten Schwefeleisen so ganz durchdrungen sind (wie dies von den Meteor-Steinen hinlänglich bekannt ist, von den Eisen-Massen aber bei Gelegenheit der Erklärung der achten und neunten Tafel bemerkt werden wird), der, wie es scheint ziemlich allgemein angenommenen, Meinung, als wäre dieser flüssige Zustand auf dem so genannten trockenen Wege, durch Hitze, hervor gebracht, und das Produkt eines gewöhnlichen Schmelz-Prozesses. Noch mehr aber streiten diese Grunde und manche Erscheinungen beim Niederfalle dieser Massen und Steine, und vorzugsweise bei dieser Eisen-Masse, gegen die beinahe ebenso allgemein gefasste, und selbst von unserm Chladni unterstützte Meinung, als ginge dieser Schmelz-Prozess während des Niederfalls in unsrer Atmosphäre noch fort (oder begönne vielmehr wohl gar in selber), und als kämen sie, und als wäre namentlich diese Masse, in wahrhaft durch Hitze geschmolzenem Zustande, selbst tropfbar flüssig (wie wenigstens unser Güßmann behaupten wollte) bis zur Erde gekommen. Der Umstand, dass hier, laut Urkunde, die Augenzeugen beide Massen, jede in Gestalt einer feurigen, verwickelten Kette (aus welchen Güßmann gegliederte Züge von einer Klafter Länge machte), wollen --- aber doch --- herab fallen gesehen haben (die Güßmann aus der Hohe sich ergießen lasst), mochte wohl einer optischen Täuschung, einem Licht-Phänomene zugeschrieben werden dürfen; und jener, das die Erde, worein sie fielen, rauchte und wie ausgebrannt und grünlich aussah, konnte, wenn ja alles wörtlich und als wahr und richtig bezeichnet angenommen werden soll, wohl mit mehr Grund einer zur Zeit unbekannten Einwirkung der Massen auf dieselbe zugeschrieben werden, als gerade ihrer Hitze, die doch nur die der Erde beigemengten animalischen und vegetabilischen Teile verbrennen und rauchen machen konnte, indes nicht einmal eines bemerkten Geruches erwähnt wird. Die flache, wie hingegossene Gestalt, und die wellenförmigen Unebenheiten aber, welche, nebst obigen Punkten der Urkunde, unser Chladni als ganz deutliche Beweise, dass die Materie in geschmolzenem Zustande zur Erde kam, geltend machen zu müssen meint, scheinen mir gerade dagegen zu sprechen. Die Masse müsste, meines Bedünkens, ungleich flacher und platter, wenn sie hingeflossen, oder mehr oder weniger konisch sein, wenn sie (wie Güßmann will) in die Erde eingegossen worden, und beides ohne alle Verspritzung, was kaum denkbar ist, und doch der Fall war, vorgegangen wäre.\\
Geradezu aber, und besonders in diesem individuellen Falle, spricht gegen eine solche Annahme: dass die Massen so tief in die Erde gedrungen waren, da sie doch von den Seiten her keinen Widerstand fanden sich auszubreiten, und dass dieses Auffallen von solcher Höhe und gewaltsame Eindringen heißer und flüssiger Metall-Massen ohne alle Verspritzung erfolgte; dass ferner an der ganzen einen großen Masse keine Spur sich findet von fest anklebender Erde, oder, was bei einer solchen Voraussetzung wohl der Fall sein müsste, von eingekneteten und eingeschmolzenen Sandteilchen und Steinchen, mit welchen sie doch in Contact gekommen sein muss; dass endlich keine der Massen beim Ausgraben warm befunden wurde, ein Umstand, den man anzumerken gewiss nicht unterlassen hätte, wenn er vorhanden gewesen wäre. Leider wird in der Urkunde nicht bemerkt, wann die Massen eigentlich ausgegraben wurden; aber eben daraus und aus der ganzen Erzählung lässt sich abnehmen, dass es auf der Stelle (bei der kleineren Masse heißt es auch wirklich: sogleich) oder doch in jedem Falle noch an demselben Tage geschah. Nun aber ereignete sich das Factum Abends um 6 Uhr, und da man wohl schwerlich das Eintreten der Nacht wird abgewartet haben; so geschah die Ausgrabung wohl höchst wahrscheinlich innerhalb den ersten zwei Stunden. Eine durch Hitz geschmolzene und im Flüsse sich befindende Eisen-Masse von solchem Volumen würde aber wohl kaum in 24 Stunden soweit ausgekühlt gewesen sein, dass man sie hätte berühren können.}

Noch sind an dieser ebeneren Fläche der Masse zwei Stellen bemerkenswert: die eine befindet sich am oberen Teile an der rechten Seite derselben am aufsteigenden Rande der größten Vertiefung, in Gestalt einer Aushöhlung oder Grube von rundlichtem Umrisse, und 6 bis 7 Linien Weite nach Außen, welche gleichsam durch einen starken, von oben und von der Seite her nach Innen und gegen die Vertiefung wirkenden Eindruck hervor gebracht worden zu sein scheint, indem die eine Wand sehr schief und mit sanft verlaufendem Rande einwärts läuft, die entgegen gesetzte aber etwas schief aufwärts steigt, und gegen die Vertiefung hin einen aufgeworfenen abgerundeten Rand bildet. Diese Grube verenget sich etwas gegen ihren Grund, welcher einen ovalen Umriss von 4 1/2 Linie Länge zu 2 3/4 Linie Breite hat, und geht in eine Tiefe von 2 bis 3 Linien, die aber kaum unter das Niveau der tiefsten Stelle jener großen Vertiefung reicht. Die Seitenwände dieser Grube haben ein raues, ockriges Ansehen, den Grund aber schließt eine glatte, nur etwas porös scheinende, matt metallisch glänzende, eisengraue Ebene von graphitähnlichem Ansehen, welche in der Mitte etwas verbrochen ist, und hier wieder eine ockrige Beschaffenheit zeigt.

Die zweite Stelle befindet sich ganz am untern Rande der Masse, wo ein Stück einst gewaltsam und absichtlich abgebrochen worden zu sein scheint. Es zeigt sich hier ein rauer, etwas hakiger, zerklüfteter, und durch Rost verunstalteter Bruch; an einer kleinen Stelle daselbst aber ein deutlicher, wenigstens zweifacher Durchgang von Blättern von beträchtlicher Dicke, metallischem Ansehen und Glanze, und lichtstahlgrauer, ins silberweiße fallender Farbe.

Auf der konvexen Fläche sind zwei, gegen die obere Ecke der Masse isoliert stehende Erhabenheiten, auf 1/4 bis 1/2 Zoll Tiefe, und auf der ebenen Fläche ist ein Stück von beträchtlicher Ausdehnung (bei 5 Zoll lang, 1 bis 2 1/4 Zoll breit und bei 3/4 Zoll dick) von der Oberfläche der Masse am Rande der abgerundeten Ecke der rechten Seite, zum Behufe technischer und analytischer Versuche, abgesägt worden, wo nun das Innere der Masse zu Tage liegt. Die solcher Gestalt erhaltenen Abschnittsflächen zeigten roh eine dichte, derbe Masse von metallischem Glanze, und lichtstahlgrauer, ins silberweiße fallender Farbe, deren Dichtheit und Gleichförmigkeit im Gefüge nur hie und da durch zarte gezackte Risse und kleine Klüfte, und noch mehr durch häufig und zerstreut eingemengte, meist mikroskopisch kleine körnige Partikelchen von metallischem Ansehen, stärkerem Glanze und weißerer Farbe, --- welche, mit mehr und weniger ockriger Substanz verbunden, zum Teil auch jene Risse und Klüfte erfüllen, --- unterbrochen erschien. Eine der kleinen Abschnittsstellen auf der konvexen Fläche der Masse, welche mit dem Gerbstahl poliert wurde, zeigt eine spiegelnde Oberfläche von beinahe silberweißer, ins stahlgraue fallender Farbe, oder vielmehr einer Farbe, welche jener des polierten Platins sehr ähnelt. Die beiden andern durch jene Abschnitte erhaltenen Flächen wurden mit Salpetersäure geätzt, um das merkwürdige kristallinische Gefüge darzustellen, das sich bei dieser Behandlung am deutlichsten ausspricht, und wovon bei der Erklärung der darauf Bezug habenden Tafeln insbesondere die Rede sein wird.

Obgleich von diesen, im Vergleich zur Dicke der Masse, nur oberflächlichen Stellen nicht geradezu auf eine durchaus gleiche Beschaffenheit im Innern geschlossen werden kann, welches überzeugend zu machen ohne wesentliche Beeinträchtigung der, gerade im ganzen Zusammenhange, so merkwürdigen Form und Beschaffenheit dieser Masse nicht geschehen konnte; so berechtiget doch zu dieser Annahme einerseits die Übereinstimmung des absoluten Gewichtes mit dem Volumen derselben, nach dem bekannten spezifischen Gewichte, andererseits die bereits gemachte Erfahrung bei ähnlichen Massen, wenn gleich nicht faktisch erwiesenen, doch unbezweifelbar gleichen meteorischen Ursprunges (den Elbogner und Lénartoer Gediegeneisen-Massen) welche teils, beinahe durch ihre Mitte, teils selbst nach mehrfachen Richtungen durchschnitten wurden, und durchaus eine, im Wesentlichen, gleichförmige Beschaffenheit zeigten.\footnote{Man wird die Umständlichkeit in der Beschreibung dieser, an sich sowohl als ihrer vielseitigen Beziehungen wegen, höchst merkwürdigen Masse, dem Bestreben zu Gute halten, jedem entfernten Forscher, der sie nie, vielleicht keine ähnliche je zu Gesicht bekommen dürfte, die möglichst vollkommenste anschauliche Kenntnis (wozu die bildliche Darstellung, der Unvollkommenheit der Kunst und der Beschaffenheit des Gegenstandes wegen, leider nur wenig beitragen konnte) von derselben zu verschaffen, und ihn in den Stand zu setzen, über so manch Rätselhaftes und Paradoxes, das uns die Erklärung des Ursprungs und der Bildung meteorischer Massen, und der meisten ihr Erscheinen und Niederfallen begleitenden Umstände, so schwer, ja unmöglich zu machen scheint, und worüber, vorzüglich was die Eisen-Massen betrifft, diese als Prototyp und als zur Zeit einzige, erwiesener Massen, meteorischen Ursprungs, einiges Licht geben kann, eine Mutmaßung fassen, oder wenigstens die zum Teil ziemlich widersprechenden Folgerungen und Behauptungen, zu welchen die mehr oder weniger genaue, richtige und unbefangene authoptische Betrachtung und Beurteilung derselben bei Andern bereits Veranlassung gegeben hat, und ohne Zweifel in der Folge noch geben wird, prüfen und würdigen zu können. Eine erschöpfende Genauigkeit bei Beschreibung dieser Masse schien mir umso notwendiger, als eine vor der Hand sehr unbedeutend und ganz unwesentlich scheinende Kleinigkeit in der Folge bei Auffassung oder Beurteilung, Verteidigung oder Widerlegung einer Ansicht oder Erklärung, oft wichtig und entscheidend sein kann, die schwierige Behandlung eines so massiven Klotzes aber eine oftmalige Wiederholung ähnlicher Betrachtungen, vernachlässigter Nachforschungen wegen, nicht wohl gestattet. Dagegen glaubte ich die Bekanntmachung der Resultate der analytisch-chemischen und physisch-technischen Untersuchungen für eine künftige Veranlassung versparen zu sollen, da dieser Gegenstand eigentlich nicht zum Zweck der gegenwärtigen gehört, und eine Ausarbeitung voraussetzt, die nur mangelhaft und unvollkommen hätte zu Stande gebracht werden können, da es an der benötigten Muße gebrach, indem sie nicht nur eine Wiederholung und Erneuerung aller früher (1808) gemachten, peremtorisch abgebrochenen, sondern eine Menge ganz neu anzustellender Versuche, wozu die in dieser Zwischenzeit erhaltenen Materialien Stoff genug lieferten, notwendig gemacht hätte.}

Die bildliche Darstellung zeigt diese merkwürdige Masse, von der konvexen Fläche betrachtet, in natürlicher Größe.
\clearpage
\section{Zweite Tafel.}
\subsection{Tabor.}
\paragraph{}
Einer der größten Steine von dem sehr bekannten und ziemlich ergiebig gewesenen Steinregen,\footnote{Ungeachtet der Ergiebigkeit dieses Steinregens, indem sich derselbe doch über einen Flächenraum von einer halben Stunde in der Länge, und einer Viertelstunde in der Breite erstreckte, und derselbe Beobachter von seinem Standplatze aus, von wo er den einen Stein fallen sah, noch deren vier in das Getreide niederfallen hörte (die folglich in seiner Nähe, und die Steine daher im Durchschnitt überhaupt ziemlich dicht gefallen sein müssen), und viele der Steine groß und von bedeutendem Gewichte waren (von 5 bis 13 Pfund), und obgleich die Begebenheit zu jener Zeit viel Aufsehen erregte, und durch Zeitungs- und wissenschaftliche Nachrichten bekannt gemacht wurde; so scheinen doch gegenwärtig nur wenige Belege mehr davon, und meistens nur in Bruchstücken, nachweisbar vorhanden zu sein. Außer einigen Privaten in Prag, und vielleicht noch an einigen Orten in Böhmen, und Hrn. Chladni, sind meines Wissens nur das Universitäts-Museum in Pesth, die De Drée'sche Sammlung in Paris, und das Mus. britan. in London (welches das von Born beschriebene Stück mit dessen Sammlung durch Grevilles Vermächtnis erhielt), im Besitze von solchen.} der sich am 3. Julius 1753 um 8 Uhr Abends bei Tabor (eigentlich um Strkow, nächst Plan, einem zur Herrschaft Seltsch gehörigen, eine Stunde von Tabor entfernten Dorfe) in Böhmen ereignete, von beinahe 5 Pfund am Gewichte, und welcher im Momente der Begebenheit, vor einem, nach der Hand als Augenzeuge amtlich vernommenen Knechte (Math. Wondruschka) auf 30 Schritte Entfernung niederfiel, und ohne sich merklich zu versenken, bloß die Erde aufwarf, auch sogleich von dem Beobachter aufgehoben und der Ortsobrigkeit übergeben wurde.

Es wurde dieser Stein von dem damaligen, zu jener Zeit in Tabor, der Kreisstadt des Bechiner Kreises, residierenden königl. Böhmischen Kreishauptmanne, Grafen Vinc. v. Wratislaw, gleich nach der Begebenheit, die derselbe aus eigenem Antriebe amtlich und förmlich an Ort und Stelle untersuchte, mit einem umständlichen Berichte an das königl. Böhmische Kammer-Präsidium zu Prag, und von diesem an die k. k. allgemeine Hofkammer nach Wien eingesendet.

Der Stein ist vollkommen ganz, und um und um mit Rinde bedeckt, die nur an einigen kleinen Stellen etwas abgestoßen, und hie und da abgebrochen worden ist.

Es zeichnet sich derselbe besonders durch eine anscheinende Regelmäßigkeit\footnote{Diese Regelmäßigkeit, auf die ich bereits in meinen Aufsätzen in Gilberts Annalen, 1808, aufmerksam gemacht habe, und die nun auch Hr. D. Chladni bewährt und einer Beachtung wert gefunden hat, ist umso merkwürdiger, da hierin eine Übereinstimmung oder doch eine auffallende Annäherung zwischen vielen Steinen, nicht nur von einer und derselben Begebenheit (demnach zwischen Bruchstücken ein und desselben Meteors), sondern auch von, nach Zeit und Ort, sehr verschiedenen Ereignissen, und selbst zwischen solchen Statt findet, die sowohl in ihrem Aggregats- als Kohäsions-Zustande, als sogar im qualitativen und quantitativen Verhältnisse der nächsten und wesentlichsten Bestand- und Gemengteile bedeutend voneinander abweichen (und kaum können dies in diesen Beziehungen irgendwelche mehr als z. B. die Steine von Tabor und von Stannern), und da dieselbe auf einen Grund-Typus hinzudeuten scheint, der jenem sehr nahe kommt, welcher der ähnlichen Bildung (Struktur, Absonderungs-Zerspaltungsform --- Figurierung ---) einiger terrestrischer, der Trapp-Formation angehörigen Fossilien, welchen die Meteor-Steine in mehrfachen Beziehungen überhaupt sehr verwandt sind, zum Grunde liegt.} in seiner Form aus. Er bildet nämlich eine deutliche, nur etwas verschoben und ungleichseitig vierseitige, abgestumpfte niedere Pyramide,\footnote{Da jener Regelmäßigkeit kein Kristallisation-Gesetz zum Grunde liegen kann, und demnach die vorkommenden Flächen und Kanten keineswegs mit wahren Kristallisation-Flächen und Kanten verglichen werden dürfen, wie sie denn auch ihrer zufälligen Beschaffenheit, der Eindrücke und Verdrückungen wegen, wenigstens nicht mit der gehörigen Genauigkeit, weder geometrisch gedeutet, noch goniometrisch bestimmt werden können; so durfte die Darstellung und Beschreibung der Formen auch nur deskriptiv, nach der auffallendsten und am leichtesten zu versinnlichenden Ähnlichkeit mit einer bekannten geometrischen Figur, keineswegs aber kristallologisch genommen werden. Wollte man letzteres, so müsste man die Form dieses Steines als eine verschobene und ungleich vierseitige Säule mit schief aufgesetzter Endfläche betrachten. Bemerkenswert scheint übrigens doch zu sein, dass zwei Seitenkanten an diesem Steine, mit möglichster Genauigkeit an gleichen Punkten gemessen, einen gleichen Winkel von beiläufig 98°, und darin eine Übereinstimmung mit ähnlichen Kanten von drei verschiedenen säulenförmigen Basalten des Kabinettes zeigten, die damit verglichen wurden; so wie sich auch ein ganz ähnlicher Winkel von einer Seitenkante am nächst zu beschreibenden Steine von L'Aigle, und ein ähnlicher am Steine von Lissa fand. Überhaupt messen die Winkel der schärfern Kanten dieses Steines zwischen 75 und 95°, und die der stumpferen zwischen 105 und 125°. Zwei Steine von diesem Ereignisse, welche der um die Geschichte desselben so verdiente, in der Zwischenzeit verstorbene D. Mayer, und einer, welchen Graf Thun in Prag besaß, und welche mir die gefälligen Besitzer einst zur Ansicht einschickten, hatten ebenfalls eine ziemlich regelmäßige Gestalt. Der eine, 10 Loth schwer, war rhomboidal; der andere, der nur 3 Quäntchen wog, bildete eine vollkommene, scharfkantige, nur etwas schiefe, sonst fast gleichseitig dreiseitige Pyramide; und der dritte, von 1 Pfund 10 Loth, einen sehr verschobenen Rhombus, dem in der Folge zu beschreibenden Steine von Lisa sehr ähnlich.} deren Grundfläche 4 1/2 Zoll in Länge und Breite, die obere Endfläche 3 Zoll in beiden Durchmessern, und deren Höhe bei 3 Zoll misst.

Die Grundfläche ist fast ganz eben, und nur an einem Rande, wo die Kante schief und etwas ungleich abgestumpft ist, von der horizontalen Ebene abweichend. Sie zeigt mehrere große länglichte, aber sehr seichte Eindrücke.

Zwei Seitenflächen, welche beinahe senkrecht auf die Grundfläche aufgesetzt sind, und mit derselben stumpfe und etwas abgerundete und geschweifte Grundkanten bilden, sind kleiner als die beiden andern, etwas konvex, haben wenige, kleine, ziemlich seichte Eindrücke, und stoßen in eine sehr abgerundete gemeinschaftliche Seitenkante zusammen.

Die beiden andern größeren Seitenflächen erheben sich unter einem ziemlich spitzen Winkel schief von der Grundfläche, und stoßen in eine ziemlich scharfe gemeinschaftliche Kante zusammen, welche mit den Kanten der Grundfläche eine starke hervorspringende Ecke bildet. Die eine dieser Flächen, die größte von allen, ist sehr gewölbt, und hat nur sehr wenige rundliche seichte Eindrücke; die durch sie mit der Grundfläche gebildete Grundkante ist schief und ungleich abgestumpft, die mit der anstoßenden Seitenfläche gebildete Seitenkante stumpf zugerundet. Die andere oder vierte Seitenfläche ist etwas konkav, sonst flach und eben, und zeigt nur einen großen, aber sehr seichten, sanft verlaufenden, und einen ovalen, starken, tiefen Eindruck, in dessen Grunde ein großes Korn Metall steckt. Die von dieser Fläche mit der Grundfläche gebildete Grundkante ist abgerundet, gegen die eine Ecke hin aber ziemlich scharf, übrigens ungleich, etwas geschweift und eingedrückt im Verlaufe; die mit der anstoßenden Seitenfläche gebildete Seitenkante ist aber, im ganzen etwas gebogenen Verlaufe, ziemlich scharf.

Die obere Endfläche entspricht der Form nach der Grundfläche, nur ist sie kleiner, und der Richtung der Seitenflächen nach, wovon zwei fast senkrecht unter einem Winkel von beinahe 90, zwei aber schief unter etwa 75° von der Grundfläche aufsteigen, aus dem Mittel geschoben. Sie ist übrigens ziemlich stark vertieft, und hat viele, zum Teil große und ziemlich tiefe Eindrücke. Die von den Seitenflächen her mit derselben gebildeten Endkanten sind alle etwas geschweift, verdrückt, gebogen, und unregelmäßig im Verlaufe, aber doch ziemlich scharf, nur die von der konvexen großen Seitenfläche her gebildete, ist stark verdrückt und etwas breit abgerundet.

Die Rinde ist durchaus gleichförmig dieselbe, und so wie sie bei Meteor-Steinen von ähnlicher Beschaffenheit der Masse, bei einem solchen Aggregats-Zustande und einem gleichen qualitativen und quantitativen Verhältnisse der Bestand- und Gemengteile, zumal bei einem ähnlichen bedeutenden Gehalte an Gediegeneisen, durchgehends gefunden wird; nämlich: von schwärzlich-brauner, hie und da, mehr oder weniger, mit eisengrau und ockergelb und bräunlich gemischter Farbe, sehr schwachen, matten, hie und da schimmernden, stellenweise matt metallischem Glanze, und ziemlich glatter, nur hie und da fein und verworren, kurz und runzlicht-aderiger, größten Teils aber klein und platt körniger, narbiger oder warziger Oberfläche, mit ziemlich häufig eingestreuten eisengrauen, metallisch glänzenden Punkten, und größeren oder kleineren Flecken, als den vorragenden und abgeplatteten Spitzen und Zacken des eingemengten Gediegeneisens. Ihre Dicke beträgt 1/12 bis 2/12, selten 3/12 einer Linie. Ihre Härte ist bedeutend, indem sie mit dem Stahle leicht und ziemlich wacker Funken gibt. Sie wirkt an allen Stellen sehr kräftig auf die Magnetnadel, und setzt eine ziemlich empfindliche auf einen halben Zoll Entfernung lebhaft in Bewegung.

Sie gleicht in allen diesen Eigenschaften am meisten jener der Meteor-Steine von Eichstädt, Timochin, Barbotan, L'Aigle, Apt, Charsonville, Berlanguillas, Toulouse \emph{zc.}

An einigen Stellen, namentlich an drei Ecken der Grundfläche, und an einer der oberen Endflächen, und auf zwei Plätzen an den Grundkanten dieses Steines, zeigt sich etwas unvollkommene Rinde, das ist, Rinde, die sich nicht vollkommen ausgebildet hat, keine vollkommen zusammenhangende Kruste bildet, und die Steinmasse nicht ganz bedeckt, sondern nur in Tropfen, oder in, aus solchen zusammen geflossenen Adern oder Flecken dieselbe teilweise deckt.\footnote{Es scheint nicht, dass die unvollkommene Rinde an diesen Stellen der späteren Entstehung derselben, durch Lostrennung oder Absprengung eines Stückes, und folglich dem Mangel des benötigten Zeit-Moments zu ihrer Bildung, welches am gewöhnlichsten wohl der Fall sein dürfte, sondern vielmehr der individuellen Beschaffenheit und dem besonderen Mengungsverhältnisse der Grundmasse an diesen Stellen, welche der Rindenbildung mehr Widerstand leisteten, zuzuschreiben sei, wie denn auch diese Stellen nur sehr klein sind, und keinen Verlust der Masse erkennen lassen. Ich verweise übrigens hinsichtlich dieser Beschaffenheit der Rinde, welche sich mehr oder weniger beinahe auf jedem einzelnen Meteor-Steine findet, wie ich zuerst bemerkt habe, und welche umso merkwürdiger ist, da sie uns am ersten über die höchst rätselhafte, und zur Zeit noch gar nicht befriedigend erklärte Entstehung und Bildung der Rinde an den Meteor-Massen überhaupt Aufschluss geben könnte, auf die Erklärung von Fig. 3 und 4 der sechsten Tafel, und hinsichtlich der mannigfaltigen Beschaffenheit derselben überhaupt auf jene sämtlicher Darstellungen auf der vierten, fünften und sechsten Tafel, und im Allgemeinen auf meinen Aufsatz in Gilberts Annalen B. 31, und bitte damit zu vergleichen, was, hinsichtlich ihrer Entstehung und Bildung, Hr. Professor v. Scherer an denselben Orte, und Hr. D. Chladni in seinem neuesten Werke vorgebracht haben.}

Die Abbildung des Steines, welche die Versinnlichung der auffallend regelmäßigen Form und der Beschaffenheit seiner Oberfläche zum Zwecke hat, ist von einer Ansicht desselben genommen, in welcher sich erstere und ihre Ähnlichkeit mit einer bekannten Figur, insbesondere aber ihre Übereinstimmung mit andern ähnlich gestalteten Meteor-Steinen am deutlichsten ausspricht. Der Stein ist diesem zu Folge auf seiner Grundfläche (ihn als Pyramide betrachtend) liegend, von der einen breiten, konvexen Seitenfläche etwas gewendet vorgestellt, um den ganzen Umriss, eine zweite Seitenfläche mit der verlängerten Kante und der vorspringenden Ecke, und die obere Endfläche ersichtlich zu machen.

\subsection{L'Aigle.}
\paragraph{}
Einer von den größeren Steinen von dem besonders ergiebigen Steinregen,\footnote{Im strengeren Sinne; denn es fielen doch zwischen zwei und drei Tausend Steine auf einen Flächenraum von höchstens 2 französ. Quadrat-Meilen, und zwar auf drei Explosions-Punkte beschränkt, die zusammen wohl kaum den fünften Teil dieses Flächenraums betroffen haben möchten. Das Gesamtgewicht, nach einem ähnlichen Maßstabe, wie bei dem Ereignisse von Stannern, geschätzt, dürfte wohl 30 bis 40, vielleicht 50 Zentner betragen haben, da viele der Steine 3 bis 5, mehrere selbst zwischen 10 und 17 Pfund wogen. Außerdem, dass dieses Ereignis, eben dieser Ergiebigkeit und der günstigen Umstände wegen, --- dass sich dasselbe nämlich in einer so bewohnten und kultivierten Gegend, und bei hellem Tage zutrug, --- nicht nur das meiste Aufsehen in neuester Zeit erregte, und die schlummernde, bisher nur von Zeit zu Zeit durch minder bedeutende Vorfälle ähnlicher Art, und oft aus weiter Ferne her, schwach angeregte Aufmerksamkeit auf diese wunderbaren, und wie sich‘s bei Erwachung dieser bald zeigte (denn noch in demselben Jahre wurden drei ähnliche beobachtet, und eine davon selbst noch innerhalb den Grenzen des alten Frankreichs, --- bei Apt, Departement Vaucluse, Oktober 1803 ---), keineswegs so seltenen Naturerscheinungen, erweckte, sondern auch nicht wenig beitrug, durch eine, auf Veranlassung des National-Instituts in Paris, von einem berühmten Physiker (Biot) an Ort und Stelle vorgenommene legale und wissenschaftliche Untersuchung und Bewährung des Factums, den noch ziemlich allgemein vorherrschenden Unglauben an die Realität solcher Begebenheiten zu verscheuchen; so ist es auch, aus eben diesen Gründen und durch den Spekulations-Geist eines Pariser Mineralien-Händlers (Lambotin), dasjenige, wovon die meisten Belege erhalten wurden und in die Welt kamen.} der sich am 26. April 1803, Nachmittags gegen 1 Uhr, zu L'Aigle (Departement de l'Orne, der ehemaligen Normandie) in Frankreich (etwa 25 franz. Meilen westlich von Paris) ereignete, von beinahe 2 Pfund am Gewicht.

Es ward derselbe noch im Laufe desselben Jahres, in welchem sich die Begebenheit zutrug, in Wien zu Kaufe geboten, und von dem damaligen Direktor, Abbé Stütz, für das k. k. Mineralien-Kabinett angekauft.

Er ist vollkommen ganz und um und um überrindet, nur ist er hie und da an den Kanten etwas abgestoßen, und eine Ecke ist abgebrochen, die sich aber dabei befindet.

Obgleich dieser Stein auf den ersten Anblick sehr unregelmäßig geformt zu sein scheint, die Flächen sehr uneben und ungleich, und die Kanten sehr verdrückt sind; so ist doch bei näherer Betrachtung desselben eine bestimmte, und, wie es scheint, nur zufällig verunstaltete Grundform unverkennbar, und auffallend die Übereinstimmung mit dem vorher beschriebenen Steine von Tabor.

Er bildet nämlich ebenfalls eine verschoben und ungleichseitig vierseitige, abgestumpfte, niedere Pyramide, deren Grundfläche etwas über 3 Zoll, die obere Endfläche 2 1/2 Zoll, in beiden Durchmessern, und deren Höhe beinahe 2 1/2 Zoll misst.

Die Grundfläche ist sehr gewölbt, und ebenfalls durch Abstumpfung einer Kante, die aber hier besonders stark ist, so dass gleichsam eine neue Fläche durch dieselbe gebildet wird, sehr, und umso mehr verunstaltet, als auch die gegen überstehende Kante einiger Maßen abgestumpft und stark verdrückt ist. Übrigens hat diese Fläche nur wenige seichte Eindrücke.

Von den Seitenflächen sind ebenfalls zwei aneinanderstoßende klein, fast senkrecht, etwas konvex, und haben nur wenige breite, seichte Eindrücke. Die beiden andern größeren erheben sich unter einem etwas spitzigen Winkel schiefer, und stoßen in eine ziemlich scharfe gemeinschaftliche Kante zusammen, welche mit den Kanten der Grundfläche ebenfalls eine hervorspringende Ecke bildet. Die eine dieser Flächen ist ebenfalls konvex, und ihr entspricht die abgestumpfte Kante der Grundfläche; die andere ist konkav: gerade wie beides am vorhin beschriebenen Steine von Tabor der Fall ist. Auch diese beiden Flächen haben nur sehr wenige kleine und seichte Eindrücke.

Die obere Endfläche entspricht zwar der Form nach, obgleich sie ziemlich scharf begrenzt ist, nicht der Grundfläche, da diese durch Abstumpfung und Verdrückung der Kanten sehr verunstaltet ist; dagegen vollkommen der gleichnamigen am Steine von Tabor: drei Schenkel des auf ähnliche Art verschobenen ungleichseitigen Vierecks, welches dieselbe bildet, sind nämlich ziemlich gleich, der vierte aber ist viel kürzer; übrigens ist sie kleiner als die Grundfläche, und ebenfalls, durch ungleiche Erhebung der Seitenflächen von der Grundfläche, aus dem Mittel geschoben. Sie ist stark vertieft, und hat viele, meistens ziemlich tiefe, zum Teil zusammen gedrängte, aber kleine Eindrücke.

Auch die oberen Endkanten stimmen an beiden Steinen darin überein, dass die von der konvexen Seitenfläche mit der oberen Endfläche gebildete, die stumpfeste, die von der konkaven die schärfste, die beiden andern etwas abgerundet sind.

Das Winkelmaß der meisten Kanten, insoweit dasselbe einiger Maßen bestimmbar ist, fällt zwischen 80 u. 115°.\footnote{Herr Graf v. Fries allhier besitzt zwei Steine von diesem Ereignisse, wovon der eine, beinahe vollkommen ganze und über 3 Pfund schwere, in seiner Form auffallend mit dem hier beschriebenen übereinstimmt, selbst in dem Umstande, dass zwei Seitenflächen mit einer Ecke verlängert sind; der andere aber von 24 Loth am Gewichte, obgleich unvollkommen, sich doch auch jener Form sehr nähert.}

Die Rinde ist genau und in jeder Beziehung dieselbe, wie bei dem Steine von Tabor, nur im Ganzen etwas glatter, mehr klein und platt narbig als aderig, und etwas lichter braun, mit mehr bräunlichen und gelblichen Ockerflecken, aber fast ohne Spur von Gediegeneisen. Ihre Dicke ist im Ganzen fast noch etwas geringer; an Härte und Wirkung auf den Magnet kommt sie aber genau mit jener am Taborer-Steine überein.

An mehreren kleinen Stellen der Grundkanten, an den Kanten und an einer Ecke der oberen Endfläche, und an der größeren Ecke der Grundfläche, zeigt sich unvollkommene Rinde; aber nur an der letzteren Stelle scheint sie die Folge eines Verlustes an Masse, durch spätere Lostrennung eines Stückes, zu sein.

Die Darstellung dieses Steines hat gleiche Zwecke, wie jene des vorhin beschriebenen Steines von Tabor, demnach sind dabei auch gleiche Rücksichten genommen, und derselbe auf seiner --- angenommenen --- Grundfläche liegend, von der einen breiteren, gewölbten Seitenfläche, etwas gewendet, vorgestellt worden, um den ganzen Umriss, die andere breite konkave Seitenfläche mit der verlängerten Kante und der vorspringenden Ecke, und die obere Endfläche zur Ansicht zu bringen.

\subsection{Eichstädt.}
\paragraph{}
Ein verschoben vierseitig pyramidales Bruchstück, 7 Loth schwer, von dem am 19. Februar 1785, nach 12 Uhr Mittags, bei Eichstädt in Franken, so viel bekannt, einzeln gefallenen Steine von 5 Pfund 22 Loth am Gewicht, welches um das Jahr 1789 von dem Domherrn v. Hompesch zu Eichstädt, dem damaligen Direktors-Adjunkten des k. k. Mineralien-Kabinettes, Abbé Stütz, mitgeteilt wurde, der es daselbst niederlegte.\footnote{Es wurde dieses Stuck, wegen des offenbaren Gehaltes an Gediegeneisen, als des merkwürdigsten Gemengteiles desselben, und mit ihm, aus gleichem Bestimmungsgrunde, der Stein von Tabor (so wie in der Folge der Stein von L'Aigle, und das Bruchstuck vom Mauerkirchner Meteor-Steine), der Agramer Eisen-Masse, und den vorhandenen Stücken vom sibirischen Eisen, beigesellt, und die ganze Suite, bei der eben um jene Zeit vorgenommenen neuen systematischen Einrichtung des Kabinettes, mit der Suite der Magneteisen-Steine vereinigt, in einen Schrank eingereihet.\\
Die Erhaltung dieses Stückes gab zu einem Aufsätze Veranlassung, welchen Abbé Stütz noch in demselben Jahre, 1789, in Form eines Briefes, in das eben angefangene periodische Werk eines von Born und Trebra gestifteten montanistischen Vereines (Bergbaukunde 2. Band, Leipzig 1790) einrücken ließ, und welcher nicht nur die früheste umständlichere Nachricht von diesem Ereignisse, sondern auch die durch dasselbe angeregte und motivierte Bekanntmachung der höchst merkwürdigen Urkunde über die Agramer Eisen-Masse, und zugleich auch eine Mutmaßung über den wahrscheinlichen Ursprung solcher angeblich aus der Luft gefallenen Massen enthält, die den damaligen Ansichten und dem allgemein herrschenden Unglauben --- wenigstens an eine ursprünglich überirdische Entstehung derselben --- entsprechend, und in dieser Voraussetzung gerade bei diesen zwei dem Verfasser näher bekannt gewordenen Vorfallen (Agram nämlich und Eichstädt, als wo nur einzelne Massen fielen) wirklich am annehmbarsten war. Eine Mutmaßung, die übrigens schon 20 Jahre früher von den Pariser Akademikern, mit Lavoisier an ihrer Spitze ausging, und 12 Jahre später noch (1802) von einem bekannten französischen Physiker (Patrin) bei Gelegenheit der Howard'schen Resultate und Folgerungen, und gegen dieselben, verteidiget wurde.  
Bruchstucke von diesem Eichstädter Steine gehören übrigens zu den seltensten und am wenigsten bekannten von allen Meteorolithen neuerer Zeit, indem die Total-Masse so unbedeutend war, und die Begebenheit selbst erst spät allgemeiner bekannt wurde. (Nämlich lange nach Stütz, 1805 erst, gab Prof. Pickel zu Eichstädt Nachricht davon in v. Molls Annalen.) Ein großes Stuck davon befindet sich am Berg-Collegium in München, ein kleines besitzt Herr v. Moll daselbst, und kleine Fragmente finden sich meines Wissens in den durch Vollständigkeit in dieser Partie ausgezeichneten Sammlungen des Marquis De Drée in Paris, und des jüngst verstorbenen L. R. Lavaters in Zürich. Klaproth opferte ein erhaltenes Bruchstück der Analyse, und Chladni suchte vergebens ein Fragment für seine Sammlung aufzutreiben.}

Obgleich dieses Bruchstück, dem Gewichte nach, nur den 26sten Teil des ganzen Steines beträgt, so lässt sich doch aus den noch daran vorhandenen natürlichen, mit Rinde bedeckten Flächen welche ohne Zweifel Seitenflächen waren --- und aus deren Richtung, so wie aus der gemeinschaftlichen Kante, in welche dieselben zusammen stoßen, nicht nur auf eine regelmäßige, sondern selbst auf eine vierseitig pyramidale, und somit den vorhin beschriebenen Steinen von Tabor und L'Aigle sehr ähnliche Form, welche dieser Stein, als ganz, gehabt haben dürfte, mit aller Wahrscheinlichkeit schließen.

Die beiden überrindeten Flächen erheben sich nämlich schief unter einem Winkel von 72° von der angenommenen breiteren, freilich hier gebrochenen, Grundfläche (wie dies bei einer der größeren schiefern Seitenflächen des Taborer Steines wirklich beiläufig auch der Fall ist), und verschmälern sich offenbar nach oben, lassen also keinen Zweifel über die ursprünglich pyramidale Form des Steines.

Sie stoßen ferner unter einem Winkel von 116° beiläufig, in eine gemeinschaftliche Kante zusammen (bemerkenswert, dass am Taborer Steine bei einer stumpf abgerundeten gemeinschaftlichen Kante zweier Seitenflächen ein ähnlicher Winkel von 115° vorkommt); verlängert man sich nun diese beiden Seitenflächen, wovon hier nur ein Teil, und zwar im Mittel, von 12 und 15 Linien vorhanden, nach ihrer offenbaren Richtung bis an ihre höchst wahrscheinliche ursprüngliche Grenze von Ausdehnung in die Breite, d. i. auf etwa 4 Zoll (welche Größe\footnote{Nach Stütz Nachricht, die sich auf eine schriftliche Mitteilung des B. Hompesch gründet, hatte der Stein ungefähr einen halben Schuh im Durchmesser. (Chladni gibt, wahrscheinlich aus einem kleinen Versehen im Niederschreiben, einen Schuh an.) Dieses kann, nach den Gewichtsverhältnissen, nur insofern gegründet sein, als man damit den längsten meinte, etwa von einer Ecke queer zur entgegen gesetzten gemessen, und dann müsste selbst noch, wie oben erwähnt, eine Ecke etwas verlängert gewesen sein, und wenn der Stein wirklich pyramidal war, dessen Höhe kaum mehr als 3 Zolle betragen haben.} der Stein, als Cubus genommen, nach seinem absoluten Gewichte und dem spezifischen = 3,7 beiläufig gehabt haben möchte); so kommt, wenn man kein sehr ungleichseitiges Prisma, oder ganz willkürlich, eine polyedrische Gestalt sich denken will --- wogegen dieser so sehr regelmäßige Teil des Ganzen, und in gewisser Beziehung das angegebene Maß des Steines selbst, streitet -- ein verschobenes Viereck heraus, das höchst wahrscheinlich ungleichseitig war, und eine vorspringende Ecke hatte, weil sonst- nach obigen Gewichtsverhältnissen -- beinahe bei keiner andern denkbaren Form des Steines, mit welcher sich die Gestalt dieses Bruchstücks vereinigen ließe, ein Durchmesser von 6 Zoll (wie doch ausdrücklich angegeben wird) sich ergeben könnte.

Die beiden überrindeten Flächenreste sind übrigens fast ganz flach und eben, besonders die eine; die andere hat nur ein paar etwas seichte Stellen, die man kaum Eindrücke nennen kann.

Die Rinde ist im Ganzen wie an den Steinen von Tabor und L'Aigle, nur etwas dunkler schwarzbraun, und mehr kurzaderig-runzlich als narbig, und am ähnlichsten jener an den Steinen von Timochin und Tipperary. Sie ist merklich dicker als an irgendeinem mir bekannten Meteor-Steine (auch hierin kommt, wenigstens stellenweise, die an den Steinen von Timochin und Tipperary ihr am nächsten), zumal an einer dieser Flächen, wo sie beinahe eine halbe Linie erreicht.\footnote{Stütz gibt aus Versehen, weil er wahrscheinlich vergaß die Betrachtung mit einer Handlupe, die wohl drei bis vier Mahl vergrößert haben mag, angestellt zu haben, die Dicke auf 2 Linien an.}

Ihre Härte ist etwas geringer als die der Rinde der Steine von Tabor und L'Aigle, doch gibt sie ziemlich leicht am Stahle Funken; dagegen wirkt sie merklich stärker auf die Magnetnadel, und setzt dieselbe fan auf 3/4 Zoll Entfernung in Bewegung. (Auch in diesen beiden Eigenschaften steht ihr die Rinde an den Steinen von Timochin und Tipperary am nächsten.) Und sie gibt dadurch nicht allein, sondern auch durch häufige, etwas erhabene eisengraue metallische Punkte und kleine Flecke von abgeplatteten Spitzen und Zacken, den starken Gehalt dieses Steines an Gediegeneisen zu erkennen.\footnote{Es ist dieser Meteor-Stein nicht nur der Gehaltreichste an Gediegeneisen, wie dies auch das spezifische Gewicht bewährt (das nach meiner Wiegung zwischen 3,680 und 3,730 schwankt, und worin ihm nur die Steine von Tipperary nach Higgins, und von Timochin nach Klaproth gleich zu kommen scheinen, und die Steine von Charsonville nach Hauy, und von Tabor nach eigener Wiegung --- denn der Bournon'schen Gewichtsangabe zu 4,28 liegt offenbar ein Versehen oder der Umstand zum Grunde, dass das gewogene Stück zufällig ein großes Eisenkorn einschloss --- nahe kommen), sondern er enthalt dasselbe auch in den größten, massivsten (obgleich immer noch sehr zarten), und hie und da wirklich ästig verbundenen und zusammen hängenden Zacken, wie sich am deutlichsten an einer abgeschliffenen Fläche erkennen lässt. Es bedarf in der Tat wohl kaum mehr eines Zwischengliedes, um den Übergang der Masse dieses Steines in jene des sibirischen Eisens (zumal in die dichteren, weniger zelligen, und mehr erdig-ockrigen Partien desselben, und der angeblich norwegischen und sächsischen Massen im Ganzen) sinnlich nachzuweisen, umso weniger, als in derselben bereits auch der olivinartige Gemengteil (wofür man, nach äußerem Ansehen, Art der Einmengung, nach den physischen Eigenschaften und chemischen Bestandteilen, das mandelsteinartig eingemengte, gleichsam in rundlichte Zellen eingeschlossene und meist von Gediegeneisen umgebene Fossil --- das sich mehr oder weniger und in verschiedenen Graden von Ausbildung, wie bei Erklärung der siebenten Tafel gezeigt werden wird, in allen Meteor-Steinen findet --- zu erkennen nicht anstehen kann) so sehr prädominiert, dass derselbe mit den Metallteilchen gut 2/3 der Gesamtmasse beträgt.} 

Die Darstellung dieses Bruchstückes ist, der Absicht gemäß, und nach den bereits erwähnten Rücksichten, von den beiden mit Rinde bedeckten natürlichen Flächenresten, und von der gemeinschaftlichen Kante, in welche sie zusammenstoßen, genommen.

\subsection{Siena.}
\paragraph{}
Ein Bruchstück, oder vielmehr höchst wahrscheinlich (nach Größe, Form, Richtung und Ausdehnung der vorhandenen, natürlichen, mit Rinde bedeckten Flächen) wenigstens die Hälfte eines (ursprünglich etwa 3 bis 4 Loth schwer gewesenen) mittelgroßen Steines, von 7 Quäntchen am Gewichte, von dem am 16. Junius 1794, Abends nach 7 Uhr, bei Siena im Toskanischen Statt gehabten beträchtlichen Steinniederfalle.\footnote{Es ist dieses einer der Steinniederfalle neuerer Zeit, von welchem die Produkte ziemlich bekannt und verbreitet wurden, obgleich man die Realität der Begebenheit, trotz einer gepflogenen legalen Untersuchung, und die Herkunft und den überirdischen Ursprung der Steine zur Zeit des Ereignisses selbst, sehr bezweifelte. Allein die Begebenheit machte großes Aufsehen, da sie bedeutend war (es fielen einige hundert, aber meist nur kleine, oder doch nur mittelgroße, einige Lothe, auch nur wenige Quäntchen schwere Steine --- nur einzelne wenige wogen 3 bis 7 Pfund --- auf einen Flächenraum von 2 bis 3 italienischen Meilen), und sich bei Tage und vor vielen Augenzeugen ereignete; von angesehenen Gelehrten, Tata, Soldani, Spallanzani, viel darüber geschrieben wurde, und mehrere angesehene und gelehrte Engländer, Thomson, Hamilton, Lord Bristol, sich eben damals in Italien befanden, welche dem Gegenstande, der zu großen Debatten Veranlassung gab, noch mehr Zelebrität im Auslande verschafften. Es finden sich demnach Belege von diesem Ereignisse in vielen Sammlungen, namentlich im Mus. brit. zu London, der De Drée'schen Sammlung zu Paris, und in jenen Chladnis, Lavaters, Blumenbachs, Klaproths \emph{zc.}}

General Tihavsky, der sich eben damals zur Zeit des Ereignisses in Neapel befand, erhielt dieses Stück von dem ebenfalls da anwesenden gelehrten Engländer Thomson, welchem es von Soldani aus Siena zugeschickt wurde, und brachte es bei seiner Rückkehr mit nach Wien; aber erst als der Steinfall bei Stannern die Aufmerksamkeit der Physiker, zumal in Wien, neuerdings und so mächtig in Anspruch nahm, ward es zur Sprache gebracht, und von dem gefälligen Besitzer auf mein Ansuchen dem kaiserlichen Kabinette zum Geschenke gemacht.

Es ist zwar an diesem Steine an zwei Stellen, und zwar, wie es scheint, mit bestimmter Vorsicht, Masse abgeschlagen worden, und die beiden solcher Gestalt entstandenen, ziemlich großen, und unter einem Winkel von 85° zusammen stoßenden frischen Bruchflächen lassen zwar an und für sich ihre ursprüngliche Gestalt, Beschaffenheit, Richtung und Ausdehnung nicht wohl erraten; doch lässt sich aus der Form des vorhandenen Stückes, und den drei mit Rinde bedeckten Seitenflächen, und der noch ganz vollkommenen Endspitze, mit aller Wahrscheinlichkeit darauf schließen, und es scheint nach dieser Ansicht die eine dieser Bruchflächen die vierte größere gewölbte Seitenfläche, die andere die untere End- oder Grundfläche des Steines gewesen zu sein. Und bei dieser Annahme erscheint die ursprüngliche Form dieses Steines nicht nur sehr regelmäßig als verschobene und ungleichseitig vierseitige Pyramide mit durch drei Flächen zugespitzter Endspitze, sondern auffallend übereinstimmend mit jener des auf der vierten Tafel vorgestellten großen Steines von Stannern, umso mehr, als die Grundfläche ebenfalls ein ähnlich verschobenes Viereck mit einer stark vorspringenden Ecke gebildet zu haben scheint, und die Endspitze durch eine ähnliche Richtung und Ausdehnung der Zuspitzungsflächen ebenfalls aus dem Mittel gerückt ist, und durch die zwei breiteren gegen über stehenden Zuspitzungsflächen zu einer Kante gebildet wird.

Die vorhandenen, mit Rinde bedeckten Seitenflächen, stehen ziemlich senkrecht auf der als Grundfläche betrachteten Bruchfläche: die eine, breiteste, ist fast eben; die nächste, kleinste von allen, welche mit voriger unter einem Winkel von etwa 80° jene gemeinschaftliche Kante bildet, auf welche die Zuspitzungsfläche aufgesetzt ist, ist etwas konkav; die dritte, welche unter einem sehr stumpfen Winkel von beinahe 135° mit letzterer zusammenstößt, ist etwas gewölbt. Die eine auf die Kante aufgesetzte Zuspitzungsfläche bildet ein auf eine Ecke gestelltes Rhomboid, ist die kleinste und etwas vertieft; die beiden andern sind breiter und größer, sehr unregelmäßig gestaltet, und, zumal die eine, fast ganz eben. Sie stoßen unter einem Winkel von 90° in die gemeinschaftliche Endkante zusammen. Alle Flächen haben nur wenige, kaum bemerkbare, seichte, kleine Eindrücke.\footnote{Die kaiserl. Sammlung besitzt außer diesem noch zwei vollkommen ganze, obgleich nur sehr kleine Steine von dieser Begebenheit. Der eine, um und um mit vollkommener, und nur an einer Ecke mit unausgebildeter Rinde bedeckte, der nur ein Quäntchen wiegt, zeigt der Form nach, trotz seiner Kleinheit, eine auffallende Ähnlichkeit mit den beschriebenen Steinen von Tabor und von L'Aigle, indem er, selbst hinsichtlich der gewölbten Grundfläche, und der einen stark vorspringenden Ecke, eine ähnliche, verschoben und ungleichseitig vierseitige, abgestumpfte, niedere Pyramide bildet. Der andere, etwas größere, von 2 1/4 Quäntchen am Gewichte, der nur an einem Ende etwas verbrochen ist, und an einer Fläche und an zwei andern kleinen Stellen unvollkommene Rinde zeigt, hat eine Form, die sich jener des nächst zu beschreibenden Steines von Lissa sehr nähert. Die Rinde an diesen beiden Steinen, die vielleicht lange dem Einflüsse der Witterung ausgesetzt waren, zeigt, obgleich sie eben so dünne, zart und rissig ist wie an dem oben beschriebenen, durch das ganz matte Ansehen und eine mehr braune, mit Rostflecken gemengte Farbe, einige Ähnlichkeit mit jener der Steine von L'Aigle.}

Die Rinde ist besonders zart und dünne, beinahe kohlschwarz, etwas ins Graue ziehend, von wenigem und mattem, aber etwas seidenartigen, stellenweise schimmernden Glanze, und von gar keinem Ansehen, das einen Metallgehalt verriete. Sie ist übrigens sehr zart rau, fein und eng, kurz und verworren, runzlicht-aderig, und voll zarter Risse, welche unregelmäßige Felder bilden. Sie hat die meiste Ähnlichkeit einerseits mit der Rinde an den Steinen von Lissa, Agen, York, andererseits mit jener an den Steinen von Parma und Benares, und zeigt überhaupt von dem geringen Metallgehalt der Masse, welchen auch das spezifische Gewicht vermuten lässt (3,3 bis 3,4). Sie gibt am Stahle nur schwer und schwache Funken, und wirkt auch nur schwach auf die Magnetnadel, kaum auf 2/12 Linie Entfernung.

Die Abbildung zeigt diesen Stein auf die eine, als untere End- oder Grundfläche betrachtete Bruchfläche aufgestellt, von der gemeinschaftlichen Kante, in welche die einen zwei mit Rinde bedeckten Flächen zusammenstoßen, und auf welche die eine Zuspitzungsfläche aufgesetzt ist, die mit den beiden andern breiteren, welche schief auf den Seitenflächen aufsitzen, die kantige Endspitze bildet.
\clearpage
\section{Dritte Tafel.}
\subsection{Lissa.}
\paragraph{}
Der größte und einzig ganz und vollkommen erhaltene von den vier bei Lissa (zwischen den Dörfern Strattow und Wustra, 4 bis 5 Meilen O. N. O. von Prag) im Bunzlauer Kreise in Böhmen am 3. September 1808, Nachmittags um halb 4 Uhr, gefallenen und im Falle beobachteten und aufgefundenen Steinen.

Er wiegt 5 Pfund 19 Loth.

Es wurde derselbe von vier Augenzeugen, in deren Nähe er niederfiel, im Auffallen beobachtet, gleich aufgehoben und an das Oberamt zu Lissa abgeliefert, welches, nachdem es am 8. September eine förmliche Untersuchung des Factums vorgenommen, und eine offizielle Anzeige davon an das königl. Kreisamt zu Bunzlau erstattet hatte, denselben bis zu der in Folge des kreisamtlichen Berichtes, von Seite des königl. Böhmischen Landes-Guberniums veranlassten wissenschaftlichen Untersuchung, welche am 17. November Statt fand, aufbewahrte, und dann an die Untersuchungs-Kommission abgab, von welcher derselbe mit den diessfälligen Berichten nach Wien eingesendet wurde.\footnote{Bruchstücke von Steinen, als Belege dieser Begebenheit neuester Zeit, möchten wohl zu den seltensten und am schwersten zu erhaltenden gehören. Denn fürs erste war der Steinfall von sehr geringer Bedeutung, es fielen nämlich nur vier Steine, die zusammen kaum 18 Pfund wogen, und wenn gleich unter den gewöhnlichen tumultuarischen Erscheinungen, doch ohne großes Aufsehen zu erregen, und nur vor wenigen Augenzeugen; so wie denn auch die ganze Begebenheit schwerlich beachtet worden, noch weniger zur öffentlichen Notiz gekommen sein würde, wenn nicht, erst drei Monate früher, und zwar kaum auf 20 Meilen Entfernung, eine ähnliche, der Steinfall bei Stannern, Statt gehabt, oder vielmehr, wenn nicht diese vorausgegangene Begebenheit durch die veranlassten amtlichen Untersuchungen, die selbst zu jener Zeit noch im Gange waren, und sich sogar, einiger Nebenerscheinungen wegen, über die Grenzen Böhmens erstreckten, die Aufmerksamkeit der Lokal-Behörden, und selbst des Landvolks in jener Gegend aufgeregt gehabt hätte. Andererseits wurden die gefallenen Steine nur wenig zerstückelt, und erhielten bald eine fixe Bestimmung. II. k. k. HH. die Erzherzoge Rainer und Johann erhielten große Bruchstücke für Höchstderen Sammlungen, ebenso Se. Excellenz Herr Graf v. Wrbna; kleine Stücke blieben zum Angedenken in Kloster zu Lissa, in den Händen einiger Beamten, und im Besitze des Hrn. D. Reuß von Bilin. Diese möchten, mit den beiden Stücken der kaiserl. Sammlung, allein schon über 10 Pfund am Gewichte betragen. Von dem Reste befinden sich, meines Wissens, kleine Fragmente in den Sammlungen Chladnis, Klaproths und De Drées, und ein Bruchstück von etwa 7 Loth in der Sammlung der mineralogischen Gesellschaft zu Jena.} Dieser Stein ist, bis auf einige kleine Stellen an den schärfern Kanten, wo die Rinde etwas abgestoßen ist, und zwei Ecken, wo ursprünglich ein Stück abgeschlagen worden war, doch so, dass die Form des Steines keineswegs gelitten, und der Verlust der Masse kaum 5 bis 6 Loth betragen haben mag, vollkommen ganz und durchaus mit der gewöhnlichen Rinde bedeckt.

Seine Gestalt ist nicht minder auffallend regelmäßig als jene der beschriebenen Steine von Tabor und von L'Aigle, und noch mehr die Ähnlichkeit, die hierin zwischen allen dreien Statt findet.

Er bildet nämlich ebenfalls eine deutliche, verschoben und ungleichseitig vierseitige, stark abgestumpfte, niedere Pyramide, die nur etwas mehr als an den beiden vorigen in die Breite gezogen ist, so dass die beiden Endflächen ein mehr länglichtes Viereck bilden.

Die größere End- oder Grundfläche\footnote{Den Stein von dieser Ansicht und bei dieser Haltung betrachtet, in welchen sich nämlich dessen Regelmäßigkeit und die Ähnlichkeit mit einer geometrischen Figur am auffallendsten ausspricht und am deutlichsten beschreiben und darstellen lässt.\\
Herr Bergrath Reuß, welchem bei Gelegenheit der wissenschaftlichen Untersuchung des Factums, zu welcher derselbe beauftragt wurde, und bei Ansicht dieses Steines die Regelmäßigkeit der Form desselben nicht entgangen war, ob er gleich durch keine ähnliche Beobachtung aufmerksam gemacht worden zu sein scheint, betrachtete den Stein kristallologisch, folglich in einer andern Haltung, nämlich der Länge nach, die beiden Endflächen als Seitenflächen nehmend, und beschreibt ihn demnach --- kristallographisch (in Gehlens Journal für Chemie, Physik und Mineralogie, B. 8. S. 447, 1809) als eine unregelmäßige fünfseitige Säule (die beiden Abstumpfungflächen der Grundkanten als einzelne Seitenflächen betrachtend), mit sehr ungleichen Seitenflächen, und an welcher eine Endfläche schief angesetzt (eine der schmälern gewolbtern Seitenflächen), die andere mit zwei sehr ungleichen Flächen zugeschärft ist (die, jener gegen über stehende, keineswegs gedoppelte, sondern bloß durch große und tiefe Eindrücke verdrückte und verunstaltete Seitenfläche).} hat über 6 Zoll im längeren, und 4 1/2 Zoll im schmälern Durchmesser, die kleinere oder obere Endfläche 4 1/2 zu 3 Zoll, und die Seitenflächen haben 3 1/2 Zoll Höhe.

Die Grundfläche ist sehr unregelmäßig, und durch viele, zum Teil ziemlich große und tiefe Eindrücke, vorzüglich aber durch starke Abstumpfung der beiden Grundkanten der gegenüberstehenden breiteren Seitenflächen sehr verunstaltet, indem durch diese gewisser Maßen zwei schiefe Flächen gebildet werden, die fast in der Mitte der Grundfläche zusammenstoßen. (Es ist bemerkenswert, dass die stärkere Abstumpfung, gerade wie beim Taborer und L'Aigler Steine, dieselbe breite und gewölbte Seitenfläche trifft; besonders auffallend aber ist übrigens die Ähnlichkeit hinsichtlich der doppelten Abstumpfung und der Gewölbtheit der Grundfläche mit dem letzteren.)

Von den Seitenflächen sind ebenfalls zwei größer und breiter; auch ist die eine davon konvex, und durch viele ziemlich große und tiefe Eindrücke sehr verunstaltet; die andere konkav, mit sehr wenigen kleinen seichten Eindrücken. Diese beiden Flächen, welche in Hinsicht der Beschaffenheit ihrer Oberfläche zweien aneinanderstoßenden am Taborer Steine so ähnlich sind, grenzen hier nicht aneinander, sondern stehen sich gegen über, und sind mehr senkrecht als schief aufgestellt. Die von beiden mit der Grundfläche gebildeten Kanten sind, wie bereits bemerkt, stark schief abgestumpft; die mit der oberen Endfläche gebildeten aber ziemlich scharf. Von den beiden andern Seitenflächen, die etwas schiefer aufsteigen, ist die eine ziemlich gewölbt, hat viele kleine, nicht sehr tiefe Eindrücke, aber eine große und ein paar kleine Vertiefungen, die von einem bruchstückweisen Verluste der Masse (durch spätere Lostrennung oder Absprengung) vor der Rindenbildung herzurühren scheinen, und welche diese Fläche sehr verunstalten; die andere ist mäßig gewölbt, sonst eben, und wenige seichte Eindrücke abgerechnet, besonders glatt. Beide bilden mit der Grundfläche sehr zugerundete, mit der oberen Endfläche dagegen besonders scharfe Kanten. Die gemeinschaftliche Seitenkante, in welche jene letztere ebenere Seitenfläche mit der angrenzenden, konkaven, breiteren Seitenfläche zusammenstoßt, und welche besonders scharf ist (der Winkel = 80-85°), bildet mit den Grundkanten dieser Flächen ebenfalls eine stark hervorspringende Ecke, wie dies bei den Steinen von Tabor und von L'Aigle der Fall ist.

Die obere Endfläche bildet ein ziemlich regelmäßiges, länglichtes, verschobenes Viereck, entspricht ziemlich dem Mittel der Grundfläche, ist aber wegen schiefer Richtung der Seitenflächen beträchtlich kleiner, fast flach, nur etwas konkav, und durch viele aber kleine und sehr seichte Eindrücke uneben gemacht. Sie gleicht jener am Taborer und L'Aigler Steine auch darin, dass drei Schenkel des Vierecks bedeutend größer sind als der vierte; übrigens ist sie länglichter.

Das Winkelmaß schwankt, obgleich es sich wegen starker Ungleichheit, Eindrückung und Verdrückung der Kanten nur an wenigen Stellen approximativ bestimmen lässt, nur zwischen 80 und 110°.\footnote{Ein kleines, 3 Loth schweres Bruchstück eines ursprünglich ebenfalls bei 5 Pfund schwer gewesenen, aber in mehrere Stücke zerschlagenen Steines von diesem Ereignisse, zeigt die Reste von zwei überrindeten Flächen, wovon die eine besonders flach, eben und glatt ist, und, von einer als Basis angenommenen Bruchfläche, unter einem Winkel von etwa 84°, die andere, etwas unebenere, vertieftere, eingedrücktere, und, der Rinde nach, rauere, unter 60° aufsteigt, und welche, unter einem Winkel von beiläufig 65 bis 70°, in eine besonders scharfe gemeinschaftliche Kante zusammen stoßen, die wieder von derselben Basis unter einem Winkel von 50 bis 55° aufsteigt, daher wohl die hervor springende Ecke jenes Steines gebildet hat, der nach diesen Indizien höchst wahrscheinlich eine ähnliche verschoben vierseitige Pyramidal-Form, wie der beschriebene, gehabt haben dürfte.\\
Der Stein im Besitze Sr. k. H. des Erzherzogs Johann, im Johanneo zu Grätz, --- welcher 1 Pfund 7 Loth wiegt, und beinahe vollkommen ganz ist, obgleich er dem ersten Anblicke nach nur ein großes Bruchstück zu sein scheint, indem eine große Fläche nur mit sehr unvollkommener Rinde bedeckt, oder vielmehr gleichsam nur angeflogen ist, --- stellt ein etwas verschobenes vierseitiges Prisma vor; und das Bruchstück in der Sammlung Sr. Excellenz des Hrn. Grafen v. Wrbna, von 22 Loth am Gewicht, lässt wenigstens auf eine ähnliche rhomboidale Form des Steines, von dem es abgeschlagen wurde, schließen.}

Die Rinde hält, dem Aggregats-Zustande und dem quantitativen Verhältnisse der Gemengteile gemäß, nach welchen diese Steine gleichsam ein Verbindungsglied zwischen zwei darin, und folglich dem äußern Ansehen nach ziemlich stark abweichenden Reihen von Meteor-Steinen bilden, das Mittel zwischen jener an den Steinen von Tabor, L'Aigle, Eichstädt \emph{zc.}, und jener der Steine von Siena, Parma, Benares \emph{zc.}, am ähnlichsten ist sie aber der Rinde an den Steinen von York und Agen, mit welchen diese Steine auch in obigen Beziehungen die meiste Ähnlichkeit haben.\footnote{Ich behalte mir vor, bei einer andern Veranlassung über diese Reihenbildung, Ähnlichkeit und Übergänge der verschiedenen Meteor-Steine umständlicher zu sprechen, und verweise inzwischen auf die Erklärung der siebenten Tafel.}

Sie ist nämlich hier, und namentlich an diesem Steine, schwarz, beinahe kohlschwarz, ohne allem metallischockerbraunen Ansehen, im Ganzen zwar mehr matt als glänzend, aber doch stellenweise von einem seidenartigen Schimmer, und, obgleich sehr zart, doch mehr runzlicht als narbigt, oder warzig rau. Obgleich sie beim ersten Anblick in diesen Beziehungen gleichförmig über den ganzen Stein ausgedehnt zu sein scheint; so zeigt doch eine genauere Betrachtung und Vergleichung einige Verschiedenheit. An einer Hälfte dieses Steines, und zwar an der oberen Endfläche, an der breiten konkaven, und der kleineren verunstalteten Seitenfläche (welche Flächen, nach obiger Beschreibung, auch in Betreff der übrigen Beschaffenheit ihrer Oberfläche mit einander übereinstimmen), zeigt sie sich ganz auf die beschriebene Weise; an der Grundfläche dagegen, der breiten konvexen und der andern kleineren, ebenfalls gewölbten Seitenfläche (die ihrer übrigen Beschaffenheit nach wieder mit einander übereinstimmen), erscheint sie mehr braun, mit einem schwachen, etwas ins Kupferrote ziehenden Schimmer, zumal in den Eindrücken, im Ganzen aber matter und glatter, wenigstens weniger aderig; auch scheint sie hier etwas dünner zu sein. Eine kleine, aber kaum beschreibbare Abweichung, zeigt in allen Beziehungen die eine kleinere, am meisten gewölbte und ebenste Seitenfläche, so dass demnach dieser Stein, hinsichtlich seiner Oberfläche, eine dreifache Verschiedenheit, gewisser Maßen drei Seiten, zeigt.\footnote{Von dieser, wie mir deucht, höchst merkwürdigen, und von mir zuerst an den Steinen von Stannern beobachteten Verschiedenheit der Oberflache sowohl, als insbesondere der Rinde an ein und demselben Steine, wird bei Beschreibung der in dieser Beziehung besonders ausgezeichneten ganzen Steine von Stannern, und bei Erklärung der Figuren auf der fünften und sechsten Tafel, die Rede sein. Zeigt sich gleich an diesem Steine von Lissa diese Verschiedenheit, zumal der Rinde, nicht so auffallend (wie es auch bei ihrer Beschaffenheit in Allgemeinen als Folge des Aggregats-Zustandes und des qualitativen und quantitativen Verhältnisses der Gemeng- und Bestandteile, und insbesondere des Eisengehaltes wegen nicht anders sein kann, und noch weniger bei jenen Meteor-Steinen der Fall ist, deren Gehalt an --- Gediegen --- Eisen noch weit beträchtlicher befunden wird); so zeigt sie sich doch, was in anderer Hinsicht nicht minder merkwürdig ist, wie es auch von ganz anderen Ursachen herrührt, um so auffallender zwischen den einzelnen Steinen von dieser Begebenheit. An dem einen kleinen Bruchstucke der Sammlung nämlich ist die Rinde noch weit schwarzer, noch mehr seidenartig schimmernd, zumal an der einen Flache, und, äußerst zart zwar, aber sehr ausgezeichnet, runzlicht-aderig, und überhaupt der Rinde der Steine von Parma und Benares gar sehr ähnlich; dagegen die Steine in Besitze Sr. k. H. des Erzherzogs Johann, und Sr. Excellenz des Hrn. Grafen v. Wrbna, eine Rinde zeigen, die beinahe ganz jener an den Steinen von Tabor, L'Aigle u. s. w. ähnlich, matt, braun und weit glatter ist. Und ebenso der Rinde entsprechend und mit gleicher Annäherung, ist auch die innere Beschaffenheit und das Ansehen der Masse im Bruche an diesen Steinen verschieden. Diese Verschiedenheit, sowohl in Hinsicht der Beschaffenheit der Oberfläche und Rinde, als auch der Masse im Innern, die offenbar von einer Verschiedenheit im Aggregats- und Kohäsions-Zustande, und wenigstens des quantitativen Verhältnisses der Gemengteile abhängt, findet sich übrigens nicht bloß bei den Steinen von dieser Begebenheit, sondern auch bei mehreren andern, namentlich bei jenen von Stannern und Siena, insbesondere auch bei jenen von L'Aigle (wie auch Hr. Chladni bemerkte), und mochte vielleicht bei den meisten gefunden werden, wenn man Gelegenheit hatte, so viele Stein und Bruchstücke von ein und demselben Ereignisse vergleichen zu können, wie es bei diesen der Fall war; und sie findet sich nicht bloß bei verschiedenen einzelnen Steinen desselben Niederfalles, ob sie gleich auch als Bruchstucke einer Hauptmasse, der Feuerkugel, betrachtet werden, sondern bisweilen selbst bei Bruchstucken eines und desselben Steines, so dass sich solche oft unähnlicher sind, wie dies vorzüglich bei obigen Steinen von Lissa und bei manchen von L'Aigle der Fall ist, als Bruchstücke von Steinen von, nach Zeit und Ort, sehr entfernten Begebenheiten.}

Die Dicke der Rinde ist übrigens im Ganzen, wie an den meisten Meteor-Steinen, etwa zwischen 1/12 bis 3/12 Linien. Nur an einzelnen kleinen Stellen, hie und da an den Kanten, zeigt sich eine Spur von unvollkommener, unausgebildeter Rinde, wo die Masse des Steines mehr oder weniger verändert (etwas gebräunt) zu Tage liegt, und es das Ansehen hat, als wenn die flüssige Rindenmasse über diese Stellen sich nicht hätte ausbreiten, nicht zusammenfließen können. In einem kleinen, aber tiefen Eindrucke an einer der Flächen, findet sich eine solche Stelle, wo die Masse des Steines ganz und gar unverändert ist, und den frischesten Bruch zeigt, indes doch der sie begrenzende Rindenrand deutlich erkennen lässt, dass es kein künstlicher Bruch ist.

Ihre Härte ist kaum geringer als die der Rinde an den Steinen von Tabor und L'Aigle; aber auf die Magnetnadel wirkt sie bedeutend schwächer.

Die Abbildung stellt den Stein nach der Ansicht und Haltung, nach welchen die Beschreibung genommen, auf der größeren End- oder Grundfläche liegend vor, so dass, mit dem ganzen Umrisse, die eine ausgezeichnetere, breitere, konkave Seitenfläche, die obere Endfläche, und zum Teil noch die zwei kleinen Seitenflächen, wovon die eine mit der vorderen die verlängerte Kante und vorspringende Ecke bildet, zu ersehen sind.
\clearpage
\section{Vierte Tafel.}
\subsection{Stannern.}
\paragraph{}
Der größte\footnote{Außer einem, von Joseph Wurschy von Neustift, in derselben Gegend, in einem Wäldchen, etwa 2500 Klafter nördlich von der Kirche von Stannern, gefundenen Steine (Nr. 61 des Planes), welcher 13 Pfund gewogen haben soll, aber in kleine Stücke zerschlagen wurde, ließ sich, trotz allen mittel- und unmittelbaren lang fortgesetzten Nachforschungen, kein ähnlicher an Größe weiter nachweisen. Die nächsten an Gewicht waren schon Steine zwischen 3 und 5 Pfund, und deren möchten wohl kaum mehr als jene 6 bis 7 gefallen und aufgefunden worden sein, welche der Plan nachweiset.} von den bei Stannern in Mähren, am 22. Mai 1808, Morgens gegen 6 Uhr, gefallenen Steinen,\footnote{Obgleich dieser Steinfall gerade keiner von den bedeutendsten war, indem nach den genauesten Nachforschungen, die wohl bei keiner Begebenheit der Art so umständlich und fortgesetzt angestellt wurden, kaum mehr als 100 Steine zu einem Gesamtgewicht von höchstens 150 Pfund gefallen sein dürften; so sind doch die Belege davon ebenso, und beinahe allgemeiner noch, wenigstens zweckmäßiger, verbreitet, als jene vom Steinregen zu L'Aigle, der doch in jeder Beziehung zwanzig bis dreißig Mahl ergiebiger war. Man hat dies den Einleitungen zu verdanken, welche bei diesem Ereignisse zur gehörigen Untersuchung des Factums, zum Einsammeln, und dann zu einer zweckmäßigen (unentgeldlichen) Verteilung der entbehrlichen Steine und Bruchstücke an die bekanntesten öffentlichen Sammlungen, und an die vorzüglichsten Privat-Sammler und Schriftsteller aus dieser Partie in ganz Europa, getroffen worden sind, und es wäre wohl sehr zu wünschen, dass von den Regierungen aller Staaten bei ähnlichen Ereignissen auf gleiche Art verfahren werden möchte. Auf diese Weise könnte sehr leicht eine ähnliche (gewiss sehr wichtige, und wie wir überzeugt zu sein glauben, in der Folge sicher noch zu sehr bedeutenden Aufschlüssen führende) Zusammenstellung der Produkte (der ausgezeichnetsten, und in irgend einer Beziehung merkwürdigen Steine und Bruchstücke) eines jeden vorfallenden Ereignisses der Art, an einen bestimmten, zweckmäßigen Platz (an irgend einer öffentlichen wissenschaftlichen Anstalt im Staate), und eine ähnliche Verbreitung und Verteilung der entbehrlichen Stücke an andere ähnliche Plätze (öffentliche Museen und Privat-Sammlungen) --- womit einerseits die nicht minder wichtige und notwendige, größtmöglichste und vollständigste Zusammenstellung solcher Produkte von verschiedenen Ereignissen, an verschiedenen Orten, und zur ausgebreitetsten Benutzung, andererseits eine sichere und dauernde Aufbewahrung derselben für Mit- und Nachwelt erzielt würde --- bewirkt, und damit am meisten zur seinerzeitigen Aufklärung dieser, in so vielfachen Beziehungen rätselhaften, Naturerscheinung beigetragen werden. Der bisherigen Vernachlässigung solcher Maßregeln ist es zuzuschreiben, dass wir von achtzig bis hundert Tausend ähnlichen Ereignissen, die sich, nach einem höchst wahrscheinlichen Kalkül, seit unserer Zeitrechnung bloß in Europa zugetragen haben möchten, kaum von einem Hundert derselben hinlänglich beglaubigte Nachrichten, und von diesem kaum von drei und dreißig (und diese beinahe ausschließlich von Ereignissen aus der neuesten Zeit, von den letzten 70 Jahren) nachweisbare, materielle Belege besitzen, und dass wir, nach Jahrtausenden, jetzt in diesem Jahrhunderte erst, nicht nur die ersten Schritte zur Aufklärung zu machen, sondern selbst noch den Unglauben an die Realität dieser ebenso auffallenden als wunderbaren Phänomene, die sich seit Menschengedenken, und keineswegs so selten, auf unserem Planeten ereigneten und immerfort ereignen, zu bekämpfen haben.} welcher ganz erhalten wurde.

Es ward derselbe erst gegen Ende des Monats Julius jenes Jahres, also zwei Monate nach dem Ereignisse, indem er in ein Kornfeld gefallen war und da verborgen blieb, von Katharina Pauser und ihrem Manne, einem Taglöhner von Neustift, im Beisein noch einiger Arbeitsleute, auf dem Felde des Neustifter Bauers, Jacob Achatzi, N. N. O. vom Markte Stannern, und zwar bei 3000 Klafter von der Kirche, fast am äußersten Ende (kaum 250 Klafter vom äußersten Punkte, wo noch ein Stein gefallen war) des befallenen Flächenraums gegen N. (Situations-Plan Nr. 59), zufällig während des Kornschneidens aufgefunden.

Er steckte fest in der Erde, und nur eine Ecke desselben ragte hervor, welche die Aufmerksamkeit des Taglöhnerweibes auf sich zog, indem es das geschnittene Korn zusammenraffte und in Garben band. Die Erde war sehr trocken und fest, und der Mann hatte Mühe, den Stein herauszubringen. Im Herausheben brach die in der Erde stecken gebliebene Spitze, oder vielmehr die eine obere Ecke ab. Das Gewicht desselben ward beiläufig auf 9 3/4 Pfund geschätzt, wie es sich auch im Plane angegeben findet; der Stein wiegt aber wirklich 11 Pfund und 10 Loth Wiener Commercial-Gewicht.

Außer einigen feinen und seichten Rissen, und hie und da etwas abgeschlagenen Kanten und Ecken, ist derselbe vollkommen ganz und durchaus mit Rinde bedeckt.

Es stellt derselbe eine wenig verschobene, und beinahe gleichseitig vierseitige Pyramide vor, deren etwas aus der Mitte gerückte Endspitze durch drei neue, auf den Seitenflächen aufsitzende, unvollkommene Flächen schief zugespitzt ist. Die Grundfläche ist fast eben, und hat wenige große, seichte, breit verlaufende Eindrücke. Die von ihr mit den fast senkrecht aufsteigenden Seitenflächen gebildeten Kanten sind meistens etwas verdrückt und abgerundet, eine jedoch ist sehr scharf, und bildet einen Winkel von 90°. Eine Seitenecke ist besonders hervorspringend, und nur wenig abgerundet und auffallend ist die Ähnlichkeit der Grundfläche dieses Steines, zumal in Hinsicht dieses Umstandes, mit jener der zuvor beschriebenen Steine von Tabor, L'Aigle, und selbst von Lissa, so wie die der Form des Steines im Ganzen, mit jener des Steines von Siena.

Zwei Seitenflächen, welche unter einem Winkel von beiläufig 100° in eine ziemlich scharfe Kante zusammenstoßen, die mit den Kanten der Grundfläche jene hervorspringende Ecke bildet, sind fast ganz flach und eben, nur etwas vertieft, und haben sehr wenige seichte, sanft verlaufende Eindrücke. Die zwei entgegen gesetzten Seitenflächen stoßen in eine stumpfere und verdrückte gemeinschaftliche Kante zusammen, und bilden ähnliche Kanten mit den vorigen Seitenflächen und mit der Grundfläche. Sie sind konvex, zumal die eine derselben, und durch häufigere, zum Teil tiefe Eindrücke, sehr uneben.

Die drei unvollkommenen Zuspitzungsflächen, wovon die eine, größere, fast gerade auf der einen gewölbteren Seitenfläche aufsitzt, und mit derselben eine sehr verdrückte, undeutliche Kante unter einem sehr stumpfen Winkel bildet, die beiden andern, kleineren, aber auf den etwas vertieften Seitenflächen schief, und so aufgesetzt sind, dass sie mit jener eine außer die Mitte fallende Zuspitzungs-Endkante bilden, --- wovon die abgebrochene Spitze die eine Ecke ausmachte, --- haben die Beschaffenheit der Oberfläche mit den korrespondierenden Seitenflächen gemein.

Die Rinde\footnote{Was die merkwürdige Beschaffenheit der Rinde an den Meteor-Steinen von Stannern im Allgemeinen, die auffallende Verschiedenheit derselben, nicht nur an verschiedenen einzelnen Steinen, sondern selbst oft, und zwar sogar gewöhnlich an einem und demselben Steine, und die große Mannigfaltigkeit hinsichtlich der besonderen Beschaffenheit ihrer Oberflache, und was endlich die Folgerungen betrifft, die sich aus der genauen vergleichenden Betrachtung derselben ziehen lassen; so verweise ich auf das, was Herr Professor von Scherer und ich im 31. Bande von Gilberts Annalen darüber umständlich vorgebracht haben, und wozu die gegenwärtigen Darstellungen (zumal die Figuren der fünften und sechsten Tafel) gewisser Maßen als versinnlichende Belege dienen sollen.} ist fast durchaus über den ganzen Stein von gleicher, und zwar von der gewöhnlichsten Beschaffenheit, wie sie an diesen Steinen überhaupt zu sein pflegt, ziemlich gleich dick, dicht und fest, etwas fettig, und ziemlich stark glänzend, rein dunkelschwarz, und von der rauen, einfach und verworrenen, runzlicht-aderigen Art (A. a. 2. Gilberts Annalen Bd. 31, S. 56); nur an den gewölbteren, unebeneren Flächen nähert sie sich der blattförmig gezeichneten (ebendas. A. a. 3), und ist hier matter, etwas weniger schwarz, und, wie es scheint, etwas dünner.

Sie ist nirgends abgesprungen, aber auch an keiner Stelle zeigt sich, trotz der bedeutenden Oberfläche dieser großen Masse, eine Spur von der unvollkommenen Art (ebendas. S. 58. D.).

Viele Runzeln und Adern, zumal an den Kanten, sind stark erhaben, scharf und faltenähnlich. Säume der Rinde finden sich an diesem Steine nirgendwo, wohl aber an den schärfern Kanten, wo die stark aderige Rinde von zwei Flächen zusammenstoßt, deutliche Nähte.

Die Dicke derselben weicht, so wie überhaupt bei diesen Steinen im Allgemeinen, nicht von der gewöhnlichen Dicke der Rinde an andern Meteor-Steinen ab, und beträgt im Ganzen 1/12 bis 2/12 Linie.

Ihre Härte ist nur sehr gering, und nur schwer, und bloß an einzelnen Stellen (an diesem Steine wohl an gar keiner) lassen sich der Rinde dieser Steine überhaupt mit dem Stahle einzelne schwache Funken entlocken; eben so wenig zeigt sie eine merkliche Wirkung auf die Magnetnadel; nur gepulvert bleiben einzelne Atome an der Spitze hängen.

Es zeigt sich zwar allenthalben an diesem Steine, in den Furchen und Vertiefungen des Adergeflechtes der Rinde, etwas Erde\footnote{Diese Erde lässt sich inzwischen selbst da, wo sie am festesten an- und eingedrückt zu sein scheint, doch ziemlich leicht und ohne Verletzung der zartesten Adern und Runzeln, mit einer scharfen Bürste wegbürsten, und mit einem nassen Schwamme vollends rein wegwaschen, so dass keine Spur in irgendeiner Beziehung von ihrem früheren Dasein zurückbleibt. Ein Umstand, der wohl, mit manchen andern Beobachtungen, sehr gegen die Annahme des flüssigen oder doch weichen Zustandes der Rinde, selbst noch im Momente des Auffallens der Steine, streitet.} eingedrückt, --- was bei dem tiefen und gewaltsamen Eindringen des Steines in das Erdreich, und bei den wiederholten Regengüssen, welche in der ziemlich langen Zwischenzeit bis zu dessen Auffinden Statt hatten, wohl nicht anders sein konnte, --- am meisten jedoch an den konvexen Flächen, auf welche der Stein auch, vermöge seines Schwerpunktes, aufgefallen sein müsste, falls dieser nicht etwa durch eine rotierende Bewegung des Steins im Falle, --- welcher inzwischen einerseits die Beschaffenheit der Rinde, wenn diese als flüssig, andererseits die Form des Steines, wenn die Masse weich gedacht werden soll, --- widerspräche, turbiert worden wäre.\footnote{Ich bemerke das noch sichtliche Ankleben von Erde an diesem, wie insbesondere an allen folgenden ganzen Steinen von Stannern, absichtlich mit Genauigkeit, weil dasselbe hier --- wo es sich übrigens, der eigentümlichen Rauigkeit der Oberfläche wegen, auch deutlicher zeigen und länger erhalten konnte als an irgend einem andern Meteor-Steine --- mit vollkommenster Verlässlichkeit, die wahren Auffallspunkte der einzelnen Steine --- inzwischen aber auch jene Stellen, welche bei tieferem Eindringen derselben in den Grund nebenher noch mit Erde in Berührung kamen, --- bezeichnet, indem die meisten dieser Steine (nur den eben beschriebenen und die beiden folgenden kleinsten ausgenommen) unmittelbar während der Begebenheit, oder doch nur wenige Tage nach dem Ereignisse, in welcher Zwischenzeit noch keine Abänderung in der ursprünglichen Lagerung derselben, noch eine zufällige Veränderung mit der umgebenden Erde Statt gefunden haben konnte, aufgehoben und unmittelbar aus der ersten Hand, von dem Auffinder selbst, erhalten worden waren.}

Der Stein ist, auf der Grundfläche liegend, so dargestellt, dass sich die eine Seitenfläche mit der aufsitzenden Zuspitzungsfläche in gerader, die zwei anstoßenden Seitenflächen aber, wovon die eine mit ersterer die etwas verlängerte Seitenkante und die vorspringende Ecke bildet, in schiefer Ansicht zeigen.
\clearpage
\section{Fünfte Tafel.}
\subsection{Erste Figur.}
\paragraph{}
Einer der kleineren, aber vollkommen ganz erhaltenen Steine von dem Ereignisse bei Stannern, der 5 Loth 1 Quäntchen wiegt, und sich durch eine besonders regelmäßige Form auszeichnet.

Er ward durch das von der Untersuchungs-Kommission veranlasste absichtliche Aufsuchen der gefallenen Steine, am 28. Mai von einem Landmanne zwischen dem Markte Stannern und dem Dorfe Lang-Pirnitz, oder vielmehr ganz nahe an diesem letzteren Orte, im südlichen Teile des befallenen Flächenraums (und zwar etwa 2600° südlich von der Kirche von Stannern, und kaum 1500° vom äußersten Punkte, wo noch ein Stein in diesem Teile gefallen war, dagegen über 5000° von der Fallstelle des vorhin beschriebenen Steines entfernt) aufgefunden. (Situations-Plan Nr. 19.)

Es ist derselbe vollkommen ganz, um und um überrindet, und bildet eine unvollkommene, dreiseitige Pyramide, deren Ähnlichkeit, obgleich sie sich, durch Abrundung und Abstumpfung der Ecken und Kanten zum Teil beinahe einer Kugelform nähert, mit der Form des großen, zuvor beschriebenen Steines unverkennbar, und umso auffallender ist, als sich an der Grundfläche dieses Steines, durch Abplattung und Breitdrückung einer Ecke, die Tendenz zu einer ähnlichen (vielleicht ursprünglich gewesenen und nur abgeänderten) verschoben und ungleichseitig vierseitigen Pyramidal-Form, die an jenem ausgesprochen ist (an den Figur 2 und 5 dieser Tafel vorgestellten Steinen aber auch nur in einem ähnlichen Grade angedeutet erscheint), nicht verkennen lässt.

Die stark konvexe und unebene Grundfläche des Steines stellt nämlich ein ungleichschenkliches Dreieck vor, dessen Ränder mit den drei ziemlich senkrecht aufsteigenden, fast ganz ebenen, nur etwas vertieften Seitenflächen stumpfe Kanten bilden, und dessen stumpfe Ecken den abgerundeten Seitenkanten entsprechen. Die eine dieser Ecken ist aber gleichsam platt und breit gedrückt, und geht, zugerundet, unmittelbar in eine ebenfalls breit gedrückte und abgerundete, beinahe zu einer vierten Seitenfläche gestaltete Seitenkante über, die bogenförmig, allmählich sich verdünnend, gegen die Endspitze verläuft.

Nach dem andern Ende des Steines verschmälern sich die Seitenflächen, und endigen sich in eine etwas nach einer Fläche hin- und selbst etwas übergebogene (folglich ebenfalls, und zwar sehr stark, außer das Mittel der Grundfläche fallende) dreiseitige, ziemlich scharfe Spitze, die durch zwei sehr unvollkommene und ungleiche, schief auf die Seitenflächen aufgesetzte Flächen zugespitzt, und gewisser Maßen kantig zugeschärft wird.\footnote{Die von mir in Gilberts Annalen von diesem Steine schon früher gegebene Beschreibung, B. 31, S. 36, D., spricht die Form desselben nicht deutlich genug aus.}

Nur auf der Grundfläche finden sich einige einzelne, ziemlich seichte und kleine Eindrücke.

Die Rinde ist über den größten Teil des Steines, und eigentlich durchaus eine und dieselbe, und zwar von gleicher zart strahlig-aderige- Beschaffenheit (A. b. 1. Gilberts Annalen, Bd. 31, S. 57), und einem, mit dieser Beschaffenheit stets verbundenen, stellenweise (wo nämlich die oberste Schichte abgesprungen oder abgestoßen ist, was bei dieser Art Rinde gewöhnlich Statt findet) matten, im Ganzen aber starken, seidenartigen, schimmernden Glanze, und beinahe kohlschwarzer Farbe. Die feinen erhabenen Strahlen sind zwar kurz und oft unterbrochen, und verwirren sich oft hin und wieder, zumal bei ihrem Ursprunge, wo sie ein Geflecht bilden; doch scheinen sie von der Spitze aus über die Seitenflächen gegen die Grundfläche hin ihre Hauptrichtung zu nehmen, an deren Kanten, zumal von zwei Flächen her, sie sich verdicken, anhäufen und als ein gezackter, ziemlich scharf abgeschnittener Rand enden, ohne einen Saum oder eine Naht zu bilden.

Auf der Grundfläche zeigt sich zwar dieselbe Rinde, ihrer Hauptbeschaffenheit nach, allein nur in Spuren, denn die oberste Schichte, die auf den Seitenflächen nur hie und da an kleinen Stellen abgestoßen ist, scheint hier ganz zu fehlen, und ihre Oberfläche erscheint beinahe matt, nur wenig schimmernd, und mehr braun als schwarz. Allein bei Betrachtung unter der Lupe zeigt sich, dass die obere Schichte doch nicht abgerieben oder abgestoßen ist, --- in welchem Falle solche Stellen ganz matt, porös und gleichsam schwammig erscheinen, --- sondern dass sie nur in einer andern Modifikation vorhanden ist, nämlich Statt Runzeln und Adern, großen Teils bloß erhabene Punkte und Körner bildend.

Von eingedrückter Erde zeigt sich an der schmählern, der gebogenen, breit gedrückten Seitenkante entgegen gesetzten Seitenfläche die meiste Spur, aber auch hier nur in den zarten Zwischenräumen der erhabenen, scharfen Adern, und, wie am gewöhnlichsten, in den vertieften mikroskopischen Punkten und Poren der Oberfläche.

Die Abbildung, welche diesen Stein als Musterstück solcher von geringerer Größe bei vollkommener Integrität und von ausgezeichneter Form darstellen soll, zeigt denselben, auf einer Seitenfläche liegend und mit der Endspitze nach unten gekehrt, um mit dem so viel als möglich ganzen Umrisse die gewölbte Grundfläche, und die eine, breiteste, Seitenfläche --- gegen welche die Spitze gebogen ist --- mit ihren Seitenrändern --- wovon der eine die gebogene, breit gedrückte Kante bildet --- ersichtlich zu machen.

\subsection{Zweite Figur. a. b.}
\paragraph{}
Ebenfalls einer von den kleineren, bei Stannern gefallenen Steinen, 4 Loth 1 Quäntchen wiegend, welcher ganz erhalten worden ist, und eine auffallend regelmäßige Form zeigt.

Es wurde derselbe, am andern Tage nach dem Ereignisse, von einem Landmanne auf einem Haberfelde zwischen Lang- und Klein-Pirnitz, ebenfalls im südlichen Teile des befallenen Flächenraums (und zwar etwa 2400° südlich von der Kirche von Stannern, beiläufig 700° östlich von der Fallstelle des vorhin beschriebenen ähnlichen Steines, und ziemlich in gleicher Entfernung vom äußersten Fallpunkte in S.), flach aufliegend und einen starken Zoll tief in das Erdreich eingedrungen gefunden, und am 28. Mai mir selbst zu Lang-Pirnitz, wo ich auf der Fahrt nach Stannern angehalten hatte, um vorläufige Erkundigungen einzuziehen, auf mein Verlangen überlassen. (Situations-Plan Nr. 16.)

Er ist vollkommen ganz und durchaus überrindet, nur eine Ecke ist etwas abgestoßen, und ein kleines Stück der oberen Endspitze abgeschlagen; der Verlust an Masse kann indes kaum 2 Quäntchen betragen.

Es stellt derselbe eine etwas verschobene, aber ziemlich gleichseitig dreiseitige, oder vielmehr eine ungleichseitig vierseitige, etwas verlängerte Pyramide vor. Er zeigt nämlich eigentlich zwar nur drei ziemlich gleich breite Seitenflächen; allein eine derselben ist, durch eine, obgleich nur unvollkommene Kante, die sich aber an der Grundfläche doch durch eine deutliche Ecke ausspricht, der Länge nach in zwei sehr ungleiche Hälften geteilt.

Diese solcher Gestalt geteilte Seitenfläche ist im Ganzen etwas konvex, und durch verhältnismäßig sehr große Eindrücke sehr uneben, ja durch einen besonders großen und tiefen gegen die Basis hin, welcher beinahe einem Verluste an Masse, durch spätere Lostrennung oder Absprengung eines Stückes (wenn diesem nicht zum Teil die Gleichförmigkeit der Rinde widerspräche) zugeschrieben werden könnte, gewisser Maßen verunstaltet. Die beiden andern Seitenflächen, welche mit dieser beiderseits unter einem ziemlich stumpfen Winkel, in eine sehr stumpfe, verdrückte und ausgeschweifte, unter sich aber in eine beinahe schneidend scharfe, aber im Verlaufe, durch Eindrücke von den Flächen her, mehrere Male gebogene gemeinschaftliche Kante, unter einem ziemlich spitzen Winkel zusammen stoßen, sind ziemlich flach, eher etwas vertieft, und haben zwar ziemlich viele, aber nur seichte und breit verlaufende Eindrücke, die mehr den Unebenheiten einer natürlichen Bruchfläche des Steines, als den gewöhnlichen Eindrücken gleichen. Nach dem einen Ende zu verschmälern sich die Seitenflächen allmählich, und gehen, nachdem sich die eine unvollkommenere Kante, welche die konvexe Seitenfläche teilte, mit der nächsten zu vereinigen scheint, in die Spitze über, welche, obgleich sie abgebrochen ist und ursprünglich fehlt, nach der Richtung der Flächen stumpf und dreiseitig, und etwas gegen die konvexe Fläche gebogen gewesen sein dürfte.

Die Grundfläche ist fast flach, nur etwas vertieft, sonst vollkommen eben, und bildet ein sehr ungleichseitiges, verschobenes Viereck, indem jeder Seitenfläche --- selbst den beiden sehr ungleich geteilten Hälften der einen konvexen --- eine Kante, und jeder Seitenkante --- selbst der unvollkommenen, jene Fläche teilenden --- eine, wenn gleich stumpfe, Ecke entspricht. Die mit den beiden Hälften der konvexen Seitenflächen gebildeten Kanten sind sehr stumpf, jene mit den zwei andern Seitenflächen aber ziemlich scharf, und da diese Seitenflächen mit ihrer gemeinschaftlichen Kante sich nach diesem Ende des Steines hin beträchtlich verlängern; so erhält die Grundfläche dadurch eine ganz schiefe Richtung gegen die viel kürzere konvexe Seitenfläche, und die durch jene verlängerte Seitenkante mit den beiden Grundkanten der Seitenflächen gebildete Ecke springt bedeutend vor, und scheint (da sie verbrochen ist) ziemlich scharf gewesen zu sein.

Die Rinde ist an diesem Steine besonders merkwürdig, und zeigt eine wesentliche und auffallende Verschiedenheit nach den verschiedenen Flächen desselben.

Auf der konvexen Seiten- und der mit dieser auch im Übrigen übereinstimmenden Grundfläche ist sie von der sehr rauen, runzlicht-aderigen Art (A. a. 1. Gilberts Annalen Bd. 31, S. 56), mit dem gewöhnlichen Glanze, der durch matte Stellen --- wo nämlich die oberste Schicht abgesprungen ist --- unterbrochen wird, und von mehr brauner als schwarzer Farbe. Auf den beiden andern Flächen dagegen ist sie ganz glatt, sehr dicht, fest und gleichförmig, sehr schwach aderig, und nur sehr undeutlich und unvollkommen blattförmig gezeichnet, pechschwarz und sehr fettig glänzend. (B. 2. ebendas. S. 57.) Von der konvexen Seiten- und der Grundfläche über die Kanten her, bildet die dortige rauere Rinde auf die Rinde dieser Flächen herüber undeutliche und nicht scharf begrenzte Säume.\footnote{Diese höchst merkwürdige, und, wie mir däucht, für die Erklärung der Bildung der Rinde sowohl, als der Formierung (Vereinzelung) der Steine sehr wichtige Eigenheit derselben, Säume zu bilden, spricht sich am deutlichsten an dem gleich zu beschreibenden, und vorzugsweise deshalb (übrigens auch der Große, Vollkommenheit und Form wegen) auf derselben Tafel Figur 5 abgebildeten Steine aus, mit welchem dieser, und zwar nicht nur in der Form, --- sogar in den einzelnen Unregelmäßigkeiten derselben, --- sondern auch in der ganzen Beschaffenheit und Art der Überrindung, die auffallendste Ähnlichkeit und Übereinstimmung zeigt.}

Von unvollkommener Rinde zeigt sich keine Spur an diesem Steine, und von eingedrückter Erde nur etwas an der Grundfläche, auf welche der Stein, vermöge seines natürlichen Schwerpunktes auch aufgefallen sein müsste.

Figur 2. a. stellt diesen merkwürdigen Stein, auf den beiden glatten Seitenflächen und ihrer gemeinschaftlichen Kante liegend, von der konvexen, unebenen und unvollkommen geteilten Seitenfläche, und der mit derselben in schiefer Richtung verbundenen Grundfläche vor;

Figur 2. b. zeigt denselben aber, auf jener Seitenfläche ruhend, von der gemeinschaftlichen, schneidend scharfen Kante, in welche die beiden andern Seitenflächen zusammenstoßen.

\subsection{Dritte Figur.}
\paragraph{}
Einer der kleinsten, und doch vollkommen überrindeten Steine von dem Ereignisse bei Stannern, von kaum 2 1/2 Quäntchen am Gewichte.

Es ward derselbe einige Zeit nach der Begebenheit, in Folge nachträglicher amtlicher Aufforderung an das Landvolk jener Gegend, die etwa noch verborgen liegenden Steine aufzusuchen und abzuliefern, an das k. k. Kreisamt zu Iglau eingebracht, und von diesem mit mehreren andern eingesendet.

Da dieser Stein zu klein und unbedeutend schien, so ward weder der Finder namentlich angezeigt, noch in dem späterhin aufgenommenen Situations-Plane die Stelle angedeutet, wo derselbe aufgefunden wurde; indessen doch in dem Einbegleitungsschreiben bemerkt: dass derselbe aus der Gegend von Lang-Pirnitz, demnach aus dem südlichen Teile des befallenen Flächenraums, eingebracht worden sei.

Er ist vollkommen ganz, und nur an einer Seite etwas abgeschlagen, so dass der Verlust an Masse etwa ein halbes Quäntchen betragen haben möchte; außer dieser Stelle ist er um und um mit Rinde bedeckt.

Er bildet eine etwas verdrückte, verschoben aber ziemlich gleichseitig vierseitige, sehr abgestumpfte und niedere Pyramide, und gleicht mit dieser Form, die sich ziemlich deutlich auf den ersten Blick ausspricht, und im Kleinen, sehr dem Steine von Tabor; nur ist die Grundfläche (so wie bei den Steinen von Lissa und L'Aigle) durch eine sehr starke schiefe Abstumpfung einer Kante, wodurch die Fläche gleichsam in zwei Hälften geteilt wird, verunstaltet, wodurch sich die Form von dieser Seite mehr jener des Steines von L'Aigle nähert. Die beiden Endflächen, den Stein in dieser Haltung betrachtet, sind sehr uneben, sonst ziemlich flach; die obere kleinere und etwas aus dem Mittel gerückte zeigt einige kleine, aber ziemlich tiefe Eindrücke; die untere größere, mehr länglicht viereckigte, wird durch die neue, durch die Abstumpfung gebildete, sehr stumpfe Kante, welche die Fläche der Quere nach in zwei ziemlich gleiche Hälften teilt, gewölbt gemacht. Eine der Seitenflächen ist beinahe senkrecht aufgesetzt, ganz flach und eben; die gegenüberstehende etwas schief, konvex und uneben; die dritte sehr schief aufsteigende etwas konkav, und dies eigentlich durch ein paar verhältnismäßig sehr große, aber seichte und sehr breit verlaufende Eindrücke; und die vierte, dieser gegenüberstehende, ist die verbrochene. Alle diese Flächen bilden sowohl unter sich als vorzüglich mit der Grundfläche, am wenigsten mit der oberen Endfläche, ziemlich scharfe Kanten. Die Rinde scheint auf den ersten Anblick über den ganzen Stein durchaus von ganz gleicher Beschaffenheit zu sein, ist es wohl auch im Wesentlichen, zeigt aber doch bei näherer Betrachtung einige untergeordnete Modificationen.\footnote{Wenn man sich die Rinde bei ihrer Entstehung, während ihrer Bildung, und wenigstens einige Zeit während des Falles des Steines, in einem mehr oder weniger flüssigen Zustande denken will (und das muss man wohl, wenigstens bei den Meteor-Steinen von Stannern), und zumal, wenn man (was man, wie mir däucht, weniger muss noch soll) die Rinde-bildende Potenz aus der Luft selbst (den durch Kondensation ausgepressten und durch Reibung erzeugten Wärmestoff) nehmen will; so muss jeder Stein an seinen verschiedenen Flächen oder Seiten, je nachdem sie, nach dessen Richtung im Falle (wenn auch eine Achsenbewegung dabei Statt fände, welcher jedoch, ohne der Form zu erwähnen, die Beschaffenheit der Rinde, diese als flüssig angenommen, an den meisten Steinen offenbar widerspricht), mehr oder weniger dem Luftstrome entgegen gestellt waren, wenigstens eine zwei-, ja wohl dreifache kleine, untergeordnete Modifikation der Rinde, wenn diese auch über den ganzen Stein von einer und derselben Hauptbeschaffenheit sein sollte --- was sie in einzelnen Fällen auch wohl sein kann --- erkennen lassen. Und dies scheinen wirklich die Meteor-Steine von Stannern, deren Rinde, vermöge ihrer ganz eigentümlichen Natur und Beschaffenheit, vorzugsweise, ja bis jetzt beinahe ausschließlich geeignet ist diese Modifikationen auszusprechen, zu bestätigen. Ein anderes ist es um jene Hauptverschiedenheiten der Rinde, deren ich in meinem Aufsätze, in Gilberts Annalen Bd. 31, erwähnt und vier aufgestellt habe; diese rühren von ganz anderen Ursachen her (von der ursprünglichen Form und der individuellen Beschaffenheit der Oberflache der Steine; von der Kraft und der Dauer des Rinde-bildenden Prozesses, die durch Hohe, Richtung und Schnelligkeit des Falles bei den verschiedenen einzelnen Steinen mannigfaltig verschieden sein, ja selbst bei ein und demselben Steine durch wiederholte Zerplatzungen oder Lostrennung einzelner Stucke im Falle wieder abgeändert werden können u. s. w.); und es können deren an einem und demselben Steine (wie dies der vorhin beschriebene bewährt, und die Figur 5 und Tafel 6 Figur 3 und 4 abgebildeten noch deutlicher zeigen) ebenfalls zwei auch drei vorkommen. Dass jede derselben nach obigem ihre eigenen Modifikationen haben müsse, ergibt sich von selbst, und welche Komplikationen aus dem zufälligen Zusammentreffen mehrerer von diesen und jenen notwendig entstehen müssen, dies lässt sich denken.}

An der oberen End- und der einen schiefen Seitenfläche ist sie nämlich von einer ganz eigenen Art, die gleichsam das Mittel hält zwischen der strahlig und runzlich-aderigen. Der Grund ist matt und etwas graulich-schwarz, und die Adern, welche mehr vereinzelt stehen, verlängert und nur selten etwas ramifiziert sind, wenig zusammenhängen, und daher kein eigentliches Netz oder Geflecht bilden, sind pechschwarz oder pechbraun, mit einem ähnlichen fettigen Glanze. Sie sind ziemlich stark und grob, so dass die Oberfläche ziemlich rau erscheint; unter der Lupe erscheinen sie aber wie gekörnt, und aus einzelnen mehr oder weniger dicht aneinander gereiheten und zusammenfließenden Kügelchen oder Tröpfchen gebildet, wie kleine Perlenschnüre (sehr ähnlich der unvollkommenen Rinde D. 2; aber nicht auf frischer Bruchfläche, sondern auf schon überrindetem Grunde; eine Anomalie, die ich bei keinem Steine von Stannern wieder finde). An der untern End- oder Grundfläche und an der konvexen Seitenfläche zeigt sich dagegen die Rinde zwar von einer ähnlichen, aber schon mehr ausgesprochenen, dichter strahlig-aderigen Beschaffenheit, von dunkelschwarzer Farbe, und starkem, etwas seidenartigem Glanze (fast genauso, wie die Rinde an den Seitenflächen des zuvor beschriebenen, und Figur 1 abgebildeten Steines), und zeigt offenbar einen Übergang in oder vielmehr aus jener zuvor beschriebenen. An der ebenen Seitenfläche endlich erscheint sie beinahe kohlschwarz, von fettigem, etwas schillerndem Glanze, und zart runzlicht und verworren, klein und sehr dicht-aderig, unverkennbar als Modifikation oder höherer Grad der letzteren.

Sie bildet übrigens nirgendwo Säume oder Nähte, aber unvollkommen, und zwar im höchsten Grade (D. 3), findet sie sich an ein paar äußerst kleinen Stellen, und auf einem, verhältnismäßig, bedeutend großen Flecke an der oberen Endfläche. An allen diesen Stellen scheint aber bloß die bereits gebildet gewesene Rinde, nicht aber ein Stück der Masse des Steines, abgesprungen zu sein. Nur an der Grundfläche zeigt sich etwas Spur von Erde.

Dieser höchst merkwürdige Stein ist auf seiner Grundfläche liegend vorgestellt, um dessen obere Endfläche --- welche am regelmäßigsten ist, und seine Form am besten ausspricht --- die eine gewölbte Seitenfläche von vorne, und die schiefe von der einen Seite, zur Ansicht zu bringen.

\subsection{Vierte Figur.}
\paragraph{}
Der kleinste, und doch vollkommen ganze und durchaus überrindete, bei Stannern gefallene Stein, der kaum 56 Gran wiegt.

Auch dieser ward erst einige Zeit nach dem Ereignisse eingebracht, und vom k. k. Kreisamte zu Iglau mit der Anzeige, dass er ebenfalls in der Nähe des Dorfes Lang-Pirnitz, also im südlichen Teile des befallenen Flächenraums, aufgefunden worden sei, hierher eingesendet; im Situations-Plane aber eben so wenig, wie vom vorigen Steine, der scheinbaren Unbedeutendheit wegen, der Finder genannt, oder die Fallstelle angegeben.

Es zeigt derselbe einen eiförmigen Umriss, da er aber sehr plattgedrückt, und bei einer Länge von 11 und einer Breite von 8 Linien, im größten Durchmesser, an der dickesten Stelle kaum 4 1/2 Linie misst, eine mandelförmige Gestalt. Diese Gestalt nähert sich jedoch --- indem die beiden Flächen auf der einen Seite in eine scharfe Kante zusammen stoßen, an der entgegen gesetzten aber durch einen ziemlich breiten Rand verbunden sind, der eine dritte, obgleich weit schmälere, Fläche bildet --- einem ungleichseitigen Prisma, und damit auffallend, obgleich im winzig Kleinen, der Form des nächst zu beschreibenden, Figur 5 abgebildeten, Steines; nur dass an diesem Steine das Prisma von zwei Flächen her stark zusammen gedrückt, und die dritte Fläche die schmälste ist, und dass diese sich allmählich in den Rand der andern Seite, der gemeinschaftlichen Kante der beiden andern Flächen, verliert, ohne mit denselben Endflächen zu bilden.

Die beiden größeren Flächen sind etwas konvex, allmählich gegen ihre gemeinschaftliche, fast schneidend scharfe, Kante schief abnehmend, und, zumal die eine, durch ziemlich tiefe ungleichförmige Eindrücke, die ebenso, wie an den gleichartigen Flächen jenes Steines, mehr den Unebenheiten einer natürlichen Bruchfläche als gewöhnlichen Eindrücken gleichen, sehr uneben; die schmale Fläche ist noch weit gewölbter und unebener, zumal nach einem Ende hin, wo ein verhältnismäßig beträchtliches Stück der Steinmasse sich gleichzeitig losgetrennt zu haben scheint, und eine bedeutende Vertiefung zurück ließ.

Die Rinde scheint auch an diesem Steine durchaus von einerlei Beschaffenheit zu sein, und ist auch wirklich von einerlei, und zwar von der glatten Art, von dunkelschwarzer Farbe und starkem fettigem Glanze, ganz ähnlich jener an den konkaven Flächen des zuvor beschriebenen und Figur 2 b, und des nächst zu beschreibenden, Figur 5 abgebildeten, Steines; nur scheint sie fast durchaus dünner zu sein; denn sie zeigt einen Grad von Durchscheinenheit, der selten vorkommt, so dass der, wie es scheint, schwerer in Rinde umwandelbare weiße Gemengteil der Steinmasse in Gestalt einzelner gelblicher und bräunlicher Körner durchscheint; und die Adern sind etwas stärker und faltenartiger, doch ohne die Oberfläche rau zu machen oder ein Netz zu bilden. Offenbar zeigt sich aber auch an diesem, doch so kleinen Steine eine Modifikation oder Abstufung der Hauptbeschaffenheit der Rinde; denn unverkennbar ist sie an der einen breiten Fläche, dichter, dunkler und glänzender, und von hier ist sie auch in Gestalt eines unvollkommenen Saumes über den scharfen Rand auf die entgegen gesetzte Fläche, und zum Teil auch über den stumpferen Rand der auf einer Seite an dieselbe grenzenden schmalen Fläche, welche in allen Beziehungen mehr mit jener übereinstimmt, übergeflossen. Von unvollkommener Rinde findet sich keine Spur, und nur gegen das eine etwas dickere und breitere Ende des Steines zeigt sich etwas Erde an der schmalen Fläche.

Die Abbildung zeigt den Stein im ganzen Umrisse auf der einen breiten, stärker überrindeten Fläche liegend, mit dem scharfen --- der schmalen Fläche entgegen gestellten --- Rande nach vorne gekehrt, um den Rindensaum auf der einen einiger Maßen ersichtlich zu machen.

\subsection{Fünfte Figur.}
\paragraph{}
Einer der größten Steine von dem Ereignisse bei Stannern, 3 Pfund 18 Loth wiegend.

Er ward am Tage (29. Mai) der an Ort und Stelle abgehaltenen Untersuchungs-Kommission, bei angeordneter Aufsuchung der gefallenen und ganz außer Acht gelassenen Steine, von einem Bauersweibe auf einem Felde zwischen Stannern und dem Dorfe Falkenau, beinahe im Mittelpunkte des befallenen Flächenraums (und zwar etwa 600° östlich von der Kirche von Stannern, und etwa 3000° vom äußersten Punkte in N., und etwa 4000° vom äußersten Punkte in S., wo die entferntesten Steine gefallen waren), auf ziemlich festem Boden, flach aufliegend und nur sehr wenig in die Erde eingedrungen, gefunden. (Situations-Plan Nr. 45.)

Es ist derselbe vollkommen ganz, und durchaus mit Rinde bedeckt; nur an ein paar kleinen Stellen ist diese etwas abgeschlagen, und an dem einen Ende ist ein kleines Stück ausgebrochen, doch so, dass der Verlust an Masse kaum auf 1 Loth angeschlagen werden kann.

Der Umriss des Steines ist eiförmig, mit stark abgestumpften Enden; er bildet aber eigentlich ein vollkommenes, nur etwas ungleichseitig dreiseitiges, gegen die beiden Enden verschmälertes Prisma, und stellt solcher Gestalt ein Segment eines Eies vor.

Er zeigt nämlich drei Hauptflächen, die unter ziemlich spitzen Winkeln zusammenstoßen, und ziemlich scharfe Kanten bilden, und von welchen die etwas breitere konvex, und die beiden andern ein wenig vertieft sind. Nach den beiden Enden hin verschmälern sich diese Flächen, aber ungleich, so dass die konvexe mit einer der konkaven mehr nach dem einen, die andere konkave mehr nach dem andern Ende zu abnimmt. Die beiden Enden sind stark abgestumpft, und durch eine Fläche geschlossen, so dass man diese als End-, jene als Seitenflächen betrachten kann.

Die eine dieser Endflächen, die man als die größere und regelmäßigere, als die Grundfläche dieses Steines ansehen mag, ist flach, nur etwas vertieft, und bildet ein vollkommenes, aber stark verschobenes, und sehr ungleichseitiges Viereck; drei Ecken desselben entsprechenden Seitenkanten, und folglich die Ränder den Seitenflächen, die vierte Ecke aber, welche etwas stumpfer ist, fällt gegen die Mitte der breiteren Seitenfläche, von deren Teilung durch eine Kante sich der Anfang zeigt. (Die Tendenz zur vierseitigen Säule, oder, da das andere Ende schmäler zuläuft, und dort die Endfläche kleiner und ganz unregelmäßig ist, zur ungleichseitig vierseitigen Pyramide, ist unverkennbar, und besonders auffallend die Ähnlichkeit und Übereinstimmung dieses Steines mit dem zuvor beschriebenen, und Figur 2. a. b. abgebildeten, ungleich kleineren, nur mit dem Unterschiede, dass dieser gegen das eine Ende ungleich mehr verschmälert ist, und daher eine vollkommen konische Gestalt hat.)

Die beiden etwas vertieften Seitenflächen, die in eine gemeinschaftliche, scharfe, etwas verdrückte und wellenförmig ausgeschweifte Kante (genauso wie an jenem Steine; auch ist sie länger als wenigstens eine der beiden andern Seitenkanten, und bildet an der Grundfläche die vorspringendste Ecke) zusammen stoßen, haben nur wenige, und sehr seichte, aber ziemlich große und breit verlaufende, unförmliche Eindrücke, die (ebenso) mehr den Unebenheiten einer natürlichen Bruchfläche, als den gewöhnlichen Eindrücken gleichen, und bilden zum Teil, oder liegen in größeren, stärkeren Vertiefungen, welche von ungleichförmiger, aber gleichzeitiger Lostrennung einzelner Stücke der Masse herzurühren scheinen.

Die breitere, konvexe Seitenfläche, welche mit jenen Flächen ziemlich scharfe, und hie und da besonders dünne, übrigens sehr verdrückte und ausgeschweifte Kanten bildet, zeigt weit wenigere und noch seichtere Eindrücke von gewöhnlicher Art, so dass sie fast eben erscheint; nur gegen das obere Ende hin ist sie durch beträchtliche Vertiefungen (die wahrscheinlich ebenfalls durch eine ungleichförmige, aber auch mit der Entstehung der ganzen Fläche gleichzeitige Lostrennung einzelner Stücke entstanden sein mögen) gewisser Maßen verunstaltet.

Die obere Endfläche, welche mit den Seitenflächen sehr undeutliche und unvollkommene Kanten bildet, entspricht der Beschaffenheit der Oberfläche nach, vollkommen dem oberen, verunstalteten Teile der konvexen Seitenfläche; die untere Endfläche aber (die mit den Seitenflächen ziemlich scharfe, und nur mit der einen sehr schmalen konkaven eine platt und sehr breit gedrückte Kante bildet), hat nur einige sehr seichte Eindrücke, und zeigt in jeder Beziehung eine, obgleich nur wenig bedeutende Abweichung von allen übrigen Flächen.

Die Rinde ist an diesem Steine (sowie an jenem Fig. 2. a. b.) ganz besonders merkwürdig, und zeigt (genauso wie an diesem in jeder Hinsicht) eine sehr wesentliche und auffallende Verschiedenheit nach den verschiedenen Flächen, oder vielmehr nach den Seiten des Steines.

An den beiden konkaven Seitenflächen ist sie nämlich von gleicher, und zwar von der (in Gilberts Annalen Bd. 31, S. 57, sub B. 2. beschriebenen) glatten, nur sehr schwach aderigen Art; sehr dicht, fest und gleichförmig, pechschwarz und sehr fettig glänzend. Nur hie und da zeigen sich wenig erhabene kleine Adern und Ramifikationen, die nur selten zusammen hangen, und nur eine Anlage zu blattförmigen Zeichnungen, ohne bestimmte Richtung, bemerken lassen. Sie bildet weder Nahte noch Saume.

An der konvexen Seitenfläche dagegen, auch an dem oberen, verunstalteten Teil derselben, ist die Rinde besonders ausgezeichnet, von der sehr rauen, runzlicht und faltig-aderigen Art (ebendas. S. 56, A. a. 1.), zwar dicht und fest, aber sehr ungleichförmig, da fleck- und stellenweise die oberste, raue Schichte derselben fehlt, wo sie lockerer, porös, matt, und mehr braun als schwarz erscheint; sonst beinahe kohlschwarz, und von ziemlich starkem, nur durch jene Stellen unterbrochenen, aber mehr seidenartigen, schillernden Glanze. Abgesehen von der Erhabenheit einzelner starker Runzeln und Falten, hat sie im Ganzen keine beträchtlichere Dicke als jene an den entgegen gesetzten Flächen. Die erhabenen, ziemlich scharfen Adern, Runzeln und Falten, bilden ein ziemlich enges, unregelmäßiges Netz, oder ein verworrenes Adergeflecht; aber, obgleich einige mehr verlängerte Adern, zumal hie und da auf den Rücken der Erhabenheiten, welche einige Vertiefungen begrenzen, ausgezeichnet und besonders scharf sind; so sind dieses doch keine Nähte, da sie immer nur von einer Seite her gebildet werden, und keine bestimmte Richtung haben. Dagegen ist die Rinde an allen Kanten dieser Fläche, sowohl gegen die beiden andern Seitenflächen, als auch gegen die untere End- oder Grundfläche hin, obgleich hier schwächer und undeutlicher, angehäuft, verdickt, und über die Kanten selbst geflossen, so dass sie an jenen Flächen eine Art von Saum bildet, der wie eine doppelte Lage von Rindenmasse, über eine Linie breit, sich auf dieselben und über die denselben eigentümliche Rinde hinein zieht, genau dem Laufe der Kanten folgt, und ziemlich scharf abgeschnitten endigt. An einer der Seitenkanten ist diese Saumrinde, und zwar gerade an den zwei hervorragendsten Stellen, auf einen halben Zoll Länge, wieder gegen die Fläche zurückgedrückt, gerade als wenn der Stein mit diesen Punkten gegen einen harten Körper gestoßen wäre, der die (noch nicht ganz erstarrte ?) Rinde zurückgebogen hätte. Die Beschaffenheit der Rinde an der oberen Endfläche stimmt ganz, so wie die übrige Beschaffenheit der Oberfläche des Steines hier, mit der Beschaffenheit beider an dieser konvexen Fläche überein; jene dagegen an der untern End- oder Grundfläche weicht hierin, obgleich nicht sehr auffallend, von jener an beiden Seiten des Steines ab. Sie ist nämlich bei weitem nicht so glatt und fettig glänzend, wie die an den konkaven Seitenflächen; aber auch nicht so rau und runzlicht-aderig und schimmernd, wie die an der konvexen, sondern überhaupt mehr von der gemein-aderigen, an den Steinen von Stannern am gewöhnlichsten vorkommenden Art (A. a. 2). Von der konvexen Fläche her bildet die dortige, über die Kante auf diese Fläche nur etwas übergeflossene Rinde, auch nur einen undeutlichen, unvollkommenen Saum; von den beiden konkaven Flächen her aber steht die Rinde gleichsam an den Rand der gemeinschaftlichen Kante frei an, und von jener dieser Fläche geschieden.

Von unvollkommener Rinde findet sich an diesem ganzen großen Steine nur eine und selbst etwas zweideutige Spur, an einer kleinen Stelle auf der scharfen gemeinschaftlichen Kante der konkaven Seitenflächen.

Nur an diesen letzteren Flächen, auf welche der Stein auch wirklich aufgefallen zu sein scheint, da er namentlich auf denselben liegend gefunden wurde, und zum Teil an den beiden Endflächen, findet sich Erde in die kleinen, seichten, ohne dieselbe kaum sichtbaren, Zwischenräume des schwachen Adergeflechtes, und in die vertieften Punkte und Poren der Rinde an- und eingedrückt; auf der konvexen Fläche dagegen, die bei dem geringen Eindringen des, doch über 3 Zoll dicken, Steines in das Erdreich hoch genug über dasselbe hinausragte, so dass nicht leicht ein Regenguss Erde darüber schlemmen konnte, zeigt sich trotz der Rauigkeit der Oberfläche keine Spur davon. (An dem Steine Fig. 2 finden sich erstere, sonst ganz gleich beschaffene Flächen, ganz rein von Erde, wovon sich hier überhaupt nur etwas an der Grundfläche zeigt.)

Die Abbildung zeigt diesen besonders ausgezeichneten Stein, auf der konvexen Seitenfläche liegend (wie Figur 2. b. den ähnlichen), von den beiden glatten, konkaven Seitenflächen und ihrer gemeinschaftlichen Kante (nur gestürzt, die Grundfläche nach oben, und etwas gewendet), um den merkwürdigen Rindensaum, von der konvexen Fläche her, an einem Seitenrande (in das nötige Licht gebracht) ersichtlich zu machen.
\clearpage
\section{Sechste Tafel.}
\subsection{Erste Figur.}
\paragraph{}
Einer von den großen Steinen von dem Steinfalle bei Stannern, 2 Pfund 12 Loth schwer.\footnote{In dem, dem Situations-Plane angeschlossenen Verzeichnisse der aufgefundenen Steine, wird das Gewicht, wahrscheinlich weil nur nach Erinnerung geschätzt, da das Verzeichnis mehrere Monate später aufgenommen wurde, nur auf 2 Pfund angegeben.} Vollkommen ganz, und durchaus überrindet.

Es ward derselbe am 29. Mai im Verfolg des angeordneten Aufsuchens der gefallenen Steine, ganz nahe an dem Marktflecken Stannern selbst, ziemlich im Mittelpunkte des befallenen Flächenraumes (und zwar kaum 500° westlich von der Kirche von Stannern, und bei 3000° vom äußersten Punkte in N., und bei 4000° vom äußersten Punkte in S., wo die entferntesten Steine gefallen waren) aufgefunden, und an die Untersuchungs-Kommission abgegeben. (Situations-Plan Nr. 26.)

Die Gestalt dieses Steines scheint beim ersten Anblicke höchst unregelmäßig zu sein, denn viele große, unförmliche und zum Teil ziemlich tiefe Eindrücke, die offenbar vom Verlust an Masse durch spätere Zersprengung und gleichzeitige Lostrennung mehrerer Stücke, vor der Rindebildung im Ganzen, herrühren, verunstalten die Flächen, verdrücken die Kanten und unterbrechen deren Richtung, so dass der Stein, zumal derselbe gegen das eine Ende hin etwas verschmälert, und hier von zwei Seiten her stark zusammen gedrückt ist, keilförmig erscheint. Inzwischen ist doch die verschoben dreiseitige oder die unvollkommen und sehr ungleichseitig vierseitige prismatische Grundgestalt unverkennbar, und an der einen Endfläche deutlich ausgesprochen, und die Ähnlichkeit mit den oben beschriebenen und Figur 2 und 5 auf der vorigen Tafel abgebildeten Steinen nachweisbar. Man kann nämlich vier Seiten- und zwei Endflächen, die zum Teil von ziemlich scharfen, wenn gleich sehr ausgeschweiften, und hie und da unterbrochenen Kanten begrenzt werden, deutlich unterscheiden.

Zwei der Seitenflächen, die sich gegen Überstehen, sind sich fast ganz gleich; sie sind breiter als die übrigen, und länglich-viereckig. Gegen das eine, obere Ende sind sie nur wenig verschmälert, aber vertieft und abgeplattet, weil der Stein hier so zusammengedrückt ist, dass er kaum einen Zoll dick erscheint; gegen das andere, untere Ende sind sie etwas mehr verschmälert, aber konvexer, wie denn der Stein hier 3 Zoll dick ist. Sie sind sehr uneben, voll großer, zum Teil ziemlich tiefer, aber sanft sich verlaufender Eindrücke.

Die dritte, zwischen jenen liegende Seitenfläche ist gegen das untere Ende fast so breit, wie die beiden vorhergehenden, nach oben aber sehr verschmälert, weil der Stein von den andern Seiten her so stark zusammengedrückt ist. In der Mitte ist sie etwas vertieft, sonst flach und ebener als jene, da sie nur wenige, zwar große, aber sehr seichte Eindrücke von gewöhnlicher Art hat. Die vierte, der letzteren gegen über liegende Seitenfläche endlich ist unvollkommen, oder gleicht vielmehr einem breit gedrückten Rande. Sie ist sehr schmal, sehr uneben und konvex, und bildet mit den beiden breiten Seitenflächen, zwischen welchen sie liegt, undeutliche, sehr verdrückte, unterbrochene und abgerundete Kanten. Sie gleicht sehr der unvollkommenen vierten Fläche des Figur 2 a auf der vorigen Tafel dargestellten Steines, und noch mehr, ihrer ganzen Beschaffenheit nach, der schmalen Fläche an dem daselbst Figur 4 abgebildeten Steine.

Die obere Endfläche ist undeutlich und unbestimmbar, oder vielmehr sie erscheint, weil der Stein von zwei Seiten her so sehr zusammen gedrückt ist, bloß als ein breiter, abgerundeter Rand. Die untere Endfläche dagegen bildet eine vollkommene, etwas verschobene und ungleichseitig vierseitige Fläche, deren ziemlich spitze Ecken den Seitenkanten, und die ziemlich scharfen Kanten den Seitenflächen entsprechen, und die mit der gleichnamigen Fläche der oben beschriebenen Steine, Figur 2 und 5, große Ähnlichkeit zeigt. Sie ist ziemlich stark ausgehöhlt, und durch etwas kleinere, aber stärkere Vertiefungen als die übrigen Flächen, uneben gemacht.

Nach dieser Beschaffenheit der Oberfläche lässt dieser Stein drei Verschiedenheiten nach seinen verschiedenen Flächen erkennen, wovon die eine Seitenfläche für sich die eine, die derselben entgegen gesetzte, schmale, in Verbindung mit der untern Endfläche, die andere, und die beiden breiten, sich gegenüberstehenden Seitenflächen zusammen, die dritte zeigen.

Die Rinde ist an diesem Steine ganz besonders ausgezeichnet und merkwürdig. Sie ist durchaus von derselben Hauptbeschaffenheit und von gleicher, und zwar von der rauen, ganz vollkommen blattförmig gezeichneten Art (A. a. 3. Gilberts Annalen Bd. 31, S. 56), zeigt aber doch, nach den verschiedenen Flächen, die sie bedeckt, unverkennbar eine zwei-, zum Teil dreifache Modifikation.

An der untern Endfläche, von wo aus die blattförmigen Zeichnungen ihre Richtung nach aufwärts über die Seitenflächen nehmen, erscheint sie noch sehr unvollkommen und undeutlich blattförmig, mehr verworren, runzlicht-aderig, von etwas graulich-schwarzer Farbe, und etwas mattem, fettigem Glanze; an der schmalen, konvexen Seitenfläche ist sie von gleicher Farbe und ähnlichem Glanze, aber schon deutlich blattförmig gezeichnet, und die Blätter streichen gerade nach aufwärts gegen das obere Ende, und bald schlagen sie, bald die Blätter der angrenzenden breiten Seitenflächen über die abgerundeten Kanten. An den beiden breiten Seitenflächen ist sie schon ausgezeichnet blattförmig, und die Richtung der Blätter geht von der Grundfläche nach aufwärts und etwas schief, größten Teils gegen die schmale Seitenfläche hin. Ihre Farbe zieht sich mehr ins Pechschwarze, und ihr Glanz ist etwas stärker und mehr fettig, und beides umso mehr, je mehr sie sich dem oberen Ende und der vierten Seitenfläche des Steines nähert. Auf dieser letzteren endlich ist sie besonders ausgezeichnet und großblätterig, und die Blätter streichen, wie vom Mittel der untern Endfläche aus, schief aufwärts in entgegen gesetzter Richtung nach den angrenzenden breiteren Seitenflächen, und mehr oder weniger selbst über die gemeinschaftlichen scharfen Kanten, so dass sie hier teils am Rande frei anstehen, teils, auf jene Flächen ganz überschlagend, einen mehr oder weniger deutlichen und unterbrochenen Saum auf denselben bilden. Die Rinde ist übrigens auf dieser Fläche von einer sehr dunkel-, fast kohlschwarzen Farbe, mehr seidenartigem, schimmerndem Glanze, und von zarterer Beschaffenheit (die Adern sind nämlich viel feiner und schärfer), und nähert sich überhaupt sehr der strahlig-aderigen Art; auch scheint sie, wo nicht im Ganzen, dünner, doch gleichförmiger zu sein, wenigstens ist sie nicht so, wie an den übrigen Flächen, stellenweise, am Rande der Blätter --- zumal wo sich dieser über den Rücken von Erhabenheiten zieht --- verdickt, angehäuft, und gleichsam wie Ölfarbe mit einem groben Pinsel hingeschmiert. An der oberen Endfläche, die eigentlich, wie bereits erwähnt, bloß einen stumpfen, abgerundeten Rand bildet, ist die Rinde bis gegen das Mittel derselben hin dick, fast wulstig (hie und da wohl auf eine halbe Linie) angehäuft, besonders glatt, und von den breiten Flächen her, wie erstarrendes Pech, gleichsam angeflossen, und zwar scheint es, den abgestoßenen Stellen nach, die obere glatte Schichte zu sein, welche sich hier verdickt hat.

Von unvollkommener Rinde findet sich nur an einer äußerst kleinen, kaum bemerkbaren Stelle, auf jeder der breiten Seitenflächen, eine Spur, wo offenbar die Rinde nicht zusammengeflossen war.

Angedrückte Erde zeigt sich nur an der schmalen konvexen Fläche, und gegen die untere Hälfte der an sie grenzenden breiten Seitenflächen, welches auch die Stellen sind, auf welche der Stein, kraft seines natürlichen Schwerpunktes, gefallen sein sollte.\footnote{Und dieser Richtung im Falle scheint die Modifikation der Rinde auf den verschiedenen Flächen sehr zu entsprechen.}

Die Abbildung stellt diesen ausgezeichneten Stein, auf der schmalen Seitenfläche liegend, etwas schief gewendet vor, und zeigt die eine ebenere Seitenfläche fast in gerader, eine der breiteren in schiefer, und die untere Endfläche in saucierter Richtung.\footnote{In Gilberts Annalen Bd. 31 ist von diesem Steine Tafel 1 Fig. 2 bereits eine verkleinerte und skizzierte Abbildung, aber von einer der breiten Seitenflächen, gegeben worden; ich habe daher absichtlich hier eine andere Fläche zur Darstellung gewählt.}

\subsection{Zweite Figur.}
\paragraph{}
Einer der größeren von den bei Stannern gefallenen Steinen, 1 Pfund 12 Loth wiegend. Vollkommen ganz und um und um mit Rinde bekleidet.

Dieser wurde am 28. Mai in Folge der gemachten Aufforderung, die gefallenen Steine aufzusuchen, von einem Landmanne ebenfalls in der Nähe des Marktes Stannern, eigentlich bei dem Dorfe Sorez, noch mehr im Mittelpunkte des befallenen Flächenraumes, als der vorhin beschriebene (und zwar kaum 1000° O. S. O. von diesem, und etwa 600° in gleicher Richtung von der Kirche von Stannern, und beinahe in ganz gleichem Abstande von den beiden äußersten Fallstellen in N. und S.), aufgefunden. (Situations-Plan Nr. 35.)

Er stellt eine zwar etwas unvollkommene, aber nur wenig verschobene, ungleichseitig vierseitige Pyramide vor, deren abgeflachte Spitze stark aus dem Mittel gedrückt und auf eine Seite übergebogen ist.

Die Grundfläche, welche ganz flach und beinahe vollkommen eben und platt, ohne alle Eindrücke und Vertiefungen ist (ein Fall, der bei einer Fläche von solcher Ausdehnung höchst selten an einem Steine vorkommt), bildet ein etwas verschobenes und ungleichseitiges Viereck, dessen ziemlich, und gewisser Maßen ausgezeichnet gerade laufende scharfe, fast schneidende Kanten, ebenso vielen, ziemlich senkrecht aufgesetzten, nach oben verschmälerten und nach einer Seite hingebogenen Seitenflächen, und dessen Ecken ebenso vielen, ziemlich scharfen, aber sehr verdrückten und ausgeschweiften Seitenkanten entsprechen. Eine Ecke der Grundfläche ist ziemlich spitzig, und die ihr diagonal gegenüberstehende etwas abgestützt; eine dritte Ecke ist stärker, und die ihr entgegen gesetzte vierte noch mehr abgestützt, so dass durch letztere die Grundfläche beinahe fünfseitig gemacht wird. Diese Abstumpfungen gehen etwas schief von unten nach aufwärts und außen, und bilden Dreiecke, deren Basis auf der Grundfläche ruht, und deren spitzer oberer Winkel sich allmählich in die Seitenkante verliert. Solcher Gestalt wird die vierseitige Form der Pyramide durch sie nicht verändert, und die Grundfläche zeigt immer noch eine große Ähnlichkeit mit jener der meisten bereits beschriebenen Steine, so wie die Form im Ganzen, welche den Grund-Typus deutlich genug ausspricht, mit jener mehrerer derselben.

Die obere Endfläche ist nur unvollkommen, und eigentlich die horizontale Fortsetzung einer schief aufsteigenden Seitenfläche.

Zwei aneinandergrenzende Seitenflächen sind, zumal die eine, breiter als die andern, und ziemlich stark gewölbt; die beiden andern, gegen deren gemeinschaftliche, sehr verdrückte und beinahe ganz verschwundene Kante (welcher auch die am stärksten abgestützte Ecke der Grundfläche entspricht) die abgeflachte Endspitze hingedrückt und übergebogen ist, sind bedeutend schmäler und etwas vertieft.

Die an diesem Steine auf allen Flächen, außer der ganz ebenen Grundfläche, vorkommenden Eindrücke, sind von ganz eigener Art, wie ich sie an keinem Steine von Stannern (deren ich doch, mit Inbegriff der größeren Bruchstücke, bei 100 zu Gesicht bekam), noch an irgendeinem Meteor-Steine, wieder fand, ausgenommen --- obgleich nicht ganz so deutlich ausgesprochen --- an der Grundfläche des nächst zu beschreibenden. Sie sind nämlich verhältnismäßig sehr klein, aber tief und grubenartig, nicht so breit wie gewöhnliche Eindrücke und sanft verlaufend, sondern ziemlich scharf gerandet, gleichsam kantig, wie von grobkörnigen oder bröckligen Absonderungen entstanden, und geben der Oberfläche, da sie ziemlich häufig sind, ein klein-wellenförmiges Ansehen. Auf den beiden schmälern, konkaven Seitenflächen zeigen sie schon eine Modifikation; sie sind nämlich hier größer, aber seichter und mehr breit verlaufend, auch minder zahlreich. Die obere Endfläche stimmt hierin mit den andern Seitenflächen überein.

Auch die Rinde ist an diesem Steine von eigentümlicher, und der seltenen, strahlig- und netzartig-aderigen Art, aber durchaus, über den ganzen Stein, von einerlei Hauptbeschaffenheit, die nur eine Haupt- und eine dieser letzteren untergeordnete Modifikation erkennen läßt.\footnote{Und diese Modifikationen zeigen eine Übereinstimmung mit der Beschaffenheit der Oberfläche und mit der Richtung, welche die Flächen im Niederfallen des Steines, kraft dessen individuellen Schwerpunktes, höchst wahrscheinlich gehabt haben möchten.}

Auf der ebenen Grundfläche ist sie nämlich ausgezeichnet auseinanderlaufend strahlig; die ziemlich erhabenen, zarten und scharfen runzelartigen Adern laufen, wenig geschlängelt und fast gar nicht ramifiziert, von einem körnig-rauen Mittelpunkte --- der aber nicht ganz im Mittel der Fläche liegt --- strahlenförmig auseinander und gegen die Kanten hin. Die Zwischenräume zwischen diesen, eben nicht sehr gedrängten Strahlen, sind durch zartere Runzeln und Adern, die zum Teil Äste derselben sind, und durch erhabene Punkte und Tröpfchen rau. Übrigens ist die Rinde hier beinahe kohlschwarz, und von einem ziemlich starken, schimmernden, seidenartigen Glanze. An allen übrigen Flächen dagegen erscheint sie netzartig-aderig, das ist, die sehr erhabenen und scharfen, zwar strahlenförmig verlängerten, aber als Folge der Unebenheiten verschiedentlich und stark gebogenen und gekrümmten Adern bilden durch ihre Verbindung unter sich ein unregelmäßiges, weitschichtiges Netz, dessen Maschen oder Zwischenräume ebenfalls durch zartere, kürzere Adern und Runzeln rau sind. An den Erhabenheiten, welche die Vertiefungen begrenzen, sowie an den meisten Kanten, bildet die Rinde ziemlich hohe und scharfe, zart gefaltete Nähte, welche der Oberfläche ein ganz eigentümliches und besonders raues Ansehen geben.\footnote{Die Erhabenheit und Schärfe der Adern und Nähte der Rinde, insbesondere an diesem Steine, sprechen wohl sehr gegen die vermeintliche Flüssigkeit derselben, die selbst noch im Momente des Auffallens der Steine Statt haben soll; so wie andererseits die Form und die Schärfe der Kanten, nicht nur an diesem, sondern an den meisten Steinen, gegen die präsumierte Weichheit, Plastizität, teigige Schmelzung (\emph{fusion pateuse}) der Steinmasse in demselben Momente zu streiten scheinen; obgleich nicht in Abrede zu stellen ist, das sie sich eben so wenig mit dem höchst spröden, leicht brüchigen und fast zerreiblichen Zustande, in welchem, wenigstens die Steine von Stannern, selbst sehr kurze Zeit nach ihrem Falle befunden worden sind, und sich noch befinden, vereinbaren lassen, und mit welchem letzteren überhaupt die vollkommene Integrität so vieler, mitunter ansehnlicher und ziemlich gewichtiger Steine im offenbarsten Widerspruche steht.} Übrigens hat die Rinde hier eine mehr ins Graue ziehende schwarze Farbe, und einen etwas schwächeren, aber noch mehr schimmernden, seidenartigen Glanz.

An den beiden konkaven Flächen zeigt sich insofern eine kleine Modifikation von dieser letzteren Beschaffenheit der Rinde, dass sie hier etwas dunkler schwarz ist (gleichsam im Übergange von jener der Grundfläche in jene der andern Seitenflächen), schwächere Nähte, minder raue Zwischenräume, und, wenigstens gegen die Endspitze hin, eine schwache Anlage zu blattförmigen Zeichnungen zeigt.

Übergeflossen oder Säume bildend findet sich die Rinde an diesem Steine nirgendwo, und unvollkommen (und zwar im höchsten Grade, aber nur als Folge einer oberflächlichen Absprengung eines äußerst kleinen Stückes derselben) zeigt sie sich nur auf einer sehr kleinen Stelle auf einer der konkaven Flächen.

An der Grundfläche, an einer der konvexen und an einer konkaven Seitenfläche, gegen welche letztere die Endspitze gebogen ist, zeigt sich stellenweise etwas eingedrückte Erde.

Dieser, durch die seltene Art von Überrindung besonders ausgezeichnete Stein, ist von seiner --- in dieser Beziehung merkwürdigsten --- Grundflache, die zugleich dessen Form am besten erkennen macht, dargestellt.\footnote{In Gilberts Annalen Bd. 31, Tafel 2 Figur 1. 2., ist bereits von diesem Steine eine skizzierte Darstellung von zwei Ansichten gegeben worden, und zwar die eine von den beiden gewölbten Seitenflächen mit ihrer gemeinschaftlichen Kante, die andere von der Grundflache genommen.}

\subsection{Dritte Figur.}
\paragraph{}
Ebenfalls einer der größeren von den bei Stannern gefallenen Steinen, von 1 Pfund 7 Loth am Gewichte, welcher am 29. Mai, auch nahe bei Stannern selbst, zwischen den Dörfern Sorez und Falkenau, demnach ebenfalls im Mittelpunkte des befallenen Flächenraumes (und zwar nur etwa 500° mehr nördlich als der letztbeschriebene, und etwa 300 östlich von der Kirche von Stannern) aufgefunden und der Kommission übergeben wurde. (Situations-Plan Nr. 43.)

Es ist derselbe vollkommen ganz, so, wie er wirklich zur Erde gefallen, obgleich er, bei oberflächlicher Betrachtung, das Ansehen hat, als wäre ein beträchtliches Stück davon nach der Hand gewaltsam abgeschlagen, und die künstlich erzeugte Bruchfläche durch absichtliche oder zufällige Beschmutzung so verändert worden, dass sie nicht mehr vollkommen einer ganz frischen der Masse gleichet. Diese Vermutung findet noch überdies in der offenbaren Verunstaltung der Form, deren ursprüngliche größere Regelmäßigkeit noch unverkennbar ist, durch Verlust an Masse, eine auffallende Bekräftigung. Es hat mit dieser Vermutung insoweit auch die vollste Richtigkeit, dass jenes Bruchansehen und diese Formverunstaltung wirklich von einem späteren, nach der ursprünglichen Bildung (Individualisierung) dieses Steines und nach dessen totaler Inkrustierung Statt gehabten Verluste an Masse herrühre; allem es zeigt sich bei näherer Betrachtung unwiderleglich, dass dieser Verlust noch vor dem wirklichen Niederfallen oder Auffallen des Steines, und während seines Zuges durch die Luft, durch natürliche Absprengung und Lostrennung eines Stückes entstanden sein müsse, indem die vermeintlich künstliche Bruchfläche wirklich mit wahrer, obgleich nicht vollkommen ausgebildeter Rinde bedeckt erscheint.\footnote{Dieser Stein war es auch, an dem ich jene, für die in jeder Beziehung so schwierige Erklärung der Bildung der Rinde an den Meteor-Steinen, gewiss sehr wichtige Beobachtung, nämlich über das Vorkommen derselben in verschiedenen Graden von Unvollkommenheit, oft selbst an ein und demselben Steine, zuerst machte, und zu machen nicht wohl verfehlen konnte, da sie an diesem Steine so ausgesprochen und in die Augen springend ist, und welche so wie die ebenso vorkommenden Hauptverschiedenheiten und Modifikationen derselben, wie mir deucht wohl unbestreitbar, eine stufenweise und allmähliche --- ich will gerade nicht behaupten, langsame, aber doch wiederholte, fortgesetzte, und während der ganzen Periode des Falles der einzelnen Steine fortdauernde --- Bildung der Rinde voraussetzen. Es war mir dann ein Leichtes, dieses, gar nicht ungewöhnliche Vorkommen der Rinde, in an sich schwerer erkennbaren Graden, nicht nur an den meisten Meteor-Steinen von Stannern, sondern auch an jenen von andern Ereignissen, deren Rinde, ihrer Natur nach, weit weniger geeignet ist, diesen Zustand erkennen zu lassen --- daher er auch bis dahin (1808), und wie es scheint, noch bis jetzt von niemand beobachtet wurde --- aufzufinden und nachzuweisen.}

So unregelmäßig die Form dieses Steines nun auch ist, so ist doch in seiner Begrenzung durch wahre Flächen, und in deren Verbindung, Ausdehnung und Richtung, der Grund-Typus zur verschoben vierseitigen Pyramide, und damit die Ähnlichkeit mit den meisten der beschriebenen Steine deutlich genug noch ausgesprochen, und man müsste diesen Stein, trotz dessen starker Abplattung und anscheinender Zurundung, nach zwei End- und vier Seitenflächen beschreiben, zöge man auch nur die verschiedene Beschaffenheit seiner Oberfläche und die Modifikationen der Rinde in Betrachtung.

Die eine, bedeutend größere Endfläche, stellt ein verschobenes, aber ziemlich gleichseitiges Viereck vor, dessen Ecken abgestumpft, und mehr oder weniger zugerundet, und dessen ziemlich gerade laufende Ränder, die mit den mehr oder weniger schief aufsteigenden Seitenflächen ziemlich scharfe Kanten bilden, ausgeschweift sind. Sie ist in der Mitte etwas gewölbt, sonst ziemlich flach, und durch sehr viele kleine, aber ziemlich tiefe, grubenartige Eindrücke auf eben die Art und ebenso sehr uneben, wie die Seitenflächen des zuvor beschriebenen Steines.

Drey aneinandergrenzende Seitenflächen sind sehr niedrig. Die eine steigt beinahe senkrecht; die andere, unter einem ziemlich spitzen Winkel in eine deutliche, ziemlich scharfe Kante mit ihr zusammenstoßende, etwas schief; die dritte, unter einem sehr stumpfen Winkel, mit ersterer eine sehr undeutliche, ganz abgerundete Kante bildende, noch mehr schief von der Grundfläche in die Höhe. Alle haben nur wenige, seichte, aber große und breit verlaufende Eindrücke von gewöhnlicher Art.

Die vierte Seitenfläche ist, zumal in ihrem Mittel, wo sich der obere Rand in eine stumpfe Spitze verliert --- von der eine ziemlich erhabene scharfe Kante bis zum Rande der Basis läuft, und diese Fläche der Länge nach in zwei Hälften teilt, auch gewisser Maßen eine fünfte unvollkommene Ecke an der Grundfläche bildet --- beträchtlich höher als jene, und erhebt sich zwischen den beiden schiefern Seitenflächen, mit welchen sie in etwas undeutliche Kanten zusammen stößt, beinahe senkrecht von der Grundfläche Sie ist sehr uneben, ihre Unebenheiten rühren aber nicht von gewöhnlichen Eindrücken her, sondern stellen natürliche Unebenheiten einer Bruchfläche der Steinmasse selbst vor.

Die obere Endfläche endlich steigt von zwei Seitenflächen --- der einen etwas schiefen und der senkrechten, niederen --- mit welchen sie unter einem sehr stumpfen Winkel in etwas undeutliche Kanten zusammen stoßt, eine Strecke lang schief aufwärts, als wenn sie eine gewölbte Fläche bilden wollte, wird aber bald durch eine neue Fläche unterbrochen, die wie von einer zufälligen, späteren und gewaltsamen Abschlagung der Endspitze entstanden zu sein scheint. Diese Fläche hat einen rundlichen Umriss, der aber doch einiger Maßen den Seitenflächen und Kanten entspricht, erhebt sich schief gegen den Rand und die Spitze der einen senkrechten höheren, und stößt mit der vierten schiefen Seitenfläche mit einem ziemlich scharfen kantenartigen Rand zusammen. Sie sieht ebenso rau und uneben aus, wie die eine hohe Seitenfläche, und folglich wie eine gewöhnliche Bruchfläche der Steinmasse, indes ihre Basis gegen die zwei ersteren Seitenflächen hin, hinsichtlich ihrer Beschaffenheit und Eindrücke, ganz diesen gleicht. So verschieden solcher Gestalt die Oberfläche dieses Steines nach den verschiedenen Flächen desselben erscheint; so verschieden und offenbar in Übereinstimmung mit jenen Verschiedenheiten zeigt sich auf eine höchst merkwürdige Weise die Beschaffenheit der Rinde an demselben.

Auf der größeren End- oder Grundfläche desselben ist sie nämlich genau und in allen Beziehungen, so wie an den Seitenflächen des vorhin beschriebenen Steines, von der dichten, festen, rauen, netzartig-aderigen Art (A. b. 2), mit sehr erhabenen Adern, häufigen, scharfen Nähten und sehr rauen Zwischenräumen; nur zieht sich hier die Farbe mehr ins Pechschwarze, und der seidenartige Glanz nähert sich mehr dem fettigen; auch scheint die Rinde hier dünner zu sein, indem an einigen Stellen, zumal gegen die eine raue Seitenfläche hin, die untere braune Schichte, und auf der ganzen Oberfläche der, wie es scheint, schwerer in Rinde umwandelbare, weiße Gemengteil der Steinmasse (wie an dem Tab. 5 Fig. 4 vorgestellten Steine) in Gestalt einzelner und zusammen gehäufter, weißer, gelblicher und bräunlicher Körner, die kaum die Größe der Hanf- oder Hirsekörner haben, durchscheint.

An den drei, aneinandergrenzenden, auch sonst gleichartigen Seitenflächen dagegen ist sie von der gewöhnlichsten einfach-aderigen Art (A. a. 2), von dunkelschwarzer Farbe und von dem gewöhnlichen fettigen Glanze. Doch zeigt sich auch hier eine kleine Modifikation, indem an einer derselben, und zwar an der am schiefsten aufsteigenden (auch unebeneren) die Rinde glatter, glänzender, anscheinend dünner, und mit einer Anlage zur blätterigen Zeichnung sich zeigt; und was besonders merkwürdig ist, auf ihr, vom Rande der Grundfläche her, die Rinde übergeflossen erscheint und einen Saum bildet, indes sie an den beiden andern Flächen von jener Fläche her gleichförmig über die Ränder oder Kanten fortläuft. An der oberen, mit der neuen Bruchfläche gebildeten Endkante steht die Rinde dieser Fläche angehäuft, gleichsam als ein aufrechtstehender, ziemlich scharfer Rand an.

An der vierten höheren Seitenfläche erscheint die Rinde sehr ungleichförmig, da sie sehr oft in der Bildung unterbrochen worden zu sein scheint; hin und wieder ist sie deutlich aderig und rau; hie und da aber, zumal an der einen Hälfte, wo auch an der Endkante von der Grundfläche her ein Saum gebildet wird, zeigt sich eine Anlage zur blattförmig gezeichneten. Sie ist übrigens sehr dicht, schwarz und fettig-glänzend, und an den erhabensten Stellen und Punkten, so auch an der Teilungskante, dick und kompakt. An den tiefen Stellen ist sie dünner, und fehlt an manchen Plätzen sogar ganz, wo die Grundmasse mit bräunlicher Farbe zum Vorschein kommt. In dieser zeigt sich der weiße Gemengteil der Steinmasse in Gestalt von weißen Körnern, und es werden auf ihr nur einzelne oder zusammen gruppierte, und mehr oder weniger ineinander geflossene schwarze Tröpfchen Rinden-Substanz dem freien Auge sichtbar. (Niedrigster Grad der unvollkommenen Rinde. D. 1.)

An der oberen Endfläche endlich, das ist, insoweit eine solche, außer der neuen Bruchfläche, vorhanden ist, und von den beiden Seitenflächen gebildet wird, ist die Rinde ganz genau von derselben Beschaffenheit in jeder Beziehung wie an diesen letzteren, und zieht sich auch von denselben geradezu, ohne alle Unterbrechung der Adern, auf diese Fläche herüber; nur dass sie hier hin und wieder etwas abgerieben ist.

Ganz anders zeigt sich nun die Rinde an jener später entstandenen Bruchfläche, die im Ganzen ein raues, mattes, erdgrau-bräunliches Ansehen hat. Hier ist in dem bräunlichen Grunde der weiße Gemengteil nicht nur noch der Farbe nach erkennbar, und nur selten gelblich oder bräunlich, sondern selbst hie und da noch ganz erdig und fast kreideweiß, und die Rinden-Substanz zeigt sich nur, vorzüglich auf dem Rücken der scharfen, gleichsam kantigen Erhabenheiten, wie ausgeschwitzte Tropfen, die entweder einzeln dastehen, oder zu Perlenschnüren, Adern oder kleinen Flecken und Streifen zusammengeflossen sind. Gegen die Ränder hin ist die Rinden-Substanz häufiger, an den Rändern selbst aber ist sie von den angrenzenden Flächen her angehäuft, und bildet einen deutlichen Abschnitt, so dass gegen die beiden aderigen Seitenflächen hin, wo die konvex sich erhebende Endfläche in diese Bruchfläche sich allmählich verliert, durch die Rinde selbst erst ein scheinbarer Rand gebildet wird. (Mittlerer Grad der unvollkommenen Rinde. D. 2.).\footnote{Es zeigt dieser Stein demnach eine fünffache Verschiedenheit der Rinde an seinen verschiedenen Flächen, wovon drei, nämlich die an den drei niederen Seitenflächen und der Basis der oberen Endflache; dann die der vierten hohen Seitenflache und der neuen Bruchfläche; endlich die der Grundfläche --- wenn sie nicht etwa Modifikation dieser letzteren ist --- Hauptverschiedenheiten zu betrachten kommen, von welchen der Grund hauptsächlich in der ungleichzeitigen Entstehung der Flächen, und folglich der ungleichen Dauer des Rindebildungs-Prozesses zu suchen sein dürfte: --- zwei aber, nämlich die an der einen schiefern Seitenflache von jener der beiden andern, und die an der vierten hohen Seitenflache von jener der neuen Bruchfläche, wohl nur Modifikationen vorstellen, die von der Richtung des Steines im Falle, und von der dadurch abgeänderten Einwirkung des Luftstromes, herrühren möchten.} 

An der Grundfläche sowohl als an allen Seitenflächen, ist hie und da etwas, obgleich nur äußerst wenig, Erde noch anklebend.

Die Abbildung stellt diesen lehrreichen Stein auf seiner Grundfläche liegend und so vor, dass nebst den drei niederen Seitenflächen die obere Endfläche mit der unvollkommen überrindeten Bruchfläche ganz zur Ansicht kommt.\footnote{In Gilberts Annalen Bd. 31, Taf. 3, Fig. 2, ist bereits auch von diesem Steine eine Darstellung versucht worden, die aber durch die Kolorierung sehr verunstaltet worden ist.}

\subsection{Vierte Figur.}
\paragraph{}
Ein mittelgroßer Stein von der Begebenheit bei Stannern, 1 Pfund 1 Loth wiegend, welcher am Tage des Ereignisses selbst, und zwar ebenfalls ganz nahe bei Stannern, auch zwischen den Dörfern Sorez und Falkenau, demnach ebenfalls im Mittelpunkte des befallenen Flächenraumes (und zwar kaum mehr als 100° südlich vom vorhin beschriebenen entfernt) aufgefunden, und dem Pater Caplan in Stannern überbracht wurde, der ihn am 29. Mai der Kommission überreichte. (Situations-Plan Nr. 40.)

Auch dieser Stein ist vollkommen ganz, und so wie er zur Erde gekommen, erhalten worden, obgleich derselbe noch ungleich mehr als der vorhin beschriebene, auch selbst bei näherer, ja wohl ganz naher Betrachtung, das Ansehen eines großen Bruchstückes, oder der Hälfte eines entzwei geschlagenen Steines hat, wofür er auch lange Zeit von mir und jedermann gehalten wurde, indem eine ganze Seite desselben eine beinahe ganz frische, nur etwas dunkler gefärbte, gleichsam beschmutzte, Bruchfläche zeigt.\footnote{Schwerlich würde ich selbst diese Fläche für das, was sie wirklich ist, so bald erkannt haben, wenn nicht der zuvor beschriebene Stein, und ähnliche, mancherlei Abstufungen der unvollkommenen Rinde aufs klarste aussprechende Stellen an vielen andern, mich aufmerksam gemacht hätten.}

Seine Gestalt ist unregelmäßig und schwer zu beschreiben; doch bilden alle bestimmbaren Flächen, und selbst die scheinbar frische Bruchfläche, ein verschobenes Viereck, und am ganzen Steine lassen sich noch acht Ecken, acht End- und vier Seitenkanten am vollkommensten nachweisen, so dass sich die Grund- oder ursprüngliche Absonderungsgestaltung leicht denken, und die Ähnlichkeit in der Total-Form mit den meisten der zuvor beschriebenen Steine wieder nicht verkennen lässt.

Die Oberfläche aller vollkommen überrindeten Flächen --- wovon wieder zwei der an einander grenzenden Seitenflächen etwas gewölbt, die zwei andern etwas vertieft sind, die als Grundfläche zu betrachtende aber, welche der neueren Bruchfläche gegen über gestellt ist, flach und ziemlich eben erscheint --- hat wenige, aber große und breit verlaufende Eindrücke gewöhnlicher Art; ein paar tiefere, schärfer begrenzte, sind nicht sowohl bloßen Eindrücken, als vielmehr einem Verluste der Masse durch --- mit der Individualisierung des Steines und der Bildung der Rinde im Ganzen --- gleichzeitige Lostrennung einzelner kleiner Stücke zuzuschreiben.

Die Rinde ist fast durchaus dieselbe, wenigstens von einer und derselben Hauptbeschaffenheit an allen diesen Flächen, und ganz und in jeder Beziehung von der gewöhnlichsten, einfach und verworren-aderigen Art, wie z. B. an den Seitenflächen des vorhin beschriebenen Steines. Sie zeigt weder Säume noch Nähte, bildet aber hie und da ziemlich lange, scharfe und erhabene Adern, die eine ziemliche Strecke über eine Kante oder den Rücken von Erhabenheiten laufen, doch keine bestimmte Richtung haben.

An einer ziemlich großen, stark hervorragenden, sehr unebenen Stelle, eigentlich an der ganzen einen gewölbten Seitenfläche, zeigt sich --- als Modifikation --- eine Anlage zur blattförmig gezeichneten Rinde; auch scheint da die matte untere Schichte bräunlich durch, und in ihrer Nähe zeigen sich an den Kanten der angrenzenden Flächen Anhäufungen von Rinde, von diesen letzteren her, die sich Säumen nähern. Übrigens ist die Rinde von der gewöhnlichen dunkelschwarzen Farbe, und dem gemeinen, ziemlich starken, etwas fettigen Glanze.

Das Merkwürdigste an diesem Steine ist nun jene dem unbewaffneten Auge ganz rindenlos erscheinende neuere Bruchfläche, welche die größte und gewisser Maßen regelmäßigste am Steine ist.

Es bildet dieselbe, obgleich sie sich auch über einen Teil einer angrenzenden Fläche ausdehnt, ein ziemlich gleichseitiges, nur etwas verschobenes Viereck, welches von drei Seiten her durch die anstehende Rinde der angrenzenden Flächen, auf der vierten aber durch die scharfe Bruchkante der Steinmasse, ausgeschweift zwar nach den vorkommenden Unebenheiten der Flächen, aber scharf begrenzt wird. Ihre ziemlich spitzen Ecken entsprechenden Seitenkanten, und die scharfen Ränder den Seitenflächen des Steines, und sie hat ganz das Ansehen, als wäre ein noch Mahl so großer Stein zerspalten worden, und habe durch einen besonders glücklichen, ziemlich ebenen und geraden Bruch diese Bruchfläche gegeben. Sie ist sehr uneben, aber nicht von der Art, wie die überrindeten Flächen zu sein pflegen (durch meist rundlichte, allmählich sich erhebende, und sanft in die Erhabenheiten breit verlaufende, sondern durch sehr ungleichförmige und winklige, von senkrechten, oder nur wenig schiefen und ziemlich scharfkantigen Erhabenheiten begrenzte Vertiefungen), vielmehr sieht sie gerade so aus wie eine frische künstliche Bruchfläche der Steinmasse, hat aber weder das frische Ansehen, noch ganz die Farbe einer solchen, sondern ist schmutzig oder bräunlich-grau, hie und da mit bläulichweiß und aschgrau gemischt. Die Masse scheint dichter, fester und weniger rau zu sein, und wenn man sie mit der einfachen Lupe betrachtet, so sieht man hier und da, zumal an den erhabenen Stellen, an den Kanten der scharfen Erhabenheiten, und der durch Risse getrennten Partien, die angefangene Erzeugung der schwarzen Rinden-Substanz in Gestalt kleiner Tropfen, Perlenschnüre oder Einfassungen. An den Rändern stößt die Rinde der vollkommen inkrustierten Seitenflächen dicht an, so dass, wie gesagt, durch dieselbe eigentlich der wahre Rand dieser Fläche selbst erst gebildet wird; und obgleich diese Rinde hier scharf abgeschnitten und nicht viel dicker ist, als an einer künstlichen Bruchfläche, so zeigt sie doch keine Spuren eines Bruches; denn sie ist da eben so dicht und glänzend, wie an der Oberfläche, und lässt die zweite untere, poröse, matte Schichte nicht erkennen. (Haupt-Kriterium eines solchen, vor dem wirklichen Niederfalle und noch in der Luft entstandenen, natürlichen Bruches von einem künstlichen.) Offenbar ist sie an einigen Stellen, zumal gegen jene Seitenfläche hin, wo die Rinde sehr kompakt, schwarz und aderig ist, von daher wie übergeflossen oder übergedrückt, wenigstens weiter fortschreitend, so dass sie einen beträchtlichen Saum oder eine Einfassung auf dieser Fläche, über die Kante her, bildet. An einer scharfen Ecke erstreckt sich diese Einfassung bis auf 1 1/2 Linie weit auf diese Fläche hinein; die Steinmasse ist in der angrenzenden Gegend auch dunkler, und zeigt häufigere Tropfen.

Eine, dieser ganz ähnliche, aber ungleich kleinere Fläche, findet sich an demselben Steine gegen den unteren Rand der einen Seitenfläche (die von jener Fläche unter einem Winkel von beiläufig 100° abweicht), mitten in der Rinde, gerade als wenn hier ein Zoll großes (aber allem Ansehen nach nur sehr dünnes) Stück der Steinmasse, das etwa ursprünglich eine hervor stehende Ecke oder eine Erhabenheit gebildet haben mochte --- nachdem die Hauptfläche und überhaupt der ganze Stein bereits überrindet war --- und zwar ganz gleichzeitig mit jenem Stücke, das obige neuere Bruchfläche bildete, mit Gewalt abgesprengt worden wäre, und als wenn, hier wie dort, das Rinden bildende Agens (der Rindenbildungs-Prozess) nicht mehr Intensität oder Zeit genug gehabt hätte, die erzeugte Bruchfläche vollkommen zu inkrustieren (was wohl unwiderleglich, wirklich und wörtlich der Fall gewesen sein muss).

Diese beiden Flächen zeigen die unvollkommene Rinde in ihrem höchsten Grade (D. 3), und zwar von bedeutender Ausdehnung, wie ich sie, aber meistens nur auf sehr kleinen Stellen vorkommend, auf den meisten der beschriebenen Steine nachgewiesen habe.\footnote{Dieser kostbare Stein zeigt demnach eine zweifache Hauptverschiedenheit der Rinde, und zwar gerade die extremsten Punkte von ihrer Ausbildung beisammen, die wohl die entferntesten Zeit-Momente der Rindebildungs-Periode, und die heterogensten Wirkungsgrade des Rindebildungs-Prozesses zu bezeichnen scheinen --- und eine, auch wohl zwei Modifikationen; erstere nämlich an der einen gewölbtern Seitenfläche, als Modifikation der dunkleren, raueren, an den übrigen vollkommen überrindeten Flächen vorkommenden Rinde; und letztere etwa an einer der, an jene große Bruchfläche angrenzenden, obiger gerade entgegen gestellten Seitenflächen, worüber sich zum Teil jener Bruch fortsetzte, die Masse aber schon weit dunkler, und die Rinde bereits in Flecken und Streifen (D. 1) sich zeigt.}

Von eingedrückter Erde zeigt sich etwas an der, der neuern Bruchfläche entgegen gesetzten, als Grundfläche betrachteten, und an der größeren, gewölbten Seitenfläche.

Die Abbildung zeigt diesen belehrenden Stein, auf einer Seitenfläche aufgestellt, von jener merkwürdigen, großen, neueren Bruchfläche, und zwar so, dass das Licht von jener Seite einfällt, wo sich die scharfe Kante und Ecke mit dem übergeschlagenen Rindensaume befindet.\footnote{Auch von diesem Steine, und von derselben Ansicht genommen, findet sich in Gilberts Annalen Bd. 31, Taf. 3, Fig. 1, eine frühere Abbildung, die aber durch die Kolorierung gar sehr an Deutlichkeit verloren hat.}

\subsection{Fünfte Figur.}
\paragraph{}
Ein 3 1/2 Loth wiegendes Bruchstück eines großen, ursprünglich 4 Pfund schwer gewesenen Steines von Stannern, welcher am Tage der Begebenheit selbst, von dem Oberjäger von Iglau, gegen den Ort Teschen zu, am westlichen Teile des befallenen Flächenraumes von dessen Mittelpunkte, und zwar am entferntesten Punkte daselbst (etwa 1300° westlich von der Kirche von Stannern, und bei 3400° süd-westlich vom äußersten Punkte in N., und bei 4500° nord-westlich vom äußersten Punkte in S., wo die entferntesten Steine gefallen waren) gefunden, aber zerschlagen, und wovon nur die größere Hälfte, von 2 Pfund 12 Loth am Gewichte, am 29. Mai an die Untersuchungs-Kommission abgegeben wurde. (Situations-Plan Nr. 63.)

Es zeigte diese größere Hälfte des Steines, außer den frischen Bruchflächen, größten Teils eine sehr raue, grob-runzlicht-aderige Rinde von dunkelschwarzer Farbe, und dem gewöhnlichen fettigen Glanze, die aber sehr häufig und bedeutend fleck- und stellenweise abgerieben oder abgesprungen, das ist, von der obersten schwarzen, glänzenden Schichte entblößt, und hier braun, matt und zart porös war (A. a. 1. Gilberts Annalen Bd. 31, S. 56 im ausgezeichnetsten Grade). Da dieses Stück übrigens nichts Auszeichnendes hatte, so ward dasselbe zum Behufe der beabsichtigten Versuche, und um mehrere Mitteilungen machen zu können, in viele Bruchstücke zerschlagen, wovon nun dieses eines ist, welches für die Sammlung zurückbehalten wurde.

Es zeigt dasselbe, von der einen konvexen Außenseite, die oben beschriebene Rinde im vollkommensten Grade, von der andern aber eine frische Bruchfläche von der gewöhnlichen Beschaffenheit der Masse dieser Steine; nur mit dem Besondern, dass auf derselben, zwar nur gegen den Rand des Bruches, und folglich gegen die äußere Rinde hin, aber doch hie und da beinahe einen halben Zoll tief von der Oberfläche einwärts, und zwar an Stellen, wo an dieser vor dem Zerbrechen des Steines gar keine Risse oder Sprünge der Masse zu beobachten waren, ziemlich große Flecke von Rinden-Substanz mitten in oder dermal vielmehr auf der ganz unveränderten Steinmasse zur Ansicht kommen.

Diese Flecke liegen zum Teil dicht an der Oberfläche, und hängen mit der äußern Rinde wirklich zusammen, als wenn diese hineingeflossen wäre; einige liegen aber weiter ab, ganz isoliert, und sind von durchaus unveränderter Steinmasse, selbst von eingestreuten, metallisch glänzenden Kies-Bröckeln und Punkten umgeben. Einige derselben sind glänzend schwarz, wie die äußere Rinde, viele matt schwarz, wie die untere Schichte derselben zu sein pflegt, die meisten aber sind mehr oder weniger von der Steinmasse bedeckt, die beim Zerschlagen des Steines daran festblieb.

Die Größe und Gestalt dieser Flecke ist sehr verschieden, ihr Umriss ist aber nie rundlich, sondern vielmehr winkelig und vieleckig; ihr Rand scharf begrenzt und wie gebrochen, und ihre Dicke beträgt nicht mehr als die der Außenrinde. Eingeknetet in die Masse sind diese Flecke keineswegs, denn sie erscheinen nur als dünne Lagen, und verursachen, dort wo sie sich finden, eine gleichsam schalige oder schiefrige Absonderung der Steinmasse.\footnote{Obgleich ich mich zur Zeit außer Stande fühle, von der Bildung der Rinde an den Meteor-Steinen überhaupt, und insbesondere von der Entstehung derselben im Innern der Steinmasse, sowohl in Gestalt solcher Flecken (in welcher sie jedoch am seltensten, und wohl nie weit von der Oberfläche entfernt vorkommen, und füglich noch der Einwirkung des Rinde bildenden Agens von Außen her zuzuschreiben sein dürfte), als in Gestalt eingestreuter Punkte (in welcher sie inzwischen nur bei sehr lockeren Meteor-Steinen, z. B. bei jenen von Chassigny (Langres) deutlich, weniger bei den Steinen von Stannern, und bei beiden selbst höchst problematisch (ob nicht Chrom-Eisen oder Eisen-Oxyd?), bei Meteor-Steinen von festem Kohäsions-Zustande und dichtem Gefüge meinen Untersuchungen nach, selbst nicht als Spur erscheint), als vollends in Form von Adern, Gängen, Schichten und Lagen (deren Substanz man für einerlei mit jener der Rinde zu halten geneigt scheint, und von welcher bei Erklärung der nächsten Tafel die Rede sein wird), eine befriedigende Erklärung zu geben; so muss ich doch freimütig gestehen, dass ich der Ansicht meines Freundes Chladni, von der Bildung der Rinde überhaupt, und dieser im Innern (insofern ihr Vorkommen darin wirklich Statt findet) insbesondere, durchaus nicht beistimmen kann. Die Gegenwart des Schwefels (dessen Anwesenheit in der Steinmasse, wenigstens in gebundenem Zustande, übrigens nicht in Abrede gestellt werden kann), den Hr. Chladni als das Haupt-Material betrachtet, aus welchem die Rinde gebildet wurde, gibt sich in derselben auf keine Weise zu erkennen; weder durch die chemische Analyse, noch durch eine leichte Schmelzbarkeit (die im Gegenteile sehr schwer ist, da sie wenigstens 6 bis 9° Wedgwd. Hitze fordert, und die wohl, wenn man den Rindebildungs-Prozess durch Hitze geschehen lassen wollte, sehr gegen die, obgleich nur durch ein paar Fälle, in Anregung gebrachte Abfärbung der Steine, streiten möchte), weder durch den Geruch bei Erhitzung, noch durch den geringsten Grad von Wirkung auf das Elektrometer, wenn gerieben oder erwärmt; so wie andererseits die Mannigfaltigkeit der Rinde bei verschiedenen Meteor-Steinen, und die offenbare Abhängigkeit derselben von den Gemeng- und Bestandteilen der Steinmasse, gegen ein solches allgemeines Haupt-Material streitet. Die Gleichförmigkeit der Rinde, zumal hinsichtlich der Dicke, auf sonst gleichartigen, wenn gleich sich noch so sehr entgegen gesetzten Flächen, an ein und demselben Steine, und die Übereinstimmung hierin bei allen Meteor-Steinen im Augemeinen; die unwiderleglich von der Beschaffenheit der Oberfläche abhängigen Hauptverschiedenheiten derselben an ein und demselben Steine; die offenbare, allmähliche und stufenweise Ausbildung derselben; und der unverkennbare Übergang ihrer Massenteilchen in jene der Steinmasse, und umgekehrt, wo beide sich im Kontakte befinden (wie sich aus der mikroskopischen Betrachtung ergibt) u. s. w., lassen sich wohl schlechterdings nicht durch eine Übergießung oder Bespritzung von Außen her erklären. Endlich lasst sich das, nach meinen Beobachtungen nur höchst selten (meiner Überzeugung nach bisher nur an diesem einzigen beschriebenen Bruchstücke) und nie tief im Innern eines Steines sich zeigende wirkliche Vorkommen von Rinde in Gestalt von Flecken, deren Form, Beschaffenheit und Zusammenhang mit der Steinmasse (nach obigem), so wie die Art des mehr als problematischen Vorkommens derselben in Adern, Gängen und Lagen (wovon seines Ortes) wohl nicht mit der Idee einer Einknetung und Zusammenklebung vereinigen, als welche einerseits einen ziemlich tumultuarischen (Gährungs-) Prozess bei jedem einzelnen Steine nach dessen Individualisierung, Bildung und bereits schon ein Mahl vollendeter Inkrustierung, andererseits ein häufiges Zusammentreffen, Zusammenpassen und Wiedervereinen der bereits mit Gewalt losgetrennten und weit weg und aus einander geschleuderten Steine und Bruchstücke voraussetzen, mit welchen die Regelmäßigkeit und Übereinstimmung so vieler Steine in der Form (der Grund-Typus), die Beschaffenheit der Flächen und Kanten (welche beide Umstande schlechterdings keinen solchen Grad von Weichheit nach einmal geschehener Inkrustierung denken lassen), der entfernte Niederfall der einzelnen Steine voneinander (der meistens einen Zwischenraum von 2 bis 300, oft 1000 und mehr Klafter beträgt) u. s. w., im offenbarsten Widerspruch zu stehen scheinen.\\
Eher könne ich der Meinung meines Freundes, des Hrn. Prof. v. Scherer (welcher früher gleichzeitig und zum Teil gemeinschaftlich mit mir über diesen Gegenstand arbeitete, und seine Bemerkungen über die Beschaffenheit und wahrscheinliche Entstehung der Rinde an den Steinen von Stannern, in einem gleichzeitigen Aufsätze in Gilberts Annalen Bd. 31 bekannt machte), beipflichten, nach welcher die Rinde in einem Nu, und gleichsam mit Blitzesschnelle, und zwar im Momente der Vereinzelung, Individualisierung der Steine, über alle zugleich, und über deren ganzen Umfang auf ein Mahl, nur mit verschiedener Intensität der wirkenden Potenz, demnach mit einigen Modifikationen, gebildet wurde, und jene Potenz in der Elektrizität zu suchen sein mochte; wenn sich darnach einige Eigenheiten derselben, z. B. die vielen und auffallenden Hauptverschiedenheiten und häufigen stufenweisen Modifikationen und Übergänge der Rinde (deren, wie gezeigt worden ist, immer an einem und demselben, oft sehr kleinen Steine, mehrere, 2 bis 5, deutlich unterschieden, aber nicht wohl begreiflich von einer so vielfachen Verschiedenheit der Intensität, der sie auf ein Mahl erzeugenden Potenz, abgeleitet werden können), befriedigend erklären ließen; wenn ihr nicht ferner einige Erscheinungen bei dem Ereignisse selbst, z. B. das bei diesem, so wie überhaupt bei allen ähnlichen Ereignissen, wo viele Steine fielen, ganz einstimmig gleichartig beobachtete, fortgesetzte, einem kleinen Gewehr- oder Pelotonfeuer ähnliche Getose nach den Haupt-Detonationen (welches wohl nur von einem wiederholten, sukzessiven Zerplatzen und Zerspringen der einzelnen Steine während ihres Falles hergeleitet werden kann); das so ausnehmend schiefe und sanfte Auffallen mancher einzelner, ziemlich großer Steine, so dass sie kaum merklich die Erde streiften und eine Strecke fortrollten (welches eine horizontale Wurfbewegung voraussetzt, die sich mit der Höhe, auf welcher die Hauptzerplatzung vorging, der gegenwirkenden Schwerkraft wegen, schlechterdings nicht verträgt, daher eine spätere Zerplatzung eines einzelnen Steines im Falle, auf minderer Höhe, und die Lossprengung eines Stückes davon in solcher Richtung vorausgesetzt werden muss) u. s. f. --- in Wege stunden; und wenn es endlich nicht ganz an allen Wahrnehmungen fehlte (worauf insbesondere und mit Vorbedacht bei der Untersuchung der Begebenheit zu Stannern alle Rucksicht genommen wurde), die das Spiel oder den Einfluss der Elektrizität bei diesen Ereignissen nur einiger Massen bewahren konnten. Dagegen bin ich mit diesen beiden scharfsinnigen Physikern vollkommen einverstanden, wenn sie behaupten, die Rinde der Meteor-Steine sei das Produkt eines Prozesses, das mit keinem Produkte der uns bekannten natürlichen und künstlichen Schmelz-Prozesse (wenn jener Rinde bildende ja in die Reihe solcher zu stellen sein sollte) einige Ähnlichkeit habe, weshalb wir uns auch zur Zeit keinen richtigen Begriff von ihrer Bildung machen können.}
\clearpage
\section{Siebente Tafel.}
\paragraph{}
Die Abbildungen auf dieser Tafel haben die Darstellung und Versinnlichung der inneren Beschaffenheit der Steinmasse einiger, der in dieser Beziehung ausgezeichnetsten Meteor-Steine, des Aggregats-Zustandes derselben und ihrer wesentlichsten Gemengteile zum Zwecke, und in Hinsicht dieser letzteren insbesondere, die Darstellung des allgemeinsten, auffallendsten und sehr wesentlichen, nämlich des mehr oder weniger kugelichten, porphyrartig in der übrigen Steinmasse erscheinenden, erdigen Gemengteiles, und zwar in den verschiedenen Graden seiner Ausbildung, die von einer kaum erkennbaren Ausscheidung bis zu dessen ausgesprochenstem Zustande --- als olivinartige Substanz im sibirischen Eisen --- Übergänge nachweisen lassen, und deren sich oft mehrere, nicht nur in verschiedenen Steinen eines und desselben Niederfalles, sondern selbst in einem und demselben Bruchstücke, beisammen finden.\footnote{Um eine deutliche Ansicht von dem so sehr verschiedenen Aggregats-Zustande der Steinmasse sowohl, als insbesondere von dem so sehr abweichenden, wechselseitigen, quantitativen Verhältnisse der Gemengteile, und von deren mannigfaltigen Beschaffenheit und Zustand zu gewinnen, ist es durchaus notwendig, an jedem Meteor-Steine oder an einem Bruchstücke von demselben, eine Bruchfläche schleifen und polieren zu lassen; doch muss dieses mit der Vorsicht geschehen, dass bei der Behandlung so wenig Feuchtigkeit und so wenig Schmirgel, oder sonstiges Schleif- oder Polier-Pulver, als nur immer möglich, angewendet, letzteres aufs vollkommenste sogleich weggewaschen, und das Stück dann schnell und gut getrocknet werde, um das eigentümliche Ansehen nicht durch eine fremdartige Substanz, oder durch beförderte Oxydation des enthaltenen Eisens, mehr oder minder verunstaltet zu erhalten.}

---

Alle bisher bekannten, eigentlichen Meteor-Steine, sind gemengte Massen, und alle authoptisch mir davon bekannten 34,\footnote{Namentlich Bruchstücke von Steinen von den Vorfällen bei Ensisheim, Verona, Tabor, Laponas (Bresse), Lucé, Mauerkirchen, Sigena, Eichstädt, Charkow, Barbotan, Siena, York, Salés, Benares, L'Aigle, Apt, Eggenfeld, Glasgow, Doroninsk, Alais, Timochin, Weston, Parma, Stannern, Lissa, Tipperary, Charsonville, Berlanguillas, Toulouse, Erxleben, Chantonnay, Limerick, Agen und Chassigny (Langres), als von welchen auch ähnliche Belege sich notorisch im Besitze öffentlicher Sammlungen oder bekannter Privat-Eigentümer befinden. Es sollten und werden wohl auch von noch mehreren Vorfällen neuerer Zeit, vielleicht von 20 bis 30 außer obigen, derlei Belege vorhanden sein und sich in den Händen von Privat-Besitzern befinden, die aber leider nicht verlässlich bekannt sind.} nach Zeit und Ort des Niederfallens verschiedene, nur mit Ausnahme jener von Alais, Erxleben, Chassigny (Langres), und zum Teil jener von Chantonnay, welche ein ganz eigentümliches Ansesehen, selbst im Ganzen\footnote{Ein Ansehen, wodurch sie sich nicht nur unter sich, sondern auch von allen übrigen bisher bekannten Meteor-Steinen so sehr unterscheiden, dass man sie wohl nicht leicht für solche erkennen möchte, wenn nicht einerseits ihre faktisch erwiesene Herkunft und die Haupt-Resultate der chemischen Analyse, und andererseits selbst einige, wenn gleich nur einseitige, und oft nur in Übergängen nachweisbare, oryktognostische Verwandtschaft hinsichtlich einzelner Gemengteile, oder irgend einer Zustandsveränderung der Masse bei andern, unbezweifelbaren Meteor-Steinen, für sie das Wort sprechen und gewisser Maßen Bürgschaft leisten möchten. (So z. B. das stellenweise Filzig-Faserige der Grundmasse der Steine von Eggenfeld, Mauerkirchen, Benares, Parma, Siena, und das zum Teil Unausgesprochene und Undeutliche des kugelichten Gemengteiles bei so vielen Meteor-Steinen, für jene von Stannern; --- die individuelle Beschaffenheit dieses letzteren Gemengteiles bei vielen andern Meteor-Steinen, und die Ähnlichkeit darin mit der Hauptmasse jener von Chassigny, für diese; --- die Ähnlichkeit der Substanz der in vielen Meteor-Steinen vorkommenden Adern und Gänge, für die von Chantonnay, und zum Teil von Alais; --- endlich die bei manchen Meteor-Steinen hie und da erscheinenden spatartigen, schillernden Stellen, für jenen von Erxleben.)} haben, und beziehungsweise auch der von Stannern, lassen viererlei Gemengteile, selbst dem freien Auge, und gewöhnlich sehr deutlich ausgesprochen, erkennen.

Zwei dieser Gemengteile sind erdiger, zwei davon metallischer Natur.

Der eine erdige Gemengteil hat ein mehr oder weniger mattes, mageres, und, nach der verschiedenen Feinheit und Gleichförmigkeit des Korns --- das vom groben bis zum äußerst feinen, dem unbewaffneten Auge kaum unterscheidbaren, abweicht --- und nach dem mehr oder minder dichten und festen Kohäsions-Zustande --- der vom leicht zerreiblichen bis zum schwer zersprengbaren und Funkengeben geht --- und insofern derselbe nicht --- was jedoch selten und nur stellenweise der Fall ist --- eine besondere, faserige, spätige oder blätterige Textur zeigt, ein mehr oder minder raues, sandsteinartiges Ansehen, und eine lichter oder dunkler aschgraue, selten ins Weiße oder Gelbliche, meistens ins Bläuliche ziehende Farbe.

Es kann dieser Gemengteil, rücksichtlich der übrigen, seiner Gleichförmigkeit wegen, und da er meistens mehr oder weniger, und oft sehr bedeutend über alle übrigen zusammen, oder doch über jeden derselben einzeln genommen, an Menge vorwaltet, als Haupt- oder Grundmasse angesehen werden, und dies umso füglicher, als alle übrigen Gemengteile aus dieser Masse gebildet oder ausgeschieden worden, aus ihr entstanden oder hervor gegangen sein dürften, als zu welchem Schlusse nicht nur die physiologisch-oryktognostischen, sondern insbesondere die physisch-chemischen Untersuchungen, auf deren Resultate gehörigen Ortes hingedeutet werden wird, zu berechtigen scheinen.

Die Abweichungen dieser Grundmasse in obigen Eigenschaften, obgleich sie in den extremsten Gliedern sehr auffallend sind, gehen durch Zwischenglieder so allmählich in einander über, dass zuletzt aller Abstand verschwindet; besonders merkwürdig aber ist, dass mehrere dieser Abweichungen, zumal in Dichtheit und Farbe, und zwar oft in einem sehr merklichen Grade, nicht selten bei Steinen von einem und demselben Ereignisse, ja selbst bei Bruchstücken eines und desselben Steines vorkommen, so dass sich solche, zumal wenn ähnliche Abweichungen hinsichtlich der übrigen Gemengteile, wo sie noch weit gewöhnlicher und ungleich mannigfaltiger sind, zugleich Statt finden, oft mehr voneinander unter sich, als von Bruchstücken ganz anderer, nach Zeit und Ort des Niederfallens sehr verschiedener, Meteor-Steine unterscheiden.\footnote{Diess ist z. B. vorzüglich bei den Steinen von Chantonnay, L'Aigle, Barbotan, Weston, Charsonville, Agen, Lissa, und zum Teil selbst bei denen von Stannern der Fall, und manche Bruchstucke eines einzelnen dieser Steine sind sich weit unähnlicher, als es oft Bruchstücke von Steinen von Eichstädt und Timochin, von Apt und Berlanguillas, von York, Glasgow und Toulouse, von Tipperary und Limerick, von Siena und Parma gegen einander sind, ja oft sind jene manchen von diesen mehr ähnlich, als sie es unter sich selbst sind.}

Im Bruche gibt diese Masse nach dem verschiedenen Kohäsions-Zustande --- wenn dieser oder vielmehr der durch die übrigen Gemengteile vermittelte Aggregats-Zustand nicht so locker ist, dass sie bröcklig oder sandsteinartig körnig zerfällt, was jedoch höchst selten der Fall ist --- größere oder kleinere, unbestimmt eckige und ziemlich scharfkantige, und an den äußersten Kanten bisweilen selbst etwas durchscheinende Bruchstücke, und geschliffen nimmt sie nicht selten einen bedeutenden und andauernden Grad von Politur an.

Nach obigem Maßstabe ist die Masse auch mehr oder weniger leicht, wenn ganz rein, meistens sehr leicht, zu Pulver zu stoßen, und zuletzt zum feinsten Pulver zerreiblich.

Das gröbere Pulver unter dem Mikroskope betrachtet, zeigt, auch bei vollkommen erdigem Ansehen der Masse im Ganzen (wie bei den Steinen von Siena, Benares, Stannern), ein Gemenge von mehr oder weniger kristallinischen, durchscheinenden, zum Teil durchsichtigen, unbestimmt eckigen, ziemlich scharfkantigen Körnern, von krystallweißer, gelblicher, gelblichgrüner und grünlicher, in einander übergehenden Farben, meistens in größerer, und von halb kristallinischen, teils halb durchscheinenden, teils ganz undurchsichtigen, grauen, blau- und rauchgrauen Partikelchen, gewöhnlich in geringerer Menge. Erstere scheinen in diese, diese in andere, meistens doch nur in einem sehr geringen Verhältnisse, oft nur einzeln vorhandene, schwarze, glänzende kleine Massen überzugehen, die ein etwas schlackiges und der Kohlenblende ähnliches Ansehen haben. Gewöhnlich zeigt sich noch eine vierte Art von Massenteilchen in jenem Gemenge, obgleich meistens nur in sehr geringer Menge, bisweilen jedoch vorwaltend, als weiße oder grauliche, mehr erdige, undurchsichtige, oder doch nur schwach und teilweise durchscheinende, dem verwitterten Feldspate ähnliche Teilchen, welche, oft innig mit den Partikelchen der zweiten Art verbunden, in andere übergehen, die eigentlich nicht mehr der Hauptmasse anzugehören scheinen, und von welchen bei Gelegenheit des einen metallischen Gemengteiles der Steinmasse (des Gediegeneisens und der damit verbundenen Rostflecke) die Rede sein wird.

Die kleinen schwarzen Massen sind etwas schwerer zu Pulver zu stoßen, und lassen beim Zerreiben gewöhnlich ein kleines Metallteilchen zurück, das sich auf dem Ambosse, obgleich etwas schwer, fletschen lässt, auch werden sie von der Magnetnadel angezogen; die grauen Partikelchen werden es nur in so ferne, als sie mit jenen oft innig verbunden sind; die kristallinischen durchsichtigen aber gar nicht.

Aus dieser Beschaffenheit\footnote{Obige Beschreibung ist das Resultat einer mühsamen, schon 1808 vorgenommenen, vergleichenden, mikroskopischen Betrachtung von zehn verschiedenen Meteor-Steinen, die mir damals zu Gebote standen (namentlich des von Eichstädt; der von Tabor, Barbotan und L'Aigle; von Ensisheim und Lissa; und der von Siena, Mauerkirchen, Benares und Stannern), welches wohl als allgemein geltend angesehen werden kann (da ich in dieser Zwischenzeit keine Muße fand, diese Untersuchungen weiter fortzusetzen), indem es aus der Vergleichung von so vielen, in den wesentlichen Beziehungen so sehr voneinander abweichenden Steinen, die nach meiner Ansicht vier Übergangesreihen in der Sippschaft bilden, abgezogen ist.} der Massenteilchen dieses einen, die Grundmasse der Meteor-Steine konstituierenden Gemengteiles, so wie aus jener, gleich zu beschreibenden des zweiten erdigen Gemengteiles, die sich bei manchen Meteor-Steinen noch weit deutlicher, und selbst im Ganzen schon, ohne mikroskopische Untersuchung der integrierenden Massenteilchen ausspricht (wie bei den Steinen von Erxleben und Chassigny), und aus den offenbaren Übergängen beider in einander, so wie aus den Resultaten der Analysen,\footnote{Abgesehen von den metallischen Gemengteilen, stimmt bekanntlich nicht nur das qualitative, sondern selbst das quantitative Verhältnis der chemischen Bestandteile der Steinmasse der meisten bisher bekannten Meteor-Steine ziemlich genau mit jenem des terrestrischen Olivins zusammen. Kieselerde ist ebenso wie bei diesem der vorwaltendste Bestandteil, der in der Regel wohl nur zwischen 30 und 50 Perzent abweicht, und Talkerde ist höchst wahrscheinlich ein ebenso beständiger, nur im quantitativen Verhältnisse etwas mehr, zwischen 2, im Allgemeinen doch wohl nur zwischen 10 und 30 Perzent variierender Bestandteil. Der sehr unbeständig scheinende Gehalt an Alaun und Kalkerde (im Allgemeinen von 1 bis 3 Perzent --- mit Ausnahme der Steine von Stannern, wo er auf Rechnung jenes an Talkerde eingetreten zu sein scheint ---) ist doch viel zu gering, als dass er für entscheidend und für etwas mehr geltend gemacht werden könnte, als höchstens vielleicht für eine Annäherung an ein anderes, mit dem Olivin geognostisch verwandtes Fossil, nämlich den Augit.\\
Obgleich ferner der eine als Grundmasse angenommene Gemengteil nur höchst selten, selbst kaum \emph{en gros}, ganz rein und für sich (nach oben beschriebener Beschaffenheit der Massenteilchen aber auf keine Weise vollkommen abgeschieden) chemisch untersucht werden kann; so fand sich doch, wo dieses einiger Maßen möglich war (wie bei den Steinen von Benares durch Howard und Bournon), ein höchst unbedeutender Unterschied selbst im quantitativen Verhältnisse der Bestandteile zwischen diesem und dem andern, doch sehr ausgeschiedenen, und schon mehr als Olivin ausgesprochenen Gemengteil, nämlich in diesem nur um 2 Perzent Kieselerde mehr, und 3 Perzent Talkerde weniger als in der Grundmasse.\\
Von dem Verhältnisse dieses olivinartigen Gemengteiles in den Meteor-Steinen zur olivinartigen Substanz im sibirischen Eisen --- und von jenem dieser zum terrestrischen Fossil dieses Namens, wird gleich bei Beschreibung des ersteren die Rede sein.} ergibt sich nicht nur die nahe Verwandtschaft, oder vielmehr die Identität beider, sondern auch die wahre Natur der Steinmasse im Ganzen, als Olivin in verschiedenen Graden von Ausbildung und Charakterisierung, wofür sie bereits auch Hausmann und Stromeyer erkannt und ausgesprochen haben.

Der zweite erdige Gemengteil der Steinmasse hat teils ein mattes, von der Grundmasse zum Teil oft nur wenig verschiedenes, mageres, meistens aber doch glatteres, dichteres Ansehen, und unterscheidet sich von derselben gewöhnlich mehr oder weniger, obgleich oft nur allmählich und übergehend, durch ein weit feineres gleichförmigeres Korn, größere Festigkeit und Härte, die vom Wacker-Feuerschlagen und Glasritzen nur bis zum Leichtzersprengbaren herabsinkt, und durch eine dichtere Textur, die bis ins Spätige und Kristallinische geht, und mit welcher der Glanz, ein Mittel zwischen Fett- und Glasglanz, zunimmt, und die Undurchsichtigkeit bis ins Durchscheinende, und selbst ins Durchsichtige übergeht.

Die Farbe geht aus dem verschiedenen Grau der Grundmasse, mit der sie inzwischen oft ganz gleich, nur meistens etwas lichter oder dunkler ist, ohne merklicher Abhängigkeit von, und ohne regelmäßige Übereinstimmung mit obigen Eigenschaften, unter vielen und allmählichen Abstufungen (Nuances) von Höhe und Tiefe, licht und dunkel, und in sehr mannigfaltigen, ebenso allmählich in einander übergehenden Modifikationen (Teintes) der Hauptfarben, aus dem Gelblichen oder Graulichen, einerseits, obgleich seltener, ins Wachs- und Honiggelbe, andererseits und am gewöhnlichsten ins Lauch-, seltener ins Spargel- und Pistazien-, am häufigsten ins Oliven- und Öl-, bis ins Schwärzlich-Grüne, und aus dem Bläulich-Grauen ins Perl- und Schiefer-Graue- und ins Lavendel- bis ins Schwärzlich-Blaue.

Es zeigt sich dieser Gemengteil bald mehr, bald weniger ausgeschieden, schärfer oder schwächer begrenzt, und nach Verhältnis obiger Eigenschaften, zumal nach den verschiedenen Graden seiner Dichtheit und der Intensität und Beschaffenheit der Farbe, mehr oder weniger ausgesprochen und von der Grundmasse ausgezeichnet, bisweilen aber auch kaum erkennbar von derselben geschieden, aus ihr oder in sie gleichsam übergehend, mehr oder minder häufig, in Massen von sehr verschiedener Größe und Gestalt, und höchst ungleichförmig in der Grundmasse verteilt.

Bei weitem am gewöhnlichsten ist das quantitative Verhältnis dieses Gemengteils zur Grundmasse nur gering, nur höchst selten nähert sich dasselbe der Hälfte, gewöhnlich beträgt es zwischen 1/5 bis 1/10 von der Gesamtmasse, oft aber auch noch weit weniger, und nicht selten findet sich dieser Gemengteil nur in einzelnen, wenigen, sehr zerstreuten Massen, scheint aber, wenn gleich oft sehr undeutlich ausgesprochen, nie ganz zu fehlen\footnote{So finden sich z. B. in der lockern, leicht zerreiblichen Meteor-Masse von Alais rundlichte Körner von beträchtlicher Dichtheit und Härte eingemengt.}; dagegen scheint er bisweilen, obgleich nur höchst selten, entweder ganz innig mit der Grundmasse gemengt zu sein, oder dieselbe beinahe ganz zu vertreten, und ausschließlich ganze einzelne Steine eines und desselben Meteors, und selbst ganze Meteor-Massen zu bilden.\footnote{Wie dies bei den merkwürdigen Steinen von Erxleben und Chassigny der Fall ist, die sich eben dadurch von allen bisher bekannten Meteor-Steinen so sehr unterscheiden, dass außer den zart eingesprengten Metallteilchen in dem einen, ersteren, auch gar keine Ähnlichkeit mit irgend einem andern bekannten Meteor-Steine nachzuweisen wäre, wenn nicht doch hie und da in einem oder dem andern die ausgezeichnete Masse jener Steine, aus der ihr Ganzes besteht, wenigstens als einzelner Gemengteil erschiene. Und so auffallend demnach, sowohl nach den Resultaten der von mir neuerlichst vorgenommenen mikroskopischen Untersuchung der Massenteilchen, als noch mehr nach jenen der chemischen Analyse der Steinmasse beider (nach Klaproth und Stromeyer von dem einen, nach Vauquelin vom andern) einerseits die Ähnlichkeit im Wesentlichen der Beschaffenheit und des Gehaltes mit allen übrigen Meteor-Steinen ist; noch umso mehr auffallend ist wohl andererseits nach denselben die ganz besondere Übereinstimmung hierin gerade zwischen diesen beiden Steinen, da sie doch unter sich, nach allen äußern und physischen Merkmahlen (das spezifische Gewicht allein ausgenommen, welches bei beiden ziemlich gleich ist, = 3,600 nach Klaproth bei jenem von Erxleben, und = 3,550 nach eigener Wiegung, bei jenem von Chassigny, obgleich dieser keine Spur weder von Gediegeneisen, noch von Kies oder Schwefeleisen zeigt, die beide in jenem häufig vorhanden sind), beinahe noch mehr als von allen andern Meteor-Steinen abweichen. (Inzwischen gerade nicht mehr als ihre beiderseitige Masse zu tun pflegt, wenn sie als isolierter Gemengteil, einzeln oder vereint, in einem andern Meteor-Steine vorkommt.)}

Selten sind diese Massen bedeutend groß, und ebenso selten ganz unförmlich oder vieleckig gestaltet; gewöhnlich, zumal bei höheren Graden von Dichtheit und bedeutender Intensität von Farbe, sind sie nur klein, höchstens von einigen Linien im größten Durchmesser, und dann meistens ziemlich spitzeckig und scharfkantig, ungleichseitig dreieckig, rhomboidal und trapezoidal, oder scharf gerandet und oval, oder mehr oder weniger zugerundet; am häufigsten aber und zwar, obgleich gerade nicht immer im Verhältnisse mit der Dichtheit und Farbe, doch stets bei den höchsten Graden derselben, und vorzugsweise bei den grünen Farben-Tinten, sehr und selbst äußerst klein, und vollkommen zirkelrund.

Im letzteren Falle, zumal wenn der Kohäsions-Zustand der Grundmasse an und für sich nicht sehr bedeutend ist, ist der Aggregats-Zustand zwischen diesem Gemengteil und jener so locker, dass diese Massen, umso mehr, wenn sie vollkommen kugelicht sind, beim Zerbrechen oder Zerschlagen des Steines (wo sie sonst, bei minder vollkommener Ausscheidung und festerem Zusammenhalte der Steinmasse, mitbrechen oder halbkugelicht über die Bruchfläche vorragend, sitzen bleiben) teils von selbst aus der Grundmasse herausfallen, teils mit leichter Mühe aus derselben heraus gebrochen werden können, und dann, ihrem Volum und ihrer Form entsprechende Gruben (runde Zellen, wie der Olivin im sibirischen Eisen), deren Boden und Wände verdichtet, und gleichsam abgeglättet sind, und wahren Absonderungsstellen gleichen, zurück lassen, so dass es wirklich das Ansehen hat, als wären diese Kugeln in die übrige Masse eingeknetet worden.\footnote{Ich kann nicht umhin, hier auf eine ganz ähnliche Bildung und Absonderung, gleichzeitig entstandener und gleichartiger, oder doch nur wenig veränderter Massen terrestrischer Fossilien hinzuweisen, nämlich auf jene, in dieser Beziehung höchst merkwürdigen, kugelichten Basalte, Thon- und Klingstein-Porphyre, welche, zumal letztere, im Innern ihrer Grundmasse ähnliche, oft vollkommen sphärische Kugeln, von 4 bis 5 Zoll in Durchmesser, von vollkommen homogener Natur, nur etwas in der Farbe verändert, und von größerer Dichtheit und Feinheit im Korne als die Hauptmasse, eben so fest eingeschlossen, oder mehr oder weniger scharf abgesondert, und nicht selten eben es vollkommen ausgeschieden und lose, mit geglätteter Oberfläche und verdichteten Wänden der Gruben, eingeschlossen enthalten.} Die Kugeln selbst sind in diesem Zustande meistens vollkommen sphärisch, und haben eine mehr oder weniger dunkle, grünlich oder bräunlich-graue Farbe, einen schwachen, etwas fettigen, meistens nur schimmernden Glanz, und eine sehr glatte Oberfläche, indes sie sonst, auf niederer Stufe von Ausbildung und Ausscheidung, wenn sie auch aus der Grundmasse hervorragen, mehr uneben und gleich dieser gefärbt, ganz matt und rau sind, indem sie von Massenteilchen derselben, die innig mit ihrer Oberfläche zusammen hangen, bedeckt erscheinen. Nach den verschiedenen, sehr mannigfaltigen und sehr abweichenden Graden von Dichtheit und Festigkeit, sind die Massen dieses Gemengteiles, mehr oder weniger, leicht zersprengbar, aber nie zerreiblich, im Gegenteile nicht selten ziemlich schwer zersprengbar, und in dem Maße, als dieselben dadurch und durch die übrigen Eigenschaften von der Beschaffenheit der Grundmasse sich unterscheiden, und vollkommen ausgeschieden erscheinen, zeigt sich der Bruch, der im unvollkommensten Zustande noch rau und erdig, doch immer dichter ist als jener der Grundmasse, immer feiner, dichter, ebener, glatter, und geht endlich in einen vollkommen dichten, flachmuschlichen über. Sie zerspringen nach allen Richtungen (und erscheinen auch so von selbst, oft in viele kleine Stücke, zersprungen auf geschliffenen Flächen) in unbestimmt eckige, ziemlich scharfkantige, meistens ganz undurchsichtige, nicht selten aber auch mehr oder weniger an den Kanten durchscheinende, bisweilen ganz durchscheinende, und, obgleich nur selten und einzeln, selbst ganz durchsichtige Bruchstücke, von einem schwachen, etwas fettigen Glanze, der sich mit zunehmender Durchscheinenheit, zumal bei lichtern, grünlichen und gelblichen Farben, immer mehr und mehr dem Glasglanze nähert; und in diesem Zustande geben dergleichen Bruchstücke nicht nur ziemlich leicht Funken am Stahle, sondern ritzen auch etwas das gemeine Glas.\footnote{Es wollen Manche an Massen dieses Gemengteiles in Meteor-Steinen --- so wie an der olivinartigen Substanz im sibirischen Eisen --- (wovon seines Ortes die Rede sein wird) wo nicht eine vollkommene und ausgesprochene Krystall-Form, doch wenigstens einzelne, wahre Kristallisation-Flächen beobachtet haben. (So Calmelet und Gillet de Laumont, eine prismatische Form mit rhomboidaler Grundlage, die sogar ganz mit einer Abänderung aus der Krystall-Suite des Augits (\emph{Pyroxene H.}) übereinstimmen soll, in einem Steine von Chassigny; so Chladni etwas Kristallähnliches, als ein regelmäßiges Parallelogramm, in einem Steine von Siena, und Kristallisation-Flächen an einer bedeutend großen Masse dieses Gemengteiles in seinem Bruchstückchen vom Steine von Eggenfeld.) Ich habe mich von der Gründlichkeit dieser Angaben noch nicht vollkommen überzeugen können, und was ich zur Zeit von solchen angeblichen Krystall-Formen und angenommenen Kristallisation-Flächen (namentlich beim sibirischen Eisen) gesehen habe, kann ich vor der Hand bloß als Absonderungsflächen erkennen.}

Nach den verschiedenen Graden von Zersprengbarkeit lassen sich die Massen dieses Gemengteiles auch mehr oder weniger leicht, nie aber so leicht wie die Grundmasse, im Gegenteile meistens schwer, und gewöhnlich sehr schwer, oft nur auf einem Ambosse, zu Pulver stoßen, und nur selten, und dann erst, wenn schon sehr verkleinert, vollends zerreiben. Die Massenteilchen erscheinen unter dem Mikroskope, nach der verschiedenen Beschaffenheit, die sie ursprünglich in ihrem Zusammenhange, in allen obigen vielseitigen Beziehungen, von Farbe, Durchscheinenheit u. s. w. zeigten, höchst mannigfaltig, doch zeigen sie, solcher Gestalt verkleinert und einzeln, immer lichtere und fast durchaus mehr ins Grünliche ziehende Farben, und mit diesen einen höheren Grad von Durchscheinenheit und scharfkantigere Bruchflächen, alles aber im Großen in einem geringeren Grade als die oben beschriebenen Massenteilchen der Grundmasse, zumal als jene der mehr kristallinischen ersterer Art, von denen sie sich übrigens noch durch ein minder kristallinisches Ansehen und durch einen mehr fettigen Glanz unterscheiden, übrigens aber, und zwar durch die halbkristallinischen Massenteilchen zweiter Art der Grundmasse, in dieselben überzugehen, oder aus denselben hervor gegangen zu sein scheinen. Sie zeigen übrigens, sowohl in diesem als im konkreten Zustande, eben so wenig als jene, wenn nicht durch zufällig eingemengte Metallteilchen vermittelt, die geringste Wirkung auf die Magnetnadel.

Alle obigen, so mannigfaltigen Verschiedenheiten im Ansehen, Verhalten und Vorkommen, so wie das so sehr abweichende quantitative Verhältnis dieses Gemengteiles, scheinen in keinem absoluten Wechselverhältnisse mit oder in einer direkten Abhängigkeit von der physischen Beschaffenheit der Grundmasse (von der Dichtheit, Farbe u. s. w. derselben) zu stehen; wohl aber scheint das quantitative Verhältnis der entfernteren Bestandteile (zumal der Talk- und Kieselerde) der Steinmasse im Ganzen, darauf einigen Einfluss zu haben\footnote{Bei allen Meteor-Steinen, bei welchen dieser Gemengteil häufiger, auch wohl nur deutlicher ausgesprochen, oder in einem vollkommeneren Zustande vorkommt (wie bei jenen von Eichstädt, Tabor, Benares, Eggenfeld), scheint (insofern auf alle Analysen in dieser Beziehung anzugehen ist) die Talkerde in einem größeren Verhältnisse = 17 bis 23 Perzent vorhanden zu sein. Am auffallendsten ist dies bei den Steinen von Erxleben und Chassigny, deren ganze Masse aus diesem Gemengteil, in einem ziemlich ausgesprochenen Zustande, besteht, und von welchen der Gehalt an Talkerde mit 23,58 bis 26,50 und 32 Perzent ausgewiesen wird. Es ist zwar von manchen noch der Gehalt als bedeutend (so von jenen von Apt mit 14, von Lissa mit 22, von Yorkshire mit 24?) angegeben, wo doch dieser Gemengteil \emph{en masse} nur selten und schwach ausgesprochen erscheint. Allein hier mag es an der Unvollkommenheit der Ausscheidung, und an der innigeren Verbindung der Massenteilchen liegen, welche letztere dieses auch (wenigstens bei den Steinen von Lissa) bewähren. Der sehr abweichende Gehalt dieses Gemengteiles sowohl als überhaupt der ganzen Steinmasse, an Eisen, und wohl auch der veschiedene Zustand, in welchem sich dasselbe in beiden befindet, durfte vielleicht den wesentlichsten Einfluss auf die meisten Zustandsverschiedenheiten haben.}; das Meiste dürfte jedoch wohl von besonderen Zustandsveränderungen der Steinmasse im Ganzen abhängen.\footnote{Diess scheinen wohl jene in obiger Note berührten Fälle, wo die Ausscheidung und der Zustand dieses Gemengteiles dem quantitativen Verhältnisse der Bestandteile der Steinmasse nicht entspricht, und überhaupt die so mannigfaltigen Zustandsverschiedenheiten desselben, die oft weder mit dem Gehalte, noch mit der Beschaffenheit der Steinmasse in irgendeinem Kausal-Verhältnisse stehen, insbesondere aber die Steine von Erxleben und Chassigny, zu bestätigen.} Sehr merkwürdig aber ist, dass dieser Gemengteil, sollte er auch in einem noch so geringen Verhältnisse vorhanden sein, in einem und demselben Steine sich höchst selten, wenn je, durchaus von ganz einerlei Beschaffenheit findet, abgesehen selbst von Form und Größe; dass er im Gegenteile gewöhnlich, selbst in einem und demselben Bruchstücke eines Steines, sollte dieses auch nur ein paar Zoll Oberfläche bieten, wenigstens in zwei oder drei, oft aber in noch mehreren, und nicht gar selten in einer ganzen Suite von Zustandsveränderungen in allen oben angeführten Beziehungen erscheint: vom unvollkommensten, kaum von der Grundmasse unterscheidbaren Zustande, bis zum vollkommenst ausgebildeten, scharf geschiedenen, vollkommen glasartigen; und nicht minder merkwürdig ist es, dass er sich ebenso und oft in einzelnen Zustandsverschiedenheiten, ganz ausnehmend ähnlich, bei, nach Zeit und Ort des Niederfallens, sehr verschiedenen, übrigens im Ganzen oder in andern Beziehungen mehr oder minder sich ähnlichen, Steinen zeigt, und solcher Gestalt einerseits die Unterscheidung solcher, sich oft ganz ähnlicher Steine oder Bruchstücke verschiedener Abkunft --- die sonst durch ihn, gerade der vielen Modifikationen wegen, in welchen er vorkommen kann, am leichtesten wäre --- sehr schwer und unsicher macht; andererseits aber einen und oft ausschließlichen Anhaltspunkt zur Wiedererkennung und Nachweisung einer Analogie zwischen sonst gar sehr heterogen scheinenden Massen darbietet; so wie er denn auch die Homogenität der Materie, die Gleichförmigkeit des Bildungs-Prozesses und die Allgemeinheit der Herkunft aller dieser Massen bewährt, und den vorzüglichsten Charakter der natürlichen Versippung derselben begründet. Und so wie einerseits diese mannigfaltigen Modifikationen und die unverkennbaren Übergänge derselben in einem und demselben Bruchstücke, so wie die Übereinstimmung darin in verschiedenen, der Grundmasse und allen Beziehungen nach oft sehr voneinander abweichenden Steinen, und das allmähliche, oft kaum erkennbare Hervortreten dieses Gemengteiles aus der Grundmasse --- die Homogenität desselben mit dieser bewähren, welche auch die Analyse bestätiget,\footnote{Wie bereits in einer früheren Note gezeigt worden ist.} und auf eine bloße Zustandsveränderung der Masse, durch welche diese Umbildung oder Ausscheidung in verschiedenen Graden bewirkt wird, schließen lassen; so scheint wohl andererseits auch aus denselben, so wie aus der Suite der oryktognostischen Merkmahle,\footnote{Gefüge, Festigkeit, Harte, Bruch, Bruchstucke, Durchscheinenheit, Glanz, und vollends die Farbenreihe, die, wie vorzüglich die Massenteilchen zeigen, Grün immer zum Typus hat, welche den mannigfaltigen Zustandsverschiedenheiten und ihren allmählichen Übergangen entsprechen.} und den Resultaten der physischen\footnote{Das spezifische Gewicht kann der Kleinheit der Massen wegen nicht wohl bestimmt werden, auch muss dasselbe nach den verschiedenen Zustandsveränderungen notwendig abweichen, und nach dem sehr abweichenden Gehalte an verlarvtem sowohl, als selbst an mechanisch eingemengtem metallischen Eisen (der bei diesem Gemengteile in den Meteor-Steinen gewöhnlich ungleich großer ist, als bei der olivinartigen Substanz im sibirischen Eisen) sehr verschieden sein. Das spezifische Gewicht der olivinartigen Substanz im sibirischen Eisen (= 3,263 bis 3,3 nach Bournon) stimmt aber ganz genau mit jenem des terrestrischen Olivins überein (= 3,225 nach Werner; 3,265 nach Klaproth). Die Schmelzbarkeit, die Graf Bournon mit einem Kügelchen aus einem Steine von Benares erprobte (wo dieser Gemengteil zwar besonders ausgeschieden, aber eben in keinem hohen Grade von Ausbildung vorkommt), ist ebenso schwer, wie die der olivinartigen Substanz im sibirischen Eisen und die des terrestrischen Olivins.} und chemischen\footnote{Insofern die Zustandsverschiedenheiten dieses Gemengteiles von dem Mischungsverhältnisse abhängen, insofern mag wohl auch dieses sehr mannigfaltig sein, inzwischen wich dasselbe nach Howards Analyse bei einer Masse der Art aus einem Steine von Benares nur höchst unbedeutend von jenem ab, welches er bei Zerlegung der olivinartigen Substanz aus dem sibirischen Eisen erhielt, und zwar --- wohl zu bemerken --- selbst weniger, trotz der Verschiedenheit beider Massen im Äußern, als das von Klaproth bei derselben Substanz gefundene von dem seinigen. (Howard erhielt nämlich aus dem kugelichten Gemengteile des Steines von Benares 50 Perzent Kiesel- und 15 Perzent Talkerde, und aus der olivinartigen Substanz des sibirischen Eisens 54 Perzent Kiesel- und 26 Perzent Talkerde; Klaproth dagegen aus derselben Substanz von ersterer 41, von letzterer aber 38 Perzent. Den Hauptunterschied macht der Gehalt an Eisen, wovon Howard aus der kugelichten Masse 34 Perzent, aus letzterer nur 16, und Klaproth 18 Perzent erhielt.) Und noch unbedeutender ist die Abweichung im Mischungsverhältnisse zwischen dieser und dem terrestrischen Olivin (in welchem die Kieselerde 50 bis 52, die Talkerde 37 bis 38, und das Eisen 10 bis 12 Perzent betragt); auffallend dagegen die nahe Übereinstimmung darin zwischen allen drei Substanzen und der Gesamtmasse der Steine von Erxleben und Chassigny. (Klaproth erhielt aus ersterem --- nebst etwas Kalk- und Alaunerde, Nickel, Mangan, Chrom und Schwefel --- 35 1/2 Perzent Kiesel- und 26 1/2 Perzent Talkerde und 31 Perzent regulinisches Eisen; Stromeyer aus demselben --- nebst den gleichen Nebenbestandteilen und 3/4 Perzent Natrum --- 36 1/3 Perzent Kiesel- 23 1/2 Perzent Talkerde und 24 1/2 Perzent metallisches und 5 1/2 Perzent oxydulirtes Eisen. Vauquelin fand im letzteren --- ohne Nebenbestandteile, außer 2 Perzent Chrom --- 33 Perzent Kiesel- 32 Perzent Talkerde und 31 Perzent Eisenoxyd.)} Untersuchungen, die vollkommenste Identität dieses Gemengteiles, trotz dessen anscheinender Verschiedenheit, nach den verschiedenen Graden seiner stufenweisen Ausbildung, nicht nur in allen eigentlichen Meteor-Steinen, sondern auch mit der olivinartigen Substanz\footnote{Schon Graf Bournon hat auf diese Identität aufmerksam gemacht. Und so wie einzelne Massen dieses Gemengteiles in den Meteor-Steinen Zustandsverschiedenheiten zeigen, die ganz vollkommen und in allen Beziehungen jenen der olivinartigen Substanz im sibirischen Eisen entsprechen --- wobei bemerkenswert ist, dass solche oft in Steinen vorkommen, wo dieser Gemengteil im Allgemeinen gerade nicht am vollkommensten ausgesprochen ist (wie z. B. in jenen von Siena und Eggenfeld, in welchen Bournon und Chladni auch vollkommen durchsichtige, glasartige, gelblich-grüne Massen desselben beobachteten) --- ebenso finden sich bei dieser (wie an seinem Orte erwähnt werden wird) Zustandsverschiedenheiten und Übergänge, die sich in manche jenes Gemengteiles verlaufen. Nur ist das Verhältnis gerade entgegengesetzt, und Zustandsverschiedenheiten, die hier am häufigsten vorkommen, sind dort am seltensten, und umgekehrt.} im sibirischen Eisen, hervor zu gehen, und man kann demnach wohl ohne Anstand diesen Gemengteil, von welcher Beschaffenheit er auch immer in den Meteor-Steinen erscheinen mag, insofern er nur in einer der ihm zukommenden Eigenschaften von der Grundmasse sich unterscheidet und erkennbar ausgeschieden erscheint (\emph{a potiori}) mit gleichem Namen bezeichnen.\footnote{Kugelicht kann man ihn im Allgemeinen nicht wohl nennen, da er bei weitem nicht immer, kaum vorherrschend, in dieser Form vorkommt.}

Hinsichtlich der Steinmasse im Ganzen modifiziert dieser Gemengteil, nach seinem verschiedenen quantitativen Verhältnisse, nach dem Grade seiner Ausbildung, der Art seiner Ausscheidung und seines Zusammenhanges mit der Grundmasse, und nach seinen so mannigfaltig abweichenden Eigenschaften, nicht nur oft den Kohäsions- und Aggregats-Zustand, sondern bestimmt auch damit und durch die Form und Begrenzung seiner einzelnen Massen, das Gefüge und äußere Ansehen derselben, welches, wie sich am deutlichsten auf geschliffenen Flächen ausspricht, wo derselbe nach dem Grade seiner Dichtheit und Festigkeit eine bedeutende Politur annimmt, bald Granit- oder Porphyrartig, bald\footnote{Aber nur beziehungsweise, der scheinbaren Einknetung wegen.} Breccie- oder ganz vollkommen Mandelsteinartig, bald Marmorartig erscheint.

Von den beiden metallischen Gemengteilen erscheint der eine, und zwar auf frischen, rohen Bruchflächen der Steinmasse, mehr oder weniger häufig, und mehr oder minder deutlich ausgesprochen, dem Gesicht und Gefühl erkennbar, in Gestalt einzelner, hervorragender, größerer oder kleinerer, mehr oder weniger rundlichter und glatter, oder eckiger, rauer Körner, oder ebenso beschaffener, gröberer oder feinerer Zacken, die zum Teil mit anklebenden erdigen Massenteilchen bedeckt, fest von der Masse eingeschlossen, innig mit ihr verbunden und gleichsam verwachsen sind, und von mehr oder weniger licht eisen- oder stahlgrauer Farbe und metallischem, obgleich meistens nur schwachem Glanze.

Geritzt geben diese Körner oder Zacken die Geschmeidigkeit und Weichheit der Materie zu erkennen, und dabei einen stark glänzenden, lichtern, ins Silberweiße ziehenden Strich.

Mit Gewalt aus der Masse gebrochen, worin sie bei weitem größten Teils ohne Verbindung unter sich eingeschlossen, bisweilen aber doch durch feine Äste einiger Maßen mit einander verbunden (wie z. B. in den Steinen von Eichstädt, Timochin, Tabor) zu sein scheinen, lassen sie sich auf einem kleinen Amosse sehr leicht --- obgleich nicht immer gleich, oder wenigstens nicht gleichförmig leicht --- \footnote{Gewöhnlich blieb, zumal von jenen mikroskopischen, ganz eingehüllten Eisenkörnern, von welchen bei den Massenteilchen der Grundmasse die Rede war, ein kleiner Eindruck auf meinem stählernen Ambosse (ein Ingredienz von Dumotiezs \emph{Nécessaire minéralogique}) zurück. Wahrscheinlich rührt diese partielle Sprödigkeit und Härte (die übrigens allem Meteor-Eisen, auch in den derbsten Massen, aus demselben Grunde --- wie seines Ortes gezeigt werden wird --- eigen ist) von mikroskopisch bei- oder eingemengtem Schwefeleisen her (welche Vermutung hier, so wie dort, wo sie noch durch überzeugendere Grund unterstützt werden kann, die etwas schwerere, wenigstens ungleichförmig leichte, Schmiedbarkeit, und die etwas schwierige Schweißbarkeit, so wie die Entwicklung von Schwefel-Wasserstoffgas bei der Auflösung dieses Eisens in Säuren, bestätigen). Vielleicht zum Teil auch von dessen Oxydation oder Verbindung mit Chrom; denn die Verbindung mit Nickel scheint demselben vielmehr den höheren Grad von Weichheit, Zähigkeit und Dehnbarkeit zu geben, worin derbe Massen im Ganzen jedes Schmiede-Eisen übertreffen. Auch Klaproth fand bei seinen Analysen dem, mittelst des Magnets ausgezogenen, Gediegeneisen, immer sehr viel Schwefelkies fest adhärierend.}, ohne zu reißen oder zu springen, zu den dünnsten Blättchen strecken, fletschen, deren meistens sehr gezackter Rand die ursprünglich uneben und zackig gewesene Oberfläche und Gestalt des Eisenteilchens bezeichnet. Bey diesem Fletschen springen nicht nur die fest angeklebt gewesenen erdigen Massenteilchen ab,\footnote{Diese scheinen oft mehr bloß oberflächlich anzukleben, und aus Vaquelins und Klaproths Beschreibung ihres Verfahrens bei den von ihnen vorgenommenen Analysen (indem sie gewöhnlich bei der Auflösung des aus der gepulverten Meteor-Steinmasse mit dem Magnete ausgezogenen, und sorgfältig von allen erdigen Teilchen gereinigten Eisens, noch 10 bis 20 Perzent erdige Bestandteile erhielten, und wie ersterer bei dem Steine von Charsonville ausdrücklich bemerkt, es sehr schwer hält, das Gediegeneisen ganz von der Talkerde zu reinigen), und vollends aus Laugiers neuester Zerlegung des sibirischen Eisens (nach welcher dieses, von allen erdigen Teilchen mechanisch vollkommen gereiniget, 16 Perzent Kiesel- und 15 Perzent Talkerde enthalten soll), scheint hervor zu gehen, dass die erdigen Bestandteile als Metalle oder Metalloide mit dem metallischen Eisen in irgend einem Verhältnisse chemisch verbunden sind. Eine Mutmaßung, die durch das auffallend geringe spezifische Gewicht, durch die, wie scheint, schwächere Wirkung auf den Magnet, und durch die Grunde, welche Klaproth bestimmt haben, alles in den Meteor-Steinen vorkommende Metall als regulinisch anzunehmen, wovon in der Folge die Rede sein wird, noch mehr Gewicht erhalt. Bekanntlich erhielt auch Daniell bei Untersuchung des Gusseisens, und Berzelius bei der Analyse eines gemeinen Schwefelkieses, Kieselerde (Silicium).} sondern es zeigt sich gewöhnlich auch ein schwarzes Pulver, das mehr oder weniger dem Magnete folgt. (Höchst wahrscheinlich Eisen-Oxydul oder Schwefeleisen, welches letztere, wie seines Ortes gezeigt werden wird, nicht nur dem sibirischen, sondern selbst den dichtesten und derbsten Meteor-Eisenmassen häufig eingemengt ist.) Es zeigen sich übrigens jene Körner, Zacken und gefletschten Blättchen sehr wirksam (doch wie mir däucht gefunden zu haben,\footnote{Ich will dies vor der Hand noch nicht als ausgemacht behaupten, bis ich im Stande bin, durch genauere Versuche, die eine eigene Vorrichtung notwendig machen, die Beobachtung zu bewahren.} etwas schwächer als gewöhnliches weiches Eisen und selbst als derbes Meteor-Eisen) auf die Magnetnadel, und bewähren sich durch alle diese Eigenschaften, so wie durch die Resultate der Analyse, als regulinisches Eisen.\footnote{Und zwar stets mit Nickel legiert, so dass diese Verbindung als charakteristisch für alles Meteor-Eisen im regulinischen Zustande angenommen wird, und es daher sehr befremdend wäre, wenn die Steine von Agen, nach Vauquelin, da sie doch sichtlich bedeutend viel Gediegeneisen führen, keine Spur von jenem Metalle enthielten. Nach den neuesten, zum Teil absichtlich in dieser Beziehung vorgenommenen Analysen Stromeyers, scheint das Mischungsverhältnis dieser Metall in den verschiedenen Meteor-Steinen und Eisenmassen ziemlich gleichförmig, nämlich in ersteren von 7 bis 10, in letzteren zwischen 10 und 11 Perzent des Nickels zum Eisen, und im Allgemeinen jenes des Nickels bedeutend höher zu sein, als es bisher von den meisten Analytikern angegeben wurde. (So hatte Klaproth in der Total-Masse des Steines von Erxleben nur 1/4 Perzent Nickel gefunden, indes Stromeyer 1 1/2 Perzent fand, und ersterer in der Masse eines Steines von Timochin, bei einem, selbst ausgewiesenen, Gehalte von 17 1/2 Gran regulinisch vorhanden gewesenen Eisens, kaum 1/2 Perzent; so fand derselbe im sibirischen Eisen nur 1 1/2, im mexikanischen 3 1/4, im Elbogner 2 1/2, im Agramer 3 1/2 Perzent Nickel. Howard, Vauquelin und N. A. Scherer geben bei den von ihnen vorgenommenen Analysen höchst ungleichförmige, zum Teil viel zu groß, zum Teil viel zu gering scheinende Verhältnisse von diesem Metalle an.) Dass, wie Stromeyer meint, Nickel mit Eisen, als Oxyd, auch in den erdigen Gemengteilen chemisch enthalten sei, ist deshalb, im Allgemeinen wenigstens unwahrscheinlich, weil Vauquelin in den Steinen von Chassigny, und Moser und Klaproth in jenen von Stannern durchaus keine Spur davon auffinden konnten, jene aber, welche Vauquelin in letzteren bemerkte, wohl in dem vorhandenen Schwefeleisen, und der Gehalt, den Howard davon in den abgesondert zerlegten erdigen Gemengteilen der Steine von Benares fand, ohne Zweifel in den mikroskopisch eingemengten Gediegeneisen- und Kiesteilchen enthalten gewesen sein dürfte.}

Das spezifische Gewicht dieses Eisens ist bedeutend geringer als jenes vom Roh- und Stabeisen sowohl, als insbesondere vom derben Meteor-Eisen.\footnote{Von ersteren kann man im Durchschnitt wohl 7,2 bis 7,7 annehmen, von letzteren fand ich dasselbe, und namentlich vom mexikanischen, kroatischen, böhmischen, ungarischen und peruanischen Eisen zwischen 7,600 und 7,830. Höchst merkwürdig ist, dass jenes vom sibirischen Eisen gleichsam das Mittel zwischen letzteren und jenem des Eisens aus Meteor-Steinen halt; ich fand dasselbe = 7,540, nur etwas geringer als Karsten, der es mit 7,573, wie es auch Werner und Hausmann annahm, angibt. Graf Bournon gibt es mit 6,487 an, diesem mochte aber wohl eine Irrung zum Grunde liegen.} Graf Bournon fand es bei jenem aus einem Steine von Tabor = 6,146, und ich bei einem großen Korne und einem Blättchen aus einem Steine von L'Aigle zwischen 6,00 und 6,60.

Auf geschliffenen und polierten Flächen erscheint dieser Gemengteil noch ungleich deutlicher, da durch Schnitt und Politur die kleinsten Metallteilchen rein und spiegelicht glänzend zur Ansicht kommen. Er zeigt sich hier nun erst in seiner wahren Menge,\footnote{Es ist sehr zu beklagen, dass die Analytiker bisher so wenig Rücksicht auf den Gehalt der Meteor-Steine an mechanisch eingemengtem Gediegeneisen genommen, und denselben gewöhnlich nur im Ganzen, bald, wie Vauquelin alles als Oxyd, wie es aus der ganzen Masse durch die Operation erhalten wurde, bald, wie Klaproth alles als regulinisch, nach Kalkül, angegeben haben; so dass man weder von dem Zustande, in welchem sich das Eisen in der Steinmasse befand, ob ganz rein und gediegen, oder mehr oder weniger mit Schwefel vererzt, als Kies- oder Schwefeleisen, oder mehr oder weniger mit Oxygen verbunden, als Oxyd oder Oxydul, noch weniger von den gegenseitigen quantitativen Verhältnissen etwas erfahrt, wie dies bis jetzt beinahe nur aus Stromeyers musterhaften, leider nur wenigen Analysen, zu ersehen ist. Es wäre zu wünschen, dass sie jedes Mahl das mechanisch eingemengte Gediegeneisen, so genau und rein wie möglich, aus der fein gepulverten Steinmasse von bestimmtem Gewichte mittelst einer Magnetnadel ausziehen, und dieses für sich angeben und untersuchen möchten.} die gewöhnlich nicht gering ist, so dass er nach einer beiläufigen, oberflächlichen Abschätzung bisweilen 1/3 oder 1/5, d. i. 0,20 bis 0,30 (wie z. B. in den Steinen von Eichstädt, Timochin, Tabor, Charsonville \emph{zc.}) der ganzen Masse beträgt, meistens aber doch nur den zehnten oder zwanzigsten Teil des Ganzen ausmachen möchte, d. i. 0,10 bis 0,05 (wie in den Steinen von L'Aigle, Lissa \emph{zc.}), und oft auch in äußerst geringer Menge, so dass er kaum 1/50 der Masse beträgt, = 0,02 (wie in den Steinen von Mauerkirchen, Siena, Benares \emph{zc.}), ja selbst noch weniger (wie in den Steinen von Parma, Eggenfeld), zuletzt ganz und gar fehlt (wie in den Steinen von Chassigny und Stannern).\footnote{Hinsichtlich des merkwürdigen Wechselverhältnisses, welches zwischen dem Gehalte der Meteor-Steine an solcher Gestalt mechanisch eingemengtem --- ganz ausgeschiedenem --- regulinischen Eisen und jenem an Eisen in mehr oder weniger geschwefeltem und oxydierten Zustande zu bestehen, und der ganz besonderen Verbindung, in welcher ersteres (?) mit den erdigen Bestandteilen der Steinmasse verbunden, vollkommen verlarvt, vorzukommen scheint, verweise ich auf die Bemerkungen bei der Abhandlung der übrigen Gemengteile (des Schwefeleisens und des Eisenoxydes).}

Es zeigt sich derselbe hier teils, zumal wenn häufig vorhanden --- und in diesem Falle meistens ziemlich gleichförmig verteilt --- in zarten, äußerst feinen, zum Teil mikroskopischen Punkten (den Ausgängen senkrecht gegen die Oberfläche stehender Zacken), teils in größeren oder kleineren, stärkeren oder schwächeren, mehr oder weniger zahnigen oder zackigen, und klein- und feinästigen, gebogenen, winkeligen Adern, Linien und Flecken (den Durchschnitten mehr oder weniger horizontal gegen die Oberfläche liegender Zacken), die bisweilen durch zarte Zweige, mehr oder minder vollkommen, einzeln wenigstens, mit einander verbunden sind\footnote{Ein, obgleich nur noch höchst unvollkommener und sehr unterbrochener Zusammenhang, der aber doch schon einige Ähnlichkeit mit dem Eisengerippe der sibirischen, zumal der sächsischen und jener angeblich aus Norwegen stammenden Eisenmasse zeigt.}; teils --- obgleich seltener, und meistens nur, wo der Gehalt im Ganzen geringe --- in einzelnen beträchtlich großen, rundlichten, ovalen, keilförmigen, mehr oder weniger dreieckigen, gewöhnlich scharf begrenzten, und gar nicht zackigen Flecken von einigen Linien im Durchmesser (den Durchschnitten von größeren, gewöhnlich so gestalteten, und meistens platt gedrückten Körnern oder Massen, in welchen sich dieser Gemengteil, bisweilen von Erbsen- bis Haselnuss-Größe, und von 20 bis 30 und mehr Gran am Gewichte --- wie in den Steinen von Ensisheim, L'Aigle, Barbotan, Salés \emph{zc.} eingemengt findet.\footnote{Auf solche, oft ganz im Innern der Steinmasse verborgen liegende, größere Eisenmassen, und überhaupt auf das mechanisch eingemengte Gediegeneisen, wenn es im Ganzen nicht sehr häufig vorhanden ist, indem dasselbe sonst sehr ungleichmäßig verteilt zu sein pflegt, muss bei Bestimmung des spezifischen Gewichtes, so wie bei der Analyse eines Bruchstückes, besondere Rücksicht genommen werden.}

Auf bloß geschnittenen, rohen, noch unpolierten Flächen zeigt sich die Farbe dieser Eisenteilchen --- die hier ihre Weichheit und Geschmeidigkeit durch Erhabenheit und durch Streifung ihrer Oberfläche (welche das nicht ganz gleichförmig vorrückende, schneidende Instrument, Rad oder Säge, bewirkte) bewähren, und bisweilen gekörnt, körnig angehäuft, fast wie geträuft erscheinen --- mehr oder weniger licht eisen- stahl- oder zinkgrau, und der Glanz, rein metallisch zwar, aber etwas schwach; auf polierten Flächen dagegen zieht sich erstere mehr oder weniger ins Silberweiße, und letzterer wird sehr stark und spiegelnd.

Diese Eisenteilchen kommen übrigens in beiden erdigen Gemengteilen eingestreut vor, in dem olivinartigen doch offenbar ungleich weniger und zarter, und, wie es beinahe scheint (namentlich bei den Steinen von Charsonville, Apt, Toulouse), um so sparsamer, je unvollkommener derselbe ausgesprochen, und je mehr ähnlich er noch der Grundmasse ist\footnote{Howard gibt inzwischen, da er doch beide erdige Gemengteile des Steines von Benares möglichst getrennt und für sich analysierte, in beiden ein ganz gleiches Verhältnis von Eisenoxyd an, nämlich 34 Perzent.}; aber, wie es auf der andern Seite scheint, zumal im vollkommeneren Zustande desselben (wie bei den Steinen von Eichstädt, Timochin, Benares), mehr um ihn herum angehäuft, die Massen desselben gleichsam umgebend, einschließend.\footnote{Wieder eine Annäherung des Entwickelungszustandes der Steinmasse der Meteor-Steine und ihrer Gemengteile an die sibirische Eisenmasse.} Am häufigsten möchten sie wohl oft, in dem ganz besonderen Zustande, fest in die offenbar veränderten Massenteilchen der Grundmasse (wie oben bei dieser erwähnt worden ist) eingehüllt vorkommen.

Von diesem metallischen Gemengteile hängt, obgleich nicht ganz ausschließlich (da das eingemengte, und selbst, wie scheint, das chemisch mit den erdigen Teilchen verbundene, mehr oder weniger oxydierte Eisen, zumal aber das eingemengte Schwefeleisen, bisweilen einige Wirkung zeigen), die Wirkung der Meteor-Steine im Ganzen auf die Magnetnadel ab, die demnach nach dem so sehr abweichenden quantitativen Verhältnisse desselben von dem ganz Unmerklichen bis ins sehr Starke geht, und jener des massiven Gediegeneisens sich nähert. Es modifiziert derselbe ferner, vermöge seines verschiedenen quantitativen Verhältnisses (wobei jedoch der Gehalt der Masse an Eisen-Oxyd --- an verhülltem oder gar verlarvtem Eisen --- und an Schwefeleisen zu berücksichtigen kommt), nicht nur das spezifische Gewicht der verschiedenen Meteor-Steine, und selbst --- der oft sehr ungleichen, daher wohl zu berücksichtigenden Verteilung und Einmengung wegen --- der Bruchstücke eines und desselben Steines, sondern auch insbesondere, und nach Maßgabe der Beschaffenheit der Teilchen (ob gröber oder feiner, glatter oder zackiger), durch mechanische Zusammenhaltung und durch eine (vielleicht erst in der Folge durch Oxydation in der Atmosphäre) vermittelte innigere Verbindung aller Gemengteile unter sich, die Dichtigkeit und Festigkeit, den Kohäsions- und Aggregats-Zustand der ganzen Masse.

Es ist demnach dieser Gemengteil, zumal derselbe in so vielseitigen Beziehungen, insbesondere im quantitativen Verhältnisse, und in Größe, Form und Verbindung seiner einzelnen Massen, so auffallende Verschiedenheiten zeigt, für die verschiedenen Meteor-Steine sehr charakterisierend, wie er denn auch das Ansehen derselben, zumal auf polierten Flächen, sehr mannigfaltig modifiziert, und manche oft ausschließlich dadurch erkennbar und unterscheidbar macht.

In Begleitung dieses Gemengteiles, und zwar wo nicht ausschließlich, doch vorzugsweise nur desselben,\footnote{Die Fälle, wo dasselbe auch bei deutlich ausgesprochenen Schwefeleisen-Massen Statt hat, scheinen mir größten Teils zweifelhaft. So viel ist gewiss, dass die Erscheinung verhältnismäßig höchst selten ist bei Steinen, die wenig Gediegeneisen, und doch viel, selbst sehr viel Schwefeleisen enthalten, wie die von Parma, Mauerkirchen, Siena, Benares, und gar nicht, wo das Gediegeneisen ganz fehlt, wie bei jenen von Chassigny und Stannern, obgleich bei letzteren der Gehalt an Schwefeleisen nicht unbedeutend ist. Inzwischen glaubte Klaproth doch diese Erscheinung der Verwitterung der Kiespunkte (des fein eingesprengten Schwefeleisens) zuschreiben zu sollen.} und wo nicht ursprünglich, doch stets in der Folge, wenn die Steine einige Zeit der atmosphärischen Luft ausgesetzt waren,\footnote{Klaproth, der die Meinung hegte, dass die Meteor-Massen und ihre Gemengteile durchaus keiner Einwirkung des Oxygens ausgesetzt waren, als etwa der momentanen während des schnellen Durchzuges durch unsere Atmosphäre --- vor der sie übrigens auch durch die blitzschnell erzeugte oberflächliche Rinde sogleich geschützt wurden --- und daher durchaus keinen Oxydations-Zustand irgend eines Gemengteiles annehmen zu dürfen glaubte, schrieb diese Erscheinung ausschließlich der späteren Einwirkung der atmosphärischen Luft zu; inzwischen scheinen, abgesehen von den leicht zu machenden Einwürfen gegen jene vorgefasste Meinung im Allgemeinen, mehrere Beobachtungen auch gegen diese daher rührende Folgerung zu sprechen. Mehrere ganze und durch vollkommene Überrindung vor dem Eindringen der atmosphärischen Luft geschützte, und dabei ziemlich dichte und kompakte Steine, die ich selbst zu zerschlagen Gelegenheit hatte, und fünf verschiedene, von welchen während meiner Anwesenheit in Paris beträchtlich Stucke --- freilich nach Steinschneiderart, aber schnell und mit möglichster Verwahrung gegen Durchnässung --- abgeschnitten wurden, zeigten im Innern ihrer Masse dieselbe Erscheinung. Dasselbe beobachtete Bergrath Reuß an einem von ihm zerschlagenen Steine von Lissa, kaum noch drei Monat nach dem Falle. Dagegen zeigen Bruchstücke von mehreren, an Gediegeneisen sowohl als Schwefeleisen ziemlich reichhaltigen Steinen, die seit vielen Jahren der atmosphärischen Luft, und selbst häufiger Betastung ausgesetzt, auch an einer Fläche angeschliffen und poliert worden waren, noch bis zur Stunde keine Spur davon.} erscheinen auf unvollkommen überrindeten, ursprünglich oder späterhin zufällig von Rinde entblößten Flächen, zumal auf frischen Bruchflächen, und zwar nach Maßgabe der Menge, Größe und Gestalt der Eisenteilchen (vielleicht auch nach der individuellen Beschaffenheit derselben),\footnote{Wie einige an Gediegen- und Schwefeleisen ziemlich reichhaltige Steine, z. B. die von Erxleben, Tipperary, Limerick, und zum Teil selbst einige von Lissa, zu beweisen scheinen, die kaum eine Spur zeigen.} mehr oder weniger häufige, größere oder kleinere, verschieden gestaltete Flecke, oft nur zarte Punkte, von matter, licht ockergelber, durch eine Reihe von Abstufungen ins Gelblich- und Rötlichbraune, bis ins Dunkelbraune verlaufender Farbe, wahre Rostflecke, die sowohl durch die verhältnismäßige Menge, als zum Teil auch durch die Gestaltung und Größe, umso mehr ein charakteristisches Merkmal für viele Meteor-Steine abgeben, als sie, zumal auf rohen, unpolierten Flächen, weit mehr als die Metallteilchen selbst auffallen, auf polierten Flächen aber der ganzen Steinmasse ein ausgezeichnetes, marmoriertes Ansehen geben, so wie sie wohl auch den Zusammenhalt und Kohäsions-Zustand derselben als vermittelndes Bindungsmittel zu verstärken scheinen.

Es ist bemerkenswert, dass diese Rostflecke, wie es scheint, nie auf den, zumal polierten, Flächen der Eisenteilchen, auch wenn sie Jahre lang der Luft ausgesetzt waren --- wobei sie kaum etwas von ihrer Politur einbüßten --- (wie denn auch das Meteor-Eisen überhaupt, vielleicht wegen der Verbindung mit Nickel, nicht so leicht rostet, und auch mehr der Einwirkung der Säuren widersteht, als gemeines Eisen), sondern immer nur auf ihrer rauen Oberfläche und am Rande derselben, insbesondere aber in den erdigen, von mehreren Eisenteilchen eingeschlossenen Zwischenräumen,\footnote{Es scheint demnach, dass es nicht die Gediegeneisenteilchen selbst sind, welche diese Oxydation erlitten haben, sondern vielmehr die Atome von Eisen-Oxydul und vielleicht von Schwefeleisen, welche jene einhüllen.} die sie oft ganz durchdringen, erscheinen, und dass sie oft, wie mir däucht, in Folge der Zeit, einen etwas fettigen Glanz, unvollkommen blätterige oder schalige Absonderungen, und ein Pflinz- oder Eisenspatartiges, bisweilen fast Harzähnliches Ansehen gewinnen.

Die Massenteilchen dieser Flecke, die bei manchen an Gediegeneisen sehr reichhaltigen Meteor-Steinen (wie z. B. bei jenen von Eichstädt und Timochin) beinahe die größere Hälfte der Grundmasse betreffen, zeigen sich unter dem Mikroskope teils als erdige, ockrige, gelbe, pommeranzen- und rötlich-gelbe, zum Teil aber als spatartige, und dann glänzende oder schillernde, ins Dunkelgelbe und Rotbraune ziehende bis ins Glasige, und dann ins Rothe verlaufende, kleine, gleichsam zusammen gekittete Massen, die mit mikroskopisch zarten Metallteilchen gemengt sind, und zum Teil in jene Feldspatartigen Massenteilchen, welche bei der Grundmasse erwähnt worden sind, übergehen. Sie werden von der Magnetnadel lebhaft angezogen, und lassen sich äußerst schwer, zum Teil gar nicht zerstoßen, halten selbst am Ambosse mehrere starke Schläge aus, und geben dann ein mehr oder weniger feines, lichter oder dunkler gelbes oder rötlich gelbes, erdiges Pulver, das zum Teil noch retractorisch ist, und ein oder mehrere Blättchen hartes und sehr zähes, metallisch glänzendes, licht eisengraues Gediegeneisen.\footnote{Aus dieser Beschaffenheit der Massenteilchen scheint wohl hervor zu gehen, dass diese Rostflecke kein Erzeugnisse einer schnellen und oberflächlichen, und bloß durch die atmosphärische Luft bewirkten Oxydation des Gediegeneisens, und noch weniger die Folge einer bloßen Verwitterung des Schwefeleisens sein können; dagegen geben vielmehr Ansehen, Glanz, Härte, Sprödigkeit, und die Eigenschaft im Wasser nicht merklich, oder doch nur zum Teil, die Farbe zu ändern, Veranlassung, dieselben mit Eisenspat zu vergleichen.}

Der andere metallische Gemengteil, der wohl nie ganz fehlen möchte,\footnote{Außer etwa bei den Steinen von Chassigny, wo sich durchaus nichts dafür zu erkennen gibt, und bei deren Analyse auch Vauquelin keine Spur von Schwefel finden konnte.} obgleich er gewöhnlich in einem ungleich geringeren, oft äußerst geringen, und offenbar gerade mit Zunahme des vorhergehenden in einem abnehmenden Verhältnisse\footnote{Dieses merkwürdige Wechselverhältnis spricht sich bei den meisten Meteor-Steinen sehr auffallend aus. So findet man bei den an Gediegeneisen sehr reichhaltigen Steinen von Eichstädt, Timochin, Tabor, Charsonville, kaum ein deutliches Korn von Schwefeleisen, und es ist dasselbe äußerst zart eingesprengt; dagegen erscheint es bei den an jenem minder reichhaltigen von Ensisheim, Salés, Lissa \emph{zc.} schon weit mehr und in größeren Massen; bei den eisenarmen Steinen vollends von Siena, Mauerkirchen, Benares, und besonders Parma, und den ganz eisenfreien von Stannern auffallend häufig und in ausgezeichnet großen Massen.} vorhanden, und oft, zumal auf rohen unpolierten Flächen, äußerst schwer zu erkennen und vom vorigen zu unterscheiden ist, zeigt sich auf solchen Flächen mehr oder weniger häufig und deutlich, in äußerst zarten, meistens mikroskopischen, teils einzeln eingestreuten, teils mehr oder weniger zusammen gehäuften Punkten und Körnern, seltener in größeren, bröcklich oder körnig zusammen gehäuften Massen von sehr verschiedener, ganz unregelmäßiger Gestalt, und mehr oder minder dann verbrochen, zerrissen und zerklüftet, und bei diesem letzteren Vorkommen, von einem unebenen, feinkörnigen, bisweilen versteckt blätterigen, seltener unvollkommen und klein muschlichen Bruche, nicht selten mit kristallinischen Facetten, und unbestimmt eckige, ziemlich scharfkantige Bruchstücke gebend.

Es haben diese Körner und Massen stets ein rein metallisches Ansehen, und auf rohen Flächen der Steinmasse, zumal die kleinsten derselben, gewöhnlich einen starken, oft sehr starken, spiegelnden, metallischen Glanz, und eine mehr oder minder rein zinkgraue, oft auch beinahe zinn- oder silberweiße, gewöhnlich aber ins Rötliche --- beinahe wie der Kupfernickel --- meistens doch ins Gelbliche, Speis- oder Messing-Gelbe ziehende Farbe.\footnote{Diese oft sehr auffallenden Abweichungen in der Farbe, auf welche schon Chladni aufmerksam machte, und deren nicht selten mehrere in einem und demselben Bruchstucke eines Steines vorkommen, scheinen wohl, zumal sie in einigem Einklange mit den übrigen Eigenschaften, als mit der Harte und der Retractibilität stehen, au fremdartige Beimischungen (Nickel, Chrom, Mangan, Silicium), oder doch auf ein verschiedenes Verhältnis vom Schwefel zum Eisen, oder auf eine anderweitige Zustandsverschiedenheit dieses letzteren hinzudeuten.} Größere Massen erscheinen bisweilen, obgleich nur selten, matt oder doch minder glänzend, und dunkelgrau oder bräunlich, auch tombakbraun, rostbraun oder kupferrötlich, und bisweilen auch pfauenschweifig, dunkelblau, rot und Messinggelb angelaufen.

Geritzt geben diese Metallteilchen sogleich ihre Sprödigkeit zu erkennen, wodurch sie sich von den vorigen sehr auffallend unterscheiden, und mit Gewalt aus der Steinmasse gebrochen, aus welcher sie sich mehr oder weniger leicht, stückweise aussprengen lassen, kann man sie auch mehr oder minder leicht zum feinsten Pulver zerstoßen und zerreiben, das dann eine mehr oder weniger matte und schwärzliche Farbe annimmt. Jene Sprödigkeit, so wie das ganze Ansehen und Verhalten dieser metallischen Massen sowohl im Ganzen als in ihren Massenteilchen, die Entwickelung von Schwefel-Wasserstoffgas bei Behandlung mit Salzsäure, und vollends die Resultate der Analysen, geben die Natur dieses metallischen Gemengteiles als Eisen- oder Schwefelkies, und zwar, nach letzteren, wegen des geringen Verhältnisses des Schwefels zum Eisen,\footnote{Die meisten bisherigen Analysen von Meteor-Steinen lassen zwar nur durch einen bestimmt, oft auch ganz unbestimmt, und selbst nur als Spur angegebenen Gehalt an Schwefel auf die gewesene Gegenwart von geschwefeltem Eisen als Gemengteil derselben, keinesweges aber auf dessen quantitatives Verhältnis zur Steinmasse, am wenigsten vollends auf dessen individuelle Zusammensetzung und auf das Verhältnis des Schwefels zum Eisen in demselben schließen. Inzwischen hat doch Howard schon das letztere naher bestimmt, indem er 14 Gran Kies aus einem Steine von Benares für sich analysierte, und --- obgleich mit unberechenbarem Verluste an Schwefel --- 2 Gran desselben mit 10 1/2 Gran Eisen verbunden, demnach beiläufig 20 Perzent Schwefel fand. Aus Stromeyers neuesten Analysen der Steine von Erxleben und Köstritz ergibt sich (aber freilich nach stöchiometrischem Kalküle, wobei es wohl in Frage stehen dürfte, ob bei diesen rätselhaften Produkten so ganz zuversichtlich darauf anzugehen sein mochte) bei ersteren ein Gehalt an Magnetkies von fast 8, bei letzteren von beinahe 7 Perzent, und bei beiden ein gleiches --- freilich präsumtives --- Mischungsverhältnis von 58 Schwefel zu 100 Eisen (wie es Berzelius für den terrestrischen Magnetkies statuiert hat). Schon Howard hat Nickel --- und zwar in einem auffallend großen, unwahrscheinlichen Verhältnis --- von beinahe 10 Perzent mit diesem Schwefeleisen in Verbindung gefunden, und da Vauquelin wenigstens (Moser und Klaproth nicht) eine Spur von jenem Metalle auch in der Masse der Steine von Stannern fand, die doch gar kein reines Gediegeneisen enthalten, so dürfte es wohl einen beständigen Bestandteil desselben ausmachen.} und da er auch meistens mehr oder weniger auf den Magnet wirkt,\footnote{Lange aber nicht aller, wie schon Graf Bournon ausdrücklich von jenem aus den Steinen von Benares, und Klaproth von dem, selbst speisgelben, aus jenem von Lissa und Erxleben bemerkt, und ich auch von jenem aus den Steinen von Siena und Mauerkirchen behaupten kann, von welchem auch nicht die kleinsten Atome von der Magnetnadel in Bewegung gesetzt werden; übrigens in sehr verschiedenen Graden. Äußerst schwach z. B. wirkt jener aus den Steinen von Parma, und mehr \emph{en masse} als im Pulver, vielleicht bloß in Folge der umgebenden oder anhangenden Atome von Gediegeneisen oder Eisen-Oxydul; hier und da einiger aus der Masse der Steine von Siena und Lissa, etwas starker; äußerst stark dagegen und selbst in den kleinsten Atomen, jener aus den derben Gediegeneisen-Massen. Und ich glaube bemerkt zu haben, dass der verschiedene Grad von Retractibilität dieses Kieses überhaupt mit der Menge und Masse des vorhandenen Gediegeneisens in einem Verhältnisse stehe. Ob derselbe übrigens von den oben erwähnten verschiedenen metallischen Beimischungen, oder von einer Zustandsverschiedenheit des Eisens, oder von dem Mischungsverhältnisse des Schwefels zum Eisen abhänge, will ich vor der Hand dahingestellt sein lassen, und nur die Analytiker darauf aufmerksam gemacht haben.} als Schwefeleisen im Minimum oder als Magnetkies, zu erkennen.

Auf geschliffenen und polierten Flächen erscheint auch dieser Gemengteil ungleich deutlicher, da die kleinsten Teilchen mehr zur Ansicht kommen (obgleich viele während des Schnittes ihrer Sprödigkeit halber ausgesprengt werden mögen), und sich besser, ja oft ausschließlich nur hier, von jenen des Gediegeneisens unterscheiden lassen, indem sie immer einen etwas schwächeren Glanz (wahrscheinlich als Folge des Anlaufens durch die angewendete Feuchtigkeit während des Schnittes) und eine dunklere, stets ins Stahl- oder Zink-Graue fallende, und meistens ins Rötliche oder Gelbliche ziehende Farbe haben, und sich gewöhnlich (zumal wenn in etwas größeren Massen), rissig, zersprungen und zerklüftet, oder äußerst zartkörnig angehäuft zeigen. Sie sind übrigens mehr oder weniger häufig, sehr ungleichförmig durch die ganze Steinmasse zerstreut, und ebenso wie die Gediegeneisenteilchen in der Grundmasse sowohl, als, und zwar in einem ähnlichen geringeren Verhältnisse, im olivinartigen Gemengteile, und erscheinen als äußerst zarte, oft mikroskopisch feine Punkte, entweder einzeln oder gruppiert, und in größeren oder kleineren, teils zart und vielfach ausgezackten und ausgeschlitzten, teils scharf begrenzten, dichten Flecken.

Von dem ganz mikroskopisch feinen Vorkommen dieser Kiesteilchen und deren innigen Verbindung mit den Gediegeneisenteilchen, ist bereits oben bei diesen Erwähnung geschehen; so wie auch, dass sie nur selten, wenn je, unmittelbar von Rostflecken begleitet sind.

Es ist dieser Gemengteil\footnote{Dieser Gemengteil ist es vorzüglich, der die Erklärung, selbst mancher Nebenerscheinungen und Veränderungen, welche mit diesen Massen offenbar in unserer Atmosphäre erst vorgehen, so schwierig macht, und zu den widersprechendsten Hypothesen Veranlassung gab. So ließe sich z. B. --- wie es denn auch, jenes und manches andern Einspruches ungeachtet, ziemlich allgemein geschieht --- das Leuchten, Glühen, Funkensprühen und endliche Zerplatzen der Feuerkugeln, und vollends die Bildung der Rinde (anscheinend! das ausgesprochene Produkt eines gewöhnlichen Schmelz-Prozesses) über die vereinzelten Bruchstucke derselben, durch --- unter mehr oder weniger annehmbaren Voraussetzungen zulässliche --- Entwickelung oder Freimachung von Warmestoff am kürzesten und leichtesten erklären, wenn nicht das häufige Vorkommen dieses Gemengteiles in der ganzen Masse, und selbst an der Oberfläche, und ganz dicht unter der Rinde der Steine, und namentlich auch in den ganz reinen und derben Gediegeneisen-Massen, im ganz unveränderten Zustande seiner oft aufs höchste ausgesprochenen metallischen Beschaffenheit bei dessen leichter Zerstörbarkeit durch Hitze dagegen stritte, umso mehr als diese, wenigstens in unserer Atmosphäre, dem Einflüsse des Oxygens, und bei der schweren Schmelzbarkeit der Stein- und vollends der Eisenmassen (welche letztere gerade durch ihr scheinbar geschmolzenes Ansehen manche Physiker verleiteten, sie geschmolzen flüssig bis zur Erde gelangen zu machen), einen Grad voraussetzen wurde, mit dem sich das Bestehen eines Schwefeleisens schlechterdings nicht vereinbaren ließe.} für manche Meteor-Steine sehr charakteristisch (zu welchem Ende aber notwendig eine Fläche des Steines abgeschliffen werden muss), teils durch seine Menge (wie für die Steine von Benares, Lissa, Parma \emph{zc.}), oder durch seine Seltenheit (wie für jene von Eichstädt, Timochin, Tabor, Charsonville \emph{zc.}), teils durch die Größe oder ausgezeichnete Farbe seiner Massen (wie für die Steine von Parma, Stannern, Mauerkirchen, Benares \emph{zc.}).

Außer jenen vier, strengen Sinnes zur Wesenheit der Meteor-Steine, als gemengten Massen, gehörigen, dem freien Auge mehr oder weniger leicht unterscheidbaren Gemengteilen, findet sich, wenigstens bei vielen, wo nicht allen, noch ein fünfter, der aber, auf rohen sowohl als auf geschliffenen Flächen der Steine, meistens nur mit Hülfe eines Vergrößerungsglases, und selbst dann nur schwer und sparsam, am leichtesten noch und am häufigsten in der gröblich gepulverten Steinmasse unter dem Mikroskope aufgefunden werden kann, und der in Gestalt äußerst zarter, unförmlicher, sehr ungleichförmig verteilter und einzeln eingestreuter, nur höchst selten in äußerst kleinen Partien zusammen gehäufter, von der Masse fest eingeschlossener Punkte oder Körner von matter, schwärzlich-brauner oder schwarzer Farbe erscheint. Es zeigen sich diese Körner leicht zerreiblich, und geben ein gleichförmiges Pulver; sie werden mehr oder minder lebhaft von der Magnetnadel angezogen, und sind wohl ohne Zweifel für ein Oxyd oder Oxydul von Eisen,\footnote{Bekanntlich hat Klaproth, dem wir in Deutschland die frühesten Analysen, und im Ganzen --- wo ich nicht irre --- die von sieben verschiedenen Meteor-Steinen verdanken, die Vermutung geäußert: es käme das Eisen in allen Meteor-Steinen, ohne Ausnahme, selbst in jenen, wo sich durchaus keine Spur davon, weder physisch noch oryktognostisch, als rein und gediegen zu erkennen gibt (wie z. B. in jenen von Stannern --- wovon er doch selbst ein Stück analysierte ---), stets nur im regulinischen Zustande vor, und dass selbst jenes --- wie auch der Nickel und das Mangan --- das sich in einem größeren oder geringeren Anteil, auch chemisch ausgesprochen, im offenbar oxydierten Zustande fände, nicht ursprünglich so in denselben enthalten gewesen, sondern erst --- so wie die Rostflecke --- später als Folge der Oxydation des zuvor frei und gediegen vorhanden gewesenen, in der atmosphärischen Luft entstanden sei; dass dagegen alles physisch und oryktognostisch unerkennbare und chemisch mit den erdigen Gemengteilen verbundene Eisen regulinisch in diesen (im oxygenfreien Zustande mit den einfachen Erden verbunden), in einer gegenseitig sich durchdringenden Mischung (wodurch auch dessen Wirksamkeit auf den Magnet aufgehoben werden kann) demnach bloß verlarvt --- sich befinden möchte. Es ist nicht in Abrede zu stellen, dass die Motive, welche diese Mutmaßung veranlassten (die höchst wahrscheinliche Herstammung dieser Massen aus Regionen, wo, ebenso wahrscheinlich, kein Oxygen vorhanden sei; --- das häufige Vorkommen des so rein ausgesprochenen, ganz unveränderten, und doch so leicht zerstörbaren Schwefeleisens in denselben; --- die Ermangelung irgend einer Anzeige von Oxygen-Gehalt bei den wiederholten Analysen; --- und endlich die Resultate des Kalküls bei Bestimmung des quantitativen Verhältnisses der verschiedenen Bestandteile der von ihm zerlegten Steine ---), dieselbe gerade nicht abnötigten, im Gegenteil manche Einspruche gestatten (wovon gleich einen z. B. der Zustand der übrigen Gemengteile, jener der erdigen Bestandteile, als Oxyde metallischer Basen, machen dürfte), und dass damit die bestimmt ausgesprochenen Befunde anderer Analytiker im offenbaren Widerspruche stehen, welche den Gehalt der Meteor-Steine an oxydiertem Eisen und andern Metallen (Mangan, Chrom, Nickel), und zwar nicht bloß im Zustande von mechanischer Einmengung (in welchem Falle derselbe etwa nach Klaproth als Produkt späterer Erzeugung angesehen werden könnte), sondern ganz verlarvt und chemisch mit den erdigen Bestandteilen verbunden, unwiderleglich dartun. (So erklärte Howard allen Gehalt an Eisen der von ihm analysierten Steine --- insofern sich dasselbe nicht als gediegen oder geschwefelt aussprach, --- so Vauquelin --- der, meines Besinnens, sogar an irgend einem Orte, alles, selbst das vollkommen regulinisch vorkommende Meteor-Eisen (wahrscheinlich der beobachteten partiellen Sprödigkeit und in eben dem Grade schweren Schmiedbarkeit wegen) stets als etwas oxydiert erklärt --- ebenso, und namentlich den ganzen, allem Ansehen nach durchaus verlarvten, doch 31 Perzent betragenden Eisengehalt des Steines von Chassigny; so Moser und derselbe jenen von 27 bis 29 Perzent --- wovon nur wenig auf den vorhandenen Kies fällt, und eben so wenig sich als freies Oxyd ausspricht --- der Steine von Stannern, für vollkommen oxydiertes Eisen; so gibt endlich Stromeyer den Gehalt an wahrhaft --- aber nur oxydulirten --- Eisen der Steine von Erxleben und Köstritz auf 5 Perzent an.) Inzwischen verdient doch, meines Erachtens, Klaproths Vermutung noch alle Beachtung und besondere Aufmerksamkeit, und dies umso mehr, als dieselbe durch die --- oben in einer Note bei den Gediegeneisen --- bereits erwähnten Umstände (der innigen, wenn gleich nur mechanisch scheinenden Verbindung der Eisen- und Erdeteilchen, selbst in den mikroskopischen Massenteilchen, --- der selbst auf chemischem Wege erst möglichen, vollkommenen Scheidung beider, --- dem bei verschiedenen Meteor-Eisen so merklich abweichenden, und bei jenem aus Meteor-Steinen so auffallend geringen spezifischen Gewichte, und den anscheinend verschiedenen Graden von Retractibilität desselben ---) neue Bekräftigung zu erhalten scheint, und in der, dem chemisch ausgewiesenen oder sinnlich wahrnehmbaren Gehalte an Eisen, oft nicht entsprechenden Angabe des spezifischen Gewichts mancher Steine, und selbst, wie mir däucht, in obigen und manch andern, ziemlich sich widersprechenden Resultaten, insbesondere aber in jenen der, gewiss höchst verlässlichen Analysen Stromeyers (nach welchen ein nur sehr geringer Teil --- und zwar bei anscheinend nur wenig Gediegeneisen und Kies führenden Steinen --- von Eisen, und dies nur im Minimum oxydiert, dagegen ein bedeutenderer Anteil an regulinischen ausgewiesen wird, als nach jenem Anscheine erwartet werden sollte, wovon demnach der Überschuss in den erdigen Gemengteilen verlarvt enthalten sein müsste) einige Bestätigung finden möchte. Dass Silicium jene Verbindung wenigstens vermitteln müsste, dürfte wohl ebenso gut hier, als bei den von Daniell und Berzelius gefundenen ähnlichen Verbindungen von Kieselerde mit metallischem Eisen, und von Laugier in sibirischen, vorauszusetzen sehn, worauf vielleicht schon der besondere Zustand, in welchem alle obige Analytiker die Kieselerde in den Meteor-Steinen überhaupt gefunden haben, hindeutet.} von Mangan etwa zum Teil, und vielleicht auch von Chrom anzusehen. 

Höchst selten, und nur bei einigen Meteor-Steinen (nach meiner Überzeugung und deutlich nur bei jenen von Chassigny und Lissa) erscheint dieser Gemengteil in etwas größeren, ebenso zerstreuten Massen von beinahe pechschwarzer Farbe, und ziemlich starkem, etwas fettigem Glanze, die wenig oder gar nicht auf den Magnet wirken.

Für Partikelchen von Rinde-Substanz, wofür sie, wenigstens zum Teil, Chladni anzusehen geneigt ist, kann man diese Körner, zumal ersterer Art, nicht wohl erkennen, da sie nicht nur in ganzen Ansehen und durch ihre Retractibilität (vorzüglich bei Steinen, wo es die Atome der Rinde gar nicht sind, wie z. B. bei jenen von Stannern, wo sie doch gerade am häufigsten vorkommen) sich davon unterscheiden, sondern auch die Art des Vorkommens und der Einmengung aller --- so sparsam und vereinzelnt, und überhaupt so selten, --- so mikroskopisch zart und isoliert, gar nicht in die Steinmasse übergehend (wie dies doch bei der oberflächlichen Rinde im Kontakte mit jener so auffallend Statt hat), und in einem gekörnten Zustande --- mit jeder möglich denkbaren Art von Entstehung und Bildung von Rinde-Substanz mitten in der Steinmasse, namentlich aber mit jener durch Einknetung, im Widerspruche steht. Leichter könnte man diese Körner, wenigstens letztere, dunklere, glänzende, mit dem olivinartigen Gemengteile oder mit der Substanz, die auch in Adern vorzukommen pflegt, und von welcher sogleich die Rede sein wird, verwechseln, mit welchen diese aber auch (zumal jene in den Steinen von Lissa) ziemlich gleicher Natur sein möchten. Am häufigsten und deutlichsten, und zwar größten Teils von bedeutender, dem freien Auge wenigstens erkennbarer Größe, kommen derlei Körner in der Masse der Steine von Chassigny vor. Diese scheinen aber eben so wenig oxydiertes Eisen als Rinde-Substanz zu sein. Dem ersteren widerspricht nämlich die pechschwarze Farbe, der starke, etwas fettige Glanz, das kristallinische Ansehen und die gänzliche Unwirksamkeit auf die Magnetnadel, ausgenommen in einzelnen wenigen, mikroskopisch kleinen Splittern, insofern auch die ganze Steinmasse einige Wirksamkeit äußert; dem letzteren aber --- nebst obigen Gründen in Betreff der Art des Vorkommens und der Einmengung --- Farbe, Glanz, und die ganze Beschaffenheit, verglichen mit der oberflächlichen, ganz eigenen Rinde dieser Steine, die überdies, obgleich schwach, doch merklich genug auf den Magnet wirkt.

Der ausgezeichnete Gehalt dieser Steine an Chrom, von 2 Perzent, welches Metall hier, nach Vauquelin, rein und regulinisch vorkommen soll, lässt mit allem Grunde vermuten, dass es dieses Metall sei, welches hier auf solche Art erscheint, indes dasselbe in den übrigen Meteor-Steinen, wo es bisher, fast durchgehends zwar, aber nur als Spur, oder in der sehr unbedeutenden Menge von 1/4 bis 1 Perzent gefunden worden ist, wahrscheinlich auf gleiche Art eingestreut, aber, nach Stromeyers Vermutung, immer als Oxyd und in Verbindung mit Eisen, als wahres Chromeisen, vorkommt.

Dass jene Atome von oxydiertem Eisen am häufigsten und mikroskopisch zart, in Begleitung und inniger, wenn gleich mechanischer Verbindung mit den eingemengten Gediegeneisenteilchen, selbst bei deren mikroskopischen Erscheinen in den Massenteilchen der erdigen Gemengteile, im Gefolge letzterer, und wahrscheinlich in Gesellschaft von ähnlichen Kies-Atomen, vorkommen, dieselben gleichsam einhüllen, und sich erst bei Fletschung derselben als schwarzes, mehr oder weniger retractiles Pulver zu erkennen geben, und dass es vorzüglich diese Atome sein möchten, von welchen die Rostflecke in der Steinmasse vorzugsweise herrühren --- ist bereits bei Beschreibung des Gediegeneisens bemerkt worden, und dass dieselben einen wesentlichen Einfluss auf den Kohäsions-Zustand, den Magnetismus, auf das spezifische Gewicht, und mittelbar wenigstens, auf das äußere Ansehen der Steinmasse im Ganzen haben müssen, ergibt sich aus ihrer Natur und dem hierüber Vorgebrachten.\footnote{Das quantitative Verhältnis dieses oxydierten Eisens im freien Zustande, als wahrer Gemengteil, kann übrigens --- dem Ansehen nach --- im Allgemeinen nur sehr gering, und, zumal bei Steinen, von welchen ein bedeutender Gehalt an Eisen im Ganzen, chemisch ausgewiesen, aber wenig, oder vollends gar keiner als regulinisch oder geschwefelt, oryktognostisch ausgesprochen ist, im Verhältnis zum chemisch gebundenen oder verlarvten, --- nur höchst unbedeutend sein.}

Höchst merkwürdig aber ist wohl das Wechselverhältnis, welches --- insoweit aus dem äußern Ansehen und den Resultaten der, leider in dieser Beziehung nicht hinlänglich befriedigenden, Analysen geschlossen werden kann --- zwischen dem Gehalte der verschiedenen Meteor-Steine an Eisen in diesem mehr oder weniger oxydierten Zustande (als Oxyd oder Oxydul --- wenn es ja bei diesen rätselhaften Fossilien keine anderen Mischungsverhältnisse zwischen Eisen und Oxygen --- so wie zwischen Eisen und Schwefel --- geben sollte --- als man bei den ähnlichen Verbindungen in terrestrischen Fossilien als Norm annehmen zu dürfen sich berechtigt glaubt), und jenem in ausgesprochen regulinischem zu bestehen scheint, indem ersterer --- offenbar oder verlarvt --- in dem Maße vorwaltet, als letzterer --- wenigstens offenbar --- in einem geringeren vorhanden ist.\footnote{Der Total-Gehalt an Eisen in allen Zuständen und Verbindungen, in welchen dasselbe vorzukommen pflegt (rein metallisch, und zwar frei, oryktognostisch ausgesprochen, oder als solches vielleicht auch verlarvt; mehr oder weniger mit Schwefel vererzt als Eisen- oder Magnetkies, und mehr oder weniger mit Oxygen verbunden, als Oxyd oder Oxydul, und als solches wieder mechanisch eingemengt oder chemisch gebunden), zusammen genommen, und alles auf Oxyd reduziert, wie es durch die Analyse der Steinmasse ohne mechanische Absonderung gewonnen wird, weicht bei allen bisher bekannten, dem Ansehen nach auch noch so verschiedenartigen Meteor-Steinen nicht sehr ab, schwankt gewöhnlich nur zwischen 30 und 40, und steigt nur in höchst seltenen Fällen bis gegen 50 Perzent von der gesamten Steinmasse. Davon beträgt das regulinische, sinnlich wahrnehmbare, wenn es nicht, was jedoch sehr selten der Fall ist (wie bei den Steinen von Chassigny, Stannern, Alais?), ganz fehlt: von 1 bis 19 Perzent --- wahrscheinlich wohl noch etwas mehr; --- das geschwefelte, wenn es nicht, was jedoch noch seltener der Fall ist (wie bei jenen von Chassigny, Alais??), ganz fehlt: von 1 bis etwa 12 oder 15; und das oxydierte endlich --- wovon jedoch in keinem Falle mehr als einige wenige Perzente mechanisch eingemengt sein dürften --- das Ganze oder den Rest jenes Total-Gehaltes. Jene Steine, welche dem Ansehen und dem spezifischen Gewichte nach am reichhaltigsten an mechanisch eingemengtem, zumal gediegenem Eisen sind, enthalten im Ganzen eben nicht bedeutend mehr als jene, wo sich wenig oder selbst gar nichts oryktognostisch und physisch nachweisen lässt. So steht von ersteren, deren spezifisches Gewicht = 3,7 ist (den Steinen von Eichstädt, Timochin, Charsonville), der Total-Gehalt an erhaltenem Oxyd beiläufig zwischen 36 und 43, bei letzteren, deren spezifisches Gewicht zwischen 1,9 und 3,3 ist (den Steinen von Alais, Stannern, Benares, Eggenfeld, Parma \emph{zc.}), zwischen 30 und 40 Perzent. (Merkwürdig ist, dass das spezifische Gewicht der Steine von Chassigny, bei welchen sich doch keine Spur von mechanisch eingemengtem Eisen oder --- außer jenen sparsamen, schwarzen Atomen --- von einem andern Metalle findet, und deren Eisengehalt, nach Vauquelins Ausweis, selbst nur 31 Perzent an Oxyd beträgt, beinahe das Mittelgewicht der Meteor-Steine überhaupt übersteigt, indem dasselbe nach eigener Wiegung 3,55 beträgt.) Bei jenen an Gediegeneisen besonders reichen Steinen endlich, und namentlich bei jenen von Eichstädt, verhält sich der Gehalt an Eisenoxyd zu dem an Gediegeneisen, nach Klaproths Angabe (die aber nicht befriedigend ist, indem er das gediegene Metall mit dem Magnete auszog, daher vieles, was in den erdigen Teilchen verhüllt war, mit in die Auflösung von diesen brachte, und durch die Operation als Oxyd erhielt), wie 16,50 : 19, und bei jenen von Timochin (bei gleichem Verfahren) wie 25 : 17,60, oder nach N. A. Scherer, wie 17,50 : 17,75. (Von den Steinen von Charsonville gibt Vauquelin den Total-Gehalt an Eisen mit 25,8 regulinisch an, wie er ihn nach Kalkül des durch die Operation im Ganzen erhaltenen Oxydes herausbrachte). Bei den an Gediegeneisen besonders armen Steinen dagegen, und namentlich bei jenen von Benares, nach Howard, verhält sich der Gehalt an Eisenoxyd zu dem an ersterem, wie 34 : 2; bei jenen von Siena, nach Klaproth, wie 25 : 2,25; bei jenen von Mauerkirchen, nach Imhof, wie 40,24 : 2,33; und bei jenen von Eggenfeld, nach demselben, wie 32,54 : 1,8 (wobei freilich auch nach Klaproths Methode verfahren worden sein mochte). Nach Stromeyers ungleich genaueren und umsichtigern Analysen ergab sich für die Steine von Erxleben und Köstritz, die dem Ansehen nach (erstere mehr) zu den mittel reichhaltigen an Gediegeneisen gehören, und deren spezifisches Gewicht zwischen 3,6 und 3,5 steht, ein Verhältnis von 5,57 und 4,89 an Oxydul zu 24,41 und 17,48 an metallischem Eisen, mit Inbegriff des Schwefeleisens.}

Noch kommen bei Betrachtung der Steinmasse der Meteor-Steine im Allgemeinen zwei ebenso auffallende als merkwürdige Beschaffenheiten zu erwähnen, die, wenn sie gleich nicht zu ihrer Wesenheit gehören, und sich gerade nicht bei allen Steinen finden, doch sehr häufig erscheinen, und als bedeutende Zustandsveränderungen der Steinmasse, wo nicht als heterogene Gemengteile, anzusehen kommen, und deren höchst rätselhafte Entstehung und Bildung einerseits mit mancher der gangbaren Theorien über die Herkunft und die ursprüngliche Entstehung und Bildung der Massen selbst, sehr im Widerspruche stehen, andererseits in der Folgezeit, wenn sie bei vervielfältigten Beobachtungen und weiteren Untersuchungen einst befriedigend sollten erklärt werden können, manche Aufklärung in letzterer Beziehung erwarten lassen dürften.

Die eine dieser Zustandsveränderungen der Steinmasse ist das Vorkommen derselben als scharf begrenzte Adern oder Gänge von verschiedener Mächtigkeit und Dicke; die andere bezeichnet das verschiedene Aussehen derselben auf scheinbaren, zum Teil wirklichen Absonderungsflächen von verschiedener Ausdehnung, mitten im Innern der Steine.

Das erstere Vorkommen findet sich --- wie ich mich nun überzeugt habe --- bei sehr vielen, und höchst wahrscheinlich, mehr oder minder häufig und deutlich ausgesprochen, wohl bei den meisten Meteor-Steinen.\footnote{Ich habe zuerst (1808) auf das rätselhafte Vorkommen dieser Adern in der Masse der Meteor-Steine bei Gelegenheit der Beschreibung jener von Stannern, obgleich sie in diesen nur äußerst selten und gewisser Maßen unvollkommen sich zeigen, aufmerksam gemacht. Beinahe gleichzeitig erwähnte ihrer Herr Bergrath Reuß bei Beschreibung der bei Lissa gefallenen Steine, in welchen sie am häufigsten vorzukommen scheinen. Erst an den Steinen von Charsonville machten Bigot de Morogues, Hauy und Vauquelin dieselbe Beobachtung, und in ihrer Beschreibung (1811) als von etwas ganz Neuem und Merkwürdigem, Erwähnung davon. In der Folge (1814) gaben die Steine von Agen Gelegenheit zur Erneuerung dieser Beobachtung, welche inzwischen Chladni und ich bereits an vielen, zum Teil lang bekannt gewesenen, älteren Meteor-Seinen zur Genüge gemache hatten. Nach neuerlichster Untersuchung kann ich sie nun an, mitunter ziemlich kleinen, Bruchstucken von folgenden Meteor-Steinen nachweisen: von Lisa, Agen, Doroninsk, Charsonville, Chantonnay, Ensisheim, L'Aigle, Barbotan, Yorkshire, Laponas, Sigena, Toulouse, Salés, Apt, Tipperary, Weston, Stannern; und bei den meisten übrigen mir außer diesen noch bekannten Meteor-Steinen möchte es wohl nur an der individuellen Beschaffenheit des vorhandenen Bruchstucks oder seiner Bruchflache liegen, dass ich nicht dasselbe zu tun im Stande bin.} Es zeigen sich nämlich auf rohen, und noch deutlicher auf geschnittenen, zumal geschliffenen Flächen der Steinmasse von einigem Flächeninhalte, einzelne oder mehrere, oft sehr viele, kürzere oder längere, gerade laufende oder bogenförmig gekrümmte, auch mehrfach gebogene Adern, von sehr verschiedener, bald im ganzen Verlaufe gleichförmiger, bald allmählich abnehmender, bald sehr und stellenweise jäh und stark ab- oder zunehmender Breite und Mächtigkeit, und zwar vom Haarfeinen, kaum dem freien Auge sichtbaren, bis --- was jedoch höchst selten --- zu 3 Linien, welche nach allen Richtungen, und oft von einem Rande der Fläche bis zum andern entgegen gesetzten, und zwar an einem oder dem andern --- aber nicht immer mit dem breiteren Ende --- bisweilen auch an beiden Rändern, aber auch sehr oft an keinem, an die etwa da befindliche Rinde anstehend, oft aber auch ganz isoliert und frei im Mittel der Fläche oder der Steinmasse verlaufen. Es sind diese Adern teils, obgleich selten, ganz einfach, teils mehr oder weniger, oft sehr häufig ramifiziert, und es gehen die Äste und Zweige von verschiedener Stärke und Länge, übrigens von ähnlicher Beschaffenheit, wie der Hauptstamm, unter oft sehr spitzigen Winkeln, nach allen Richtungen von demselben ab, und verlaufen auf ähnliche Weise gegen die Ränder oder mitten in der Masse; sie sind nicht selten wieder zerästelt, durchsetzen und durchschneiden sich, münden sich in einander ein, oder laufen zum Teil auch eine Strecke parallel --- wie dies alles nicht selten selbst die Hauptstämme, wenn deren mehrere vorhanden sind, zu tun pflegen --- und bilden oft ein ziemlich enges, sehr ungleiches Netz oder Adergeflecht. Oft gehen diese Adern, als Gänge, in eine beträchtliche Tiefe mit gleicher oder abnehmender, auch wohl veränderlicher Mächtigkeit, oft bei ansehnlicher Dicke des Stückes, auf einige Zolle; aber dieselbe Ader nicht durchaus auf gleiche Tiefe. Manche scheinen wohl bis an die Oberfläche des Steines zu gehen, die bei weitem meisten aber nicht, und viele nur auf eine höchst unbedeutende Tiefe, so dass nach diesem und obigem manche --- und dies möchte wohl die meisten treffen --- ganz auf das Innerste der Steinmasse beschränkt sind, und mit der Oberfläche und der Außenrinde in gar keiner Verbindung stehen; andere nach einer oder zweien, wieder andere vielleicht nach allen Richtungen ganz durchgehen. Nie scheinen diese Gänge auf irgendeine Tiefe ganz senkrecht, sondern immer mehr oder weniger schief durch die Steinmasse zu setzen.

Die Masse, welche diese Adern und Gänge bildet, ist im Wesentlichen, die Farbe abgerechnet, von der übrigen Steinmasse im Allgemeinen nicht verschieden, indem sie im Gegenteile eine in jeder Beziehung ganz ununterbrochene Fortsetzung von dieser ausmacht, und außer der scharfen Begrenzung durch die Farbe, durch gar nichts, das z. B. einem Salbande gliche, geschieden ist, sondern vielmehr unmittelbar in dieselbe übergeht. Sie zeigt dieselbe Textur, dieselbe Beschaffenheit der Oberfläche sowohl im Bruche als im Schnitte, dasselbe, meistens doch ein etwas feineres, Korn, nur mehr Dichtheit, Festigkeit und Härte --- beiläufig so wie der olivinartige Gemengteil in einem mittleren Grade von Ausbildung --- daher sie auch geschliffen --- so wie dieser --- eine höhere Politur und einen etwas fettigen Glanz annimmt, und sie enthält ebenso wie die übrige Steinmasse, Gediegeneisen eingesprengt; vom olivinartigen Gemengteile, nach der gewöhnlichen Art seiner Ausscheidung und Begrenzung, konnte ich aber nie etwas in ihr bemerken. Nur, wiewohl höchst selten, und an einzelnen Stellen besonders breiter Adern, zeigt sie eine schwache Anlage zu einer schiefrigen Textur. Sie zeigt denselben, gewöhnlich nur etwas, höheren Grad von Wirkung auf den Magnet wie die übrige Steinmasse, aber einen merklich geringeren als die Rinde desselben Steines.

Das beinahe einzige und wesentlichste Merkmal, wodurch sich die Masse dieser Adern und Gänge von der übrigen Steinmasse unterscheidet, ist die Farbe. Diese ist schwärzlich, oft beinahe schwarz, gewöhnlich aber graulich- oder bläulich-schwarz, oder bläulich- und mehr oder weniger dunkel schiefer-grau, nie so pech- oder kohlschwarz, wie die Rinde an manchem solcher Steine, am wenigsten braun, wie diese an den meisten, und ohne allem metallischen Ansehen; dagegen oft genauso, wie der olivinartige Gemengteil im ausgesprochneren Zustande in denselben oder in andern Meteor-Steinen zu erscheinen pflegt. Nur auf polierten Flächen zeigt diese Gangmasse einen mehr oder weniger ausgezeichneten, etwas fettigen, und dem olivinartigen Gemengteile im ausgesprochneren Zustande ähnlichen Glanz, auch, wenigstens bei dem einen Steine von Stannern, wo auch die Farbe den dunkelsten Partien jenes Gemengteiles entspricht, ein ähnliches, zerrissenes und zersprungenes, gleichsam gekörntes Ansehen.

Es ist bemerkenswert, dass sich diese Adern und Gänge am häufigsten und ausgezeichnetsten in solchen Meteor-Steinen finden, die sich --- mit Ausnahme jener von Stannern, wo sie übrigens nur an einem unter so vielen gesehenen Bruchstücken, und auch hier nur in einem unvollkommenen Zustande beobachtet wurden --- durch eine beträchtliche Dichtheit, Festigkeit und Innigkeit des Kohäsions-Zustandes sowohl als des Aggregats-Zustandes auszeichnen (wie die Steine von Lissa, Agen, Charsonville, Chantonnay, Ensisheim \emph{zc.}), und gerade in solchen, wo der olivinartige Gemengteil nur sehr wenig, oder doch nur als solcher unvollkommen ausgesprochen und nicht sehr mannigfaltig erscheint (wie dies ebenfalls bei den genannten Steinen der Fall ist).

Vauquelin und Chladni halten die Substanz dieser Adern und Gänge für ganz homogen mit der Rindenmasse, inzwischen ergibt sich aus obigem, dass sie in der Farbe stark, in der Textur und übrigen Beschaffenheit aber ganz und gar von dieser abweicht (wie sie denn auch gar keine Porosität und nirgend einen Übergang in die Steinmasse zeigt, welches beides, wenigstens nach meiner Ansicht hinsichtlich ihrer Entstehung und Bildung, so gut wie bei der oberflächlichen Rinde der Fall sein müsste), dagegen ungleich mehr Ähnlichkeit mit der übrigen Steinmasse, zumal mit dem einen Gemengteile derselben erkennen läßt.\footnote{Vauquelin erklärte die Entstehung dieser Adern und Gänge, nach obiger Voraussetzung und in Annahme eines wahren Schmelz-Prozesses zur Erzeugung der Rinde (durch Erhitzung in der Luft während des Durchzuges und Niederfallens der Steine), wie jene der Außenrinde: durch Verbrennung des Eisens und Verschlackung der Erden durch die atmosphärische Luft, welche durch einen Riss, den der Stein im Glühen bekam, und der nach der Hand wieder zusammengebacken wurde, in die Masse eingedrungen war. Allein gegen diese Mutmaßung streiten --- wenn man auch jene Annahme hinsichtlich der Bildung der Rinde im Allgemeinen zugeben könnte --- nicht nur die erwähnte Verschiedenheit der Substanz dieser Adern von jener der wahren Rinde, sondern die ganze Beschaffenheit und alle Eigenschaften jener, welche durchaus die Idee verbannen, dass sie, zumal späterhin, durch Risse oder Sprünge der Steinmasse entstanden sein können. So die Umstände: dass diese Gänge bisweilen nach allen Richtungen durch die ganze Masse durchsetzen, daher diese an solchen Stellen notwendig zerfallen sein müsste; dagegen oft ganz in der Mitte mit gar keiner oder nur äußerst schwacher Verbindung nach Außen erscheinen, wo demnach keine Luft eindringen konnte; oft nach Außen äußerst dünn, haarfein, im Verlaufe nach Innen aber bei einer Linie dick, was gerade bei einem Risse umgekehrt sein müsste; bald im ganzen Verlaufe von gleicher, bald von sehr und wiederholt abweichender Dicke sich zeigen; dass sie ästig, verworren, beinahe ein Netz bildend, sich durchkreuzen, durchschneiden u. s. w., folglich einzelne Stücke umschließen, die sich hätten lostrennen müssen; dass viele zu fein für Risse, nach der Beschaffenheit der Steinmasse, auch oft zu grob --- bis 3 Linien dick --- als dass von Außen der Riss nicht sichtbar geblieben sein oder das Stück sich nicht losgetrennt haben sollte, u. s. w.\\
Chladni meint dagegen (wie bereits oben bei Erklärung der fünften Figur der vorhergehenden Tafel erwähnt worden ist), es wären diese Gänge oder (nach seiner Ansicht) Lagen durch das zufällige Zusammentreffen und Zusammenkleben bereits losgesprengter und schon überrindet gewesener Bruchstücke von Steinen, während ihres Niederfallens, mit ihren Flächen aneinander, entstanden. Allein außerdem, dass (wie an jenem Orte bemerkt worden ist) ein solches Zusammentreffen nicht wohl denkbar, ein solches Zusammenpassen, ein so festes, inniges Vereinigen und Zusammenkleben zweier, nach Ausdehnung, Bruch, Umriss u. s. w. gewiss oft ganz verschiedenartigen Flächen zweier Steine, oder --- wie es nach der netzartigen Durchkreuzung jener Adern anzunehmen nötig wäre --- der Flächen gar vieler Bruchstücke gleichzeitig, gar nicht begreiflich ist; so stehen mit dieser Meinung nicht nur alle obigen Wahrnehmungen, am offenbarsten wohl jene, dass diese Lagen nur selten nach allen Dimensionen des Steines ganz durchsetzen, oft gar nicht nach Außen irgendwo anstehen, sondern ganz im Mittel der Masse eingeschlossen sind, --- sondern insbesondere noch folgende im Widerspruche: die Feinheit und oft haarscharfe Gleichförmigkeit dieser Gänge, da doch die Bruchflächen und selbst die überrindete Oberflache der Steine immer sehr uneben sind, und die dünnste Rinde wenigstens drei Mahl dicker zu sein pflegt; dagegen oft wieder die Dicke derselben, welche jene der gewöhnlichen Rinde bisweilen ums Sechsfache übersteigt; keine Spur von einer doppelten Schichte, die sich doch erkennen lassen müsste, wo sie von zwei überrindeten Flächen zusammen traf; keine Spur von Porosität oder vom Übergange der Massenteilchen der Substanz derselben in jene der übrigen Steinmasse, die sich doch an der Außenrinde so deutlich aussprechen u. s. w. Übrigens kommt gegen beide Meinungen zu bemerken: dass diese Adern und Gänge sich oft, selbst in einem kleinen Bruchstücke, in solcher Menge finden, dass sie sich schlechterdings nicht von so vielen Rissen und Sprüngen, am wenigsten aber von ebenso vielen Absonderungen und Wiedervereinigungen herleiten lassen, und dass sie sich, wie bereits oben bemerkt worden ist, gerade am häufigsten und deutlichsten bei solchen Steinen zeigen, die den festesten Kohäsions-Zustand und das dichteste Gefüge haben, bei welchen sich daher am wenigsten Risse und Zertrümmerungen erwarten ließen, wie denn auch bei den meisten dieser Meteor-Massen gar keine oder nur eine geringe Vereinzelung Statt fand (so fielen die Massen von Ensisheim, Chantonnay, York --- und diese trotz ihrer bedeutenden Große --- von Tipperary, Apt, Sigena, ganz und unvereinzelt, die von Laponas, Charsonville, Lissa, nur als zwei, drei oder vier Stücke herab); endlich, dass sie bisweilen in solchen Steinen vorkommen, bei welchen selbst die Außenrinde im Ganzen nur wenig oder unvollkommen gewesen zu sein scheint (wie bei den Steinen von Ensisheim und Chantonnay). Bigot de Morogues wollte gefunden haben, dass die Substanz dieser Gänge (die er übrigens für ganz verschieden von der Rinde hält), wenigstens in den Steinen von Charsonville, ein bedeutend geringeres spezifisches Gewicht hätte, als die übrige Steinmasse. Er fand nämlich jenes dieser letzteren = 3,637, dagegen das eines Stückes, worin eine Ader von jener Substanz vorkam, die, nach seiner Schätzung, etwa 1/15 des Ganzen betrug, = 3,635, und berechnet nach diesem (übrigens höchst geringen Abstand und nach einem Kalkül, gegen den sich viel einwenden ließe) das eigentümliche Gewicht derselben auf 3,592, und (auf gleiche Weise nach einer --- wahrscheinlich des zufällig größeren Eisengehaltes wegen --- höheren Gewichtsangabe der Steinmasse von Hauy = 3,712) gar nur auf 2,457, und will daraus auf eine Ähnlichkeit dieser Substanz mit der Masse der Steine von Alais schließen. Die offenbar größere Dichtheit dieser Ader-Substanz gegen die übrige Steinmasse, bei übrigens ganz gleicher Beschaffenheit, gleichem Eisengehalte und s. w. machte mir jenen, dem an sich höchst unverlässlichen Kalküle zum Grunde liegenden, reellen Befund selbst höchst unwahrscheinlich, und ich wollte mich demnach durch eigene Wiegung ähnlicher Stücke von demselben Steine überzeugen. Ich fand das spezifische Gewicht eines 27 1/2 Gran wiegenden, von Rinde sowohl als Ader-Substanz ganz freien Stuckes der Masse eines Steines von Charsonville = 3,571; jenes dagegen eines 26 1/4 Gran schweren Stuckes von demselben Steine, welches zwar keine Rinde, aber eine, über eine Linie breite, ganz durchsetzende Ader von jener Substanz einschloss, die wenigstens 1/5 des Ganzen betrug (was demnach ein drei Mal so auffallendes Resultat geben konnte, als das von Bigot de Morogues untersuchte), = 3,658.}

Ich wäre vor der Hand geneigt, die Entstehung dieser Adern und Gänge, hinsichtlich des Momentes, für ganz gleichzeitig mit der Bildung der übrigen Steinmasse und der Bildung und Ausscheidung ihrer Gemengteile, und, hinsichtlich der Art, für ganz gleichartig mit jener der übrigen Gemengteile, und insbesondere des noch mehr und bezeichneter ausgeschiedenen und in der Wesenheit noch weit mehr abweichenden olivinartigen zu halten; die Substanz derselben aber für homogen mit der Steinmasse, nur etwa mit einer kleinen Zustandsveränderung oder Modifikation in der Art der Ausscheidung, und dieselbe überhaupt zum Teil mit dem olivinartigen Gemengteil, zum Teil mit jener Zustandsveränderung der Steinmasse, von der sogleich die Rede sein wird, für ein und dasselbe anzusehen.

Das andere Vorkommen der Steinmasse von ungewöhnlichem Ansehen findet sich, wie es scheint, nicht minder häufig, und wo nicht immer, doch meistens auch bei jenen Steinen, wo obige Adern sich zeigen, so wie sich diese dagegen notwendig immer in irgendeiner Richtung zeigen müssen, wo jenes Vorkommen Statt hat. Es besteht dieses aber in einer mehr oder weniger dicken und massiven Schichte oder Lage, gewöhnlich aber nur in einem äußerst feinen und dünnen Häutchen, oft nur zarten, durch die Steinmasse hie und da bisweilen selbst unterbrochenen Anfluge, von einer dichteren, scheinbar fremdartigen Masse, welche in Gestalt von größeren oder kleineren, ganz unregelmäßigen, gar nicht scharf begrenzten Flecken, oder mehr oder minder breiten, bandartigen, oft sehr scharf abgeschnittenen Streifen auf einer Bruchfläche erscheinen, und bisweilen dieselbe ganz bedecken, und die --- wie sich oft an derselben Bruchfläche, wenn sie groß und sehr uneben ist, noch mehr aber an entgegen gesetzten Bruchflächen eines größeren Stückes zeigt --- ganz nach Art jener Adern und Gänge, und auf ähnliche Weise hinsichtlich ihrer Ausdehnung in Bezug auf das Innere und die Oberfläche des Steines, in verschiedenen, nicht selten sich durchkreuzenden und schneidenden Richtungen durch die Steinmasse durchsetzen.

Es zeigen diese Flecke und Streifen, wenn sie sehr dünn und zart, zumal anflugartig sind, die gewöhnlichen Unebenheiten der natürlichen Bruchfläche des Steines, und ziehen sich gleichförmig über dieselbe hinüber; wenn sie aber von einiger Dicke sind, erscheinen sie ebener und glatter, und unterscheiden sich solcher Gestalt auffallend von der übrigen Bruchfläche des Steines. Im ersteren Falle haben sie gewöhnlich ein streifiges, bisweilen selbst ein, mehr oder weniger deutlich ausgesprochenes, obgleich unvollkommen schiefriges Ansehen, das selbst die Steinmasse angenommen zu haben scheint; im letzteren aber eine Anlage zu blätterigen Ablösungen; in beiden Fällen endlich bilden sie mehr oder minder vollkommene, natürliche Absonderungsflächen, oder ähnliche Stellen, nach welchen sich der Stein auch leicht zu spalten scheint. Letzteres doch nur dann, wenn ein bedeutender Metallgehalt ins Mittel tritt. Die Masse selbst hat im frischen Bruche, entweder, obgleich seltener, ein mattes erdiges, von der übrigen Steinmasse, zumal dem olivinartigen Gemengteil im unvollkommeneren Zustande, wenig verschiedenes Ansehen, und eine schiefergraue, ins Schwärzliche ziehende Farbe, meistens aber, und wie es scheint, bei den an Gediegeneisen reichhaltigern Steinen, ein, wenigstens ganz oberflächlich mehr oder weniger metallisches, einiger Maßen dem Graphit ähnliches Ansehen, eine lichter oder dunkler eisengraue Farbe, und einen ziemlich starken, metallischen, fleckweise schimmernden Glanz. Dieser Glanz rührt von wirklich metallischem Eisen her, das an solchen Stellen in dünnen, zarten Blättchen gleichsam angeflogen, indes dasselbe dort, wo diese Masse ein mehr erdiges, den übrigen Gemengteilen mehr ähnliches Ansehen hat, ebenso wie in diesen, körnig eingesprengt erscheint; auch zeigen sich, besonders an ersteren Stellen, sehr häufige Rostflecke; vom olivinartigen Gemengteile im ausgesonderten, mehr oder weniger kugelicht begrenzten Zustande, konnte ich aber in keiner Art des Vorkommens dieser Masse eine deutliche Spur bisher finden.

Bisweilen erscheint diese Masse, zumal im erdigen Zustande, fleckweise und in kleineren und größeren, selbst in bedeutenden Partien von ansehnlicher Größe nach allen Dimensionen, ebenso wie der olivinartige Gemengteil, nur mehr unförmlich und nicht so scharf begrenzt, von der übrigen wie gewöhnlich gemengten Steinmasse gleichsam abgeschieden, wie dies ganz besonders ausgezeichnet bei dem merkwürdigen, noch wenig bekannten Meteor-Steine von Chantonnay der Fall ist. Hier zeigt sich diese Masse, welche beinahe den größeren Teil der --- jener der meisten Meteor-Steine (zumal jenen von Tabor, Barbotan u. a.) übrigens ganz ähnlich gemengten --- Steinmasse ausmacht, von sehr dichtem, festem Gefüge, und sehr feinem, nicht unterscheidbarem Korne, von schwärzlich schiefergrauer Farbe, mattem, nur etwas schimmernden, erdigen, basaltähnlichen Ansehen, und ganz derb, im Ganzen von sehr festem Zusammenhange mit der übrigen Steinmasse, nur hie und da mit einer schwachen Anlage zur schiefrigen Textur, oder stellenweise zu schiefrigen Ablösungen, und gleich der übrigen Steinmasse mit zarten, stark glänzenden Metallteilchen eingesprengt, sonst ganz gleichförmig. Und so erscheint dieselbe hier teils in größeren und kleineren Flecken, teils in schmälern oder breiteren Adern (ganz und in jeder Beziehung jenen anderer Meteor-Steine, zumal jenen der Steine von Charsonville ähnlich), teils aber in so großen, frei anstehenden Partien, dass man ansehnliche Bruchstücke rein von dieser Masse erhalten kann, die dann der übrigen gemengten Steinmasse von gewöhnlichem Aussehen, dem ersten Anblicke nach, so unähnlich sind, wie nur immer ein derbes Basaltstück einem feinkörnigen, eisenschüssigen Sandsteine sein kann, und die wohl niemand, dem bloßen äußern Ansehen nach, für Bruchstücke eines Meteor-Steines, am wenigsten aber für solche von diesem Steine erkennen möchte, wenn ihm nicht die Ähnlichkeit dieser Masse mit jener der Adern und Gänge in andern Meteor-Steinen, und das fleck- und partienweise Vorkommen derselben in diesem vorhinein bekannt ist.\footnote{Ich fand das spezifische Gewicht jenes Teiles der Steinmasse dieses Steines von gewöhnlichem Aussehen = 3,440 bis 3,480 (das mir unerwartet gering vorkam); jenes des schwarzen Anteiles aber = 3,490 (ein unbedeutender Unterschied, der wohl auch nur, wie bei den Adern im Steine von Charsonville, in dem verschiedenen Grade von Dichtheit beider Massen seinen Grund haben dürfte). Noch ist von diesem merkwürdigen Steine keine Analyse bekannt. Vauquelin soll seit lange schon die Absicht gehabt haben, sie vorzunehmen; auch habe ich meinem geehrten Freunde, Hrn. Professor Stromeyer, ein kleines Stuck von beiden Massen dieses Steines zu diesem Ende zugesendet.}

Wo der Art Schichten, Lagen, Flecke oder Streifen dieser Masse durch Bruch oder Schnitt eines Stückes in irgendeiner Tiefe quer getroffen werden, müssen an der Oberfläche notwendig Adern erscheinen, welche die Ausgänge derselben bezeichnen, und deren Mächtigkeit oder Breite demnach durch die Dicke, und deren Tiefe durch die Ausdehnung jener in die Breite bestimmt wird. Und aus dem, was vorhin von den Eigenheiten jener Adern und Gänge, und der Beschaffenheit ihrer Masse, und hier in denselben Beziehungen von diesem Vorkommen der Steinmasse bemerkt worden ist, und vollends aus den Modifikationen und Übergängen, welche jener Stein von Chantonnay hinsichtlich beider zeigt: scheint sich wohl die Identität der Masse in beiden Arten des Vorkommens zu ergeben und die Schlussfolge ziehen zu lassen, dass demselben, so wie dem Hervortreten des olivinartigen Gemengteiles --- der in seiner Wesenheit wohl auch nicht sehr davon verschieden sein möchte --- ein und derselbe Bildungs- und Ausscheidungs-Prozess zum Grunde liege, der nur durch das, obgleich nur wenig abweichende quantitative Verhältnis der entfernteren Bestandteile der Steinmasse im Ganzen, oder etwa durch veränderte Nebenumstände abgeändert werden, und darnach jene mannigfaltigen Modifikationen veranlassen dürfte.\footnote{Es war zur Zeit nicht möglich, von dieser merkwürdigen Zustandsveränderung der Steinmasse der Meteor-Steine eine befriedigende bildliche Darstellung für gegenwärtige Bekanntmachung zu Stande zu bringen. Sie soll bei einer künftigen Veranlassung versucht werden.}

\subsection{Stannern.}
\paragraph{}
Ein, bei 4 Loth wiegendes, auf drei Seiten (den Resten von drei verbrochenen ursprünglichen Flächen) mit Rinde --- von der gewöhnlichsten Art und Beschaffenheit --- bedecktes, frisches Bruchstück eines --- allem Ansehen nach --- ursprünglich ziemlich groß gewesenen Steines von Stannern, welches mit unter denen war, die bei Gelegenheit der abgehaltenen Untersuchungs-Kommission von verschiedenen, gleich anfänglich in viele Stücke zerschlagenen Steinen, an Ort und Stelle erhalten, und welches, des ausgezeichneten Mengungszustandes der Steinmasse wegen, für die Sammlung bestimmt wurde.\footnote{Es sind nämlich aus einem Vorrate von 93 Stücken, zusammen an 46 Pfund wiegend, welcher teils unmittelbar bei Gelegenheit der Untersuchung an Ort und Stelle, teils nachträglich durch Vermittlung des k. k. Kreisamtes zu Iglau, und auf andern Wegen von diesem Steinfalle zusammen gebracht wurde, 22 Stück und mehrere kleine Fragmente, zusammen nahe an 25 Pfund, und zwar eilf ganze, mehr oder weniger vollkommen überrindete Steine, und ebenso viele größere und mehrere kleine Bruchstucke, für die kaiserliche Sammlung ausgewählt worden, insofern sie bemerkenswerte Abweichungen in der Große und Form, oder in der Beschaffenheit der Rinde und der Steinmasse zeigten.}

Es zeigt dasselbe im Ganzen den, den Meteor-Steinen von Stannern eigentümlichen, lockern, ziemlich leicht zerreiblichen Kohäsions-Zustand der Masse,\footnote{So dass sie beim schwächsten Versuche, Feuer zu schlagen, zerstiebt, und nähert sich hierin, in aufsteigender Progression --- mit Ausnahme der Steine von Alais und Chassigny, die im Ganzen noch lockerer sind --- jener der Steine von Eggenfeld, Mauerkirchen, Benares, Parma, Siena, welche letzteren unter diesen die dichtesten und festesten sind.} und auf der einen, hier vorgestellten, mit zwei Rändern an die Rindendecke anstehenden, sehr roh und grob erzeugten, frischen Bruchfläche insbesondere --- auf etwa 1 1/2 Quadrat-Zoll Oberfläche --- bei einem sehr unebenen, unbestimmt eckige und ziemlich scharfkantige Bruchstücke, andeutenden Bruch --- den gewöhnlichen, feinen, undeutlich ausgesprochenen und verworrenen, aber ziemlich gleichförmigen und innigen Aggregats-Zustand; ferner die eigene, teils bröcklig-körnige, teils gleichsam filzig-faserige Textur von äußerst feinem Korne, und endlich die, wie gewöhnlich, im Ganzen ziemlich gleichförmig gemengte Steinmasse, von mattem, mehr oder weniger erdigen, rauen, magern, beinahe bimssteinartigen Ansehen, und teils kalkweißer, teils bläulich- oder perlgrauer Farbe, in welcher die Gemengteile zum Teil so undeutlich ausgesprochen und innig gemengt, wenigstens so gleichförmig verteilt sind, dass keiner derselben vorzugsweise als Grundmasse betrachtet werden kann.

Der eine, mehr erdige, lockere und raue Gemengteil, von meistens kalkweißer Farbe, welcher aller Analogie nach für die Grundmasse angesprochen werden muss,\footnote{Sowohl dem äußern Ansehen nach, das sich an einigen Stücken --- wie selbst an diesem --- durch stärkeres Hervortreten der andern Gemengteiles (welches sich am besten auf polierten Flächen zu erkennen gibt) schon deutlich genug ausspricht, als nach den physischen Eigenschaften und chemischen Bestandteilen, in welchen sich derselbe dem gleichnamigen und vollkommen als solchen ausgesprochenen in andern Meteor-Steinen, und zwar stufenweise und nachweisbar --- oft an einem und demselben Stucke --- (wie der nächst zu beschreibende Stein zeigen wird) nähert. Vielleicht hat der große Gehalt an Tonerde (7-9 bis 14 Perzent) und an Kalkerde (9 bis 12 Perzent), und der umso geringere an Talkerde (= 2) --- wodurch sich diese Meteor-Steine so sehr von allen übrigen auszeichnen --- die Ausbildung oder Ausscheidung dieses Gemengteiles --- welchem vorzüglich Kiesel- und Talkerde zukommen --- verhindert.} zeigt sich teils in pulverigen, fast staubartigen Punkten und sehr kleinen Flecken, teils in kurzen, schmalen, nach allen Richtungen laufenden, filzig-faserigen Streifchen\footnote{Dem äußern Ansehen nach haben diese Streifchen einige Ähnlichkeit mit einer Art des Vorkommens von Werners Schmelzstein, Dipyre. Auch in den Steinen von Siena, Benares, Parma, zumal aber in jenen von Mauerkirchen und Eggenfeld, zeigt sich die Grundmasse stellen- und partienweise von gleicher Beschaffenheit.}; der andere, festere, dichtere und mehr glatte Gemengteil dagegen, von lichter und dunkler bläulich- oder perlgrauer Farbe (welcher ebenso dem mehr oder weniger kugelichten --- olivinartigen --- Gemengteile anderer Meteor-Steine entspricht), erscheint teils mehr oder weniger innig gemengt, teils mehr oder weniger scharf geschieden, und abwechselnd mit jenem, bald in ähnlichen, aber festeren und dichteren Punkten, kleinen Flecken, Körnern und kleinen Massen, bald, obgleich seltener, in ähnlichen, ebenso beschaffenen Streifchen; und beide Gemengteile so, dass bald der eine, bald der andere von denselben, stellenweise mehr oder minder vorwaltet.\footnote{Auf geschliffenen und polierten Flächen zeigt sich das Gemenge, nach dem verschiedenen Vorwalten des letzteren Gemengteiles, dessen mehr oder minder scharfen Ausscheidung und Begrenzung, Gestaltung und verschiedenen Intensität der Farbe, teils Granit- oder Porphyr-teils Marmorartig, und dieser Gemengteil fällt hier durch seine größere Dichte --- die mit der Scharfe der Begrenzung der Massen und mit der Intensität der Farbe im Verhältnisse steht --- noch mehr auf, indem er, und zwar in denselben Graden, eine ziemlich gute Politur annimmt und einen etwas fettigen Glanz zeigt.}

An der einen Seite der vorgestellten Fläche dieses Bruchstückes aber erscheint dieser letztere Gemengteil als eine bedeutend große, dreieckige, gleichsam ausgeschiedene, isolierte Masse, obgleich nicht sehr scharf begrenzt, von beinahe lavendelblauer Farbe, und ein ganz ähnlicher, nur ungleich kleinerer, aber mehr dreieckiger und schärfer begrenzter Fleck zeigt sich auf der andern Seite.\footnote{Ich verweise auf das, was in Hinsicht der beiden erdigen Gemengteile und dieses letzteren olivinartigen insbesondere, in der Einleitung zur Erklärung dieser Tafel im Allgemeinen vorgebracht worden ist, und bemerke hier nur noch, dass sich dieser unvollkommene Grad von Ausscheidung und Figurierung desselben ganz genau so, auch bei andern Meteor-Steinen (z. B. bei jenen von Siena, Ensisheim, L'Aigle u. f. w.), und nicht selten in Verbindung mit vollkommeneren Graden von Ausbildung desselben finde. Auch ist bemerkenswert, dass an einem kleinen, etwas über 4 Loth wiegenden, beinahe ganzen, mit besonders dünner, nur wenig und weitzellig-aderiger Rinde bedeckten Steine von Stannern, von welchem ein Stück abgebrochen worden war --- die ganze Masse ausschließlich aus diesem letzteren Gemengteile zu bestehen scheint, indem die ganze, doch bei 1 1/2 Quadrat-Zoll betragende Oberfläche der quer über den ganzen Stein ausgedehnten Bruchfläche ein ganz gleichförmiges Ansehen hat, und einen festen Kohäsions-Zustand, eine ebenso dichte, äußerst feinkörnige Textur, und eine licht lavendelblaue Farbe zeigt.}

Zum Teil mit freiem Auge, mehr aber doch mit Hülfe einer Lupe, entdeckt man in diesem Gemenge äußerst zarte, einzelne, matte, schwarze Körner\footnote{Auch in dieser Beziehung verweise ich auf das, in Betreff dieses mikroskopischen und unbeständigen Gemengteiles, oben in der Einleitung Gesagte, und bemerke nur, dass die Menge desselben auch hier nur höchst unbedeutend ist, und bei dem durch die Analysen ausgewiesenen Eisengehalte dieser Steine (mit Inbegriff des Schwefeleisens = 27 bis 32 Perzent) kaum in Anschlag gebracht werden kann; dass übrigens die Atome davon keinesweges mit Rinde-Substanz verwechselt werden können.} (wohl größten Teils Eisenoxyd, vielleicht auch Chromeisen), und ebenso zarte, aber hie und da zusammen gehäufte, mehr oder weniger glänzende Metallteilchen von zinkgrauer, teils ins Rötliche, teils ins Gelbliche fallender Farbe (Schwefeleisen),\footnote{Von diesem Gemengteile finden sich an andern Stücken dieser Meteor-Steine nicht selten beträchtliche Partien und Massen (häufiger und ungleich größere als bei irgendeinem andern, mit Ausnahme jener von Parma, und etwa der von Benares, Mauerkirchen und Lissa) eingemengt, wie bei Beschreibung eines zweiten, auf dieser Tafel dargestellten, und in dieser Beziehung besonders ausgezeichneten Bruchstückes, gezeigt werden wird.} ziemlich häufig eingestreut; von regulinischem Eisen findet sich aber an diesem Stücke, so wie überhaupt in den Steinen von Stannern, keine Spur,\footnote{Dieser Mangel an Gediegeneisen, wodurch sich die Steine von Stannern mit jenen von Chassigny bisher ausschließlich (denn von jenen von Alais ist es zweifelhaft, und von jenen von Agen erwähnter Maßen unrichtig) von allen bisher bekannten Meteor-Steinen auszeichnen, spricht sich auch durch das bedeutend geringere spezifische Gewicht aus (= 3,1 bis 3,2), welches nur bei jenen von Alais noch geringer ist (= 1,9); dagegen jenem der Steine von Benares, Eggenfeld, Parma, Siena, Mauerkirchen, als den, jenen von Stannern in jeder Beziehung nächst verwandtesten Meteor-Steinen (wo dasselbe zwischen 3,3 und 3,4 schwankt) --- die auch nur einen geringen Gehalt an Gediegeneisen zeigen --- am nächsten kommt. Bei den meisten übrigen Meteor-Steinen steht dasselbe zwischen 3,5 und 3,7. Vom Ensisheimer ist das spezifische Gewicht mit 3,23 zu gering angegeben worden, wie nach der ausgezeichneten Dichtheit der Masse dieses Steines und dem nicht so ganz unbedeutenden Gehalt an Gediegeneisen zu vermuten war, und beträgt nach eigener Wiegung 3,480 bis 3,490. Eine merkwürdige Abweichung in dieser Beziehung zeigt die Masse der Steine von Chassigny, deren spezifisches Gewicht --- bei gänzlichem Mangel an mechanisch eingemengtem Gediegeneisen, und selbst an Schwefeleisen --- nach eigener Überzeugung, doch 3,550 beträgt.) Noch bestimmter äußert sich übrigens der Mangel an Gediegeneisen bei diesen Steinen von Stannern durch die gänzliche Unwirksamkeit der Masse sowohl als selbst der Rinde auf die empfindlichste Magnetnadel, die nur von letzterer an einzelnen seltenen Punkten kaum merklich in Bewegung gesetzt wird, und aus der fein gepulverten Masse und Rinde nur äußerst wenige, einzelne mikroskopische Körnchen anzieht, die allem Ansehen nach Eisenoxydul sind. Da übrigens der Total-Gehalt an Eisen der Steine von Stannern nach den Analysen Mosers, Klaproths und Vauquelins zwischen 27 und 32 Perzent beträgt, das eingemengte Schwefeleisen im Durchschnitt nach einer oberflächlichen Schätzung kaum 5 Perzent der Masse, das ebenso vorhandene Oxyd aber kaum so viel betragen kann; so muss der größte Anteil des Gehaltes in den erdigen Gemengteilen chemisch gebunden (als Oxyd nach Moser und Vauquelin), oder in irgendeinem Zustande verlarvt enthalten sein.} und eben so wenig eine Andeutung von Rostflecken, die (wie bereits oben erwähnt worden ist), wo nicht ausschließlich, doch vorzugsweise das mechanisch eingemengte Gediegeneisen und dessen Umgebung zu begleiten pflegen.

Von Adern und Gängen, oder von einer andern Zustandsverschiedenheit der Steinmasse (von welchen oben in der Einleitung zur Erklärung dieser Tafel die Rede war), zeigt sich an diesem Stücke ebenfalls keine Spur, und überhaupt zeigte, unter so vielen gesehenen Bruchstücken, nur eines das Vorkommen von ersteren in den Meteor-Steinen von Stannern.

\subsection{Siena.}
\paragraph{}
Dasselbe Stück von dem Steinfalle bei Siena in Italien, welches der ausgezeichneten Form wegen bereits auf der zweiten Tafel von einer andern Ansicht gegeben worden ist, von einer polierten frischen Bruchfläche dargestellt, die mit zwei Rändern an die Außenrinde stößt, und, auf etwa 1 Quadrat-Zoll Oberfläche, bei vollkommener Abglättung, aber etwas matter und ungleichförmiger eigentlicher Politur, die innere Beschaffenheit der Steinmasse zu erkennen gibt.

Es zeigt dieselbe einen ziemlich festen Kohäsions-Zustand, der jedoch --- wie eine zweite frische, aber rohe Bruchfläche noch besser erkennen lässt --- ziemlich nahe ans Zerreibliche grenzt, und einen, zum Teil mehr oder weniger feinen, hie und da etwas undeutlich ausgesprochenen, verworrenen, zum Teil aber einen sehr grobbröckligen, und sehr auffallend ausgesprochenen, breccieartigen, im Ganzen daher sehr ungleichförmigen, aber ziemlich festen Aggregats-Zustand; eine --- abgesehen von dem breccieartigen Gemengteile --- körnige Textur von äußerst feinem Korne, und im Ganzen eine merklich, obgleich nicht sehr stark und etwas ungleichförmig, vorwaltende Grundmasse von ganz mattem, erdigen Ansehen, und licht aschgrauer, aber mehr ins Schmutzig- und Gelblich-Graue als ins Bläuliche ziehender Farbe, welche dem andern Gemengteile, anscheinend, zum Zemente dient.

Sie unterscheidet sich demnach, außer der kleinen Verschiedenheit im Kohäsions-Zustande und der Farbe, von jener des vorigen Steines durch das mehr offenbare Vorwalten der Grundmasse, und durch ein, wenigstens zum Teil, deutlicheres Hervortreten des andern (olivinartigen) Gemengteiles.

Dieser erscheint nämlich hier, teils in eben so verschieden gestalteten und eckigen, mehr oder weniger scharf --- im Ganzen jedoch durchaus schärfer --- begrenzten, ganz ähnlichen Flecken von verschiedener Größe, derselben Dichtheit und Festigkeit, gleichen, obgleich meistens mehr ins Dunkle bis ins Dunkelblaue und Bräunlich- und Schwärzlich-Graue ziehenden Farben-Tingirungen, und ähnlichem fettigen Glanze, wie die ausgezeichneteren Massen dieses Gemengteiles in jenem Bruchstücke, und überhaupt in den Steinen von Stannern; teils aber auch schon, wie in den meisten andern Meteor-Steinen, in größeren oder kleineren, rundlichten oder ovalen Massen von bestimmterer Absonderung und noch größerer Dichtheit, die demnach auf der rohen Bruchfläche unverbrochen, als erhabene Körner, zum Teil selbst als Kügelchen erscheinen. Mitunter zeigen sich der Art Massen, selbst schon von einigem Grade von Durchscheinenheit und von grünlich-grauer ins Lauchgrüne fallender Farbe, und Graf Bournon und Klaproth bemerkten selbst in Bruchstücken von Steinen dieses Herkommens ganz durchscheinende, ja vollkommen durchsichtige Körner von gelblicher und grünlich-gelber Farbe und fast vollkommenem Glasglanze.\footnote{So dass demnach dieser Gemengteil hier in allen Graden von Ausbildung, Ausscheidung und Absonderung, von dem unvollkommensten, kaum von der Grundmasse unterscheidbaren Zustande, wie bei den Steinen von Stannern \emph{a potiori} (und zum Teil bei jenen von Parma, Ensisheim, L'Aigle u. a.), durch die vollkommeneren Mittelzustände, wie \emph{a potiori} bei den Steinen von Benares, Timochin (Tabor, Barbotan, Eichstädt u. a.), bis zu dem vollkommensten, wie bei manchen andern Meteor-Steinen (\emph{a potiori} aber im sibirischen Eisen), in Steinen von einem und demselben Ereignisse, zum Teil selbst in einem und demselben Bruchstücke vorkommt.}

Von den metallischen, dem bewaffneten, so wie selbst dem freien Auge zwar deutlich erkennbaren, aber nur sparsam erscheinenden Gemengteilen zeigt sich der eine --- das Gediegeneisen --- nur in einzelnen, zerstreuten, meistens äußerst zarten Punkten oder Körnern, von licht eisengrauer, ins Silberweiße fallender Farbe, und starkem metallischen Glanze, und zwar auf der rohen Fläche als kleine Zacken, auf der polierten als Punkte oder kleine, äußerst zart zackig gerandete Fleckchen; der andere --- das Schwefeleisen --- teils in ebenso zarten und zerstreuten einzelnen Körnern, teils in kleinen Partien feinkörnig, und hie und da zu etwas größeren Massen bröcklig angehäuft, von zinkgrauer, bald ins Rötliche, bald ins Speisgelbe ziehender Farbe und ziemlich starkem metallischen Glanze.\footnote{Der geringe Gehalt an eingemengtem, regulinischem sowohl als geschwefeltem, Eisen spricht sich übrigens sowohl durch das ziemlich niedere spezifische Gewicht (= 3,3 bis 3,4), als durch die äußerst Wirkung der Steinmasse auf den Magnet aus; inzwischen ist der Total-Gehalt derselben an Eisen nicht unbedeutend, und beträgt nach Howard bei 35, nach Klaproth etwa 28 Perzent (als durch die Operation erhaltenes Oxyd). Da nun, nach einer oberflächlichen Schätzung, das sichtlich eingemengte Gediegeneisen kaum 4 bis 6 Perzent, dass ebenso vorhandene Schwefeleisen aber nur wenig mehr betragen dürfte, vom eingemengten Oxyde sich aber nur wenig Spur findet; so muss ein bedeutender Anteil jenes Gehaltes in den erdigen Gemengteilen chemisch gebunden oder verlarvt enthalten sein.}

Von mechanisch eingemengtem Oxyde oder ähnlichen Partikelchen findet sich nur äußerst wenig, und nur sehr wenige kleine Stellen von schmutzig graulich-gelber, ins Bräunlich- und Rötlich-Gelbe verlaufender Farbe, geben die Gegenwart von Rostflecken zu erkennen.

Von Adern, Gängen oder einer anderweitigen Zustandsveränderung der Steinmasse findet sich aber, weder an diesem, noch an irgendeinem der mehreren von mir gesehenen Bruchstücke von Steinen dieses Herkommens, auch nur die entfernteste Andeutung.

\subsection{Benares.}
\paragraph{}
Ein ausgezeichnetes, 4 3/4 Loth schweres Bruchstück eines, wahrscheinlich ursprünglich ziemlich groß gewesenen Steines von jenen, welche am 19. Dezember 1798, Abends, bei Krakhut in der Nähe von Benares in Bengalen gefallen sind, und welches die kaiserl. Sammlung 1807 von dem jüngst verstorbenen Charles Greville aus London zum Geschenke erhielt.\footnote{Obgleich dieser Steinfall ziemlich bedeutend und ergiebig war, auch von ansässigen Engländern das Factum gleich an Ort und Stelle untersucht, bekannt gemacht und viele Steine nach Europa versendet wurden; so finden sich doch nur wenige Bruchstücke im Besitze bekannter Anstalten oder Sammlungen. So meines Wissens nur im Pariser Museum, im Mus. brit. zu London, in De Drées, Blumenbachs und Klaproths Sammlung, wohin sie wohl sämtlich durch Greville gekommen sind.}

Es ist dasselbe von einer der größeren, rohen Bruchflächen dargestellt, welche das Innere der Steinmasse auf einer Ausdehnung von etwa 2 Quadrat-Zoll Oberfläche, und auf 1/2, 1 bis 1 1/2 Zoll und mehr Entfernung von der äußersten mit Rinde bedeckten Oberfläche des Steines zeigt.

Der Kohäsions-Zustand der Masse im Ganzen ist nur wenig fester und dichter als bei den Steinen von Stannern, und merklich geringer als bei jenen von Siena. Die Grundmasse für sich ist selbst ziemlich leicht zerreiblich, und zerstiebt beim Versuche, Feuer zu schlagen; übrigens ist sie sehr feinkörnig, doch minder so als jene der Steine von Siena. Der Aggregats-Zustand ist ziemlich locker, und bei weitem mehr als bei den Steinen von Stannern und Siena, da die Gemengteile größten Teils sehr ungleichartig sind, und der eine sehr ausgeschieden und meistens scharf abgesondert ist; übrigens fein sandsteinartig, hinsichtlich des einen; grob körnig und kugelicht, hinsichtlich des andern Gemengteiles; und im Ganzen von mandelsteinartigem Ansehen.

Die Grundmasse, die sich, obgleich sie nicht sehr bedeutend über die übrigen Gemengteile vorwaltet, doch als solche wegen der Ausgeschiedenheit und scharfen Begrenzung dieser, sehr deutlich ausspricht, und gewisser Maßen als Zement derselben erscheint --- hat ein ganz mattes, erdiges, raues, mageres Ansehen, und eine sehr licht, nur etwas schmutzig aschgraue, stark ins Weiße fallende Farbe.

Der olivinartige Gemengteil, der beinahe fast die Hälfte der Steinmasse beträgt, erscheint hier auf der rohen Fläche in Gestalt vieler, mehr oder weniger über die Oberfläche hervorragender, zum Teil kleiner und sehr kleiner, zum Teil aber auch bedeutend großer (von der Größe eines Hirsekornes oder kleinen Nadelkopfes bis zu der einer großen Erbse von 1 1/2 bis 2 1/2 Linie im Durchmesser, und selbst noch mehr), selten stumpfeckiger und bloß abgerundeter, gewöhnlich ovaler oder rundlichter, meistens aber vollkommen kugelförmiger Massen, wovon die kleineren und die minder scharf begrenzten und weniger kugelicht ausgeschiedenen fester und inniger von der Grundmasse eingeschlossen sind, und gleichsam in dieselbe übergehen, die größeren und vollkommen kugelicht abgesonderten aber bisweilen so lose sitzen, dass sie leicht aus derselben heraus fallen oder ausgebrochen werden können. Erstere sind gewöhnlich von Partikelchen der Grundmasse eingehüllt, und haben demnach wie diese ein mattes, raues, erdiges Ansehen, und eine gleiche, nur etwas dunklere Farbe; letztere, zumal die vollkommen kugelichten dagegen, haben meistens eine ganz glatte, schwach und etwas fettig glänzende Oberfläche, und eine schiefer- oder bräunlich-graue, bisweilen schmutzig lauch- oder olivengrüne Farbe. Gebrochen zeigen erstere zwar ungleich mehr Festigkeit, Dichtheit und Härte als die Grundmasse, doch bei weitem nicht so sehr wie letztere, welche ziemlich leicht Funken am Stahle geben, und deren scharfkantige Bruchstücke selbst etwas das Glas ritzen, oder dasselbe wenigstens matt machen; auch zeigen diese einen vollkommenen, flachmuschelichen Bruch, indes jener der ersteren sich in verschiedenen Abstufungen aus dem erdigen durch den dichten und ebenen nur allmählich demselben nähert. Nur wenige, selbst von den ausgeschiedensten, zeigen einigen Grad von Durchscheinenheit an den scharfen Kanten ihrer Bruchstücke, alle aber im Bruche und auf einer geschnittenen und polierten Fläche --- wo sie mehr oder weniger rissig und zersprungen erscheinen --- nach den verschiedenen Graden ihrer Dichtheit, einen mehr oder weniger fettigen, oder doch schimmernden Glanz, und eine aus dem Grauen ins Lauch- oder schmutzig Olivengrüne und ins Bräunliche ziehende Farbe. Dort, wo der Art vollkommen kugelichte und scharf ausgeschiedene Massen im Bruche ausgefallen sind, findet sich eine dem Volumen und der Form derselben entsprechende Grube in der Grundmasse, deren Wände, von übrigens mattem, erdigem Ansehen und weißlich-grauer Farbe, verdichtet und gleichsam geglättet erscheinen.\footnote{So wie diese Steine einerseits durch die Beschaffenheit der Grundmasse --- und in vielen andern Beziehungen --- jenen von Siena (und noch mehr jenen von Mauerkirchen, Parma, Eggenfeld) gleichen; so nähern sich dieselben andererseits durch die Art der Ausscheidung sowohl, als durch die Beschaffenheit des olivinartigen Gemengteiles --- wenigstens in den hier einzeln vorkommenden niederen und mittleren Graden --- den meisten übrigen Meteor-Steinen, zumal jenen von Timochin (Eichstädt, Tabor, Barbotan u. v. a.). Nur die besondere Größe einzelner Massen desselben, und die vollkommene Ausscheidung und Absonderung einiger derselben aus der Grundmasse, ist diesen Steinen ganz eigentümlich, obgleich sich auch hierin jene von Weston denselben sehr nähern.}

Die Gediegeneisenteilchen zeigen sich beinahe noch sparsamer, aber in etwas gröberen Körnern und Zacken als an den Steinen von Siena, und ebenfalls von licht stahlgrauer, […] Silberweiße fallender Farbe und metallischem Glanze; die Kies-Partikelchen dagegen zwar ebenso sparsam in zerstreuten, zarten, glänzenden, meistens gelblichen Körnern, häufiger aber in größeren Partien feinkörnig, oder als größere Massen bröcklig (in etwas stumpskantigen, minder spröden und leicht zerreiblichen Stücken) angehäuft, und mehr von zinkgrauer, etwas ins Rötliche ziehender Farbe und schwächerem Glanze.\footnote{Auch hier spricht sich der geringe Gehalt an Gediegeneisen (das kaum 3 Perzent) und an Schwefeleisen (das höchstens das Doppelte von jenem betragen möchte) durch das geringe spezifische Gewicht (= 3,35) und durch den äußerst schwachen Magnetismus der Steinmasse im Ganzen aus; doch beträgt der Total-Gehalt an Eisen auch bei diesen Steinen nach Howard und Vauquelin 34 bis 38 Perzent.} Von Rostflecken zeigt sich kaum eine Spur (obgleich doch, und zwar schon vor eilf Jahren, eine Fläche des Stückes abgeschliffen und poliert worden war), und eben so wenig von deutlich eingemengtem Oxyde. Auch von Adern und Gängen, oder einer sonstigen Zustandsveränderung der Steinmasse, findet sich durchaus keine Andeutung an diesem Stücke.

\subsection{Timochin.}
\paragraph{}
Ein charakteristisches Stück, 4 Loth 3 Quäntchen wiegend, von dem am 13. März 1807 bei Timochin (im Juchnow'schen Kreise, im Smolensk'schen Gouvernement) in Russland einzeln niedergefallenen, bei 140 Pfund wiegenden Steine,\footnote{Außer Russland dürften Bruchstücke von diesem Steine wohl sehr selten zu finden sein, und außer dem Klaproth'schen ist mir nur eines in Blumenbachs, und ein anderes in Chladnis Besitze bekannt.} welches Klaproth von einem mir zur Ansicht mitgeteilten 18 Loth schweren Bruchstücke in seinem Besitze, abschneiden zu lassen gestattete, und der kaiserl. Sammlung gefälligst überließ.

Es zeigt dasselbe das Innere der Steinmasse auf einer geschliffenen und polierten, nur an einer Seite an Rinde anstehenden Fläche, von 2 1/2 Quadrat-Zoll Oberfläche.

Der Kohäsions-Zustand der Masse im Ganzen ist nicht viel fester und dichter als bei den Steinen von Benares, aber inniger, wie es scheint, durch Vermittlung der so häufig eingemengten, rauen und zackigen Gediegeneisenteilchen, und vorzüglich der vielen Rostflecke. Die Grundmasse für sich wäre, abgesehen von letzteren, auch wohl etwas zerreiblich; in jenem Zusammenhange gibt sie aber, wahrscheinlich doch nur mittelst des häufig vorkommenden olivinartigen Gemengteiles, ziemlich leicht Funken am Stahle. Übrigens ist sie nicht sonderlich feinkörnig, weniger beinahe als die der Steine von Benares.

Der Aggregats-Zustand ist, obgleich der olivinartige Gemengteil so häufig, und zum Teil eben so scharf begrenzt (aber lange nicht so abgesondert) und kugelicht (aber viel kleiner) ausgeschieden erscheint --- und wahrscheinlich auch durch Vermittlung der Eisenteilchen und Rostflecke --- viel inniger und fester, obgleich lange nicht so, wie bei andern Meteor-Steinen (z. B. jenen von Charsonville, Salés, selbst jenen von Siena, zumal aber jenen von Ensisheim, L'Aigle u. a.), mehr sandsteinartig, von gröberem und ungleichförmigem Korne, und --- der geringen Menge und Kleinheit der weniger scharf ausgeschiedenen Massen des andern Gemengteiles wegen --- mehr von klein porphyrartigem als mandelsteinartigem Ansehen.

Die Grundmasse, welche hier sehr stark vorwaltet, --- obgleich sie sich, da sie sehr häufig und unmerklich in den andern Gemengteil übergeht, nur schwach ausspricht --- hat ein ganz mattes und erdiges, aber kein so raues und mageres Ansehen, und eine aschgraue, nur wenig ins Bläuliche ziehende Farbe.

Der olivinartige Gemengteil, der, insofern er deutlich ausgesprochen erscheint, kaum 1/6 der ganzen Steinmasse betragen möchte, zeigt sich auf dieser polierten Fläche sehr ungleichförmig zerstreut --- aber ziemlich gleichartig, und nicht sehr abweichend in Größe, Gestalt, Dichtheit, Farbe und Glanz --- in kleinen, sehr und ganz kleinen (selten von 1/2, meistens nur von 1/4 Linie im Durchmesser und noch weniger), meistens rundlichten, selbst auch vollkommen kugelichten Körnern, von grauer, ins Lauch- und schmutzig Oliven-Grüne, oder ins Braune ziehender Farbe, und schwachem, fettigem Glanze.

Es sind diese Körner zwar scharf begrenzt und ausgeschieden, aber bei weitem nicht- so, wie wenigstens viele in den Steinen von Benares (selbst nicht wie manche in jenen von Weston; dagegen genauso wie die meisten in den Steinen von Eichstädt, Tabor, Barbotan u. a.), aus der Grundmasse abgesondert, sondern innig von derselben eingeschlossen und festsitzend, so dass sie an rohen Bruchflächen nie ausgefallen oder ausgebrochen, aber auch nicht verbrochen, und mit rauer, erdiger Oberfläche, mehr oder weniger halbkugelicht, hervorragend erscheinen. Sie sind etwas schwer zersprengbar, zeigen einen dichten, ebenen Bruch, der sich mehr oder weniger dem flachmuschelichen nähert, und geben unbestimmt eckige, nur wenig scharfkantige, meistens vollkommen undurchsichtige, oder nur schwach an den Kanten durchscheinende Bruchstücke.\footnote{Ihre Beschaffenheit ist in allen Beziehungen dieselbe, wie die der ähnlichen in den Steinen von Siena, Benares, und vielen andern (und selbst im sibirischen Eisen), einzeln und selten, in vielen andern Meteor-Steinen aber, als jenen von Eichstädt, Tabor, Barbotan u. a., häufig und vorwaltend in diesem Grade von Ausbildung vorkommenden Massen dieses Gemengteiles.}

Außer diesen einzelnen, durch Farbe und Schärfe der Begrenzung mehr ausgesprochenen und auffallenden, findet sich aber noch eine Menge ähnlicher, zum Teil noch weit kleinerer Körner, die aber nur auf der polierten Fläche als Punkte oder kleine und äußerst kleine Fleckchen zur Ansicht kommen, die sich von der Grundmasse --- mit der sie innig verbunden sind, und in welche sie zum Teil überzugehen scheinen --- bloß durch eine bald etwas lichtere, bald etwas dunklere Farbe, etwas mehr Dichtheit, durch ein feineres Korn und durch ihre Figurierung -- die durch eine mehr oder weniger scharfe, oft kaum merkliche Ausscheidungslinie oder Begrenzung bestimmt wird --- unterscheiden.\footnote{Von eben der Beschaffenheit, wie dieser Gemengteil wieder einzeln in den meisten Meteor-Steinen, häufig und beinahe ausschließlich aber in andern (z. B. in jenen von Charsonville, Salés, Berlanguillas, Apt, York, Lissa u. a.) vorzukommen pflegt.}

Der Gehalt an mechanisch und sichtlich eingemengtem Gediegeneisen ist bei diesen Steinen ausgezeichnet stark, und beträgt fast 20 Perzent, oder beinahe den fünften Teil der Steinmasse.\footnote{Dieser beträchtliche Gehalt an Gediegeneisen, den Klaproth und N. A. Scherer, nach den Resultaten ihrer Analysen, auf beinahe 18 Perzent angeben, gibt sich auch durch das bedeutende spezifische Gewicht (= 3,700 --- worin diese Steine wohl nur von jenen von Eichstädt übertroffen werden dürften, und welchem sich jene von Tipperary, Tabor, Charsonville, Toulouse, Erxleben nur zu nähern scheinen ---), und durch eine sehr starke Wirkung auf den Magnet zu erkennen. Klaproth gibt übrigens noch 25, Scherer 17 1/2 Perzent als den Gehalt dieser Steine an oxydiertem Eisen an, dessen Vorhandensein ersterer den später, durch die Einwirkung unsrer Atmosphäre, entstandenen Rostflecken zuschreibt.} Die Eisenteilchen erscheinen auf den rohen Bruchflächen als einzelne, mehr oder weniger hervorragende, ziemlich starke, raue Zacken und Körner von eisengrauer Farbe und schwachem metallischen Glanze, insofern sie nicht von erdigen Massenteilchen bedeckt sind. Auf der polierten Fläche zeigen sie sich sehr häufig und ziemlich gleichförmig verteilt, als mehr oder weniger zarte Punkte, als größere oder kleinere, meistens gezackte Flecke, und als mehr oder weniger gebogene, ästige und zum Teil zusammenhängende Linien und Adern, von sehr licht stahlgrauer, ins Silberweiße ziehender Farbe, und ziemlich starkem metallischen Glanze. Dagegen ist der Gehalt an Schwefeleisen höchst unbedeutend, und selbst auf der polierten Fläche kann man nur äußerst zarte, mikroskopische Punkte, die hie und da zu kleinen Flecken angehäuft sind, und sich durch eine Zinkweiße, etwas ins Gelbliche oder Rötliche fallende Farbe, und einen etwas schwächeren Glanz auszeichnen, dafür erkennen. Besonders häufig aber zeigen sich die Rostflecke, so dass man sie nach Chladni allerdings für diese Steine (aber ebenso für die Steine von Eichstädt, Charsonville, Barbotan u. e. a.) als charakteristisch ansehen kann, indem sie beinahe die Hälfte der Steinmasse ausmachen, und derselben ein ganz eigentümliches marmoriertes Ansehen geben. Sie sind übrigens hier sehr klein, zart, matt, erdig, und von besonders dunkler gelblich-brauner Farbe.

Von Oxyd oder ähnlichen Partikelchen zeigt sich keine deutliche Spur; eben so wenig von Adern und Gängen oder einer andern Veränderung der Steinmasse.

\subsection{Charsonville.}
\paragraph{}
Ein großes, 1 Pfund schweres Stück von einem der am 23. November 1810 in der Gegend von Charsonville bei Orléans (Departement du Loiret) in Frankreich niedergefallenen Steine, welches während meiner Anwesenheit in Paris (1815) auf mein Ansuchen und mit Genehmigung der königlichen Administration des Museums der Naturgeschichte, von einem daselbst aufbewahrten Bruchstücke,\footnote{In dem, dem Werke Chladnis angeschlossenen Verzeichnisse der Meteor-Massen der kaiserl. Sammlung, ist aus Versehen dieses Bruchstück als ein ganzer Stein angegeben worden. Aus Bigot de Morogues verlässlichen Nachrichten über diesen Steinfall ergibt sich aber, dass dasselbe selbst nur ein Bruchstück, und zwar von dem einen größeren der niedergefallenen und aufgefundenen Steine war, dessen Gewicht bei 40 Pfund betrug, welches D. Pellieux zu Baugenci an den damaligen Minister des Innern (Grafen Montalivet) einsendete, von welchem dasselbe an das königl. Museum abgegeben wurde.} von 11 Pfund am Gewichte, abgeschnitten, und mir, nebst mehreren andern, für die kaiserliche Sammlung gefälligst mitgeteilt wurde.\footnote{Obgleich dieser Steinfall hinsichtlich der Zahl der gefallenen Steine nicht sehr beträchtlich war, indem deren nur drei im Falle beobachtet, und davon selbst nur zwei aufgefunden wurden; so gehört er doch der Masse nach zu den bedeutenderen, da der eine der aufgefundenen Steine bei 40, der andere 20 Pfund wog. Indessen ist mir außer obigem geteilten Bruchstücke nur noch eines in De Drées, und ein zweites in Chladnis Besitze bekannt.}

Es ist dasselbe, auf der zum Teil mit Rinde bedeckten, zum Teil verbrochenen, gewölbten Außenseite liegend, von der durch den Schnitt erhaltenen, ganz ebenen, aber noch unpolierten Fläche dargestellt, die das Innere der Steinmasse auf einer Ausdehnung von ungefähr 4 Quadrat-Zoll, und, wo am dicksten, in einer Tiefe von beinahe 1 1/2 Zoll von der äußern Oberfläche des Steines, zur Ansicht bringt.

Der Kohäsions-Zustand der Masse ist sehr fest und dicht, so dass sie sich hierin den kompaktesten und härtesten Meteor-Steinen (jenen von Ensisheim, Erxleben, Chantonnay) nähert, indes sie doch nur etwas schwer Funken gibt. Der Aggregats-Zustand ist ebenfalls sehr fest und innig, und dabei auch sehr gleichförmig --- da der olivinartige Gemengteil äußerst wenig, nur höchst unvollkommen und schwach ausgeschieden, und von der Grundmasse in allen Beziehungen nur wenig abweichend, und selbst sehr gleichförmig erscheint --- und dicht sandsteinartig, von äußerst feinem, sehr gleichförmigen Korne.

Die Grundmasse, welche hier besonders stark vorwaltet, und abgesehen von den eingemengten Metallteilchen, und ohne Lupe betrachtet, bis auf wenige Massen, in welchen sich der andere Gemengteil etwas deutlicher ausspricht, beinahe die ganze Steinmasse zu konstituieren scheint, indem sie größten Teils allmählich und sehr unmerklich in jenen übergeht --- hat ein ganz mattes, erdiges, aber, selbst auf rohen Bruchstellen, eben kein sehr raues noch mageres Ansehen, und eine aschgraue, nur wenig ins Bläuliche ziehende Farbe.

Der olivinartige Gemengteil erscheint darin nur sehr schwach und undeutlich ausgesprochen, in sehr sparsamen, einzelnen, zerstreuten, sehr und äußerst kleinen, oft kaum merklich ausgeschiedenen, oder doch nur sehr schwach begrenzten, meistens rundlichen oder ovalen, doch auch stumpfeckigen Körnern, von mattem, erdigen Ansehen, und licht aschgrauer, gelblicher, bläulicher, nur selten bräunlicher Farbe. Die meisten dieser Massen unterscheiden sich bloß durch etwas größere Dichtheit, Feinheit im Korne, und durch ihren Umriss von der Grundmasse, und gleichen zum Teil vollkommen jenen, welche in dem zuvor beschriebenen Steine von Timochin in ziemlicher Menge, einzeln aber in den meisten Meteor-Steinen, und zwar gemeinschaftlich mit andern vorkommen, die in verschiedenen und weit höheren Graden von Ausbildung und Ausscheidung sich befinden. Nur sehr wenige davon zeigen sich an den rohen Bruchstellen als vollkommen ausgeschieden oder abgesondert von der Grundmasse, in Kugel- oder Körnerform, mit vorragender konvexer Oberfläche; die meisten sind mit der Grundmasse zugleich gebrochen, und zeigen nur einen dichteren, ebeneren Bruch.\footnote{Wenn das quantitative Verhältnis der nächsten Bestandteile von mehreren Meteor-Steinen mit Verlässlichkeit angegeben wäre; so ließe sich vielleicht --- wie bereits oben erwähnt worden ist --- mit einiger Gewissheit nachweisen, dass in demselben und nicht in bloßen Zustandsveränderungen der Steinmasse, der nächste Grund der ebenso auffallenden als mannigfaltigen Abweichungen in der Menge, Beschaffenheit und in der Art der Ausscheidung und Absonderung dieses Gemengteiles liege, wie dies zum Teil aus den vorhandenen Analysen hervor zu gehen scheint. Von den meisten Meteor-Steinen nämlich, in welchen dieser Gemengteil nur schwach und unvollkommen ausgesprochen ist (wie z. B. in jenen von Stannern, Parma, Charsonville, Doroninsk, L'Aigle, Ensisheim), weisen jene einen verhältnismäßig geringeren Gehalt an Talkerde (nämlich zwischen 2 und 13 Perzent), und dabei einen nicht ganz unbedeutenden Gehalt an Thon- und Kalkerde (von ersterer 3 bis 9, von letzterer 4 bis 12 Perzent) aus; von jenen dagegen, wo derselbe häufiger, deutlich ausgesprochen, oder in einem besonders hohen Grade von Ausbildung, oder vollends vorwaltend erscheint (wie in jenen von Eichstädt, Tabor, Benares, Eggenfeld, Erxleben, Chassigny), einen weit größeren Gehalt an Talkerde (17 bis 21; 23; 26 bis 32 Perzent), aber keine Spur, oder doch nur äußerst wenig (1 1/4 und 1/2 Perzent von jenen von Erxleben), an Thon- und Kalkerde. Die Steine von Timochin hielten in beiden Beziehungen gerade das Mittel. (Klaproth gibt deren Gehalt an Talkerde auf 14 1/4, von Thon auf 1, und von Kalkerde auf 3/4 Perzent an.)}

Die Gediegeneisenteilchen werden durch ihre Menge und zum Teil durch ihre Beschaffenheit charakteristisch für diese Steine. Sie erscheinen nämlich äußerst häufig --- so dass ihre Masse zusammen genommen, nach einer oberflächlichen Abschätzung, gut den vierten Teil des Ganzen betragen möchte\footnote{Bigot de Morogues schätzt den Gehalt auf 31 Perzent. Vauquelin gibt den Gehalt des von ihm analysierten Stückes im Ganzen mit 25,8 als regulinisch an (nach Kalkül, denn er hatte nach seinen Verfahren alles Eisen daraus als Oxyd im \emph{maximum}, also etwa 36 Perzent erhalten). Es ergibt sich hieraus, dass der Total-Gehalt dieser Steine an Eisen eben nicht größer ist, als bei den meisten Meteor-Steinen, und dass, da sich dieser Gehalt, dem äußern Ansehen nach, schon in dem mechanisch eingemengten Gediegeneisen ausspricht, in diesen Steinen wenig oder gar nichts oxydiert, vererzt oder sonst verlarvt enthalten sein könne. Der starke Gehalt an Gediegeneisen bewährt sich übrigens nicht nur durch das spezifische Gewicht (das --- im Durchschnitt und mit Hinsicht auf Adern und Rinde --- zwischen 3,6 und 3,7 fällt), sondern auch durch sehr starke Wirkung der Steinmasse auf den Magnet.} --- höchst unregelmäßig zwar, aber doch ziemlich gleichförmig, und im Ganzen sehr dicht eingestreut, und auf dieser geschnittenen Fläche als etwas erhabene, äußerst zarte Punkte von licht eisengrauer Farbe und etwas mattem metallischen Glanze, die hin und wieder zusammen gehäuft und gewisser Maßen zusammen geflossen, mehr oder weniger Adern gleichende, nur selten und wenig zusammen hängende, gezackte, gekörnte und gleichsam geträufte, kleine Flecke oder Massen bilden, welche sich mit einem stählernen Instrumente sehr leicht breit und platt drücken und ritzen lassen, und dann (wie an den rohen Bruchstellen) einen höheren metallischen Glanz und eine stark ins Silberweiße fallende Farbe zeigen.

Von Kiesteilchen findet sich dagegen nur wenig Spur in äußerst zarten Punkten, von etwas stärkeren metallischem Glanze, und einer aus dem Weißen ins Messinggelbe ziehender Farbe, und noch weniger von Oxyden oder ähnlichen Partikelchen; umso häufiger erscheinen aber die Rostflecke, die durch ihre Menge sowohl --- da sie der ganzen Oberfläche ein zart marmoriertes Ansehen geben --- als durch ihre Zartheit und Farbe --- indem sie meistens als einzelne, äußerst feine Punkte, die nur stellenweise in Flecke zusammen geflossen sind, und von einer eigenen graulich-gelblichen Farbe erscheinen --- ebenfalls als charakteristisch für diese Steine angesehen werden könnten, insofern sie nicht späterhin und zufällig entstanden sind.\footnote{Es ist bemerkenswert, dass die Rostflecke an diesem Stücke in einem Zeitraume von fünf Jahren, während welchem dasselbe der Luft, dem Lichte und selbst häufiger Betastung ausgesetzt war, sich gar nicht merklich vergrößert, vermehrt, noch in irgendeiner Beziehung verändert haben.}

Das Merkwürdigste an diesem Steine, und weshalb auch dessen bildliche Darstellung versucht wurde, sind die Adern und Gänge von einer, scheinbar, fremdartigen Substanz, welche auf dessen Oberfläche erscheinen und die Steinmasse durchziehen, von welchen bereits oben in der Einleitung zur Erklärung dieser Tafel im Allgemeinen gesprochen wurde, und die bei diesen Steinen, zwar gerade nicht am häufigsten (denn ungleich häufiger zeigen sie sich bei jenen von Agen und Lissa), aber durch Stärke und Ausdehnung am ausgezeichnetsten vorkommen.

Es zeigen sich auf der geschnittenen Fläche dieses Stückes zwei solche Adern.\footnote{Bigot de Morogues, welcher Gelegenheit hatte, Bruchstücke von beiden aufgefundenen Steinen zu untersuchen, bemerkte in dem einen zwar viele, aber äußerst zarte, dem freien Auge kaum sichtbare Adern, in dem andern mehrere, aber durchaus stärkere, und darunter eine von 1 bis 3 Linien in der Breite oder Mächtigkeit, und von sehr abweichender Dicke oder Tiefe. Hauy und Vauquelin haben an dem großen Bruchstücke des Museums, welches mit letzterem Stücke Bigots von demselben Steine herstammt, nur eine Ader bemerkt, indes an dem hier beschriebenen, unmittelbar von ersterem abgeschnittenen Stücke, deren zwei vorkommen.} Die eine davon geht von einem Rande des Stückes etwas schief quer über die Fläche zum andern, die aber beide nicht die Grenzen des ursprünglichen Steines und dessen Oberfläche bezeichnen, indem sie verbrochen und rindenlos sind. Sie ist an einem Ende bei 5/4 Linien breit, verschmälert sich allmählich, und läuft gegen das andere beinahe haarfein aus. Im Verlaufe macht sie nur einige schwache und kleine Biegungen, und erscheint bald breiter, bald schmäler, so dass sie an einigen Stellen 1/2, gleich unmittelbar darauf schnell abnehmend, kaum 1/4 Linie breit ist, zeigt aber nur einen einzigen, zarten Seitenzweig im ersten Drittel ihres Laufes, der unter einem ziemlich spitzen Winkel von ihr ausgeht, schief vor- und aufwärts steigt, und sich sehr bald haarfein in die Steinmasse verläuft. In derselben Gegend zeigt sich ein ebenso zarter, aber unausgefüllter, leerer Riss oder Sprung in der Steinmasse, der quer vom Rande herkommt, und sich nahe an der Hauptader verliert, ohne mit ihr in Berührung zu kommen; ein zweiter ähnlicher zeit sich am andern Ende derselben, der eine Strecke weit schief einwärts geht. An beiden Rändern des Stückes, wo diese Ader ausgeht, und wo absichtlich ein kleines Stück abgeschlagen wurde, um den Verlauf in die Tiefe zu verfolgen, zeigt sich, dass diese Ader eine an Breite oder Mächtigkeit den beiden Ausgängen entsprechende Lage bezeichnet, die in schiefer Richtung (unter einem Winkel von etwa 60° gegen die Oberfläche) die Steinmasse auf eine Tiefe von einem halben Zoll durchsetzt.

Die zweite Ader geht von demselben Rande aus, weicht aber im Verlaufe von jener ab, und zieht ebenfalls etwas schief und quer über die Fläche gegen einen andern Rand hin, wo wirklich von Außen Rinde ansteht, in welche sie sich verläuft. Sie ist beinahe durchaus im ganzen Verlaufe haarfein, nur in ihrer Mitte bildet sie gleichsam einen ovalen Wulst oder Botzen (2 Linien lang, 1 Linie breit), der durch einen quer aus der Mitte der Fläche herkommenden schwachen Riss etwas zerklüftet ist --- und erscheint zwei Mahl etwas bogenförmig in entgegen gesetzten Richtungen geschwungen. Sie zeigt wohl hin und wieder eine Spur von Seitenzweigen, die von ihr unter verschiedenen Winkeln und in verschiedenen Richtungen ausgehen, und gegen einen Rand hin oder in die Steinmasse verlaufen --- sie sind aber mikroskopisch fein, so wie eine ähnliche Ader, die in geringer Entfernung von dieser, und fast in paralleler Richtung mit ihr, frei mitten auf der Fläche eine Strecke fortläuft; dagegen findet sich ein Netz von ähnlichen Adern, gegen den einen Rand des Stückes, die teils von diesem, teils von jener Hauptader ausgehen, und ebenso wechselseitig gegen einander sich verlaufen, unter sich verzweigen, einmünden, und verschiedentlich sich durchschneiden und kreuzen.

Alle diese Adern zeigen, sowohl auf der geschnittenen Fläche als an den, dieser entgegen gesetzten, frischen Bruchstellen, eine matte, schwärzlich- bläulich- oder dunkel schiefer-graue Farbe, durch welche allein sie sich von der übrigen Steinmasse unterscheiden. Die Substanz selbst ist gar nicht fremdartig, durch gar nichts von jener getrennt, sondern bloß durch die Farbe, durch diese aber scharf von ihr geschieden; im Gegenteil ist die Verbindung und der Zusammenhang mit derselben sehr fest und innig, so zwar, dass die Steinmasse beinahe leichter quer über als an und in der Richtung dieser Adern bricht, zumal wenn sie von einiger Dicke sind. Die Unebenheiten jener setzen sich ununterbrochen und in derselben Richtung über diese fort; der Bruch ist ganz derselbe, nur etwas dichter, und an einer, obgleich nur kleinen Stelle der breiteren Ader, zeigt sich eine Spur von unvollkommen schiefriger Textur, in perpendikulärer, aber etwas schiefer und gekrümmter Richtung. Es wirkt diese Ader-Substanz übrigens etwas stärker als die übrige Steinmasse, aber doch schwächer als die Außenrinde, auf die Magnetnadel, auch ist dicht an ihr und mitten in ihr, ebenso wie in der ganzen Masse, Gediegeneisen eingesprengt. Mit der Rinde des Steines hat sie weder der Farbe, noch weniger der Textur und übrigen Beschaffenheit nach, die geringste Ähnlichkeit. Von einer anderweitigen, dieser Substanz mehr oder weniger ähnlichen Beschaffenheit der Steinmasse, von Absonderungsflächen oder metallischem Anfluge zeigt sich an diesem Stücke keine deutliche Spur.\footnote{An einem kleinen Stücke, dass ich selbst besitze, findet sich eine Absonderungsfläche mit metallischem, graphitähnlichen Anfluge, ganz von der Art, wie an den Steinen von York, Sigena, Laponas \emph{zc.}}

\subsection{Salés.}
\paragraph{}
Ein charakteristisches Stück, 2 1/2 Loth schwer, von dem am 12. März 1798 bei Salés (nicht weit von Ville Franche, Departement du Rhone) in Frankreich\footnote{Der verzögerten Bekanntwerdung des Factums, die wir den späteren, eifrigen Nachforschungen des Marquis De Drée verdanken, und der Unbedeutendheit der niedergefallenen Masse ist es zuzuschreiben, dass nur mehr wenige Fragmente davon nachweisbar vorhanden sind, wovon sich eines im Mus. brit. zu London, aus Grevilles Vermächtnis, und ähnliche in De Drées, Blumnenbachs und Chladnis Besitze sich befinden.} einzeln gefallenen Steine, der ungefähr 20 bis 25 Pfund wog, welches die kaiserl. Sammlung der gefälligen Mitteilung des Marquis De Drée verdankt.

Es ist dasselbe von einer der größeren, abgeschliffenen Flächen dargestellt, die das Innere der Steinmasse auf einem Flächenraume von etwa 1 1/4 Quadrat-Zoll, und auf wenigstens 1 1/2 Zoll Entfernung von der äußersten Oberfläche des Steines zeigt, wo nämlich an einer Seite Rinde ansteht.

Der Kohäsions-Zustand ist beinahe eben so dicht und fest, wie am Steine von Charsonville; die Härte der Steinmasse im Ganzen doch bedeutend geringer, da sie nur schwer und schwach Funken gibt. Der Aggregats-Zustand ist zwar (des schon etwas häufiger und zum Teil mehr ausgesprochenen olivinartigen Gemengteiles wegen) im Ganzen gröber, doch beinahe eben so dicht und innig; die Textur von ebenso feinem und gleichförmigen Korne, beinahe noch in einem höheren Grade, und die ziemlich stark vorwaltende, aber im Ganzen nur wenig durch die Gemengteile herausgehobene Grundmassen von mattem, erdigem Ansehen, und von licht aschgrauer, beinahe gar nicht ins Bläuliche fallender Farbe.

Der olivinartige Gemengteil erscheint darin weit häufiger als im Steine von Charsonville, und teils, und zwar größten Teils, in ganz ähnlichen, ebenfalls nur schwach und undeutlich ausgesprochenen, sehr kleinen, schwach begrenzten und innig mit der Grundmasse verbundenen, runden, ovalen, mitunter auch stumpfeckigen Körnern und Mandeln von mattem, erdigem Ansehen, und licht und dunkler aschgrauer, mehr oder weniger ins Bläuliche ziehender Farbe, die dem Ganzen ein schwach porphyrartiges Ansehen geben; teils aber auch, obgleich in einem nur geringen Verhältnisse, in einzelnen, kleinen und größeren, scharf ausgeschiedenen und begrenzten (zum Teil selbst durch eine zarte, vertiefte Linie von der Grundmasse abgesonderten), meistens vollkommen kugelichten (ganz jenen ausgesprochenern im Steine von Timochin und vielen von jenen im Steine von Benares ähnlichen) Körnern, von dunkel bläulichgrauer, ins Lauchgrüne ziehender Farbe, etwas fettigem Glanze, größerer Dichtheit, Härte, rissiger Oberfläche u. s. w., die auch auf den rohen Bruchflächen als insitzende Kügelchen mit hervorragender konvexer Oberfläche, auch wohl schon ausgebrochen, erscheinen.

Der Gehalt an Gediegeneisen zeigt sich dagegen ungleich geringer als am Steine von Charsonville\footnote{Ich fand das spezifische Gewicht eines kleinen, rindelosen, und, nach möglichst genauer Prüfung, von größeren Gediegeneisenteilchen ganz freien Stücke = 3,434; da nun aber das in größeren Massen zerstreut eingemengte Gediegeneisen im Ganzen bald mehr betragen dürfte, als das zart eingesprengte zusammen genommen, und ersteres demnach auf die ganze Steinmasse verteilt werden müsste; so möchte das das spezifische Gewicht wohl zwischen 3,5 und 3,6 anzusetzen sein, welchem auch der wahre Total-Gehalt an Gediegeneisen, nach oberflächlicher Abschätzung (= etwa 0,08 bis 0,10) entspräche. (Vauquelin erhielt bei der Analyse 38 Perzent als Oxyd.) Abgesehen von den größeren Eisenteilchen ist die Wirkung der Steinmasse auf den Magnet auch nur schwach, ebenso wie bei den Steinen von Lissa, stärker jedoch als bei jenen von Siena und Benares.} (Timochin u. v. a.), und die Eisenteilchen erscheinen größten Teils --- außer in eben nicht sehr häufig eingestreuten, zarten Punkten und Körnern --- von seltenerer Art des Vorkommens, nämlich in beträchtlicheren Massen, die auf der polierten Fläche als unregelmäßig gestaltete, eckige, zum Teil gezackte und kleinästige, scharf begrenzte, aber fest eingeschlossene Flecke von licht eisengrauer, stark ins Silberweiße fallender Farbe, und mit starkem metallischen Glanze sich zeigen, und wovon einer der größeren hier, von ovaler, etwas keilförmiger Gestalt, 2 Linien in der Länge, und 1 1/2 in der größten Breite mißt.\footnote{De Drée fand in einem Stücke dieses Steines ein 24 Gran wiegendes Korn von Gediegeneisen.}

Kiesteilchen lassen sich nur äußerst wenige, höchst zart eingesprengt und feinkörnig angehäuft, auf der polierten Fläche durch eine mattere, aus dem Zinkgrauen etwas ins Rötliche stechende, auf den rohen Bruchflächen aber durch eine glänzendere, und mehr ins Gelbe ziehende Farbe von jenen unterscheiden.\footnote{Es ist dieser Kies sehr spröde, leicht zersprengbar, und lässt sich sehr leicht zum feinsten Pulver zerreiben, zeigt sich aber auch als solches ganz ohne Wirkung auf die Magnetnadel.} Von Oxydkörnern zeigt sich keine Spur, und von Rostflecken nur äußerst wenig. Zarte, mikroskopisch feine, schwärzliche Adern durchziehen die Masse nach allen Richtungen, ohne doch die Ränder, selbst dieser kleinen Fläche, zu berühren; von Absonderungsflächen oder einem metallischen Anfluge findet sich aber an diesem Stücke sonst keine weitere Andeutung.

\subsection{Stannern.}
\paragraph{}
Ein 13 1/2 Loth schweres Bruchstück von demselben großen, ursprünglich bei 4 Pfund schwer gewesenen Steine von Stannern, von welchem, durch Zerschlagen der davon erhaltenen Hälfte, auch das oben beschriebene und Fig. 5 der vorigen Tafel abgebildete Stück erhalten worden war.

Dieses Bruchstück --- von welchem hier des Raumes wegen nur ein Teil vorgestellt ist --- zeigt auf seiner ganzen, bedeutend großen, rohen Bruchfläche von 5 Quadrat-Zoll Ausdehnung, an allen Rändern an Rinde anstoßend, das gewöhnliche, sehr zarte und feine, und hier ganz besonders gleichförmige Gemenge der beiden erdigen Gemengteile von ganz gleicher Textur und Beschaffenheit, nur dass sich der olivinartige etwas durch Farbe und größere Dichtheit unterscheidet, ohne sich jedoch durch eine bestimmtere Form oder schärfere Begrenzung auszuzeichnen.

Das Merkwürdige an diesem Stücke ist der ausgezeichnete Gehalt an Schwefeleisen. Es ist dasselbe hier nur wenig in zarten Punkten und Körnern eingestreut, dagegen an mehreren Stellen in beträchtlichen Massen eingemengt. Eine solche fast viereckige von 1/4 Zoll Ausdehnung zeigt sich, und zwar ganz dicht, kaum auf 1 Linie Entfernung von der anstehenden Rinde an dem einen Rande, zerklüftet und in unregelmäßige, unbestimmt eckige, ziemlich scharfkantige Bruchstücke zersprungen und bröcklig angehäuft, von körniger Textur, ziemlich dunkelgrauer, weiß schimmernder, ins Rötliche stechender Farbe, und mit schwachem metallischen Glanze. An einer andern Stelle, ganz dicht an der Rinde, findet sich eine kleinere Masse, die zum Teil wie geschmolzen aussieht, von pfauenschweifigem Farbenspiele und etwas stärkerem Glanze.
\clearpage
\section{Achte und neunte Tafel.}
\paragraph{}
Der Zweck der bildlichen Darstellungen dieser Tafeln ist die Versinnlichung des merkwürdigen kristallinischen Gefüges der vorzüglichsten Gediegeneisen-Massen, deren meteorischer Ursprung teils faktisch erwiesen, teils höchst wahrscheinlich, ja unbezweifelbar ist, und deren Untersuchung in jener Beziehung mir bisher möglich war.\footnote{Es ist die Entdeckung dieser Eigentümlichkeit des Gediegeneisens, wahrhaft meteorischen Ursprunges, schon seit mehreren Jahren ziemlich bekannt; denn Herr Direktor v. Widmannstätten machte sie bereits im Jahre 1808 bei Gelegenheit der ersten physisch-technischen Versuche, die er mit der Agramer Eisenmasse vornahm, und wir waren weit entfernt sie geheim zu halten, im Gegenteile ward dieselbe allen Wissenschaftsfreunden gelegenheitlich mitgeteilt, und jene Masse, an welcher (wie bereits oben erwähnt wurde) eine bedeutende Fläche geätzt worden war, um das Gefüge darzustellen, bleib nach wie vor, und zwar seit 1809, mit den übrigen vorhandenen Meteor-Massen und der zahlreichen Suite von ausgewählten Stücken vom Steinfalle zu Stannern vereinigt, und als eine für sich bestehende Sammlung abgeschlossen, am kaiserl. Mineralien-Kabinette zur öffentlichen Ansicht ausgestellt. Noch in demselben Jahre hatte Herr v. Widmannstätten Gelegenheit, an einem ausgezeichnet schönen Ladenstücke vom sibirischen Eisen aus der Von der Null'schen Sammlung --- deren sachverständiger Besitzer sich sehr bereitwillig fand den Schnitt wund Schliff dieses kostbaren Stückes zu gestatten, da es damit von der andern Seite ein höheres Interesse gewann; --- im Jahre 1810 aber an dem Stücke vom Mexikaner Eisen, welches die kaiserl. Sammlung eben durch Klaproth erhalten hatte; dann im Jahre 1812 an der großen Gediegeneisen-Masse, welche vom Magistrate zu Elbogen in Böhmen an das kaiserl. Naturalien-Kabinett abgegeben wurde; endlich 1815 an dem Stücke vom karpatischen Eisen, welches Herr Baron v. Brudern dem kaiserl. Kabinette zum Geschenke machte --- jene interessante Entdeckung zu bewähren. Da sich jenes Gefüge auf ebenen und polierten Flächen bei der Behandlung durch Ätzung in tastbaren, und zwar nach Maßgabe der Dauer des Prozesses, in mehr oder weniger erhabenen und vertieften Figuren (\emph{en basrelief}) ausspricht; so kam Herr v. Widmannstätten gleich Anfangs, bei der Agramer Masse schon, auf die glückliche Idee, durch unmittelbare Abdrücke solcher Flächen mittelst Druckerschwärze --- die Masse selbst gleich als natürliche Form oder Stereotyp benützend --- eine vollkommen getreue und leicht vervielfachbare Darstellung zu bewirken, und der gute Erfolg dieses Verfahrens veranlasste uns 1813, von der großen geätzten Fläche der Elbogner Masse, welche das Gefüge besonders schön und deutlich zeigte, solche unmittelbare Abdrücke in hinlänglicher Menge abziehen zu machen, um sie als Belege zu einer Abhandlung zu gebrauchen, die wir damals schon über diesen Gegenstand auszuarbeiten und bekannt zu machen dachten. Allein Zeitumstände und Verhältnisse erschwerten unsere Arbeiten, die eine Reihe von mühsamen und ununterbrochenen Versuchen und Untersuchungen notwendig machten, und brachten uns zuletzt --- wie mirs 1809 mit meinen früheren ähnlichen Unternehmungen ergangen war --- ganz davon ab, so dass jene Autografe bis zu dieser Stunde, als sie endlich eine neue Veranlassung --- leider nur zu unvorbereitet und peremtorisch --- ans Tageslicht ruft, unbenutzt liegen blieben. Inzwischen wurde der Gegenstand durch mündliche Mitteilungen, zumal durch Fremde und Reisende, immer mehr und mehr bekannter, und endlich, vorzüglich teils durch Chladni selbst --- der während seines Aufenthalts in Wien, im Frühjahr 1812, Zeuge unsrer früheren und damaligen Versuche war --- teils auf dessen Anregung öffentlich zur Sprache gebracht; so äußerten Herr Gubernialrat Neumann in Prag, auf dessen Veranlassung, bei Gelegenheit seiner Nachricht von der Elbogner Masse (1812, Hesperus, Heft 9), und nach dieses letzteren Mitteilung, Schweigger (1813, Journal für Chemie und Physik, Bd. 7) ihre, und Chladni selbst (1815, in Gilberts Annalen, Bd. 50) seine Meinung und Erfahrung darüber, und auch unser Herr v. Hammer erwähnte desselben bei Gelegenheit einer Mutmaßung über die orientalischen damaszierten Klingen (1815, in den Fundgruben des Orients, Bd. 4, daraus im Hesperus Heft 9). Späterhin ward der Gegenstand vollends durch mich selbst in Gesprächen mit wissenschaftlichen Freunden, auf meiner Geschäftsreise nach Paris, 1815, in Deutschland und Frankreich verbreitet, und in der Folge durch Mitteilung von einzelnen Blättern jener autographischen Abdrücke an einige meiner Korrespondenten, dort und auch England noch genauer bekannt, und veranlasste die Äußerungen Gillet de Laumonts (Jour. des Mines, Vol. 38, Sept. 1815), und Sömmerrings (in einer Vorlesung an der königl. Bayerischen Akademie der Wissenschaften im Februar 1816, abgedruckt in der Bibl. univers. T. 7, und in Schweiggers Journal für Chemie und Physik, Bd. 20), Schweiggers (in dessen Journal, Bd. 19), und Leonhards (in dessen Taschenbuche für Mineralogie, Bd. 12).} Es zeigt sich dasselbe am schönsten und deutlichsten auf ganz ebenen, rein abgeschliffenen und fein polierten Flächen solcher Massen --- insofern diese nicht etwa durch künstliche Hitze oder durch mechanische Gewalt vorher eine Veränderung erlitten haben\footnote{Wird nämlich ein Stück einer solchen Masse, und zwar bloß kalt und nur nach einer Richtung mehr oder weniger platt gehämmert, dann erst abgeschliffen, poliert und geätzt; so zeigen sich auf licht stahlgrauem matten Grunde nur wellenförmige und verschiedentlich gebogene und gekrümmte, nach verschiedenen Richtungen, und nur zum Teil parallel verlaufende, im Verlaufe sehr ungleich begrenzte, oft fleckartig ausgebreitete, erhabene Linien, und unregelmäßige, mehr oder weniger zusammenhangende Winkelzüge von licht stahlgrauer, stark ins Silberweiße fallender Farbe und einigem Glanze. Wird ein solches Stück aber vollends heiß und nach verschiedenen Richtungen gehämmert; so erscheint eine höchst unvollkommene und verworrene Zeichnung, von der sich zuletzt, bei fortgesetzter ähnlicher Behandlung, [...] Spur verliert, und die licht stahlgraue Oberfläche durch die Einwirkung der Säure nicht verändert, sondern nur etwas, und zwar im Ganzen und gleichförmig, dunkler gefärbt und matt erscheint.} --- wenn dieselben mit Salpetersäure\footnote{Schwefel- und Salzsäure bewirken zwar dieselbe Erscheinung, aber nicht so vollkommen, und langsamer. Sehr konzentrierte rauchende Salpetersäure wirkt zwar schneller, aber oft zu tumultuarisch; man tut am besten, dieselbe, wenn man gerade nicht schnell und tief ätzen will, mit etwa zwei auch drei Teil Wasser zu verdünnen. Die zu ätzende Fläche muss in eine feste, vollkommen horizontale Lage gebracht, und mit einem, etwa eine Linie hohen Saum oder Rand von Wachs umgeben werden, damit die Säure nicht abfließe, die doch 1/4 oder 1/2 Linie hoch die Fläche gleichförmig bedecken soll. Wenn die Ätzung etwas tief zu geschehen hat, so ist notwendig die Säure zu wiederholten Mahlen zu erneuern, und dabei ist es gut, wenn man unter einem die Fläche jedes Mal mit reinem Wasser abspült, auch wohl mittelst eines Pinsels oder einer feinen Bürste abstreift, um sie von dem erzeugten Eisenoxyde, und dem, bei Verdünstung des Fluidums, darauf niedergeschlagenen salpetersauren Eisen zu reinigen, welche die Einwirkung der frisch aufgegossenen Säure verhindern würden. Soll die Ätzung sehr tief (z. B. 1/4 bis 1/2 Linie tief) eindringen; so fordert dies, auch bei jenem Verfahren, mehrere Tage Zeit, und wenn man den Prozess beschleunigen will, muss die Wirkung der Säure außerdem noch durch Wärme, auch wohl durch Zusatz von etwas Salzsäure, verstärkt werden.} übergossen werden, und diese eine Zeitlang auf die Oberfläche eingewirkt hat.\footnote{Eine Spur von dem Gefüge zeigt sich zwar schon, aber nur wie ein Hauch, und nur bei gewissen Wendungen gegen das Licht, auf einer Fläche die vorläufig aus dem Rohen geschliffen und adoucirt worden ist; sie verliert sich aber ganz wieder während des weitern Polierens, so dass eine vollends fein polierte Fläche, abgesehen von den durch Farbe, Glanz und Textur sich auszeichnenden, zerstreut eingemengten Massen der heterogenen bröcklig-körnigen Substanz, ein vollkommen gleichförmiges Ansehen von licht stahlgrauer, mehr oder weniger ins Silberweiße fallender Farbe, und von ziemlich starkem, metallisch spiegelnden Glanze zeigt. Auffallend und ausgezeichnet schön aber spricht sich das Gefüge auf solchen fein polierten Flächen aus, wenn man dieselben, wie Stahl, auf die gewöhnliche Art durch Erhitzung blau anlaufen lässt. Anstatt nämlich, dass dieselben mit den bekannten Farben, aus dem Goldgelben ins Veilchenblaue bis ins Dunkelblaue in allmählicher Progression nach der Dauer des Prozesses, gleichförmig anlaufen, zeigen sie vielmehr diese Farben, wenn der Prozess bis zum Erscheinen des Blauen gekommen ist, alle zugleich, und zwar nach den verschiedenen Teilen des Gefüges, eine ähnliche Zeichnung wie die Ätzung hervorbringend. Die Streifen nämlich erscheinen purpurrot ins Blaue, die Zwischenfelder oder Figuren bald aus dem Blauen, bald aus dem Rothen ins Goldgelbe (nach Glattheit oder Streifung derselben) verlaufend, die Ränder oder Einfassungslinien aber, so wie selbst die zartesten Schraffierungslinien, rein Goldgelb, jene Massen der körnig-bröckligen Substanz endlich von etwas matter und ins Messinggelbe fallender Farbe.} Die Einwirkung geht gewöhnlich auf der Stelle vor sich, und nach wenigen Minuten schon, oft augenblicklich, zeigt sich das Gefüge in den gleich näher zu beschreibenden geraden Streifen und winkeligen Figuren, die sich aber noch gar nicht durch Erhabenheit und Vertiefung, sondern bloß, gleichsam als ein oberflächlicher Anflug, oder vielmehr wie angehaucht, durch Farbe und Glanz aussprechen; die Streifen nämlich erscheinen matt und von sehr licht stahlgrauer, die Figuren oder Zwischenfelder dagegen, welche von jenen begrenzt oder eingeschlossen werden, zwar ebenfalls matt, aber doch --- bei schiefer Richtung der Fläche --- mit einigem Schimmer von ihrem Rande her, und von ziemlich dunkler, eisengrauer Farbe; die Ränder von beiden endlich sind von einer gemeinschaftlichen, zarten Linie eingefasst, die aber ebenfalls nur bei schräger Richtung und bei Wendungen deutlich sichtbar wird, und sich dann durch eine silberweiße Farbe, und durch einen starken, spiegelnden Glanz auszeichnet. In größeren oder kleineren Klüften, und in zarten, oft sehr feinen Rissen --- welche sich ursprünglich schon und vor der Ätzung auf der Oberfläche zeigten --- aber auch häufig zerstreut eingemengt und fest eingeschlossen, in einzelnen kleinen und äußerst kleinen Partien bröcklig ober feinkörnig angehäuft, oft auch nur als einzelne zarte Körner eingesprengt in die übrige Metallmasse, erscheint eine andere metallische Substanz --- insofern sie nicht hier und da durch Schnitt und Schliff der Fläche ausgesprengt worden ist --- von ziemlich starkem Glanze und silberweißer oder zinkgrauer, bisweilen etwas ins Gelbliche oder Rötliche ziehender Farbe, auf welche die Säure schon etwas weniger als auf die übrige Oberfläche eingewirkt zu haben scheint.

Wird die Ätzung längere Zeit fortgesetzt, so erscheinen die einzelnen Teile des Gefüges nicht nur immer deutlicher, sondern allmählich und immer mehr und mehr, und zwar in verschiedenen Graden vertieft, und es zeigen sich jene Streifen nun am tiefsten, die Zwischenfelder oder Figuren dagegen etwas weniger tief, deren Einfassungslinien aber und die Massen jener bröcklig-körnigen Substanz am erhabensten. Hat man demnach die Ätzung bis auf einen gewissen Grad\footnote{Auf etwa 1/12 Linie der tiefsten Stellen. Es darf natürlich dieser Grad nicht um gar viel überschritten werden, weil sonst die minder erhabenen Stellen im Verhältnis zu den erhabensten zu tief zu liegen kommen, und sich nur schwach oder gar nicht ausdrucken.} fortgesetzt; so ist die ganze Zeichnung eines unmittelbaren Abdruckes von der Fläche mittelst Druckerschwärze fähig, indem die erhabensten Stellen sich stark, die minder erhabenen schwächer, die tieferen dagegen sich gar nicht ausdrucken, und da sie alle regelmäßig abwechseln und unter einander verbunden sind, so erhält man solcher Gestalt nicht nur eine ganz vollkommene und genaue Darstellung der geätzten Fläche, sondern auch ein treues Bild des natürlichen Gefüges der Masse, wie sich dasselbe durch die Ätzung ausspricht.\footnote{Obgleich die Möglichkeit des Vorkommens von wahrhaft meteorischem Gediegeneisen ohne solchem Gefüge nicht geradezu in Abrede gestellt werden kann, zumal wenn dasselbe --- was jedoch nicht wahrscheinlich ist --- von einer bloßen Zustands-Modifikation des reinen Metalle, und bloß von einer regelmäßigen mechanischen Lagerung und Fügung der Grundteilchen, nicht aber von einer besonderen und eigentümlichen, chemischen oder mechanischen Verbindung mit andern Stoffen, einem eigenen Mischungs- und regelmäßigen Mengungs- und Absonderungsverhältnisse abhängen sollte; so ist doch merkwürdig, dass dasselbe noch bei allen Gediegeneisen-Massen gefunden wurde, deren meteorischer Ursprung, wenn gleich nicht --- so wie von der Agramer --- faktisch erwiesen, aber doch der vollkommensten Ähnlichkeit wegen mit dieser und nach allen physischen und chemischen Kriterien unbezweifelbar ist, und selbst bei den kleinen, mechanisch eingemengten Massen von Gediegeneisen in Meteor-Steinen --- insofern dieselben nur Größe genug hatten, um darauf ohne Veränderung ihrer Struktur (durch allzustarke Fletschung z. B.) untersucht werden zu können --- dagegen keine Spur davon bei solchen, die jenen Forderungen, eine ähnliche Herkunft zu bewähren, nicht vollkommen entsprechen, und die auch nur insofern noch ihres Ursprunges wegen mehr oder weniger für problematisch angesehen werden, als sie zum Teil an Orten gefunden worden sind, wo man keinen Grund hat natürliche Eisenlager in der Nähe, oder die frühere Existenz von Eisenhütten zu vermuten, und es sich zur Zeit nicht wohl begreifen lässt, wie sie dahin gekommen, oder durch welchen irdischen Prozess sie dort gebildet worden sein konnten: wie jene Massen von Aachen, Mailand, Cilly, Kamsdorf, Florac, u. m. a., die übrigens aber auch des als eigentümlich und charakteristisch (obgleich wohl nicht minder unter gewissen Restriktionen) für jenen Ursprung angesehenen Gehaltes an Nickel ermangeln, und daher umso billiger bezweifelt werden. Indes waren wir, trotz [...] wiederholten Versuchen, doch auch nicht im Stande, eine Spur jenes Gefüges an den uns zu Gebote stebend- [...] Stücken vom Kap'schen und dem Peruanischen Eisen zum Vorschein zu bringen, obgleich dieselben [...] ganz verlässlichen Händen erhalten worden sind --- so dass über deren Echtheit hinsichtlich ihrer Herstammu- [...] -ein Zweifel Statt finden kann --- und da doch über deren unbezweifelbar meteorischen Ursprung --- für welchen selbst das andere als entscheidend betrachtete Kriterium, nämlich der Gehalt an Nickel, und zwar in einem ganz ähnlichen quantitativen Verhältnisse, und die meisten übrigen physischen und chemischen Eigenschaften, Bürgschaft zu leisten scheinen --- vorlängst abgesprochen ist. Es frägt sich demnach noch, ob das Erscheinen dieses Gefüges als ein unbedingtes und beständiges Merkmal des meteorischen Gediegeneisens zu betrachten sei; und beinahe ebenso sehr steht es in Frage, ob es denselben, wenigstens strenggenommen, ausschließend zukomme. Denn einerseits lässt sich die Möglichkeit einer ähnlichen Zustands-Modifikation und einer gleichen Tendenz zur Kristallisation, sowie eines ähnlichen Mischungs- und Mengungsverhältnisses mit ähnlichen Stoffen (mit Schwefel zu Eisen- und Magnetkies; mit Kohle zu Stahl und Graphit; mit Silicium, Magnesium, und vielleicht selbst mit Nickel), je nachdem dieses oder jenes als nächste Ursache jener Erscheinung zu Grunde läge, bei terrestrischem und künstlich erzeugtem regulinischen Eisen nicht läugnen, in Gegenteile beweisen ersteres deutliche Anzeigen eines und zwar ganz ähnlichen kristallinischen Gefüges, im Bruche mancher Roheisen-Stücke, letzteres (nur wie es scheint, mit Ausnahme des Nickels zur Zeit noch) die Resultate mehrerer Analysen verschiedener Arten von Roh- und Frischeisen-Massen (man sehe was hierüber Herr Professor Hausmann in dem gehaltreichen Aufsätze --- \emph{Specimen Crystallographiae metallurgicae} --- vorgelesen im Mai 1818 in der königl. Gesellschaft der Wissenschaften zu Göttingen, und abgedruckt in den neuern Schriften derselben, Bd. 4, 1820, in beiden Beziehungen vorgebracht hat), andererseits zeigt beinahe jedes künstliche Roheisen (so wie namentlich auch das Cillier, des Fundortes wegen für problematisch angesehene, metallische Eisen) eine, obgleich nur entfernt ähnliche, und keineswegs so regelmäßige Figurierung, und zwar stets und in mannigfaltig abweichenden Modifikationen, die sich auch nur schwach, bloß oberflächlich und gewöhnlich sowohl nach dem Schliffe als nach der feinen Politur, durch Ätzung aber (unsern Erfahrungen nach) keinesweges vollkommener und \emph{en basrelief} (wie auch Daniells Versuche lehren --- mit deren Resultaten man übrigens die unsrer Ätzungsversuche mit dem Meteor-Eisen verwechselt zu haben scheint --- wohin wohl auch das, durch eine ähnliche Prozedur bewirkte und auf gleichem Prinzipe beruhende, Moirieren des verzinnten Bleches zu zählen sein dürfte) ausspricht: inzwischen hat doch Gillet de Laumont, seiner Versicherung nach, an einem Stücke durch Kunst geschmolzenen, reinen, regulinischen Eisen, von besonders deutlich blätterigem Gefüge (\emph{en grand lames}), tiefe, glänzende Streifen (\emph{des stries profondes}), die sich sogar ebenso (?) und zwar unter gleichen Winkeln, wie am Elbogner Eisen, durchkreuzten, durch Ätzung erhalten.\\
Nichts desto weniger dürfte denn doch das Erscheinen jenes Gefüges von der Art und Beschaffenheit, wie es sich am Agramer Eisen, als Prototyp, und diesem ganz ähnlich, und mit nur sehr unbedeutenden Abweichungen bei der Böhmischen, Karpatischen und Mexikanischen derben Eisenmasse zeigt, für das Meteor-Eisen charakteristisch, und demselben ausschließlich eigentümlich sein, so wie dasselbe auf ein Mischungs- und Mengungsverhältnis, auf eine Vereinigung und Absonderung von Bestand- und Gemengteilen nach einem bestimmten Affinitäts- und Kristallisationsgesetze, und auf einen Prozess hinzudeuten scheint, auf welche wir von nichts ganz ähnlichem, auf unsern Planeten vorkommenden, nach Analogie schließen können.}

Die neunte Tafel zeigt nun einen solchen unmittelbaren Abdruck von einer großen, auf den gehörigen Grad geätzten Fläche an der Eisenmasse von Elbogen, die ich ihres autographischen Vorzuges wegen, und da sie das zusammen gesetzteste Gefüge zeigt, nach welchem sich jenes der übrigen Massen am besten vergleichend beschreiben lässt, als Norm wähle, obgleich dieses Vorrecht, an sich und der Folgerungen wegen, der Agramer Masse, als Prototyp, gebührte.\footnote{Es war nicht möglich, von dieser und den übrigen Gediegeneisen-Massen, ähnliche, zur Bekanntmachung geeignete autographische Darstellungen ihres Gefüges auf der Stelle zu bewerkstelligen, indem die Zustandebringung viele Zeit raubend mechanische Vorarbeiten und Vorkehrungen notwendig gemacht hätte. Sie sollen für eine künftige Veranlassung vorbereitet werden. Vorläufig finden sich von denselben auf der achten Tafel mit möglichster Genauigkeit aus freier Hand lithographisch nach der Natur gefertigte Kopien.}

Bei Betrachtung dieses Abdruckes fallen nun auf den ersten Blick oben erwähnte Streifen auf, welche, da sie auf der geätzten Fläche die tiefsten Stellen ausmachen, hier unabgedruckt und weiß, und nur durch ihre Begrenzung --- durch jene erhabenen Einfassungslinien --- bezeichnet erscheinen, insofern nicht einige zart erhaben punktiert, gestrichelt oder gestreift vorkommen. Da sich diese Streifen häufig durchschneiden, durchkreuzen, und folglich sich wechselseitig und hinsichtlich ihrer Verteilung sehr unregelmäßig unterbrechen, so erscheinen sie von sehr verschiedener Ausdehnung in der Länge, und zwar hier von einer halben bis zu sieben Linien, und beinahe in allen denkbaren Zwischenmaßen; dagegen zeigen sie nur wenig Verschiedenheit in der Breite, die nur zwischen 1/4 und 1/2 Linie abweicht, und nur bei einzelnen wenigen 3/4 oder eine ganze Linie beträgt. Bey etwas genauerer Betrachtung findet man bald, dass diese Streifen regelmäßig und genau, aber ungleich an Menge und ganz unordentlich in der Aufeinanderfolge, einer drei- und zum Teil einer vierfachen Richtung folgen; dass die nach einer Richtung gehenden unter sich einen vollkommenen Parallelismus beobachten, und dass sie sich nach diesen verschiedenen Richtungen regelmäßig und unter bestimmten Winkeln durchschneiden oder unterbrechen. Die eine dieser Richtungen geht (nach der Lage der Fläche, in welcher dieselbe hier vorgestellt ist --- mit dem schmälern Teile nach oben ---) vollkommen senkrecht. Die Streifen welche ihr folgen, scheinen von allen übrigen am häufigsten und am gleichförmigsten verteilt vorzukommen, sind auch unter sich am gleichförmigsten, die schmälsten, zartesten, und am schärfsten gerandet oder begrenzt. Die andere Richtung geht schief von der Rechten zur Linken abwärts, so dass die Streifen --- welche im Ganzen minder zahlreich, ziemlich gleichförmig verteilt, aber ungleichförmiger unter sich, meistens länger und etwas breiter (so dass an Masse im Ganzen das ersetzt wird, was etwa an Menge gegen erstere gebrechen möchte), und nicht so schnurscharf gerandet sind --- die ersteren meistens unter einem Winkel von 60° (nur selten unter einem merklich davon abweichenden und dann doch immer zwischen 56 und 65 fallenden Winkel) durchschneiden. Die dritte Richtung geht jener entgegen gesetzt, schief von der Linken zur Rechten abwärts, und die derselben folgenden Streifen sind noch weniger zahlreich selbst als letztere, dagegen meistens bedeutend länger, und im Durchschnitt auffallend breiter (so dass sich das Verhältnis der Masse gegen jene wieder auszugleichen scheint), viel ungleichförmiger verteilt, noch weit ungleichförmiger unter sich, weniger scharf und sehr ungleich begrenzt --- so dass sie in ihrem Verlaufe nicht selten [...] breit, hie und da bauchig und geschweift erscheinen --- und sie durchschneiden die Streifen der se- [...] Richtung sowohl, als die der andern schiefen, unter ganz ähnlichen Winkeln wie diese jene, so dass durch ihre wechselseitige Durchkreuzung Dreiecke gebildet werden, die teils, und zwar meistens, vollkommen gleichseitig, teils gleichschenklich (wo zwei Winkel gleich sind, z. B. = 62 zum dritten = 56°), teils, obgleich nur selten, ganz ungleichseitig sind (z. B. mit Winkeln = 56, 60 und 64°). Außer diesen zeigen sich ähnliche Streifen, aber in ungleich geringerer Menge, meistens partienweise von 3, 4 bis 8 und 9 zusammen gereiht, dicht aneinander, und sehr ungleichförmig verteilt. Diese sind höchst ungleichförmig unter sich, bald kurz, bald lang, von 1 bis 6, und selbst von 9 Linien Länge, aber bedeutend breiter als alle vorigen, von 1/4 bis zu einer vollen Linie, im Verlaufe übrigens oft sehr abweichender Breite, und meistens sehr ungleichförmig begrenzt, so dass ihre Ränder oft sehr ausgeschweift und gebogen erscheinen. Ihre Richtung geht (bei obiger Lage der Fläche) schief von der Linken zur Rechten abwärts, also gleich jener der Streifen der dritten Richtung, aber nicht parallel mit dieser, sondern unter einem Winkel von beiläufig 27° mit derselben sich kreuzend, und demnach die Streifen der beiden übrigen Richtungen unter andern Winkeln als diese durchschneidend, woraus nun wieder mehr oder weniger ungleichseitige Dreiecke, und zwar von dreierlei Art erwachsen, die aber nicht zahlreich vorkommen, da der Streifen dieser Richtung verhältnismäßig nur wenige, und diese meistens partienweise zusammen gehäuft sind.\footnote{Um sich eine deutliche Ansicht und eine leichte Unterscheidung dieser verschiedenen Streifen nach ihrem meist schnurgeraden, aber oft unterbrochenen Laufe, von den verschiedenen Richtungen welche sie verfolgen, von dem Parallelismus den sie hierin halten, und von ihren häufigen Durchkreuzungen, zu verschaffen; tut man am besten, wenn man alle Streifen einer jeden Richtung, ihrem ganzen Verlaufe nach, mittelst eines Lineals mit verschieden gefärbten Zeichenstiften (Pastel- oder Wachs-Crayons) überfährt; so wie, um sich eine möglichst genaue Vorstellung von der Form der Dreiecke und der Beschaffenheit der Winkel zu verschaffen, wenn man einige dieser solcher Gestalt gefärbten Streifen über den Abdruck hinauszieht, und so weit verlängert, bis sich alle, ihrer Richtung nach entgegen gesetzten, außerhalb des Abdruckes wechselseitig durchkreuzen. Man erhält solcher Gestalt, und zwar nach einem beliebig großen Maßstabe, viererlei Dreiecke; nämlich: aus der Durchkreuzung der drei ersteren, regelmäßigen und fast ganz beständigen Richtungen, ein meistens mehr oder weniger vollkommen gleichseitiges Dreieck mit Winkeln von 60° (und wenn man will und mit Präzision verfährt, auch alle kleinen Abweichungen davon, die sich jedoch ziemlich auf Dreiecke mit Winkeln von 62, 62 und 56°, oder 60, 64 und 56° beschränken), und dann aus der Durchkreuzung der Streifen der vierten unregelmäßigern Richtung mit je zwei und zwei der vorher gehenden, dreierlei mehr oder weniger ungleichseitige und ungleichschenkliche Dreiecke (meistens mit Winkeln = 95, 60, 25 oder 98, 55, 27; ferner = 25, 120, 35 oder 30, 115, 35; endlich 60, 85, 35 oder 65, 76, 39°. --- Abweichungen, die übrigens bei oft mangelhafter Schärfe der Streifen und unmöglich zu erreichender Präzision in der Darstellung und Messung, wohl mehr von der Unvollkommenheit der Bestimmung, als von der Unregelmäßigkeit des Gefüges herrühren möchten). Die Rhomben und Trapezen, die durch einzelne Streifen entstehen, welche, einem der Schenkel jener Dreiecke parallel, diese durchschneiden und Segmente derselben bilden, zeigen dem ursprünglichen Dreiecke entsprechende Winkel und Winkel-Supplemente; demnach bei solcher Durchschneidung vollkommen gleichseitiger Dreiecke --- die hier am häufigsten vorkommen --- ein Winkel-Supplement von 60, folglich Winkeln von 120°, wie sie Gillet de Laumont, Leonhard, Schweigger u. a. bemerkt haben.}

Bei weiterer Betrachtung des Abdruckes bemerkt man ferner häufige, größere und kleinere, sehr ungleichförmig verteilte und unregelmäßig zerstreute, meistens dreieckige, bisweilen aber auch rhomboidale oder trapezoidale (keineswegs aber vollkommen viereckige --- wie zum Teil behauptet wurde --- als welche bei dieser Struktur nicht wohl vorkommen können) Figuren, Felder oder Zwischenräume, welche durch die Durchkreuzung von 3 oder 4 jener Streifen verschiedener Richtungen, oder durch das Zusammenstoßen zweier Dreiecke, gebildet werden, und notwendig gebildet werden müssen, insofern nicht jene Streifen --- was bisweilen der Fall ist --- dicht an einander stoßen, und solcher Gestalt gar keinen, wenigstens keinen dem freien Auge auffallenden, Zwischenraum lassen.

Die Form der Dreiecke und die Beschaffenheit ihrer Winkel entspricht jenen regelmäßigen Richtungen und den oben angegebenen Durchkreuzungspunkten der Streifen, und die der Rhomben und Trapezen jenen Dreiecken, insofern diese durch einzelne, irgendeiner jener Richtungen parallellaufende Streifen wieder durchschnitten, oder in Abschnitte geteilt worden sind. Es erscheinen diese Figuren oder Felder hier nicht nur im Umrisse, indem sie von jenen, ihnen sowohl als den Streifen als gemeinschaftliche Scheidewand dienenden, erhabenen, und folglich im Abdruck erscheinenden Linien begrenzt werden, sondern selbst ihrer Oberfläche nach, obgleich etwas schwächer ausgedruckt, und zwar glatt und gleichförmig, oder mehr oder wenig --- und in diesem Falle etwas stärker ausgedruckt --- mikroskopisch zart punktiert, gestrichelt oder gestreift, und dies zwar in verschiedenen, oft sich durchkreuzenden, aber stets ihren Rändern oder den Einfassungslinien und den angrenzenden Streifen parallel laufenden Richtungen.

Ferner bemerkt man, hie und da zerstreut, zwischen und auch oft mitten in den Streifen, mehr oder minder stark abgedruckte, größere oder kleinere, ganz unregelmäßig und verschieden gestaltete Flecke und Punkte, welche ähnlichen Erhabenheiten der Metall-Masse auf der geätzten Fläche, und jenen bereits erwähnten, mechanisch eingemengten Massen der heterogenen bröcklig-körnigen Substanz entsprechen.

Endlich zeigen sich in diesem Abdrucke ziemlich häufige (wohl zwischen 50 und 60) und dem Anscheine nach ganz unregelmäßig zerstreute, mehr oder weniger fleckartige, oft ziemlich große, 2, 4, 6, 8 bis 12 und 16 Linien lange, und (2-3)/12 und 1/2 bis 2 Linien breite, meistens gegen beide Enden spitz zulaufende Striche, welche die Oberfläche in sehr verschiedenen Richtungen, doch, wie es scheint, nicht ganz und gar unabhängig von jenem regelmäßigen Gefüge (indem doch wenigstens drei Richtungen vorherrschen, nach welchen auch diese Striche einen Parallelismus zeigen, obgleich kaum eine davon mit einer der Streifen koinzidiert), [...] Es erscheinen diese Striche hier größten Teils oder ganz unabgedruckt, und nur im Umrisse durch die begrenzende, abgedruck [...] Umgebung angedeutet --- indem sie beträchtlich tiefen, leeren Rissen entsprechen, die sich, wie bereits oben erwähnt wurde, in [...] Metall-Masse selbst, schon vor der Ätzung der Fläche vorfanden --- und nur zum Teil fleckig oder punktiert, insofern diese noch mit Bröckeln und Körnern obiger heterogener Substanz, die durch den Schnitt und Schliff nicht vollends ausgesprengt wurden, stellenweise ausgefüllt sind.\footnote{Ein besonderer Abdruck von der geschnittenen und polierten Fläche vor der Ätzung, gab ein reines und deutliches Bild dieser, die Gleichförmigkeit und Homogenität der Metall-Masse unterbrechenden Striche, und von deren Beschaffenheit, Verteilung und Richtung.}

Diese verschiedenen Teile in welchen sich das Gefüge durch den Abdruck ausspricht, zeigen sich nun auf der geätzten Metall-Fläche selbst, von folgender Beschaffenheit.

Die nach den vier Richtungen gehenden Streifen erscheinen bei diesem Grade von Ätzung als die tiefsten Stellen (jene Risse ausgenommen, die aber nicht durch die Ätzung zum Vorschein gebracht worden sind), und zwar alle von ganz gleicher Tiefe; die Ränder aber, die im Abdrucke deren Kontur gaben, am erhabensten, als Leisten oder dünne Zwischenwände, durch welche jene unter sich sowohl als von den Figuren oder Feldern geschieden erden, und die deren, nun zum Teil ausgeätzte, Substanz begrenzen und gleichsam einfassen, daher wir sie Einfassungsleisten nennen wollen.

Die vertiefte Oberfläche, oder die rückständige Substanz dieser Streifen, hat ein etwas raues, unter dem Mikroskope gleichsam flachnarbiges oder platt runzlicht-faltiges Ansehen, eine Zinkweiße Farbe, und einen schwachen metallischen, etwas seidenartig schimmernden Glanz; die Leisten dagegen sind vollkommen glatt, und haben eine licht stahlgraue, stark ins Silberweiße ziehende Farbe, und einen sehr starken, spiegelicht metallischen Glanz.

Einige (obgleich hier nur wenige) dieser Streifen erscheinen teils durch einzelne wenige, und dann ziemlich starke, teils aber auch durch sehr viele, dicht an einander gereihete, und dann mehr oder weniger zarte, oft mikroskopisch feine, bisweilen bloß aus zusammen gereiheten Punkten oder kurzen Strichelchen zusammen gesetzte, oft im Verlaufe aussetzende, abgebrochene, erhabene Linien --- die unter sich sowohl als den Einfassungsleisten parallel, aber nicht vollkommen geradlinig, sondern meistens etwas gebogen oder fast wellenförmig verlaufen --- der Länge nach gestreift. Es haben diese Linien, die wir zum Unterschiede Streifungs- --- oder besser, zumal sie eine entsprechende Wirkung hervor bringen --- Schraffierungsleisten nennen wollen, gleiche Höhe mit den Einfassungsleisten (daher sie auch im Abdrucke erscheinen), mit welchen sie selbst ihrer Substanz nach von ganz gleicher Beschaffenheit zu sein scheinen, wie sie denn auch dieselbe Bestimmung haben, indem sie ähnliche Streifen begrenzen, nur dass diese oft so mikroskopisch zart sind, dass jene Leisten sich fast berühren.

Die Felder oder Figuren, welche zwischen jenen Streifen liegen --- durch deren Zusammenstoßen und Durchkreuzen sie gebildet werden -- erscheinen zwar ebenfalls tiefer als die Einfassungsleisten --- die zwischen ihnen und den Streifen gleichsam die gemeinschaftliche Scheidewand bilden, und daher im Abdrucke auch zugleich die Form und Begrenzung jener bezeichnen --- aber bei weitem nicht so tief geätzt wie die Streifen, wie sich denn auch ihre Oberflache, zumal wenn diese rau oder gestreift ist, bei einem gewissen Grade von Ätzung, obgleich schwächer als die Einfassungsleisten, abdruckt.

Es haben diese Felder eine eisengraue Farbe, ein ganz mattes metallisches Ansehen, und teils eine glatte, teils aber, und zwar durchaus oder nur zum Teil, meistens gegen die Winkel zu, eine raue, mikroskopisch fein gekörnte Oberfläche; sehr viele aber haben dieselbe ganz, oder zum Teil, zart erhaben gestreift. Diese Streifung (Schraffierung) wird, so wie vorhin bei den Streifen bemerkt wurde, durch ganz ähnliche, aber gewöhnlich äußerst zarte und mikroskopisch feine, mehr oder weniger, doch meistens sehr dicht aneinander gereihete, erhabene Linien oder Leisten hervorgebracht, die, bei ihrer Menge und Zartheit, mittelst ihres Glanzes diesen Feldern oft einen seidenartigen Schimmer geben. Es laufen diese Schraffierungsleisten aber auf den einzelnen Feldern nur höchst selten bloß nach einer Richtung (wie dies bei den Streifen der Fall ist), sondern gewöhnlich erscheinen sie partienweise, und zwar parallel unter sich sowohl als mit ebenso vielen Seitenrändern, nach zwei oder drei Richtungen, die sich im Kleinen ebenso und unter ähnlichen Winkeln durchschneiden und durchkreuzen wie die Streifen im Großen (daher eine wahre Schraffierung bewirken). Sehr oft sind diese Leisten nicht nur einzeln oder partienweise solcher Gestalt unterbrochen, sondern sie selbst setzen oft aus, und lassen einen glatten Zwischenraum, oder erscheinen bloß als in eine Linie gereihte Punkte oder Strichelchen. Beinahe jedes Feld hat seine eigentümliche Schraffierung, ohne Bezug auf die nächstliegenden. Jene vertieften Streifen scheinen eine vollkommene Trennung oder Isolierung zwischen denselben zu bewirken. Es scheint dieselbe übrigens von den Rändern der Felder oder von den Einfassungsleisten her ausgegangen zu sein, wenigstens zeigen sich hier immer die meisten Leisten, auch wenn sich im Mittel oft gar keine finden und sie selbst nicht weit hinein reichen, sondern als abgebrochene Strichelchen an einem der Ränder erscheinen; inzwischen zeigt sich doch auch oft im Mittel eines Feldes die Streifung fleckweise unterbrochen; so dass z. B. mitten in einer Partie senkrecht laufender Leisten ein Fleck von ganz unregelmäßiger Form von solchen einer schiefen Richtung vorkommt. In manchen Feldern erscheint die Streifung nur in Gestalt zarter, mikroskopisch feiner, mehr oder weniger dicht und anscheinend ganz unordentlich zerstreuter, noch gar nicht in parallele Linien und nach einer bestimmten Richtung gereiheter, erhabener Punkte.\footnote{Um eine deutliche Vorstellung von der merkwürdigen Beschaffenheit der Oberfläche dieser Felder zu verschaffen, ist eine stark vergrößerte Darstellung mehrerer derselben durchaus notwendig, welche nebenher in dieser Zwischenzeit mit der gehörigen Genauigkeit zu Stande zu bringen ich nicht vermochte.} Die glatten Felder erscheinen etwas tiefer geätzt, zumal aber ist ihr Mittel bisweilen grubenartig vertieft, gleichsam eingesunken, indes sich der Rand allmählich gegen die Einfassungsleisten zu erhebt.

Die im Abdrucke bemerkten größeren und kleineren, unregelmäßig gestalteten und zerstreut in und zwischen den Streifen erscheinenden Flecke und Punkte, zeigen sich hier als erhabene Massen, und zwar größten Teils von gleicher Höhe mit den Einfassungsleisten, mitunter aber auch etwas tiefer, und daher und überhaupt bei näherer Betrachtung der Oberfläche noch ungleich häufiger als im Abdrucke, so dass die Masse ganz damit durchsäet erscheint, aber in allzu zarten Körnern, als dass sie, oft ihrer Erhabenheit ungeachtet, durch den Abdruck bemerkbar werden konnten. Die Substanz derselben zeichnet sich von der übrigen Metall-Masse durch ein bröcklig-körniges, oder doch rissiges Aussehen, eine matte, dunkeleisengraue, im Schliffe aber hier stark und beinahe ganz rein ins Silberweiße fallende Farbe und starkem spiegelnden Glanze aus.

Ähnliche, aber meistens mehr vertiefte, und daher im Abdrucke nur im Umrisse und undeutlich erscheinende, und größten Teils rundliche oder ovale Flecke von verschiedener, zum Teil bedeutender Größe (von 1/4 bis über 2 Linien im stärksten Durchmesser), zeigen sich ziemlich häufig und ganz unordentlich zerstreut, aber scharf begrenzt, zwischen den Streifen und Feldern gleichsam wie eingeknetete oder eingekeilte Massen oder Körner von matter, schwärzlich eisengrauer, durch den Schliff nur wenig veränderter Farbe, glatter Oberfläche und einem Ansehen, das zwischen jenem der Substanz der Felder und jener bröcklig-körnigen gleichsam das Mittel hält.

Die beim Abdrucke erwähnten fleckartigen Striche erscheinen hier als wahre Risse und enge Klüfte, die zum Teil ziemlich tief (oft über eine Linie), teils senkrecht, teils schief in die Masse eindringen, und die schon ursprünglich vorhanden waren und nicht erst durch die Ätzung hervor gebracht worden sind; dagegen ist wohl durch den Schnitt und Schliff der Fläche die ursprünglich in denselben enthalten gewesene, bröcklig-körnige Substanz --- die mit jener in einzelnen Körnern zerstreut eingesprengten von ganz gleicher Beschaffenheit ist --- vermöge ihrer Sprödigkeit und bröckligen Anhäufung, mehr oder weniger ausgesprengt worden, und die Risse erscheinen daher stellenweise leer und im Abdrucke demnach bloß nach ihrem, von den angrenzenden erhabenen Teilen bestimmten Umrisse, oder nur fleckweise ausgedruckt.

Eine auf der achten Tafel gegebene, mit möglichster Genauigkeit aus freier Hand lithographisch nach der Natur kopierte Darstellung eines auf ähnliche Art und in einem gleichen --- zum Abdrucke geeigneten --- Grade geätzten Plättchens von der Agramer Eisenmasse, zeigt ein ganz ähnliches Gefüge, nur mit folgenden kleinen Abweichungen.\footnote{Die Beschreiung ist teils von diesem Plättchen, teils von einer auf der Masse selbst geätzten Fläche (deren oben bei Beschreibung der Masse Erwähnung gemacht wurde), von 6 Quadrat-Zoll Ausdehnung, genommen.}

Die Streifen zeigen sich nämlich hier nur nach drei Richtungen, und zwar in den drei regelmäßigeren, nach welchen sie vollkommen parallel verlaufen, und zwar so, dass sie sich unter Winkeln von beiläufig 56, 50 und 74° kreuzen; die der vierten Richtung fehlen ganz und gar, und es finden sich demnach, als durch sie gebildete Zwischenfelder oder Figuren, nur einerlei, und zwar mit äußerst wenig Abweichung, ungleichschenkliche Dreiecke, und, aus deren Verbindung und Durchschneidung, Rhomben und Trapezen, ebenfalls von wenig Abweichung und mit leicht zu bestimmenden, jenen obiger Dreiecke entsprechenden Winkeln. Die Zeichnung erscheint solcher Gestalt viel einfacher, gleichförmiger, und zum Teil regelmäßiger, als bei der Elbogner Masse.

Die Streifen selbst, die im Ganzen jedoch merklich minder zahlreich, dagegen aber etwas stärker und breiter als an jener Masse vorkommen --- daher das ganze Gefüge ein etwas gröberes Ansehen hat --- sind übrigens ebenso ungleichförmig verteilt, und die einer Richtung auf ähnliche Art partienweise zusammengehäuft, und nach diesen Richtungen, mit auffallender Übereinstimmung, ebenso an Menge und Masse abweichend, wie an jener; auch durchschneiden und unterbrechen sie sich in einem ähnlichen Grade, und erscheinen demnach im Ganzen von ähnlicher Länge, nur, wie bemerkt, im Durchschnitte von etwas stärkerer Breite --- doch so, dass die breitesten kaum 1/2 Linie erreichen --- und mit einer ähnlichen und übereinstimmenden Abweichung in derselben nach der verschiedenen Richtung, zeigen aber nach beiden Dimensionen etwas mehr Gleichförmigkeit.

Auf der geätzten Fläche selbst zeigen diese Streifen eine etwas minder raue und narbige oder faltige, bisweilen sogar eine ganz glatte Oberfläche, eine mehr ins Silberweiße fallende Farbe, dagegen etwas weniger Glanz als die der Elbogner Masse, erscheinen aber häufiger übrigens ganz auf ähnliche Art schraffiert, und die erhabenen Ränder oder Einfassungsleisten weniger silberweiß, mehr stahlgrau, und etwas schwächer glänzend.

Die Zwischenfelder oder Figuren haben hier eine etwas dunklere, mehr schwärzlich-graue Farbe, sonst dasselbe Ansehen und dieselbe Beschaffenheit wie jene der Elbogner Masse, nur dass sie im Durchschnitte seltener und meistens nur teilweise, gewöhnlich auch bloß nach einer Richtung --- einer Einfassungslinie parallel --- gestreift, dagegen häufiger rau und zart gekörnt und nur selten ganz glatt vorkommen, daher auch die meisten nicht bloß im Umrisse, sondern mit ihrer ganzen Oberfläche im Abdrucke ausgedruckt erscheinen. Merkwürdig ist, dass einige, zumal kleinere, solche Felder ebenso erhaben, glatt und glänzend wie die Einfassungsleisten, von ganz gleichem Ansehen und gleicher Beschaffenheit, und gleichsam mit denselben zusammen geflossen erscheinen, als wenn ihre Substanz in diese übergegangen wäre.

Flecke und Punkte von der bröcklig-körnigen Substanz in den Streifen zeigen sich, sowohl im Abdrucke als auf der geätzten Fläche, im Ganzen nur sehr wenige, und ebenso finden sich auch wenigere eigentliche Risse, dagegen mehr fleckartige, sehr unregelmäßig und unordentlich zerstreute, zum Teil ziemlich große, mehr oder minder mit solcher Substanz --- die aber hier eine mehr Zinkweiße und etwas, teils ins Messinggelbe, teils ins Rötliche fallende Farbe hat --- ausgefüllte Klüfte.

Von der besonderen, in rundlichen Massen gleichsam eingekeilten metallischen Substanz, findet sich hier keine deutliche Anzeige.

Auf derselben Tafel findet sich eine auf ähnliche Art versuchte Darstellung einer ebenso geätzten Platte von der Eisenmasse von Lénarto, welche in Vergleichung mit beiden vorigen folgende Abweichungen im Einzelnen des Gefüges zeigt.\footnote{Auch diese Beschreibung ist nicht bloß nach der vorgestellten Platte, sondern nach noch zwei, in verschiedenem Grade geätzten Flächen, von 12 Quadrat-Zoll Ausdehnung, an großen Stücken von dieser Masse abgefasst.}

Die Streifen erscheinen hier ebenfalls nur nach drei Richtungen, die sich aber unter ganz andern Winkeln, nämlich meistens und mit kaum merklichen Abweichungen von beiläufig 77, 77 und 26° kreuzen, und daher gleichschenkliche, aber lang gezogene und scharf zugespitzte Dreiecke, und diesen entsprechende rhomboidale und trapezoidale Segmente zu Zwischenfeldern haben. Die Zeichnung ist demnach ebenfalls einfacher und gleichförmiger, und selbst noch mehr als an der Agramer Masse, da die Anzahl der Streifen im Ganzen noch bedeutend geringer ist und diese noch weit seltener durch Risse und Klüfte unterbrochen werden.

Die Streifen selbst, da sie im Ganzen ungleich weniger zahlreich sind, durchschneiden sich weit seltener, sind demnach um so länger, so dass die meisten von 6 bis 7, viele selbst von 12 bis 15 Linien Länge erscheinen; inzwischen finden sich doch auch viele 3/4, 2 bis 4 Linien lang. Sie haben dabei eine ungleich stärkere Breite als an den beiden vorigen Massen, die meisten zwischen (7 und 9)/12 bis zu 1 3/12 Linie, daher das Gefüge im Ganzen noch ein ungleich gröberes Ansehen hat, als das der Agramer Masse. Sie sind übrigens etwas gleichförmiger verteilt, oder, wenigstens den verschiedenen Richtungen nach, weniger partienweise zusammen gehäuft, dagegen bei weitem weniger scharf begrenzt, und selten geradlinig, sondern meistens bauchig und geschweift und oft wie ausgeflossen; so dass viele der kürzeren, bei ihrer Breite, oft als Flecke erscheinen und dadurch die Regelmäßigkeit des Gefüges stören.

Auf der geätzten Fläche haben diese Streifen ein beinahe durchaus ganz glattes, gar nicht narbiges oder faltiges, sondern nur bisweilen ein etwas streifiges Ansehen, eine zinkgrauliche, mehr ins Bläuliche als Weiße ziehende Farbe, und einen etwas stärkeren, und zwar schimmernd seiden-fast atlasartigen, metallischen Glanz. Nur wenige erscheinen gestreift, und diese nur zum Teil und durch einzelne, weit abstehende und abgebrochene Schraffierungsleisten; dagegen finden sich in denselben einzelne Körner und Massen jener bröcklig-körnigen Substanz, von allen Größen und Gestalten, als erhabene Punkte, Flecke, Winkelzüge, Linien, eingewachsen und fest eingeschlossen äußerst häufig, und von licht stahlgrauer, ins Silberweiße fallender Farbe, mit starkem, bei schiefer Richtung, metallisch spiegelndem Glanze. Die Einfassungsleisten haben hier eine etwas matte, stahlgraue Farbe.

Die Zwischenfelder oder Figuren, welche hier ungeachtet der geringeren Anzahl der Streifen, wegen gleichförmigerer Verteilung derselben, verhältnismäßig häufiger und aus denselben Gründen bei weitem größer, eben deshalb aber auch seltener als Dreiecke, mit oben angegebenen Winkelmaßen, sondern meistens in rhomboidalen oder trapezoidalen, oft sehr kleinen, Segmenten derselben erscheinen --- sind beinahe durchgehends, und zwar äußerst zart und dicht, gewöhnlich nach zwei auch drei, den Seiten parallelen Richtungen, und mit all den, oben bei der Elbogner Masse bereits erwähnten, Modifikationen, teilweise, zumal an den Rändern, oder durchaus schraffiert, oder doch durch ebenso zarte mikroskopische Punkte rau. Da jene Schraffierungsleisten und diese Punkte erhaben sind, so erscheinen auch alle diese Felder --- und daher weit mehrere als an beiden vorigen Massen --- nicht bloß im Umrisse (durch die Einfassungsleisten), sondern mehr oder weniger, ihrer ganzen Oberfläche nach, im Abdrucke ausgedruckt, und da jene Leisten und Punkte eine glänzende, ins Silberweiße fallende Farbe haben, so geben sie ihrer Menge, Zartheit und Dichtheit wegen, der Oberfläche dieser Felder, die an sich matt und dunkel eisengrau wäre, ein ähnliches Ansehen und einen seidenartigen Schimmer, wodurch selbst die ganze Fläche ein lichteres und glänzenderes Aussehen bekommt. Nur einzelne wenige und meist sehr kleine Felder zeigen sich, auch unter dem Mikroskope, ganz glatt, und dann etwas vertieft, wenigstens im Mittel, und von matter, dunkler, selbst schwärzlich-grauer, oft ganz schwarzer Farbe. Größere Klüfte oder Risse, welche mehr oder weniger mit jener bröcklig-körnigen Substanz ausgefüllt wären, finden sich hier beinahe gar nicht; dagegen --- obgleich nicht so häufig wie im Elbogner Eisen, dafür aber in größeren Partien (von 4 bis 5 Linien im Durchmesser) --- jene dichte, harte, schwärzlich-eisengraue metallische Substanz in rundlichten oder ovalen (hier bisweilen länglichten und linienförmigen) Massen fest eingeknetet, und gleichsam eingekeilt. Merkwürdig ist, dass diese für sich scharf begrenzten Massen (hier wenigstens besonders deutlich) fast durchaus und rings um ihren Rand von einem schmalen, aber ungleich breiten Saume von jener körnig-bröckligen Substanz, von gewöhnlicher Beschaffenheit, Farbe und Glanz, umgeben, eingefasst und durch denselben von der übrigen Metall-Masse fast vollkommen geschieden sind.\footnote{An dem großen, bei 37 Pfund wiegenden Stücke, welches Herr Baron von Brudern von diesem Meteor-Eisen besitzt, scheint eine Masse der Art, gleichsam wie ein an Dicke etwas abnehmender, langer, rundlichter Zapfen, durch die ganze Höhe des Stückes durchzugehen, wenigstens zeigt sich dieselbe auf der einen Abschnittsfläche als ein unvollkommen rundlichter Fleck, von 4 Linien in Durchmesser und vollkommen senkrecht unter demselben auf der entgegen gesetzten Abschnittsfläche, auf mehr als 6 Zoll Tiefe, zeigt sich ein ähnlicher (und hier einziger), etwas ovaler (von 2 1/4 : 3 3/4 Linien in beiden Durchmessern), der jenem vollkommen entspricht, und denselben aufs Haar zentriert. Es wäre denn doch ein ganz besonderer Zufall, wenn sich zwei bloß oberflächliche Flecke oder nicht tief eindringende Massen von derselben Substanz, Beschaffenheit und Form, auf zwei entgegen gesetzten und doch so weit voneinander abstehenden Flächen von beträchtlicher Ausdehnung, und wo sie, wenigstens hinsichtlich ihrer Größe, einzeln stehen, so haarscharf begegnen sollten, ohne miteinander in wirklicher Verbindung zu stehen. De Gegenfläche von jener ersteren Abschnittsfläche befindet sich an dem, 5 3/4 Pfund wiegenden Stücke der kaiserl. Sammlung, das von jenem abgeschnitten worden war, und hier fand sich auch die Fortsetzung jenes präsumierten Zapfens als ein ganz ähnlicher Fleck. In der Hoffnung, dass die Masse auch hier noch wenigstens auf einige Tiefe gehen würde, ließ ich eine 3 Linien dicke Platte, der Fläche horizontal und dicht an einem Rande dieses Fleckes abschneiden, in der Absicht, diese Masse dann aus der Platte herausbrechen und für sich chemisch untersuchen zu machen. Leider ward ich aber in meiner Hoffnung getäuscht, denn die Masse fand sich kaum auf 1/4 Linie tief eingedrungen. Da die andere Abschnittsfläche jenes Stückes von dem in Nation-Museum zu Pesth aufbewahrten, bei 134 Pfund wiegenden Hauptstücke genommen ist, so muss sich dort auf der diesem Abschnitte entsprechenden Fläche die weitere Fortsetzung oder das andere Ende jenes Zapfens finden.}

Dieselbe Tafel gibt ferner eine ähnliche Darstellung einer ebenso geätzten Fläche an dem Stücke vom mexikanischen Gediegeneisen, welches die kaiserl. Sammlung der Mitteilung Klaproths verdankt.

Es zeigt dieselbe ziemlich wesentliche Abweichungen im Einzelnen des Gefüges von den vorhergehenden, und es scheint beinahe als wäre dieses durch irgendeine mechanische Gewalt, etwa beim Lostrennen dieses Stückes von der Stamm-Masse, oder einem größeren Stücke, durch, vielleicht nach einer Richtung fortgesetztes, Meißeln, Hämmern oder Schlagen in etwas verändert worden. Die Streifen erscheinen nämlich beinahe ausschließlich nur nach zwei, und zwar oft ziemlich rechtwinkelig sich durchschneidenden, Richtungen und in diesen selbst nicht immer vollkommen parallel und sogar gekrümmt und gebogen; so dass die Zwischenfelder zum Teil sehr ungleichartige und selbst verzogene, vierseitige Figuren, Parallelepipeden, Rhomben, Rhomboiden, Trapezen, aber nie Dreiecke bilden.

Die Streifen sind übrigens eben so zart und scharf begrenzt, wie bei der Elbogner und Agramer Masse, und da sie ziemlich zahlreich und dabei gleichförmiger als bei jenen verteilt und nicht so partienweise nach einer Richtung zusammen gehäuft sind; so durchschneiden sie sich umso häufiger, erscheinen demnach im Ganzen kürzer, und bilden im Verhältnis häufige, aber kleine Zwischenfelder. Das Gefüge erhält dadurch ein viel feineres und zarteres Ansehen, so wie es auch, da eine Richtung von Streifen beinahe ganz fehlt (denn es zeigen sich nur einzelne wenige, und diese nur undeutlich in einer dritten schiefen Richtung), viel einfacher und gleichförmiger erscheint.

Auf der geätzten Fläche zeigen diese Streifen eine sehr unebene, narbige Oberfläche, äußerst selten eine Spur von Schraffierung, und eine ganz matte, schwärzlich eisengraue, nur hie und da etwas ins Zinkweiße fallende Farbe, so dass sie von den nur etwas weniger vertieften Zwischenfeldern kaum zu unterscheiden sind, die ein ganz ähnliches Ansehen, aber, insofern sie nicht schraffiert sind --- was jedoch ebenfalls nicht häufig und meistens nur zum Teil und nach einer Richtung der Fall ist --- eine glatte Oberfläche haben.

Nur die erhabenen Einfassungsleisten, die Schraffierungsleisten aber nur zum Teil, zeigen, und selbst dieses nur bei einer schiefen Wendung, eine licht stahlgraue Farbe und einen starken metallischen Glanz.

Außer einigen Körnern und kleinen Massen in den Streifen, findet sich von der bröcklig-körnigen Substanz in einzelnen kleinen Rissen und Klüften die Spur, am meisten aber in einer großen rissartigen, ganz damit angefüllten Kluft, die das Stück der Quere nach in einer etwas gebogenen Richtung, aber hier von keiner beträchtlichen Tiefe mehr, beinahe ganz durchzieht.\footnote{Ein diesem am meisten ähnliches Gefüge zeigen die größeren, zu einer Ätzung geeigneten Massen oder Körner von Gediegeneisen, welche sich bisweilen als Gemengteile in der Steinmasse von Meteor-Steinen isoliert eingeschlossen finden, aber dieses nur fleck- oder stellenweise. Es zeigen sich nämlich unter der Lupe auf der gleichförmigen, glatten, matt eisengrauen Oberfläche Stellen welche gestreift erscheinen, und zwar durch erhabene, mikroskopisch zarte, lichtere und etwas glänzende Linien, die größten Teils nach zwei sich durchkreuzenden Richtungen parallel laufen und ein enges Netz mit rhomboidalen und trapezoidalen, vertieften und etwas dunkler gefärbten, matten Zwischenfeldern bilden, wie dies z. B. jenes große Korn in dem auf der siebenten Tafel von der abgeschliffenen Fläche vorgestellten Meteor-Steine von Salés sehr deutlich zeigt, aber der mikroskopisch zarten Beschaffenheit wegen nicht dargestellt werden konnte. (Bemerkenswert ist, dass in diesem abgeschliffenen und geätzten Korne derben, gediegenen Metalle --- nebst Atomen von der bröcklig-körnigen Substanz von ins Rötliche ziehender Farbe --- zwei kleine unförmlich eckige Körner von unveränderter und von der Säure unangegriffener Steinmasse eingekeilt erscheinen.) An jenen kleinen Massen Gediegeneisen, welche in Gestalt wahrer Zacken in der Steinmasse der Meteor-Steine vorkommen, habe ich bisher durch Ätzung keine Spur eines Gefüges oder irgendeiner Heterogenität des Metalle erhalten können, und die Oberfläche derselben zeigte sich stets gleichförmig an Farbe und Glanz, jene war aber lichter und dieser stärker als an jenen größeren, derberen Massen.\\
Es dürfte wohl voreilig scheinen entscheiden zu wollen, welcher von jenen vier Metall-Massen, dem Gefüge nach --- dessen Darstellung und Beschreibung vergleichend gegeneinander zu stellen hier versucht worden ist --- in Hinsicht auf Vollkommenheit oder Vollendung in der Ausbildung, der Vorzug gebühre; inzwischen will ich mir doch erlauben eine Vermutung zu äußern. Das Gefüge der Elbogner Masse zeigt von allen unstreitig den höchsten Grad von Ausscheidung und regelmäßiger Absonderung der einzelnen, mehr oder weniger verschiedenartig erscheinenden Teile desselben, nämlich: die häufigsten, zartesten, gleichförmigsten und am schärfsten begrenzten Streifen; die meiste, und zwar der vorauszusetzenden Grund-Kristallisation --- dem regelmäßigen Oktaeder --- am vollkommensten entsprechende Regelmäßigkeit und Gleichförmigkeit der Zwischenfelder und in deren Schraffierung; die vollkommenste, häufigste und zum Teil selbst etwas regelmäßige Ausscheidung der bröcklig-körnigen, und die nicht minder häufige der ähnlichen, härteren Substanz, so wie die Ausgesprochenheit aller dieser Teile und der Substanzen aus welchen sie gebildet sind, in äußern Ansehen sowohl als in den physischen Eigenschaften, auf deren Eigentümlichkeit und Reinheit hinzudeuten scheint. Das Gefüge aller übrigen zeigt dagegen, und zwar in derselben Reihenfolge in welcher sie dargestellt und beschreiben worden sind, ungleich mehr Einfachheit; aber eben diese Einfachheit hat offenbar ihren Grund in einer minder häufigen und weniger scharfen Absonderung der homogenen Teile, und in einer mangelhaften Ausscheidung der heterogenen Substanzen und Stoffe --- die doch in der Total-Masse vorhanden zu sein scheinen --- welches einerseits in der geringeren Menge, minder scharfen Begrenzung und weniger regelmäßigen --- wenigstens der vorauszusetzenden Grundform im Allgemeinen minder entsprechenden Absonderung jener, und in der Mangelhaftigkeit oder doch geringeren Menge dieser, und in der weniger ausgesprochenen Eigentümlichkeit und Heterogenität der vorhandenen, hinlängliche Bekräftigung finden dürfte. Insofern demnach die Vollkommenheit oder ein höherer Grad von Ausbildung dieser Massen überhaupt, in der häufigeren und vollkommeneren Ausscheidung der heterogenen Bestandteile, und in der schärfern Absonderung und regelmäßigeren Lagerung der aus ihnen einzeln bestehenden, oder aus der neuen Verbindung einiger derselben gebildeten Substanzen zu suchen ist; insofern möchte wohl die Elbogner Masse unter den hier abgehandelten den ersten Anspruch darauf machen dürfen.\\
Lange nachdem diese Note schon niedergeschrieben war, und eben als dieser Bogen der Presse übergeben werden sollte, erhalte ich durch Herrn v. Widmannstätten die Resultate einiger physisch-technischer Versuche, welche derselbe auf meine Bitte mit dem uns so sehr problematisch scheinenden Kap’schen Gediegeneisen, soweit es der Drang der Zeit und der Umstände gestattete, zum Behufe dieser Ausarbeitung noch vorzunehmen die Güte hatte. Es zeigte sich nach denselben an dieser Masse weder im Schliffe, noch beim Anlaufen, noch durch Ätzung, auch nur die entfernteste Spur eines Gefüges.\\
Blank polierte Flächen zeigten denselben metallisch spiegelnden Glanz, dieselbe, das Meteor-Eisen auszeichnende, licht stahlgraue, stark in Silberweiße fallende Farbe, einen hohen Grad von Dichtheit und eine Gleichförmigkeit in dieser, die selbst nicht im Geringsten durch eine heterogene, eingesprengt oder in Rissen enthaltene Substanz unterbrochen erschien, und die sich auch im Schnitte bewhärte, bei welchem jene häufigen, harten, spröden, die Sage verwüstenden Stellen nicht beobachtet wurden.\\
Salpetersäure brachte auf solchen Flächen, und zwar ohne merkliche Entwickelung von Schwefelwasserstoffgas, nur einige größere und kleinere, meistens geflammte und allmählich sich verlaufende, selten etwas schärfer begrenzte, eisen- oder mehr oder weniger schwarzgraue, matte Flecke zum Vorschein, welche auf eine Ungleichartigkeit der Substanz und auf eine unvollkommene Ausscheidung des einen Anteiles schließen ließen. Unter der Feile und Säge zeigte sich die Masse im Ganzen vollkommen und ziemlich gleichförmig geschmeidig, wie gewöhnliches, sehr dichtes und weiches Eisen, aber nicht so weich wie die Streifen-Substanz des Gefüges der beschriebenen Gediegeneisen-Massen. Jene bemerkte Ungleichartigkeit der Substanz sprach sich aber bei Untersuchung einzelner Stücke für sich, die, so viel als bei der unvollkommenen Absonderung jener möglich war, durch mechanische Trennung erhalten wurden, sehr auffallend aus. Möglichst reine Stücke des glänzenden, lichtern Anteiles zeigten einen sehr dichten, glänzenden, weißen Bruch und einen hohen Grad von Geschmeidigkeit, so dass sich ein etwa 45 Gran wiegendes Stückchen sehr gut zu einem beinahe 3 Zoll langen Stäbchen heiß strecken ließ; Stücke vom grauen Anteile dagegen zeigten einen feinkörnigen, matten, schwarzgrauen, sehr schnell bräunlich sich beschlagenden Bruch, und gaben im Feilen einen zwar metallischen, aber grauen Strich, und nur sehr wenige Späne, sondern größten Teils ein schwarzes Pulver. Einzelne Stücken davon hielten in der Rothitze nur einige schwache Hammerschläge aus, und ließen sich damit etwas weniges zusammendrücken, zerbröckelten aber beim dritten, vierten Schlage; andere zersprangen selbst beim ersten Schlage schon. Beide Anteile ließen sich durchaus nicht härten, ersterer schien sich aber --- so viel ein Versuch im Kleinen lehren konnte --- leicht schweißen zu lassen. Beide zeigten starke Wirkung auf den Magnet, aber schwache Polarität, nahmen diese aber durch Streichen bald mehr an, und der erstere erhielt dadurch eine beträchtliche magnetische Kraft.\\
Das spezifische Gewicht der Masse im Ganzen fand Herr v. Widmannstätten = 7,318 (also beträchtlich unter jenem der übrigen von mir darauf untersuchten Gediegeneisen-Massen, bei welchen ich dasselbe, wie ich schon in einer früheren Note bemerkte, zwischen 7,600 und 7,830 fand; nämlich: von der Elbogner = 7,800-7,830; der Agramer = 7,730 bis 7,800; der Lénartoer = 7,720-7,800; der Mexikaner = 7,600-7,670; der Peruaner = 7,600-7,650; und selbst noch unter jenem des sibirischen = 7,540-7,570; aber höher als jenes der in Meteor-Steinen eingemengten Gediegeneisen-Körner = 6,000-6,600; dagegen dem angenommenen Mittelgewichte des Roheisens = 7,200-7,500 am nächsten kommend); jenes des weißen Anteiles für sich, nach dem verschiedenen Grade der Reinheit der Stücke, zwischen 7,633 bis 7,877 (also zum Teil weit über dem angenommenen Mittelgewichte des gewöhnlichen, weichen und geschmeidigen Eisens = 7,700); jenes des grauen dagegen zwischen 6,655 und 6,926 (demnach weit unter jenem des Roheisens), von welchen beiden nun das arithmetische Mittel eine der obigen ganz ähnliche Zahl gibt, und in welchen der Grund der Differenzen des Befundes Anderer zu suchen ist (so gab v. Dankelmann das spezifische Gewicht dieser Masse mit 7,708; Van Marum mit 7,654 an, indes ich es einst = 7,260 gefunden hatte).\\
Unter einem ward die uns nicht minder zweifelhafte peruanische Eisen-Masse nochmals geprüft, soweit es die Kleinheit des zu Gebote stehenden Stückes erlaubte. Auch diese zeigte keine Spur von jenem eigentlichen Gefüge; unter der Lupe erschienen aber doch auf der stark geätzten, kleinnarbigen Oberfläche viele, äußerst zarte, mikroskopisch feine, erhabene Linien, die nach mehreren, offenbar vorherrschend aber nach drei, zumal zwei, Richtungen meist gerade, nur selten etwas gebogen, und stets parallel verlaufend, sich durchkreuzen, und hie und da ein sehr enges Netz bilden, ganz ähnlich jenem auf der geätzten Oberfläche des großen Metall-Kornes in dem oben beschriebenen und auf der siebenten Tafel abgebildeten Meteor-Steine von Salés, und, stellenweise, jenem auf der geätzten Fläche des Mexikaner Eisens. Eine polierte Fläche zeigte eine besonders stark ins Weiße fallende, die geätzte aber eine beinahe zinnweiße Farbe. Unter der Feile gab sich die Masse merklich härter als die Kap'sche zu erkennen. Das spezifische Gewicht fand Herr v. Widmannstätten = 7,646.\\
So sehr nun auch diese Resultate in vielen Beziehungen von jenen abweichen, welche sich bei ähnlicher Untersuchung der übrigen, oben beschriebenen, derben Gediegeneisen-Massen ergaben, und die Zweifel über den präsumtiven meteorischen Ursprung dieser beiden vermehren (so dass selbst Herr v. Widmannstätten mehr geneigt wäre, zumal die Kap'sche, für das Produkt eines künstlichen Schmelz- und unvollkommenen, unvollendeten Verfrischungs-Prozesses anzusehen, welche Mutmaßung durch v. Dankelmanns Nachrichten von der geognostischen Beschaffenheit jener Gegend, wo diese Masse ursprünglich gefunden worden war, und wo Eisenerze aller Art und in großer Menge zu Tage stehen --- welche vielleicht einst von den Bewohnern zu Gute gemacht wurden --- auch von dieser Seite einige Wahrscheinlichkeit erhält); so finde ich doch darin keinen Bestimmungsgrund, meine sowohl in einer früheren Note über das Eigentümliche und Charakteristische des Gefüges am Meteor-Eisen, als in dieser über den verschiedenen Grad von Vollkommenheit desselben und den, diesem wahrscheinlich zum Grunde liegenden Ursachen, ausgesprochenen Ideen und Mutmaßungen abzuändern, auch selbst dann nicht, wenn auch jene Zweifel (gegen welche die ausgezeichnete Farbe, die offenbare und ganz eigentümliche Mengung, und das beträchtliche spezifische Gewicht der Masse, im Ganzen sowohl als insbesondere des einen Anteiles, vorzüglich aber der erwiesene, selbst quantitativ entsprechende Gehalt an Nickel in Berücksichtigung kommen) für nicht hinlänglich begründet erachtet werden sollten; im Gegenteile dürfte ich darin vielmehr in jedem Falle einige Bekräftigung für dieselben zu finden glauben.}

Wird nun die Ätzung solcher Flächen noch längere Zeit (z. B. bis auf eine Tiefe von 1/4 Linie) fortgesetzt; so sprechen sich die erhabenen und vertieften Stellen gegenseitig noch immer mehr aus, und es verändert sich zum Teil das Ansehen des Gefüges oder der Zeichnung im Ganzen, indem zuletzt manche der tiefen --- namentlich die Streifen --- ganz verschwinden, und bisweilen andere --- gewöhnlich ein Teil eines Zwischenfeldes --- an ihre Stelle treten.

Die Streifen erscheinen nun als seichtere und tiefere, mehr oder weniger leere Kanäle oder Rinnen (Geleise), indem oft nur die erhabenen Begrenzungslinien oder die Einfassungsleisten als ihre gemeinschaftlichen Scheidewände, als Kontur, und ihr Boden, der mit diesen von einerlei Beschaffenheit ist und mit denselben ein Ganzes, gleichsam eine Rinne bildet, welche die Substanz der Streifen selbst einschloss, noch übrig sind, letztere aber von manchen ganz ausgeätzt und von der Säure aufgelöst worden ist. Hat man demnach ein Plättchen von einer bestimmten Dicke (z. B. von 1/4 oder 1/2 Linie) einem solchen Versuche unterzogen, so erscheinen manche dieser Streifen --- insofern sie tiefer gingen als die Platte dick war, und ihre Substanz solcher Gestalt zufällig, und zwar noch über den Boden des Canals getroffen worden ist --- ganz ausgeätzt, ohne Boden, und die Einfassungsleisten stehen als die ehemaligen Scheidewände, wie Lamellen, frei da, und hängen nur mittelst ihrer Enden mit den übrigen minder angegriffenen Teilen (den angrenzenden, quer gehenden, ähnlichen Leisten, oder mit solchen von Feldern) der Masse zusammen.\footnote{Wenn zufällig --- was jedoch selten der Fall ist --- aller Zusammenhang fehlt, so fallen solche Lamellen, einzeln oder mehrere unter sich oder mit einem Zwischenfelde verbunden, ab und finden sich als solche --- unverändert in der Auflösung --- am Boden des Gefäßes worin die Ätzung geschah.} Manche, und zwar ungleich mehrere (zumal wenn die Platte 1/2 Linie dick war, da nur äußerst wenige so tief gehen), die seichter lagen, erscheinen als leere Rinnen von verschiedener Tiefe, und auf der entgegen gesetzten Seite des Plättchens (die, falls dasselbe von beiden Seiten gleichzeitig geätzt wurde, eine ganz andere Zeichnung und Verteilung der Streifen und Felder zeigt) finden sich unter ihnen Felder, oder zum Teil Streifen in einer andern Richtung, auf welchen sie mit ihrem Boden auflagen, der nun --- falls er nicht etwa wegen allzu seichter Lage der enthaltenen Substanz und der langen Dauer des Prozesses ebenfalls weggeätzt wurde -- mit den Einfassungsleisten auf denselben aufsitzt.\footnote{Es ergibt sich hieraus, dass sowohl die Streifen als Felder, so wie auch ihre gemeinschaftlichen Scheidewände --- die Einfassungs- und Schraffierungsleisten --- und überhaupt alle einzelnen, mehr oder weniger verschiedenartigen Teile des Gefüges die sich durch die Ätzung aussprechen, nur auf eine gewisse, und zwar nicht sehr beträchtliche (wie es scheint, selten über 1/2 Linie reichende), übrigens aber sehr unbestimmte und ungleichförmige Tiefe gehen und unordentlich über einander gehäuft, nur mit einiger Regelmäßigkeit in der Ausscheidung, Absonderung und Lagerung unter sich, die Total-Masse konstruieren. Am deutlichsten zeigt sich dies an zwei Würfeln (von beiläufig 4 Linien Seite), die ich aus einem Stücke von der Elbogner Masse ausschneiden und woran ich an dem einen noch eine Ecke abstumpfen ließ, und dann beide ringsherum, auf allen Flächen und Kanten, gleichzeitig und gleichförmig ätzte. Jede Fläche auf denselben zeigt nun eine andere Zeichnung oder Gruppierung der Streifen und Felder, und manche von diesen oder jenen, oft von beiden, so auch die mit der bröcklig-körnigen Substanz mehr oder weniger ausgefüllten Risse und Klüfte, setzen sich über die gemeinschaftliche Kante auf die nächstanstoßende Fläche mehr oder weniger weit fort, je nachdem sie gerade an diesen Stellen tiefer oder seichter in die Masse eingedrungen waren.}

Die Einfassungsleisten, die nun mehr oder weniger frei da stehen, zeigen --- was zum Teil bei einer minder tiefen Ätzung schon beobachtet werden kann --- (übrigens aber auch vom Schnitte abhängen mag, je nachdem dieser die Richtung derselben traf) eine, gegen die Ebene der Fläche, etwas schiefe --- doch immer unter sich parallele --- Lage, und gleichen papierblattdünnen Lamellen von der Länge der vormaligen Streifen, einem gegenseitigen Abstande welcher der Breite dieser entspricht, und von sehr ungleicher Höhe oder Tiefe, je nachdem die enthaltene Substanz mehr oder weniger in die Tiefe ging und ausgeätzt worden ist. Kurz, sie bilden paarweise die Seitenwände eines schrägen, aber gleich weiten Canals, in welchem die Streifen-Substanz eingeschlossen war, und sind nach unten durch eine ähnliche Lamelle verbunden und geschlossen, welche solcher Gestalt den Boden des Canals vorstellt. Boden und Wände haben ein etwas unebenes, gebogenes und gleichsam faltiges, oder vielmehr breit und flach gefaltetes Ansehen, eine stahlgraue, stark ins Silberweiße fallende aber meistens eisengrau angelaufene Farbe, und einen schwachen metallischen Glanz, indes der obere Rand der Wände (Einfassungsleisten) lichter und glänzender ist.

Wo zwei oder mehrere Streifen einer Richtung dicht an einander liegen und durch solche Lamellen dem Anscheine nach nur einfach getrennt sind, scheinen diese doch alle Mahl, wenigstens hie und da im Verlaufe, doppelt oder doch dicker und gleichsam aus zwei zusammen geschmolzen zu sein, schließen, in eben diesem Grade mehr oder weniger deutlich, etwas von der den Feldern oder Figuren eigentümlichen Substanz ein, und bilden kleine, oft nur linienförmige (doch immer Segmenten der vorkommenden Dreiecke entsprechende) ähnliche Zwischenfelder; so dass demnach Streifen und Felder und die verschiedenartige Substanz beider stets und regelmäßig abwechseln und jene Lamellen oder Einfassungsleisten gleichsam als Trennungsmittel dienen und die gemeinschaftliche Scheidewand bilden.

Ebenso, wie diese Lamellen, zeigen sich auch die Schraffierungsleisten, in den Streifen sowohl als auf den Figuren --- wie sie denn auch in der Tat und in jeder Beziehung ähnliche Einfassungsleisten, obgleich im mikroskopisch Kleinen, vorstellen --- nur etwas weniges minder erhaben --- zumal die auf den Figuren als die feinsten --- indem sie, vermöge ihrer Zartheit, doch etwas mehr von der Säure angegriffen worden zu sein scheinen.

Auf der Oberfläche der noch rückständigen Streifen-Substanz, oder wo diese ganz ausgeätzt ist, auf dem Boden der Streifen-Kanäle, finden sich hie und da Körner oder kleine Massen von jener, bereits früher bemerkten, bröckligkörnigen Substanz, fest aufsitzend oder gleichsam eingewachsen, und zwar teils ebenso erhaben wie die Leisten, teils tiefer, teils ganz tief, je nachdem sie ursprünglich seichter oder tiefer in der Streifen-Substanz eingeschlossen waren, und je nachdem diese mehr oder weniger ausgeätzt wurde.\footnote{Nach Beendigung eines solchen fortgesetzten Ätzungsversuches finden sich demnach auch auf dem Boden des Gefäßes viele einzelne Körner und Massen von dieser Substanz, die frei wurden, als jene der Streifen, in welche sie eingemengt gewesen, von der Säure aufgelöst worden war, und die --- so wie oben von den unzusammenhängenden Einfassungsleisten bemerkt wurde --- unverändert in der Auflösung zu Boden fielen, da sie wie diese, von der Säure (zumal von Salpetersäure im diluierten Zustande und bei langsamen Gange des Prozesses) wenig oder gar nicht angegriffen werden.}

Die Zwischenfelder oder Figuren erscheinen auch nach dieser starken Ätzung nur wenig minder erhaben als die Leisten, haben aber, abgesehen von der Schraffierung, einiger Maßen ihr Ansehen verändert, und ihre Oberfläche zeigt sich nun eisenschwarz, rau wie bestaubt, und zum Teil äußerst zart und verworren, unvollkommen faserig (beinahe wie feine ausgebrannte Steinkohlen, \emph{cox}) und matt, mit einem licht eisengrauen Schimmer.

Die Beschaffenheit des solcher Gestalt durch Ätzung zum Vorschein gebrachten Gefüges dieser Massen, die verschiedene Art der Absonderung, Lagerung und Gestaltung, und das verschiedenartige äußere Ansehen der dasselbe konstruierenden Teile, lassen demnach eine vier- und zum Teil fünffache Verschiedenheit der Substanzen erkennen und unterscheiden; nämlich: jene der am meisten von der Säure angegriffenen Streifen, die der etwas minder angegriffenen Zwischenfelder, und die der am wenigsten angegriffenen Einfassungs- und Schraffierungsleisten; ferner jene der mechanisch eingemengten, bröcklig-körnigen, ebenfalls nur sehr wenig angegriffenen, und endlich die --- der letzteren am nächsten verwandt scheinenden --- der auf ähnliche Art und gleichsam eingekeilt vorkommenden, auch nur wenig angegriffenen, rundlichen Massen. Und diese Verschiedenheit spricht sich auch noch durch andere physische Merkmahle, insbesondere durch die Härte aus.\footnote{Diesem verschiedenen Grade von Härte und Geschmeidigkeit der verschiedenen, das Gefüge konstituierenden Teile und Substanzen, ist wohl das oben erwähnte oberflächliche Erscheinen desselben nach Schliff und erster Politur einer Fläche zuzuschreiben, so wie der Umstand des Wiederverschwindens bei der feinen Politur dadurch erklärlich wird, dass jene Verschiedenheit der Wirkung des zur letzteren gebrauchten schärferen und feineren Polierpulvers nicht im Wege steht.} Die Substanz der Streifen zeigt sich nämlich (mit einer Stahlnadel geritzt, selbst nur stark gedrückt) am weichsten ( beinahe so weich wie Bley) und am geschmeidigsten, und gibt einen noch lichtern und etwas glänzenderen Strich; jene der glatten (nicht schraffierten) Felder zeigt sich, obgleich nur wenig, doch hinlänglich merklich härter, und gibt einen ähnlichen, zinkgrauen, mehr oder weniger ins Silberweiße fallenden, doch aber minder lichten und minder glänzenden Strich, der ein dunkelgraues Pulver abscheidet, welches wie ein Anflug die Oberfläche bedeckte; die Substanz der Einfassungsleisten endlich ist beträchtlich härter, wenigstens bedeutend zäher, nimmt aber doch den Eindruck an, der Farbe und Glanz unverändert zeigt, und ebenso verhält sich die der Schraffierungsleisten, insofern deren Zartheit den Versuch nicht begünstiget, was inzwischen, zumal auf der Oberfläche der Felder, wo sie oft als mikroskopisch feine Fäden erscheinen, so weit geht, dass sich eine ganze Partie derselben mit einem Striche oder Drucke zerstören lässt. Die bröcklig-körnige Substanz dagegen verhält sich beträchtlich hart und vollkommen spröde, ist etwas schwer zersprengbar, lässt sich aber doch zum feinsten Pulver zerstoßen und zerreiben, das dann schwärzlichgrau erscheint, und durchaus, ohne Ausnahme eines Stäubchens beinahe, dem Magnete folgt; die schwärzlichgrauen rundlichen Massen aber zeigen sich nicht nur derb und dicht, sondern noch härter (doch nicht Funken gebend) und schwerer zersprengbar, und zwar auch spröde, aber mit einiger Zähigkeit, so dass sich die Substanz größten Teils schwer zerstoßen und noch schwerer zerreiben lässt. Geritzt gibt ihre Oberfläche einen ziemlich scharfen, stahlgrauen, etwas ins Rötliche fallenden, schwach metallisch glänzenden Strich,\footnote{Von diesem, zumal in größeren Massen so merklich werdenden, verschiedenen Grade von Härte und Geschmeidigkeit der einzelnen, zum Teil so ungleich verteilten und hie und da partienweise so ungleichförmig angehäuften Gemengteile, insbesondere aber von der so häufig und unregelmäßig durch die ganze Metall-Masse zerstreuten bröcklig-körnigen und der ihr nächst verwandten Substanz, rührt denn auch die Schwierigkeit der technischen Bearbeitung des Meteor-Eisens im Großen und Ganzen, die von mehreren Physikern bereits bemerkte Ungleichförmigkeit in der Schmiede- und Schweißbarkeit desselben (welche Vauquelin verleitete einen geringen Grad von Oxydation dieses Eisens im Allgemeinen anzunehmen), und vorzüglich der Umstand her, dass sich solche Massen äußerst schwer mit der Säge behandeln lassen, so dass man bei mühsam fortgesetzter Arbeit vieler Tage Zeit bedarf um auch nur ein Handgroßes Stück vollends abzusägen, und dabei viele Sägen (aus Uhrfederblättern) zu Schanden arbeitet, indem man häufig auf Stellen stößt die einen großen Widerstand zeigen, die Sägen stumpf machen und die Zähne ausbrechen, indes andere sich beinahe wie Blei schneiden.} der ein schwarzes Pulver abscheidet. Das durch Zerstoßen und Zerreiben erhaltene Pulver ist nicht ganz gleichförmig, ballt sich zusammen und klebt etwas; einzelne Stückchen widerstehen dem Stößel mehr, und zeigen sich minder retractorisch als die übrigen Atome, die fast durchgehends ziemlich stark, aber doch weniger als jene der bröcklig-körnigen Substanz, auf die Nadel wirken.

Und diese Verschiedenheit, im Äußern sowohl als in den physischen Eigenschaften --- insofern letztere ohne mechanische Absonderung und partielle Untersuchung aller dieser Substanzen für sich, was nicht wohl möglich ist, ausgemittelt werden konnten --- vollends aber der so auffallend sich aussprechende verschiedene Grad von Einwirkung des Ätzmittels auf dieselben --- der denn doch von einem verschiedenen Grade von Verwandtschaft derselben zur Säure oder dem Oxygen abhängt,\footnote{Diesen verschiedenen Grad von Verwandtschaft zum Oxygen scheint wohl auch schon das oben erwähnte Resultat des Versuches mit dem Blau-Anlaufen polierter Flächen zu bestätigen. Und von demselben rührt ohne Zweifel jene von selbst entstandene, einer schwachen Ätzung gleichende Andeutung des Gefüges her, die sich durch vertiefte, den Streifen entsprechende Linien und erhabene, in jeder Beziehung mit den Figuren übereinstimmende Zwischenräume, zumal an einer der von dem unvollkommenen rindenartigen Überzuge entblößten Fläche der Masse von Elbogen ausspricht, indem diese --- so viel aus der dunkelen, mit Volkssagen und Märchen verwebten Geschichte über die Herstammung derselben bekannt ist --- durch mehrere Jahrhunderte den Injurien der Zeit und dem Einflüsse der atmosphärischen Luft und selbst des Wassers ausgesetzt war.} scheinen wohl auf eine wesentliche, und zwar auf eine chemische Verschiedenheit hinzudeuten.\footnote{Manche Physiker, welchen die Erscheinung des Gefüges bei dieser Behandlung bekannt wurde, und die eine Mutmaßung darüber äußerten (namentlich Neumann, Schweigger u. a.), glaubten den verschiedenen Grad von Einwirkung der Säuren, wodurch dasselbe zum Vorschein gebracht wird, von dem bekannten Gehalte dieses Eisens an Nickel, und von einer ungleichen Verteilung desselben in jenem --- oder vielmehr von der ungleichen Verteilung aber einer gewissen regelmäßigen Absonderung des damit legierten Anteiles von Gediegeneisen --- herleiten zu können, obgleich der von den Chemikern --- wenigstens anfänglich --- als so höchst unbedeutend angegebene Gehalt der verschiedenen Meteor-Eisen-Massen an jenem Metalle (im Durchschnitt von 1 1/2 bis 3 1/2 Perzent, nach Klaproth wenigstens) eine so auffallende Wirkung kaum erwarten ließ. Jene Vermutung hätte daher durch die Resultate späterer Untersuchungen Stromeyers, nach welchen jener Gehalt beträchtlich höher (nämlich zwischen 10 und 11 Perzent) gestellt wurde, einerseits sehr an Wahrscheinlichkeit gewonnen; allein mit diesem Befunde ergab sich auch, dass dieser Gehalt nicht nur unveränderlich und bei allen Meteor-Eisen-Massen gleich groß, sondern dass derselbe auch stets durch die ganze Masse vollkommen gleichförmig verteilt sei, und solcher Gestalt ließe sich, falls dies im strengsten Sinne zu nehmen wäre, jene Erscheinung durchaus nicht von demselben herleiten --- wie denn auch der --- von Stromeyer selbst ausgewiesene ähnliche --- Gehalt an Nickel im Kap’schen Eisen (es mag dasselbe nun terrestrischen oder wirklich meteorischen Ursprungs sein), bei ähnlicher Behandlung der Masse, sich wirklich gar nicht ausspricht ---; man müsste denn zugeben --- was jenen Resultaten unbeschadet wohl auch der Fall sein kann --- dass bei jenen Massen doch eine, beziehungsweise wenigstens, ungleichförmige Verteilung des Nickels Statt fände. Sollte nämlich der Gehalt an diesem Metalle, wenigstens zum Teil, die verschiedene Einwirkung der Säuren und somit die Erscheinung des Gefüges wirklich veranlassen; so müsste eine teilweise Verbindung desselben mit dem Eisen (folglich eine ungleichförmige Verteilung gegen dieses), und eine --- dem Gefüge entsprechend regelmäßige Absonderung dieses Nickel-Eisens von dem übrigen (womit die Gleichförmigkeit der Verteilung des Nickels gegen die Masse im Ganzen ziemlich wieder hergestellt würde) angenommen werden, und in dieser Voraussetzung könnte man von allen Teilen des regelmäßigen, eigentlichen Gefüges, die Einfassungs- und Schraffierungsleisten am passendsten dafür erkennen, als welche sich als die erhabensten, folglich als die von der Säure am wenigsten angegriffenen, durch eigentümliche Härte und Zähigkeit, Farbe und Glanz (Eigenschaften, welche, \emph{a priori} zu schließen, das Eisen durch eine solche Legierung höchst wahrscheinlich in einem solchen Grade erlangen dürfte) ausgezeichneten Teile, und umso mehr dafür aussprechen, als auch ihre Masse zusammen genommen dem chemisch ausgewiesenen \emph{maximo} des Total-Gehaltes an Nickel, und ihre ziemlich gleichförmige Verteilung in der Gesamtmasse, der Forderung in dieser Beziehung, am meisten entspräche. Die am stärksten angegriffene Substanz der Streifen könnte dann, allen ihren Eigenschaften gemäß, für reines Gediegeneisen gelten; es bliebe demnach nur noch die dritte Substanz übrig, die sich durch das Gefüge und durch ihre Eigenschaften als heterogen von jenen beiden ausspricht, nämlich die der Zwischenfelder oder Figuren. Gillet de Laumont scheint geneigt, diese für gekohltes Eisen anzusehen; und wirklich hat diese Vermutung vieles für sich, zumal da Kohlenstoff als Bestandteil mehrerer Meteor-Steine, und selbst --- nach Tennant -- des Kap’schen Eisens (in welchem derselbe vielleicht, so wie der Nickel und die übrigen Bestandteile, ganz gleichförmig in der Masse verteilt, und nicht bloß mit einem bestimmten Anteile des Gediegeneisens chemisch verbunden und mit diesem ausgeschieden, sich befindet), bereits dargetan ist.\\
Das Gefüge wäre demnach die Folge einer mehr oder weniger vollkommenen Ausscheidung der früher gleichförmig im Ganzen vermischt und gebunden gewesenen Bestandteile des Meteor-Eisens, des Nickels und Kohlenstoffes, einer neuen chemischen Verbindung derselben mit einem Teile des Gediegeneisens und einer mehr oder weniger vollkommen und regelmäßigen Absonderung und Gruppierung der solcher Gestalt verschiedenartigen, neu gebildeten Gemengteile von dem reinen Eisen.\\
Was die beiden andern metallischen Substanzen betrifft, die keine integrierenden Teile jenes Gefüges ausmachen, und die ihre Heterogenität schon durch die Art ihrer Einmengung in die übrige Masse, und durch ihr Äußeres zu erkennen geben; so spricht sich die eine, bröcklig-körnige, durch alle oryktognostischen und physischen Merkmahle deutlich genug als Schwefeleisen, und zwar als Magnetkies aus, und es ist unbegreiflich, wie bei dessen großer Menge --- indem die Massen ganz damit durchsäet sind --- der eine Bestandteil desselben, nämlich der Schwefel, bei den bisherigen Analysen solcher Massen (bis neuerlichst der sibirischen durch Laugier) so ganz aller Wahrnehmung entgangen sein konnte.\\
Die andere, derbere, dichtere, in rundlichen Massen bisweilen eingemengt vorkommende Substanz hat wohl viele Ähnlichkeit mit jener, scheint aber doch wesentlich von ihr verschieden und vielleicht eine unvollkommene Ausscheidung oder ein Rückstand des ursprünglichen Total-Gemisches aller Bestandteile der metallischen Meteor-Masse, also gekohltes und geschwefeltes, und vielleicht auch noch mit Nickel verbundenes Gediegeneisen, und demnach, hier gleichsam als Gemengteil, in demselben Zustande zu sein, in welchem die Masse des Kap’schen Eisens noch im Ganzen sich befindet.\\
Auch diese Note war bereits niedergeschrieben und zum Drucke bereitet --- der nicht länger mehr aufgeschoben werden konnte --- als ich durch Herrn Apotheker Moser die Resultate einer, wie mir däucht, sehr entscheidenden chemischen Untersuchung, welche derselbe auf mein Ansuchen und nach meinen Wünschen in dieser Zwischenzeit vorzunehmen die Güte hatte, mitgeteilt erhielt. Nach einer vorläufigen Analyse eines Stückes von der Elbogner Eisen-Masse im Ganzen, wobei sich --- nach Wollastons Verfahren vorgegangen --- in hundert Teilen ein Gehalt an Nickel von 7,29 ergab, wurden drei Plättchen, welche aus einem ähnlichen, aber größeren Stücke dieser Masse geschnitten worden waren --- jedes beiläufig von einem Zoll im Gevierte, und etwa 3/12 Linie dick, am Gewichte zusammen [...] Gran [...] --- so lange in Salpetersäure gebeitzt, bis die am leichtesten auflösliche Substanz, nämlich die der Streifen, ganz ausgeätzt und aufgelöset war. Es wurde nun das rückständige Gerippe oder Netz von Lamellen (den Einfassungsleisten) und den zum Teil ausgefüllten Zwischenräumen (der Substanz der Figuren oder Zwischenfelder) sowohl als die Flüssigkeit, welche das Ausgeätzte aufgelöst enthielt, beide einzeln für sich, nach gleichem Verfahren untersucht, und es ergab sich bei ersterem ein Gehalt an Nickel von 9,83, in letzterer nur von 4,18 in hundert Teilen. Das arithmetische Mittel von diesen beiden Zahlen gibt nun beinahe ganz genau obige Summe des Gehaltes der Masse im Ganzen. Es ist demnach wohl nicht zu zweifeln, dass wo nicht aller, doch der bei weitem größere Teil des Nickels in dem unauflöslicheren Teile der Masse, und zwar höchst wahrscheinlich in den Lamellen oder Einfassungsleisten enthalten sei; denn da bei diesem Versuche die Beitze über die Gebühr und so lange fortgesetzt wurde, bis selbst ein großer, ja der größte Teil der Figuren oder Zwischenfelder ganz durchgeätzt, und auch deren Substanz aufgelöst worden war, so konnte wohl nur wenig von jenem Nickelgehalte des Gerippes von dieser letzteren herrühren, im Gegenteil ist es weit wahrscheinlicher, dass der in der Flüssigkeit aufgefundene Nickelgehalt von derselben, oder vielmehr von der Substanz der Einfassungs- zumal der Schraffierungsleisten herzuleiten sei, deren gleichzeitige Auflösung, wenn gleich in einem geringeren Grade, schlechterdings immer unvermeidlich ist, bei diesem Versuche aber, der langen Dauer des Prozesses wegen, bedeutend gewesen sein muss. Alle im Obigen geäußerten Vermutungen, hinsichtlich des Nickels und seines Anteiles an der Bildung und Erscheinung des Gefüges, fänden sich somit bewährt, so wie wohl auch jene von der Substanz der Streifen, da sich außer Eisen und Nickel kein anderweitiger Stoff in der Auflösung ausmitteln ließ. Von Kohle oder Grafit wollte sich dagegen bei diesen Haupt- so wie bei mehreren, zum Teil absichtlich darauf vorgenommenen Nebenversuchen durchaus keine Spur finden, und da sich auch von keinem anderweitigen Stoffe, mit Ausnahme des Schwefels, weder von Silicium noch selbst von Chrom (auch nicht von Kobalt), worauf Bedacht genommen wurde, eine Anzeige ergab; so bleibt die Natur der Substanz, welche die Figuren oder Zwischenfelder des Gefüges bildet, zur Zeit noch zweifelhaft. Auf dem Boden des Gefäßes, in welchem die Beitze vorgenommen wurde, fanden sich --- nebst mehreren teils einzelnen, teils zu zwei und drei zusammenhangenden Lamellen, welche von den Ätzungsplättchen wegen Mangel an Verbindung abgefallen waren und die mit zur Untersuchung des Gerippes verwendet wurden --- mehrere unförmliche Stücke und Körner (wovon doch eines eine vollkommene Würfelform zeigte), zusammen von 4,40 Gran am Gewichte, von jener bröcklig-körnigen Substanz, die wir bereits für Schwefeleisen erkannten, als welches sie sich auch durch die Analyse bewährte, und zwar in einem Verhältnisse des Schwefels zum metallischen Eisen wie 0,30 : 4,10; ein Verhältnis das demnach weit unter jenem steht, welches für das terrestrische Schwefeleisen im \emph{minimum} als konstant angenommen wird. Auch jene dichte, härtere Substanz, welche in Gestalt rundlicher Massen eingemengt vorkommt, und namentlich jene oben erwähnte aus dem Lénartoer Eisen, erwies sich als reines Schwefeleisen, in welchem jedoch offenbar das Verhältnis des Schwefels zum Eisen --- das wegen der allzu geringen Menge, die davon zu Gebote stand, nicht genau ausgemittelt werden konnte --- ein ganz anderes ist. Es geht hieraus die Richtigkeit der schon früher gemachten Bemerkung hervor, dass das Schwefeleisen in den Meteor-Massen von ganz eigener und von sehr mannigfaltiger Art sei, und dass man bei dessen Beurteilung nicht von dem für das terrestrische Schwefeleisen festgesetzten Prinzip ausgehen, und vollends bei Bestimmung des quantitativen Verhältnisses desselben nicht stöchiometrisch vorgehen dürfe.\\
Noch muss ich bei dieser Gelegenheit des Resultates eines Versuches erwähnen, welches einerseits die zu vermuten gewesene Zerstörbarkeit dieses Schwefeleisens durch Hitze, andererseits die nicht minder \emph{a priori} wahrscheinlich gewesene, höchst schwere Schmelzbarkeit des Meteor-Eisens bestätiget, und somit meine hin und wieder geäußerten Zweifel gegen die herrschende Meinung, als wären die Meteor-Massen mehr oder weniger das Produkt eines Schmelz-Prozesses, und als kämen die Metall-Massen wohl gar im geschmolzenen Zustande selbst zu Erde, zu bekräftigen scheint. Es wurde nämlich ein drei Quäntchen schweres Stück gewöhnliches, weißes Roheisen, und gleichzeitig ein 1 Linie dickes, 40 Gran wiegendes Plättchen vom Elbogner Eisen, welches von einem, ganz mit solchem Schwefeleisen ausgefüllten Risse durchzogen war, zu schmelzen versucht. Jenes Stück Roheisen schmolz bei ungefähr 130° Wedgd. vollkommen; das Plättchen Meteor-Eisen dagegen blieb ganz unverändert, selbst an den scharfen Kanten und Ecken; aber das im Risse enthalten gewesene Schwefeleisen war ganz zerstört. Und diese Zerstörung nahm selbst schon bei sehr mäßiger Rotglühhitze ihren Anfang. Wie könnten sich demnach die feinen Atome von Schwefeleisen, mit welchen die lockere, poröse Steinmasse der Meteor-Steine, und vollends die Metall-Massen vom Mittelpunkte bis zur äußersten Oberfläche ganz durchsäet sind, so unverändert im metallischen und zum Teil selbst im kristallinischen Zustande erhalten haben, wenn erstere auch nur eine solche Hitze, welche zur Erzeugung der Rinde auf diesem Wege nötig ist, und letztere eine solche --- durchdringende und anhaltende --- welche etwa notwendig sein dürfte, ihre Masse --- oft von mehreren Zentnern --- in Fluss zu bringen, ausgestanden hätten.}

Auf der achten Tafel befindet sich endlich noch eine getreue Darstellung der abgeschliffenen Fläche an dem bereits erwähnten, in der Von der Null'schen Sammlung befindlichen, schönen Ladenstücke\footnote{Es zeichnet sich dieses (28 Loth wiegende) Stück durch ein besonders frisches Ansehen, reine und gute Erhaltung, durch Größe der Metall-Zacken, und vorzüglich durch einen auffallend reichen Gehalt am olivinartigen Gemengteile aus, so dass dieser im Ganzen, dem Volum nach, wohl mehr als der Anteil am Metalle betragen möchte. Obgleich dieser Gemengteil hier --- was sonst nur selten und im Einzelnen der Fall ist --- größten Teils in einem besonders reinen Zustande und hohen Grade von Ausbildung vorkommt, so zeigt er doch eine Menge von Abstufungen darin, und geht --- wie bereits in der Beschreibung desselben bei den Meteor-Steinen bemerkt worden ist, nur in umgekehrter Progression, in entgegen gesetztem Quantitäts-Verhältnisse und gewöhnlich mit Abnahme an Volum der Massen --- aus den lichtesten, blass gelblich-weißen und grünlichen Farben, einerseits durch wachs- und honiggelbe Tinten ins Dunkelbräunliche, Zimtbraune und selbst ins Hyacinthrote, andererseits durch spargel- und pistaziengrüne ins Schmutzig- und Olivengrüne über, und in eben dem Maße nehmen die Grade der Durchscheinenheit, vom vollkommen Durchsichtigen bis zum Undurchsichtigen ab; der Glasglanz nähert sich immer mehr und mehr dem Fettglanze; der Bruch verläuft sich aus dem flachmuschlichten, versteckt-blätterigen, in den ebenen, nicht selten mit deutlich blätterigen, oft selbst schaligen Absonderungen; die scharfkantigen Bruchstücke erscheinen stumpfer; und die Härte sinkt vom Glasritzen bis beinahe zum Weichen herab. Nur höchst selten findet sich, selbst an diesem Stücke, ein einzelnes Korn, wenn nur von einiger Größe, das, zumal im höheren Grade von Reinheit, aus jener Suite von Eigenschaften durchaus nur ein Glied zeigte; gewöhnlich finden sich deren zwei auch drei, oft sehr entfernte, meistens aber ineinander verlaufende, an einem und demselben. Sehr häufig aber, obgleich an diesem Stücke nur wenig und nur stellenweise, dagegen an den meisten Stücken die ich kenne, namentlich dem großen (5 1/2 Pfund schweren), und noch mehr an dem --- angeblich aus Norwegen herstammenden (über 2 Pfund schweren) Stücke der kaiserl. Sammlung; so auch an der, im Museum zu Gotha aufbewahrten, in Sachsen aufgefundenen ähnlichen, ästig-zelligen Eisenmasse, und wie auch Graf Bournon von dem einen größeren, mehrere Pfund wiegenden Stücke der Howard'schen Sammlung bemerkt --- bei weitem vorwaltend, findet sich dieser Gemengteil in ähnlichen, doch meistens in den unvollkommensten oder doch minder vollkommenen Graden von Ausbildung in ganz unförmlichen, größeren und kleineren Körnern und Bruchstücken, zum Teil in, dem Eisenspate in verschiedenen Abstufungen, ungemein ähnlichen Partien von blätterigem Gefüge zusammen gemengt und durch Eisenoxyd zu einer festen, kompakten Masse gleichsam zusammen gekittet, und bildet gewisser Maßen eine Grundmasse, welche von Zacken des Gediegeneisens durchwachsen ist, die hie und da als Spitzen über die Oberfläche hervorragen, aber nur höchst selten, und dann nur unvollkommene, kleine Zellen bilden. Hie und da findet sich an allen größeren Stücken der Art, und namentlich auch an einem großen (3 Pfund 19 Loth schweren) Stücke im Besitze Sr. kaiserl. Hoheit des Erzherzogs Johann (im Johanneo zu Grätz) --- das in Hinsicht auf Frischheit im Ansehen, der guten Erhaltung, der Größe der Metall-Zacken und der Ausgeschiedenheit und Reinheit des Olivines im Einzelnen, jenem aus der Von der Null'schen Sammlung keineswegs nachsteht --- in größeren oder kleineren Partien, eine ganz erdige, trockne, zum Teil ganz zerreibliche, matte, graulichweiße, der Grundmasse der Meteor-Steine vollkommen ähnliche Substanz (wie auch Graf Bournon an jenem größeren Howard'schen Stücke bemerkte), die vielleicht für verwitterte Olivin-Masse angesehen werden könnte, mancher Rücksichten wegen aber wohl richtiger für ursprünglich minder ausgebildeten Olivin, oder für wirkliche, jener der Meteor-Steine ganz ähnliche, Grundmasse zu halten sein möchte. Die abgeschliffene Fläche eines kürzlich erhaltenen kleinen Stückes der Art, welches diese erdige Substanz mit jener unförmlich gemengten, verschieden gefärbten Olivin-Masse in bedeutender Menge und nur mit einzelnen wenigen und zarten Metall-Zacken durchwachsen zeigt, würde jedermann eher für die eines Meteor-Steines, als eines Stückes vom sibirischen Eisen erkennen.} von der sibirischen Eisen-Masse, um das Gefüge zu versinnlichen, welches eine ähnliche Ätzung auf der polierten Oberfläche der durchschnittenen Metall-Zacken zum Vorschein gebracht hat. Es zeigt dasselbe zwar einige Abweichung von jenem obiger derber Eisen-Massen, im Wesentlichen aber doch dasselbe; nämlich: eine wo nicht so regelmäßige und vielfach vereinzelnte, doch eine ähnliche und ebenso scharfe Absonderung von wenigstens zwei heterogen scheinenden metallischen Substanzen. Die Oberfläche eines jeden solchen geätzten Zackens zeigt nämlich, gleichsam als Kern desselben, ein Feld von matter, eisengrauer Farbe, welches von einem zwar nicht immer gleich breiten, aber scharf abgeschnittenen, und selbst durch eine mikroskopisch feine Linie getrennten Saume von spiegelicht glänzender, stark ins Silberweiße fallender Farbe, eingefasst ist, der, indem er die Kante der Fläche oder den Rand des Umrisses vom Zacken selbst bildet, jenes Feld ringsum begrenzt. Die Form dieser Felder ist keineswegs gleichförmig und regelmäßig --- so wie jene der Zwischenfelder oder Figuren an den derben Eisen-Massen zu sein pflegt --- sondern vielmehr höchst verschieden und unbestimmt, indem sie ziemlich genau dem Umrisse des Zackens entspricht. Die Grenzlinie jener Felder folgt nämlich allen Ecken, Krümmungen und Ausbuchten des Zackenrandes, nur mit der Abweichung, dass sie nicht immer gleich weit vom äußersten Rande sich entfernt, so z. B. in den Krümmungen einen verhältnismäßig kleineren Bogen, in den Ecken meistens einen weit spitzeren und mehr gedehnten Winkel bildet; bisweilen macht aber doch der Umriss eines Feldes eine Krümmung oder Ecke, die jenem des Zackens nicht entspricht. Da nun der Saum den Rand des Zackens oder die Kante der Fläche desselben bildet und den Zwischenraum zwischen dieser und dem Felde ausfüllt; so folgt, dass derselbe ungleich breit sein müsse. Im Durchschnitte hat er eine Breite von 1/3 oder 1/4 Linie, oft jedoch kaum von 1/12 Linie; dagegen nicht selten, zumal in den Krümmungen, von einer halben, und in den Ecken bisweilen selbst von einer ganzen Linie.

Wo die Fläche eines Zackens sehr schmal ist, wo nämlich der Schnitt einen Seiten- oder Verbindungsast, oder die Schneide eines liegenden Zackens traf, da zeigt sich kein Feld oder Kern, sondern die Säume von beiden Rändern stoßen zusammen und sind bloß durch eine zarte Linie getrennt; wie sich aber diese Fläche erweitert (was sehr oft bei Seiten- und Verbindungsästen der Fall ist, indem sie sich gegen die Hauptstämme hin verdicken), so trennen sich die beiden Säume und der Kern erscheint als ein grauer Strich, der nach Maßgabe der zunehmenden Breite der Fläche immer breiter und endlich zu einem Felde wird, dessen Umriss wieder jenem der Fläche entspricht.

Da mir die Ätzung an diesem Stücke zu schwach schien, so ersuchte ich Herrn v. Widmannstätten, zum Behufe dieser Ausarbeitung an einem kleinen Stücke von diesem Eisen in meinem Besitze einige abgeschnittene Zacken stärker und bis auf jenen Grad zu ätzen, bis zu welchem jene Flächen obiger derber Eisen-Massen, um eines Abdruckes fähig zu sein, früher von ihm selbst geätzt worden waren. Es zeigte sich nun, dass die Substanz des Kernes, der nun dunkler eisengrau erschien, ganz jener der Figuren oder Zwischenfelder, die des Saumes oder Außenrandes der Zacken aber vollkommen jener der Streifen entspreche, indem sie nun nicht nur in eben dem Grade gegen erstere vertieft, sondern auch von ganz ähnlicher, zinkweißer Farbe und mit gleicher und zwar ziemlich grobnarbiger Oberfläche erschien; und dass endlich jene zarten Linien, welche zuvor zwischen Kern und Saum bemerkt wurden, vollkommen mit den Einfassungsleisten überein kommen, indem sie nun ebenso erhaben und ganz von gleicher Beschaffenheit sich zeigten. Es finden sich demnach auch an dieser Metall-Masse jene drei verschiedenartigen Substanzen, welche bei den derben Meteor-Eisenmassen das beschriebene Gefüge bilden, und zwar ebenso deutlich ausgesprochen und scharf begrenzt und ganz von derselben Beschaffenheit, nur mit dem Unterschiede, dass sie hier nicht mit jener kristallinischen Regelmäßigkeit ausgeschieden und gegenseitig gelagert sind.\footnote{Ein mit einem zweiten ähnlichen Stücke von dieser Masse vorgenommener Versuch zum Blau-Anlaufen durch Erhitzung gab nicht nur ein vollkommen entsprechendes Resultat, sondern brachte auch eine Menge höchst zarter Linien --- Einfassungs- und Schraffierungsleisten --- zum Vorschein, die sich auf dem teils violetten teils dunkelblauen Grunde durch eine schön goldgelbe Farbe auszeichneten, und die, wahrscheinlich ihrer Zartheit wegen, durch die Säure zerstört wurden, daher sich an dem geätzten Stücke nur hie und da Spuren davon finden. Und genau dasselbe zeigte ein Stückchen von jener sächsischen Masse.}

Die Oberfläche jener Zacken, welche nur fein poliert, aber nicht geätzt wurde (welches letztere am Von der Null'schen Stücke --- wie auch aus der Darstellung zu ersehen ist --- nur auf der einen Hälfte der abgeschliffenen Fläche geschah), zeigt von dieser Trennung der Substanzen, in Kern und Saum, Feld und Einfassung, so wie ähnlich behandelte Flächen an den derben Massen, noch keine Spur, sondern es hat dieselbe ein ganz gleichförmiges Ansehen, gleichen spiegelichten Glanz, und eine durchaus gleiche, sehr licht stahlgraue, stark ins Silberweiße fallende Farbe.

Die zerstreut und mechanisch eingemengte, bröcklig-körnige Substanz (das Schwefeleisen) zeigt sich aber hier wie dort und so wie bei jenen derben Massen, sehr deutlich und häufig, so dass sie hier wenigstens den sechsten Teil des gesamten Metall-Anteiles dieser Masse ausmachen dürfte, von welchem sie sich durch ihr körniges oder doch rissiges Ansehen, durch eine Zinkweiße, schwach ins Rötliche ziehende Farbe und durch einen schwächeren Glanz auszeichnet. Sie findet sich teils in einzelnen kleinen und äußerst kleinen Körnern, teils in größeren bröcklig zusammen gehäuften Partien, teils in dichteren, zart rissigen Massen, und zwar meistens am Rande der Zellen, welche durch die Metall-Zacken gebildet werden und den Olivin einschließen, und die sie oft, entweder ganz oder stellenweise und abwechselnd mit dem Eisen und zwischen dieses gleichsam eingekeilt, gleich einer, obgleich ungleichförmigen Einfassung umgibt. Bisweilen bildet sie selbst ganze Nebenzacken, Seiten- oder Verbindungsäste von den Hauptzacken oder Stämmen des Eisens; in jedem Falle ist sie aber immer durch eine zarte Furche von diesem geschieden.
\clearpage
\section{Zehnte Tafel.}
\subsection[Plan der Gegend um Stannern in Mähren.]{Plan der Gegend um Stannern in Mähren,}
\paragraph{}
in der sich, am 22. Mai 1808, jener merkwürdige Steinfall ereignete,\footnote{Umständliche Nachrichten davon --- die Resultate einer schon am sechsten Tage nach dem Ereignisse an Ort und Stelle gemeinschaftlich mit Herrn Direktor v. Widmannstätten und unter Mitwirkung des k. k. Kreisamtes zu Iglau von mir vorgenommenen förmlichen und wissenschaftlichen Untersuchung --- finden sich in Gilberts Annalen der Physik, Bd. 29, Jahrg. 1808. Leider wurde die Fortsetzung derselben --- zu welcher die nicht genug anzurühmende Betriebsamkeit jener Landesbehörde, und insbesondere die, bei dieser Gelegenheit gar sehr in Anspruch genommene, zum Glück durch das eigene Interesse den anziehenden Gegenstand lebhaft angeregte persönliche Aufmerksamkeit von Seite des Herrn Kreishauptmannes, Gubernialrates v. Huß, Materialien zu Genüge geliefert hatten (indem noch im Laufe desselben Jahres eine zwei Mahl wiederholte Durchsuchung des Flächenraumes nach den etwa verborgen liegenden und die sorgfältigste Nachforschung über die bereits aufgefundenen Steine, eine ebenso oftmalige amtliche Einberufung und Vernehmung aller Finder und Beobachter von solchen, und endlich, gemeinschaftlich mit den angrenzenden Kreisämtern, sehr umständliche Nachforschungen über die Ausdehnung und Grenzen einiger, die Begebenheit begleitender, merkwürdiger Nebenerscheinungen vorgenommen und die Resultate davon bereits eingesendet worden waren) --- so wie die Bekanntmachung vieler dahin Bezug habender Untersuchungen, Arbeiten und Versuche (als Fortsetzung jener, welche bereits im 31. Bande desselben Werkes angefangen wurde), durch die ungünstigen Zeitumstände --- den Ausbruch des Krieges von 1809 --- unterbrochen, und durch deren lange Fortdauer und Folgen zuletzt ganz unterbleiben gemacht.} von welchem viele der ausgezeichnetsten Steine hier beschrieben und dargestellt worden sind.

Es erstreckt sich dieser Plan\footnote{Es wurde dieser Plan, auf Anordnung der betreffenden hohen Landesstelle, nach den mitgeteilten Anforderungen durch den Landes-Ingenieur Herrn v. Berniere in Frühjahre 1809 vor Bestellung der Gründe und nachdem alle oben erwähnten Untersuchungen und Nachforschungen bereits vollendet waren, unter Leitung des k. k. Kreisamtes und mit Zuziehung der Ortsobrigkeiten und aller jener Individuen, welche Steine aufgefunden oder im Niederfallen beobachtet hatten, an Ort und Stelle aufgenommen, und nach einem willkürlichen aber bestimmten Maßstabe --- welcher zur gegenwärtigen Kopie genau auf die Hälfte reduziert wurde --- ausgefertiget.} über eine Gegend von 4 Meilen in der Länge (von dem Marktflecken Schelletau in S. bis zur Kreisstadt Iglau in N.), und auf 2 Meilen in der größten Breite (von den Landstädtchen Telsch und Trisch in W. bis zum Dorfe Pirnitz in O.), durch welche die mährisch-böhmische Poststraße, beinahe in gerader Richtung von S. nach N., durch den Marktflecken Stannern zieht, der ziemlich im Mittelpunkte dieses Flächenraumes (in einer Entfernung von 20 Meilen N. W. von Wien, 22 S. O. von Prag, und 13 N. W. von Brünn) liegt.

Es sind in demselben nicht nur alle innerhalb dieses Umkreises befindlichen Ortschaften in ihrer gehörigen Lage aufgeführt, Hügel und Täler, Gehölze, Waldungen, Äcker und Wiesen, Bäche und Teiche, Wege und Fußsteige nach deren verhältnismäßiger Ausdehnung angedeutet, sondern auch die einzelnen Stellen, wo Steine aufgefunden oder im Niederfallen mit Verlässlichkeit beobachtet,\footnote{Solcher, bloß im Niederfallen beobachteter und nicht wirklich aufgefundener Steine, sind in diese Tabelle eigentlich nur zwei aufgenommen worden; nämlich die beiden unter Nr. 30 und 42 im Plane angedeuteten, welche in den einen bei Stannern gelegenen Teich fallen gesehen und gehört wurden.} und wo sie von den Findern oder Beobachtern auf dem Platze selbst angegeben worden waren, mit möglichster Genauigkeit durch Punkte und fortlaufende Zahlen bezeichnet. Letztere beziehen sich auf eine dem Plane beigefügte Tabelle, welche die Nahmen der Deponenten nach ihren Wohnorten und in der Ordnung, nach welcher die amtliche Verhandlung ihrer Vernehmung gepflogen wurde, und das Gewicht der einzelnen Steine, welches teils nach wirklicher Abwiegung, teils nach einer beiläufigen Abschätzung bestimmt worden war, angibt.

Bezeichnet man die Grenzen des von Steinen wirklich befallenen Flächenraumes nach den äußersten Punkten oder den entferntesten Fallsstellen (wie dies auf der Karte durch eine punktierte Linie geschehen ist); so erhält man ein elliptisches Feld,\footnote{In der Voraussetzung --- die übrigens alle Wahrscheinlichkeit für sich hat --- dass die niederfallenden Steine Trümmer oder Bruchstücke einer Masse (des Meteors oder einer, bei solchen Ereignissen gewöhnlich --- wie auch bei diesem --- beobachteten, so genannten Feuerkugel) sind, welche, in Folge wiederholter Zersprengung oder Zerplatzung letzterer während ihres mehr oder weniger horizontalen oder vielmehr (weil sie selbst im Niederfallen ist) parabolischen Zuges durch unsere Atmosphäre, von ihr losgetrennt und nach allen denkbaren Richtungen hinweg geschleudert werden, ist die elliptische (und oft selbst --- wie gerade hier --- vollkommen und zugespitzt eiförmige) Form des Flächenraumes, auf welchen dieselben niederfallen und dessen Grenzen ihre entferntesten Fallstellen bestimmen, sehr begreiflich, und die natürliche Folge teils der Vorwärtsbewegung der Masse selbst, während jener sukzessiven Zersprengung, teils der zusammengesetzten Wurfs- und Fallsbewegung dieser von ihr weggeschleuderten einzelnen Bruchstücke, welche letztere, selbst bei ganz gleichgesetzter Wurfkraft, nach der Richtung, in welcher die Wegschleuderung geschieht, verschieden gedacht werden muss. Ein anderes ist es nämlich, wenn diese Wurfsrichtung mit der eigentümlichen Bewegung der Masse koinzidiert und solcher Gestalt die erhaltene Wurfkraft verstärkt wird, als wenn sie nach seitwärts, nach hinten oder vollends nach abwärts Statt findet, in welchen Fällen der, der Wurfkraft entgegenwirkenden, Schwerkraft der einzelnen Steine weniger Widerstand geboten, oder diese wohl gar selbst verstärkt wird. Im ersteren Falle müssen die Steine ungleich weiter vom Mittelpunkte der Explosion und weit langsamer, nämlich nach Maßgabe der Höhe, auf welcher diese vor sich ging, in einer mehr oder weniger schiefen oder parabolischen Richtung zur Erde kommen; in letzteren Fällen dagegen weit näher jenem Centro oder selbst in demselben, schneller und mehr oder weniger senkrecht niederfallen. Und in dieser so mannigfaltig modifizierten und komplizierten Fallsewegung, vollends aber in der weitern Zersprengung einzelner solcher Bruchstücke während derselben (wofür nicht nur mehrere bei solchen Ereignissen gewöhnlich beobachtete Nebenerscheinungen, sondern --- wie aus obigen Beschreibungen erhellet --- manche Beobachtungen an den Steinen selbst --- zumal rücksichtlich der verschiedenen Beschaffenheit der Oberfläche und der Rinde an einem und demselben Stücke --- zu sprechen scheinen), wodurch sie wieder abgeändert und in eine neue, ähnliche, auf eben die Art und noch mehr komplizierte aufgelöset wird, möchte wohl die Erklärung jenes rätselhaften Umstandes zu suchen sein, dass viele Steine, trotz ihres bedeutenden spezifischen Gewichtes und der beträchtlichen Höhe, in welcher deren Lostrennung von der Masse in den meisten Fällen vorzugehen scheint, so äußerst sanft auffallen, dass sie kaum die Erde aufschürfen, eine Strecke weit fortrollen oder auf weichem, lockern Boden oberflächlich liegen bleiben.} das ziemlich das Mittel jener Gegend einnimmt, den Marktflecken Stannern beinahe zum Mittelpunkte hat, bei 7000 Klafter in der Länge und über 2600 in der größten Breite misst und einen Flächeninhalt von mehr als zwölf Millionen Quadrat-Klafter begreift.\footnote{Da, wie aus dem Folgenden erhellen wird, sowohl der Zahl und Masse als dem Gewichte nach, doch wenigstens zwei Drittel der bei diesem Ereignisse niedergefallenen Steine mit hinlänglicher Verlässlichkeit ausgemittelt und die Fallsstellen derselben angegeben werden konnten, und da vorzüglich auf die Grenzpunkte alle Aufmerksamkeit gerichtet worden war; so dürfte die angegebene Lage, Richtung und Ausdehnung dieses Feldes als ziemlich richtig angenommen werden können.}

Eine innerhalb dieses Feldes von der äußersten Fallsstelle in N. (Nr. 60) bis zur äußersten in S. (Nr. 1) gezogene Linie --- welche eine der Richtung des magnetischen Meridians parallel laufende (vorausgesetzt dass bei Übertragung der Stellung der Magnetnadel auf den Plan die damals Statt gehabte Abweichung gehörig berücksichtigt wurde) unter einem Winkel von etwa 7° durchschneiden möchte --- würde dasselbe der Länge nach in zwei sehr ungleiche Hälften teilen,\footnote{Obgleich, auch in Annahme obiger Voraussetzung, eine solche, die größere Achse der Ellipse, und somit wohl auch beiläufig den Zug des Meteors bezeichnende Linie, überhaupt nur höchst unsicher auf die wahre Bahn des Meteors schließen ließe, indem dies voraussetzen würde, dass die äußersten Punkte derselben durch Steine bestimmt worden wären, die in einer ihr vollkommen entsprechenden Richtung von der Masse abgeschleudert wurden --- was wohl bei einem ähnlichen Vorfalle je erweislich sein möchte --- so wäre dies hier umso weniger zulässlich, da ein Drittel der wahrscheinlich gefallenen Steine, wenigstens ihren Fallsstellen nach, nicht ausgemittelt werden konnten, wovon doch leicht einige --- welches hier, wie aus dem Folgenden erhellen wird, wirklich höchst wahrscheinlich der Fall war --- wenn gleich noch innerhalb des Feldes, doch so zu liegen gekommen sein konnten, dass sie die Richtung jener Linie abändern würden, wenn auch keiner davon, als der Bahn vollkommen entsprechend, den wahren Endpunkt derselben bezeichnet haben sollte.} und eine Linie, welche man quer durch dasselbe, und zwar von der äußersten Fallsstelle in O. (Nr. 51) zur äußersten in W. (Nr. 63) zöge, würde jene etwas über dem Mittel ihrer Länge, dem Nordende etwas näher, durchkreuzen.

Bei einiger Aufmerksamkeit auf die Punkte, welche die Fallsstellen der Steine bezeichnen, bemerkt man bald, dass sie nicht durchaus und gleichförmig über das Feld verbreitet, sondern vielmehr deutlich in drei Gruppen verteilt sind, die durch beträchtliche, ganz freie Zwischenräume voneinander getrennt werden und in deren Mittel sie zum Teil --- wenigstens auf der Karte --- ziemlich dicht erscheinen, indes sie außerhalb desselben sehr weitschichtig und nach allen Richtungen um selbes herum zerstreut vorkommen.\footnote{Diese Gruppen oder partienweisen Steinniederfälle entsprächen nun wirklich den angenommenen sukzessiven Zerplatzungen des Meteors umso mehr, als diese selbst durch ebenso viele Haupt-Detonationen während des Ereignisses, die gleich starken Kanonenschüssen oder gewaltigen Donnerschlägen selbst auf sehr weite Entfernung --- nach gewissen Richtungen auf 10 bis 14 Meilen weit --- ziemlich allgemein vernommen worden waren, bezeichnet wurden; so wie wohl auch die gedrängtere Lage der Fallsstellen unmittelbar und gleichsam im Centro dieser Gruppen, dagegen die weite Zerstreuung vieler anderer, offenbar dazu gehöriger, um dasselbe in sehr verschiedenen Abständen, obige Schlussfolgerung in Betreff der so mannigfaltigen und komplizierten Wurfs- und Fallsbewegung der Bruchstücke, und vollends das eigene, nach einstimmiger Aussage, einem Peloton- oder kleinem Gewehrfeuer ähnliche, fortgesetzte Getöse, die Annahme einer wiederholten Zersprengung vieler einzelner Steine während ihres Falles zu bekräftigen scheinen.}

Die eine Gruppe findet sich am nördlichen Ende des Feldes, bei dem Orte Neustift und zwischen diesem und dem Orte Roschitz, und begreift vier Fallsstellen, die sich ziemlich nahe sind, so dass die einzelnen Steine kaum 300 bis 400 Klafter weit voneinander entfernt zu liegen kamen, und einen Flächenraum von etwa 200,000 Quadrat-Klafter einschließen. Die zweite Gruppe zeigt sich ziemlich genau im Mittel des elliptischen Feldes und ungleich beträchtlicher an Zahl der Fallsstellen sowohl als an Ausdehnung, in und um Stannern, bei Sorez, Falkenau und bis über Mitteldorf und Otten in W. und gegen Hasliz in O. hinaus. Sie begreift 36 Fallsstellen, wovon 16 gewisser Maßen die Hauptgruppe oder das Mittel derselben bilden, die sich zum Teil besonders nahe sind, nämlich auf 100, 200 bis 300 Klafter, und zusammen einen Flächenraum von kaum 600,000 Quadrat-Klafter einschließen; die übrigen liegen mehr zerstreut und in weit größeren Entfernungen, so dass manche 400 bis 600 Klafter voneinander und die äußersten in O. und W. (Nr. 51 und 63) von einem als wahrscheinlich anzunehmenden Mittelpunkte der Gruppe, über 1000 und bei 1600 Klafter abstehen, und so dass alle 36 Fallsstellen einen Flächenraum von nahe an 5 bis 6 Millionen Quadrat-Klafter einnehmen.\footnote{Diese beträchtlichen Fallsentfernungen setzen eine, den einzelnen Bruchstücken mitgeteilte, horizontale Bewegung und eine Wurfkraft voraus, die sich, bei der Höhe in der jene Explosionen, welche selbe bewirken sollen, vorzugehen scheinen, einerseits mit dem spezifischen Gewichte dieser Massen und der daraus resultierenden und jener entgegen wirkenden Schwerkraft, andererseits mit der leichten Zersprengbarkeit, dem lockern Kohäsions-Zustande, in welchem sie wenigstens zur Erde kommen, nicht wohl zusammen reimen lassen. Und noch mehr als diese beträchtlichen Fallsentfernungen der Steine sprechen für die Gewalt, welche die Explosionen der Masse bewirkt, die ausnehmende Stärke und die weite Ausdehnung des Getöses, das dieselben bezeichnet, und welches bei allen ähnlichen Ereignissen in einem ziemlich gleichen Grade und von auffallend gleichförmiger Art beobachtet wurde. So verbreitete sich bei diesem Ereignisse --- nach den Resultaten einer im Laufe desselben Jahres noch von dem Kreisamte zu Iglau einvernehmlich mit den angrenzenden Kreisämtern von Znaim in Mähren, Czaslau und Tabor in Böhmen, und von Korneuburg und Krems in Österreich (deren sowie aller untergeordneten Behörden tätige Mitwirkung über hundert, mit Protokollen und andern Dokumenten belegte, Amtsberichte bewährten) in dieser Beziehung gepflogene Untersuchungsverhandlung --- das Getöse --- wenigstens jenes der Haupt-Detonationen --- von Stannern aus --- den Ort selbst als Mittelpunkt angenommen --- in N. gegen Czaslau auf 4, in O. gegen Brünn auf 8, in S. gegen Stockerau und in W. gegen Tabor selbst bis auf 14 Meilen weit, und zwar mit solcher Stärke noch, dass mit demselben, wenigstens nach jenen weitern Richtungen hin, auf eine Entfernung von 8 bis 12 Meilen von jenem Mittelpunkte, eine Erschütterung der Gebäude und ein Klirren der Fenster bemerkt wurde. (Merkwürdig ist, dass die Grenzen des Flächenraumes, über welchen sich dieses Getöse ausgebreitet hatte, die ich nach den, mit den Berichten erhaltenen, sehr genauen Angaben und nach den bezeichneten Ortschaften auf eine Karte übertrug, eine ähnliche und jener des von den Steinen befallenen Flächenraumes entsprechende Ellipse gaben, deren größere Achse ebenfalls wie die von jener von N. N. W. gegen S. S. O. und derselben sehr parallel lief, und dass damit auch ganz auffallend die Richtung und Ausdehnung des in Begleitung des Phänomenes beobachteten und unbezweifelbar mit demselben zusammenhangenden Nebels überein kam, der nur auf engere Grenzen als das Getöse beschränkt war, indem sich derselbe in S. auf 8, in N. kaum auf 4, in W. nur wenig weiter, in O. nicht einmal so weit erstreckte. Dass der Nebel sowohl als vorzüglich das Getöse sich bedeutend weiter gegen S. und W. als gegen N. und O. ausgedehnt haben, mag wohl Nebenumständen zuzuschreiben sein, die leicht darauf Einfluss gehabt haben konnten, z. B. dem Luftstrome --- obgleich während der Dauer des Ereignisses, so wie selbst den ganzen Tag über, wenigstens in der niedereren Region, die Atmosphäre vollkommen ruhig war --- zum Teil auch dem Niveau des Terrains, das sich gegen O. und vorzüglich gegen N. beträchtlich erhebt --- obgleich diese Erhebung bei der Höhe, in welcher die Explosionen Statt gefunden zu haben scheinen, geradezu keinen großen Einfluss auf die Verbreitung des Schalles gehabt haben kann. ---)\\
Wenn man nun erwäget, dass der Umfang der bei diesem Ereignisse niedergefallenen Masse im Ganzen (als Feuerkugel) --- deren Form als sphäroidisch sich gedacht und die physische Beschaffenheit ihrer Substanz in dem Zustande angenommen, in welchem die einzelnen Steine als Bruchstücke derselben zur Erde kommen --- nach dem wahrscheinlichen absoluten Gewichte von 150 Pfund im Vergleich mit dem spezifischen von 3 : 1 des Wassers, kaum mehr als einen Schuh im Durchmesser (in Dampfgestalt --- bei gewöhnlicher Kompression --- etwa 6000 Kubik-Schuh körperlichen Inhalt) gehabt haben konnte (und bei den meisten ähnlichen Ereignissen muss dieser noch ungleich kleiner gewesen sein, indem die teils im Ganzen teils in nur wenigen einzelnen Stücken herabgefallene Masse oft nur wenige Pfund betrug); so möchte man sich wohl bestimmt finden von der Idee, diese Gewalt als eine bloß mechanische zu betrachten, abzugehen und dieselbe vielmehr als die Wirkung eines uns ganz fremden, großen chemischen Prozesses anzusehen, dessen Resultat vielleicht die Bildung der nächsten Bestandteile, in welchen sich uns die meteorischen Massen, wenn sie einmal zur Erde gekommen sind, zu erkennen geben, aus den uns zur Zeit noch unbekannten Urstoffen (Chladnis Ur-Materie) sein dürfte, und wobei die Explosion und Zertrümmerung der Masse nur Nebenwirkung wäre.} Die äußerste, höchst wahrscheinlich zu dieser Gruppe gehörige Fallsstelle gegen N. (Nr. 55) steht von der äußersten der vorigen Gruppe gegen S. (Nr. 61) über 1000 Klafter ab, so dass im elliptischen Felde zwischen diesen beiden Gruppen ein steinfreier Zwischenraum von wenigstens 2 Millionen Quadrat-Klafter auffällt. Die dritte Gruppe endlich findet sich gegen das südliche Ende des Feldes, zwischen und über den Orten Hungerleiden, Lang- und Klein-Pirnitz, und zeigt sich ebenfalls sehr beträchtlich an Zahl der Punkte und an Ausdehnung. Erstere beläuft sich auf 26, wovon wieder mehrere, zumal 10, sich ziemlich nahe, nur auf 100, 200 Klafter Entfernung voneinander liegen, so dass sie einen Flächenraum von kaum 2 bis 300,000 Quadrat-Klafter einschließen. Die übrigen liegen wieder mehr zerstreut und entfernter voneinander, so dass alle zusammen einen Flächenraum von etwa 2 bis 3 Millionen Quadrat-Klafter einnehmen möchten. Diese Gruppe hat sich übrigens mehr in die Länge als in die Breite ausgedehnt,\footnote{Wahrscheinlich weil der Stein Nr. 1 ziemlich horizontal und der ursprünglichen Bewegung des Meteors entsprechend, folglich mit verstärkter Wurfkraft, vorwärts geschleudert wurde, daher die schiefste Richtung oder längste Parabel im Falle beschrieb und folglich am weitesten flog, wie dessen Fallsstelle denn auch in gerader Linie bei 1600 Klafter vom annehmbaren Centro dieser Gruppe entfernt liegt.} denn der entfernteste Fallspunkt gegen S. (Nr. 1) --- der überhaupt auch sehr weit, bei 1600 Klafter, vom wahrscheinlichen Mittelpunkte derselben sich befindet --- ist vom äußersten dieser Gruppe gegen N. (Nr. 11. b.) auf 2200 Klafter entfernt, indes der äußerste gegen W. (Nr. 25) vom äußersten gegen O. (Nr. 18) nur 1300 Klafter absteht. Jener äußerste dieser Gruppe gegen N. (Nr. 11. b.) ist von dem äußersten der vorigen, mittleren Gruppe gegen S. (Nr. 62. b.) ebenfalls auf beinahe 1000 Klafter entfernt, so dass demnach auch hier, wie zwischen letzterer und der ersten Gruppe am Nordende, ein ähnlicher steinfreier Zwischenraum von beiläufig 2 Millionen Quadrat-Klafter bemerkbar wird.\footnote{Die Gleichheit dieser steinfreien Räume zwischen den Gruppen, so wie die der Abstände dieser voneinander, sowohl ihrem wahrscheinlichen Mittelpunkte als ihren Endfallsstellen nach, ist meines Erachtens sehr merkwürdig, indem sie auf gleiche Intervalle zwischen den Explosionen schließen lässt, welche übrigens auch die Aussagen über das Vernehmen der drei Haupt-Detonationen der Dauer der Zwischenzeit nach, bestätigten.}

Jene durch die äußersten Fallsstellen in N. und S. --- nach Angabe des Planes --- der Länge nach durch das Feld gezogene Linie durchschneidet eben so wenig den als wahrscheinlich anzunehmenden Mittelpunkt dieser Gruppen als jenen des Feldes im Ganzen; um diesen Forderungen zu entsprechen, müsste eine andere angenommen werden, welche in N. um einige Grade mehr westlich fiele, welches den, in mehrfacher Beziehung auch wirklich sehr wahrscheinlichen Umstand voraussetzen würde, dass am Nordende des Feldes noch einige Steine, gegen Willenz und Porenz zu, gefallen wären, die nicht zur Notiz kamen oder deren Fallsstellen wenigstens nicht ausgemittelt werden konnten.\footnote{Obgleich nach alle dem, was bereits über die Explosionen des Meteors und über die so mannigfach modifizierte und komplizierte Wurfs- und Fallsbewegung, mit welcher die von demselben weggeschleuderten Bruchstücke zur Erde kommen, als wahrscheinlich vorgebrach worden ist, es wohl unmöglich sein dürfte, den wahren Mittelpunkt dieser Gruppen von Fallsstellen, und noch mehr jenen, diesen in senkrechter Höhe zentrierenden der Explosionen selbst, mit voller Zuversicht zu bestimmen; so kann man doch mit aller Wahrscheinlichkeit ersteren dort annehmen, wo die meisten Fallsstellen und diese im Durchschnitte am dichtesten beisammen liegen, letzteren aber etwas hinter diesem Punkte, da wohl vorausgesetzt werden kann, dass die eigentümliche Bewegung der Masse auf alle von ihr getrennten Bruchstücke mehr oder weniger Einfluss gehabt habe. Der Mittelpunkt der mittleren Gruppe möchte demnach etwa 600 Klafter O. N. O. von der Kirche von Stannern und etwa 1000 Klafter N. von Sorez und 600 S. S. W. von Falkenau zu setzen sein; so dass der äußerste Stein dieser Gruppe in W. 15 bis 1600, der äußerste in O. 1000 bis 1100, der südlichste 12 bis 1600, der nördlichste etwa 800 Klafter davon entfernt zu liegen käme; Entfernungen die den denkbaren Wurfs- und Fallsbewegungen der dieser Gruppe angehörigen Bruchstücke ganz gut entsprechen möchten. Der Mittelpunkt der dritten Gruppe könnte sich um die Fallsstelle Nr. 11 a. gedacht werden, etwa 400 Klafter N. von Lang-Pirnitz, 2800 Klafter vom Mittelpunkte der vorigen; so dass der äußerste Stein in W. etwa 800, der in O. 500, der südlichste 1600, der nördlichste 600 Klafter davon zu liegen käme. Zieht man nun eine gerade Linie durch diese beiden Punkte und verlängert sie bis ans Nordende des elliptischen Feldes; so würde ihr Endpunkt hier gegen Willenz zu, etwa 600 Klafter östlich von diesem Orte, und etwa 200 Klafter W. N. W. von der Fallsstelle Nr. 60 fallen. Diese Linie --- welche etwa um 3 oder 4 Grade von der hier angegebenen Richtung des magnetischen Meridianes abwiche --- würde nun nicht nur die beiden in Hinsicht der gefallenen Steine am besten beobachteten und den Fallsstellen nach am genauesten ausgemittelten Gruppen in ihrem wahrscheinlichen Mittelpunkte durchschneiden, sondern auch die Zahl der Fallsstellen und selbst den befallenen Flächenraum --- obgleich letzteres von weniger Belang ist -- in zwei ziemlich gleiche Hälften teilen, und somit mit vieler Wahrscheinlichkeit als die wahre Bahn des Meteors bezeichnend angesehen werden können. Es würde dieselbe nur voraussetzen, dass auf jenen durch sie bezeichneten Punkt, oder vielmehr noch mehr westlich, gegen Willenz oder Porenz zu, einige Steine mit der ersten Explosion innerhalb der Grenzen des elliptischen Feldes gefallen seien. Und dies war höchst wahrscheinlich wirklich der Fall; denn nicht nur, dass von dieser Gruppe nur vier Steine ausgemittelt werden konnten, deren doch, im Verhältnis zur Zahl und Masse der übrigen, nicht gar so wenige gefallen sein können, und dass deren Fallsstellen so nahe beisammen und alle nach einer Seite hin liegen, so dass kaum ein Mittelpunkt oder eine Durchschnittslinie, am wenigsten eine solche denkbar wäre, welche jener der übrigen Gruppen nur einiger Maßen entspräche; so ist es auch sehr möglich, dass in dieser Gegend mehrere Steine unbeobachtet niederfielen oder nicht aufgefunden wurden, da diese Gegend weit weniger bevölkert ist und während der Momente des Ereignisses beinahe ganz von den Einwohnern verlassen war, die sich eben auf dem weiten Wege zur Kirche nach Stannern befanden; auch hat sich im Verfolg der Nachforschungen ergeben, dass hier nach der Hand wirklich noch einige Steine aufgefunden wurden, von welchen aber keine nähere Notiz erhalten werden könnte, so wie auch gleich Anfangs am Tage der Begebenheit selbst, mehrere Steine und Bruchstücke von Fuhrleuten, die gerade dieses Weges und namentlich von Willenz kamen und weiter zogen, von daher nach Stannern gebracht, daselbst gezeigt und weiter mitgenommen wurden, daher auch diese einer späteren Notiznehmung entgingen. Sowohl in diesem präsumtiven als in dem bestehenden Falle --- wie ihn inzwischen der Plan ausweiset --- würde der Mittelpunkt dieser Gruppe von jenem der zweiten oder mittleren in einem ähnlichen Abstande, d. i. von beiläufig 2600 bis 2800 Klafter, wie der von dieser zu jenem der dritten Gruppe zu liegen kommen, was auch die oben bemerkte Gleichheit der steinfreien Räume zwischen denselben und des Abstandes der Gruppen \emph{en masse}, so wie die Gleichheit des Zeit-Momentes im Vernehmen der, die Explosionen bezeichnenden Detonationen vermuten ließen. Der von dem Meteore während diesen Explosionen, die jene Steingruppen als Produkt gaben, auf seinem Zuge zurückgelegte Raum, würde demnach eine Strecke von 5 bis 6000 Klafter in gerader Linie betreffen, und danach einstimmigen Aussagen so vieler Augen- und Ohrenzeugen des Phänomenes, das begleitende Getöse im Ganzen 6 bis 8 Minuten dauerte, so bestimmt diese Dauer beiläufig den Zeitraum, welchen das Meteor brauchte, jene Strecke zurückzulegen; die Schnelligkeit der Bewegung scheint demnach nicht ausnehmend groß gewesen zu sein, man mag die Höhe auch als noch so beträchtlich und die Richtung des Zuges auch noch so schief oder parabolisch annehmen, auch wohl voraussetzen, dass die Zeitschätzung, wie kaum zu bezweifeln ist, um vieles zu hoch ausgefallen sein möchte.}

Aus einer Übersicht der dem Plane beigefügten Tabelle ergibt sich, dass die vier am nördlichen Ende des elliptischen Feldes niedergefallenen und die erste Gruppe bildenden Steine alle ansehnlich groß und gewichtig waren (der größte von 13 Pfund, der kleinste --- der wohl nur ein Bruchstück eines später im Falle zersprungenen Steines gewesen sein dürfte --- von 28 Loth), und zusammen bei 27 Pfund wogen. Die 36 Steine der mittleren oder zweiten Gruppe betragen dagegen am Gewichte zusammen nur etwas über 55 Pfund, und es waren meistens kleinere oder doch nur mittelgroße Steine, im Durchschnitte von 1 bis 3 Pfund (nur 8 von 2 Pfund und darüber, 3 von 3 und 2 von 4 Pfund, der größte von 4 1/2 Pfund; dagegen aber auch keiner unter 8 Loth, nur 8 unter 16 Loth, 13 unter einem Pfund). Jene, die dritte, südliche Gruppe bildenden 26 Steine endlich geben ein Gesamtgewicht von kaum mehr als 11 Pfund und waren fast durchgehends kleine und sehr kleine Steine, im Durchschnitte von 7 bis 12 Loth (12 davon unter 8, 7 unter 16 Loth, nur einer von 1 3/4, der größte etwas über 2 Pfund) der kleinste hier aufgezeichnete wog 3 1/2 Loth, und ohne Zweifel sind hier noch weit kleinere gefallen, die aber entweder nicht aufgefunden oder der Aufzeichnung nicht wert befunden wurden, wie dies die beiden auf der fünften Tafel, Fig. 3 und 4 abgebildeten, der Anzeige nach aus dieser Gegend herstammenden und folglich zu dieser Gruppe gehörigen Steine bewähren, wovon der eine kaum 2 1/2 Quäntchen, der andere kaum 56 Gran wiegt.\footnote{Den sprechendsten Beleg für die Richtigkeit dieses merkwürdigen Umstandes, der sich uns bei der Untersuchung des Ereignisses an Ort und Stelle sogleich bemerkbar machte (so wie er bereits von dem französischen Physiker Biot --- bei Gelegenheit der wissenschaftlichen Untersuchung des Steinfalles bei L'Aigle --- bemerkt worden war) --- dass nämlich an dem einen Ende der großen Achse der Ellipse, und zwar --- nach dem bei diesen beiden Ereignissen mit aller Verlässlichkeit beobachteten Zuge des Meteors und der ganzen Erscheinung --- mit der ersten Explosion, meistens große und darunter die größten Steine, am entgegen gesetzten dagegen, mit der letzten Explosion, meistens kleine und die kleinsten, im Mittel und als Produkt der zweiten Explosion, aber meistens mittelgroße Steine fielen (woraus allenfalls zu schließen wäre, dass die Masse Anfangs zäher und schwerer zersprengbar gewesen sein möchte) --- und zugleich für die Genauigkeit des Planes und der Tabelle (deren mittel- und unmittelbare Zustandebringer denn doch keine entfernte Ahndung dieses Umstandes hegen konnten), geben die wirklich vorhandenen, größten Teils lange vor der Zustandebringung jener und unmittelbar aus erster Hand erhaltenen, oben beschriebenen unverbrochenen Steine von diesem Ereignisse, deren angegebene Fallsstellen genau den Erwartungen (so wie auch vollkommen den über manche persönlich eingeholten Privat-Notizen) entsprachen, zu welchen das respektive Volumen und Gewicht derselben in diesen Beziehungen berechtigten. So ist der auf der vierten Tafel abgebildete größte Stein zunächst der äußersten Fallsstelle am nördlichen Ende des Flächenraumes, als Glied der ersten Gruppe unter Nr. 59 angezeigt, so finden sich die Fig. 5 auf der fünften und Fig. 1 bis 4 auf der sechsten Tafel abgebildeten größeren Steine sämtlich im Mittel des elliptischen Feldes, als Produkt der zweiten Explosion unter Nr. 45, 26, 35, 43 und 40; dagegen die kleineren Fig. 1. 2. der fünften Tafel, im südlichen Ende und unter der letzten Gruppe des Feldes unter Nr. 19 und 16 angedeutet, und die beiden kleinsten, Fig. 3. 4. derselben Tafel, wenigstens als in dieser Gegend, nämlich in der Nähe von Lang-Pirnitz aufgefunden, angegeben.}

Die Tabelle weiset übrigens 63 Finder und 66 Steine, und von letzteren ein Gesamtgewicht von 93 Pfund aus. Ich hatte bereits in der ersten von diesem Ereignisse gegebenen Nachricht, nach den Resultaten der an Ort und Stelle gepflogenen Untersuchung und nach Erwägung aller Umstände und Verhältnisse, die Total-Zahl der gefallenen Steine auf 100 Stück und das Gesamtgewicht derselben auf 150 Pfund geschätzt, obgleich ich damals nur von 40 aufgefundenen verlässliche Notiz, und trotz des angelegentlichsten Einsammelns der bereits aufgefunden gewesenen und der eifrigsten Betreibung des Aufsuchens der liegen gebliebenen Steine durch zwei Tage, nur 61 Stück (wovon die meisten nur Fragmente waren), am Gewichte zusammen bei 27 Pfund, aufbringen konnte,\footnote{Die Umstände und Verhältnisse, welche damals --- als noch von Seite des, weder durch besondere Neugierde, noch weniger durch Eigennutz gereitzten Landvolkes jener Gegend keine absichtliche Verheimlichung oder Zurückhaltung der aufgefundenen Steine zu besorgen war, indem man vielmehr das Förmliche der Verhandlung, das Angelegentliche des Aufsuchens und Eintreibens solcher an sich ganz wertloser (der vorherrschenden Meinung nach für angebrannte Mauerstücke eines in die Luft gesprengten Pulver-Magazines angesehene) Steine und die Vergütung für die dabei bezeigte Willfährigkeit und Bemühung höchst sonderlich fand --- bei jener Abschätzung Berücksichtigung heischten, waren: die physische und agronomische Beschaffenheit des Flächenraumes --- der zum Teil mit Gehölze und Waldungen bedeckt, größten Teils aber bebauet und in dieser Jahreszeit von der bereits [...] Saat bedeckt, das Auffinden der Steine schwierig machte --- ferner der Umstand, dass gerade während des Verlaufes des Begebenheit das gesamte Landvolk aus den umliegenden und gerade aus den, den befallenen Flächenraum begrenzenden Ortschaften auf den Gange zur Pfarrkirche nach Stannern begriffen, größten Teils schon in dieser Gegend versammelt, und von ersteren, zumal den entlegeneren in N. und S. schon ziemlich entfernt und mit dem Rücken dahin gekehrt war --- so dass folglich in jenem beschränkteren Bezirke die meisten der gefallenen Steine im Nieder- oder Auffallen (insofern dies an sich --- wie wirklich mehrenteils --- der Fall war) beobachtet und daher gleich aufgehoben oder auch später noch bald aufgefunden, in ersteren Gegenden dagegen nur wenige, von den zurück gebliebenen und zufällig gerade im Freien sich befindenden Bewohnern, im Falle bemerkt und daher mehrenteils nach der Hand nur zufällig oder durch absichtliches Aufsuchen, gefunden werden konnten. (Diess bewährte sich auch in der Folge durch die später aufgefundenen und eingelieferten Steine, die alle aus diesen Gegenden herstammten, so wie durch manche andere, die mir in dieser Zwischenzeit mittel- oder unmittelbar zu Gesicht und Kenntnis kamen, welche --- in Folge der allmählichen Aufklärung und des gereitzten Eigennutzes späterer Finder --- auf Nebenwegen in fremden Besitz geraten waren; und ohne Zweifel kommen, wo nicht alle, doch die meisten Steine jenes Drittels der wahrscheinlichen Total-Zahl, die, wenigstens den Findern und den Fallstellen nach, nicht mit Verlässlichkeit ausgemittelt werden konnten, dahin zu versetzen.) Ein dritter zu berücksichtigender Umstand war endlich jener, dass eben während und unmittelbar nach der Begebenheit --- die durch das Wunderbane und Lärmende die ganze Gegend auf ziemlich weite Entfernung, wenigstens für den ersten Tag, in Angst und Staunen versetzte --- mehrere Reisende auf der Poststraße ab und zu, und mehrere Fuhrleute des Weges von Willenz her, durch diese Gegend kamen, welche teils selbst aufgefundene, teils mitgeteilt erhaltene Steine, die ihnen denn doch mehr oder weniger sonderbar (und vielleicht nicht ganz so wie angebrannte Mauerstücke) vorgekommen sein mochten, der Merkwürdigkeit wegen und als Beleg der von ihnen ganz oder zum Teil beobachteten Erscheinung, mit sich fortnahmen. Einem dieser Reisenden --- dessen Aufmerksamkeit glücklicherweise lebhafter und von einer richtigeren Ansicht aus angeregt wurde, obgleich derselbe weder Augen- noch Ohrenzeuge des Phänomenes, sondern nur Zeuge des ersten Eindruckes war, den dasselbe einige Stunden früher unter den Bewohnern einer so beträchtlichen Strecke allgemein verbreitet hatte, und ein Bruchstück eines der gefallenen Steine mitgeteilt erhielt --- hatte man auch die erste, verlässliche und hinlänglich frühzeitige Nachricht von diesem Ereignisse zu verdanken, ohne welche die Notiz davon --- wie von so vielen ähnlichen --- höchst wahrscheinlich auf jene Gegend und den schnell verlöschenden Eindruck beschränkt, für die Geschichte und für die Wissenschaft unbenutzt geblieben wäre.} und ich fand auch späterhin, nach den nachträglich erhaltenen Notizen, keine Ursache davon abzugehen. Durch dieses Verzeichnis, das beinahe ein Jahr später zu Stande gebracht wurde --- nachdem zu zwei verschiedenen und den günstigsten Perioden (zur Schnitt- und Herbstbestellungszeit der Gründe) das sorgfältigste Aufsuchen der etwa verborgen liegenden Steine in der ganzen Gegend veranlasst und alle Individuen, welche seit dem Momente des Ereignisses Steine aufgefunden hatten, oder auch nur um die Auffindung von welchen wussten, zu wiederholten Mahlen amtlich vernommen worden waren --- erhielt ich vollends in jeder Beziehung die vollkommenste Bestätigung, und finde auch jetzt, nach einer Zwischenzeit von 12 Jahren, während welcher ich nicht versäumte, mittel- und unmittelbar meine Nachforschungen über die Besitzverbreitung der Steine von diesem Ereignisse fortzusetzen, keinen Grund, jene Annahme auch nur im geringsten abzuändern.\footnote{Unter mehr denn 40 ähnlichen Ereignissen, die sich in der neuesten Zeit, den letzten 30 Jahren, zutrugen und wovon wir nähere und verlässliche Nachrichten haben, war dieses --- sowohl der Masse im Ganzen als der Zahl der gefallenen Steine nach --- eines der bedeutendsten und ergiebigsten; nur jenes, das 1790 in der Gegend von Barbotan, und vollends jenes, welches 1803 bei L'Aigle in Frankreich sich ergab, haben dasselbe in beiden Rücksichten übertroffen, und jene die sich, 1794 zu Siena im Toskanischen, 1807 zu Weston in Nord-Amerika, 1812 zu Toulouse in Frankreich, 1813 zu Limerick in Irland und 1814 zu Agen in Frankreich zutrugen, möchten demselben gleich gestellt werden dürfen.}

Wenn man nun bedenkt, dass jene 100 Steine einen Flächenraum von mehr denn zwölf --- oder wenn man streng sein und die steinfreien Räume zwischen den Gruppen in Abrechnung bringen will --- doch wenigstens von acht Millionen Quadrat-Klafter trafen, und dass, selbst da wo sie am dichtesten fielen, die einzelnen doch 100 bis 300 Klafter voneinander zu liegen kamen\footnote{Diese weitschichtige Zerstreuung der Steine über den befallenen Flächenraum bei Ereignissen der Art, wo deren mehrere und wo sie selbst in großer Menge fielen, wie z. B. bei jenen von L'Aigle, wo zwischen 2- und 3000 über einen Flächenraum von 2 1/2 franz. Meilen in der Länge und von einer in der Breite verteilt waren; scheint einerseits auf eine beträchtliche Höhe, in welcher die Explosionen vor sich gehen, hinzudeuten, andererseits aber wohl auch die im Obigen vorausgesetzte, höchst mannigfaltig und vielseitig wirkende Wurfkraft zu bestätigen.}; so wird man es wohl nicht gar so sonderbar finden, dass bei solchen Ereignissen, wenn sich dieselben selbst bei Tage und in bewohnten Gegenden zutragen, so selten Menschen oder Vieh von den Steinen getroffen werden, so wie man wohl auch in dieser Hinsicht den Ausdruck Steinregen für nicht ganz passend erachten möchte.
\clearpage
\section{Erklärung der Titel-Vignette.}
\paragraph{}
Es stellt dieselbe einen massiven, rein aus einem Stücke von der Elbogner Meteor-Eisen-Masse geschnittenen, und ohne alle Hämmerung, bloß durch Feilen zu Stande gebrachten Siegel-Ring von antiker Form vor, dessen Oberteil oder Kranz eingedreht worden war, und der dann im Ganzen fein poliert und auf der Siegelfläche mit Salpetersäure auf eine solche Tiefe geätzt wurde, dass sich das bei solcher Behandlung zum Vorschein kommende kristallinische Gefüge der Metall-Masse nicht nur deutlich und vollkommen erkennen, sondern sich auch als Zeichnung im geschmolzenen Siegellacke gut abdrucken lässt. Als Devise ist auf jene Fläche ein Pfeil --- das in der chemischen Bildersprache zur Bezeichnung des Eisens (dem Mars geweiht) übliche Symbol --- in Verbindung mit einem Sterne, in der Richtung des Falles --- als ein die Natur und zugleich den Ursprung der Masse bezeichnendes Sinnbild --- graviert worden.

Diesem Ringe zur Seite sind zwei Würfel --- jeder von beiläufig 4 Linien Seite, und woran an dem einen noch eine der Ecken abgestumpft worden war --- dargestellt, welche ebenfalls rein aus einem Stücke jener Masse winkelrecht geschnitten, eben und scharfkantig zugefeilt, fein poliert, und dann im Ganzen, auf allen Flächen und Kanten zugleich, auf einen ähnlichen Grad geätzt wurden, um die verhältnismäßige Tiefe zu zeigen, auf welche die verschiedenen, das Gefüge der Masse konstituierenden Teile in dieselbe eindringen, und wie diese, auf einer bestimmten und im Ganzen gleichförmigen Tiefe von der Oberfläche, gegen einander gelagert, unter einander verbunden und voneinander geschieden sind. (Siehe Seite 81, Note 2.)
\clearpage
\setlength\intextsep{0pt}
\pagestyle{fancy}
\fancyhf{}
\rhead{Titel-Vignette.}
\cfoot{\thepage}
\begin{figure}[p]
\tiny Elbogen.
\includegraphics[width=\textwidth,keepaspectratio]{Figures/Elbogen.jpeg}
\end{figure}
\clearpage
\rhead{Tafel 1.}
\cfoot{\thepage}
\begin{figure}[p]
\tiny Agram.
\includegraphics[width=\textwidth,keepaspectratio]{Figures/Table1-Agram.png}
\end{figure}
\clearpage
\rhead{Tafel 2.}
\cfoot{\thepage}
\begin{figure}[p]
\includegraphics[scale=0.65,keepaspectratio]{Figures/Table2-Tabor.png}\tiny Tabor.
\includegraphics[scale=0.65,keepaspectratio]{Figures/Table2-Eichstädt.png}\tiny Eichstädt.
\includegraphics[scale=0.65,keepaspectratio]{Figures/Table2-LAigle.png}\tiny L'Aigle.
\includegraphics[scale=0.65,keepaspectratio]{Figures/Table2-Siena.png}\tiny Siena.
\end{figure}
\clearpage
\rhead{Tafel 3.}
\cfoot{\thepage}
\begin{figure}[p]
\tiny Lissa.
\includegraphics[width=\textwidth,keepaspectratio]{Figures/Table3-Lissa.png}
\end{figure}
\clearpage
\rhead{Tafel 4.}
\cfoot{\thepage}
\begin{figure}[p]
\tiny Stannern.
\includegraphics[width=\textwidth,keepaspectratio]{Figures/Table4-Stannern.png}
\end{figure}
\clearpage
\rhead{Tafel 5.}
\cfoot{\thepage}
\begin{figure}[p]
\includegraphics[scale=0.6,keepaspectratio]{Figures/Table5-Stannern-1.png}\tiny 1.
\includegraphics[scale=0.6,keepaspectratio]{Figures/Table5-Stannern-3.png}\tiny 3.
\includegraphics[scale=0.6,keepaspectratio]{Figures/Table5-Stannern-2a.png}\tiny 2a.
\includegraphics[scale=0.6,keepaspectratio]{Figures/Table5-Stannern-2b.png}\tiny 2b.
\includegraphics[scale=0.6,keepaspectratio]{Figures/Table5-Stannern-4.png}\tiny 4.
\includegraphics[scale=0.6,keepaspectratio]{Figures/Table5-Stannern-5.png}\tiny 5.
\end{figure}
\clearpage
\rhead{Tafel 6.}
\cfoot{\thepage}
\begin{figure}[p]
\includegraphics[scale=0.6,keepaspectratio]{Figures/Table6-Stannern-1.png}\tiny 1.
\includegraphics[scale=0.6,keepaspectratio]{Figures/Table6-Stannern-2.png}\tiny 2.
\includegraphics[scale=0.6,keepaspectratio]{Figures/Table6-Stannern-3.png}\tiny 3.
\includegraphics[scale=0.6,keepaspectratio]{Figures/Table6-Stannern-4.png}\tiny 4.
\includegraphics[scale=0.6,keepaspectratio]{Figures/Table6-Stannern-5.png}\tiny 5.
\end{figure}
\clearpage
\rhead{Tafel 7.}
\cfoot{\thepage}
\begin{figure}[p]
\includegraphics[scale=0.7,keepaspectratio]{Figures/Table7-Charsonville.png}\tiny Charsonville.
\includegraphics[scale=0.7,keepaspectratio]{Figures/Table7-Stannern-2.png}\tiny Stannern.
\includegraphics[scale=0.7,keepaspectratio]{Figures/Table7-Stannern.png}\tiny Stannern.
\includegraphics[scale=0.7,keepaspectratio]{Figures/Table7-Sales.png}\tiny Salés.
\includegraphics[scale=0.7,keepaspectratio]{Figures/Table7-Benares.png}\tiny Benares.
\includegraphics[scale=0.7,keepaspectratio]{Figures/Table7-Timochin.png}\tiny Timochin.
\includegraphics[scale=0.7,keepaspectratio]{Figures/Table7-Siena.png}\tiny Siena.
\end{figure}
\clearpage
\rhead{Tafel 8.}
\cfoot{\thepage}
\begin{figure}[p]
\includegraphics[scale=0.9,keepaspectratio]{Figures/Table8-pallas.jpeg}\tiny Sibirischen.
\includegraphics[scale=0.9,keepaspectratio]{Figures/Table8-2.png}\tiny Mexiko.
\includegraphics[scale=0.9,keepaspectratio]{Figures/Table8-4.png}\tiny Agram.
\includegraphics[scale=0.9,keepaspectratio]{Figures/Table8-3.png}\tiny Lénarto.
\end{figure}
\clearpage
\rhead{Tafel 10.}
\cfoot{\thepage}
\begin{figure}[p]
\includegraphics[scale=0.54,angle=90,origin=c,keepaspectratio]{Figures/Table10.png}
\end{figure}
\clearpage
\end{document}
